\section{离散系统}

\subsection{离散系统概述}

\subsubsection{离散系统的定义}

\textbf{离散系统}(数字控制系统)是指信号在时间上离散,通常涉及采样、数字处理和保持等环节。

\begin{minipage}[t]{0.52\textwidth}
\textbf{连续系统 vs 离散系统:}

\textbf{连续系统:}
\begin{itemize}
    \item 信号连续变化
    \item 微分方程描述
    \item 拉普拉斯变换分析
\end{itemize}

\textbf{离散系统:}
\begin{itemize}
    \item 信号在采样时刻定义
    \item 差分方程描述
    \item Z变换分析
\end{itemize}

\vspace{0.3cm}
\textbf{离散系统的组成:}

\begin{enumerate}
    \item \textbf{采样器}:将连续信号转换为离散序列
    \item \textbf{数字控制器}:对离散信号进行数字处理
    \item \textbf{保持器}:将离散信号还原为连续信号
    \item \textbf{被控对象}:连续系统
\end{enumerate}

\vspace{0.3cm}
\textbf{采样定理(Nyquist-Shannon定理):}

为无失真恢复连续信号,采样频率必须:
\begin{align*}
f_s \geq 2f_{\max}
\end{align*}

或采样周期:
\begin{align*}
T \leq \frac{\pi}{\omega_{\max}}
\end{align*}

其中 $f_{\max}$ 是信号的最高频率分量。

\textbf{工程实践:}
\begin{itemize}
    \item 一般取 $f_s = (6 \sim 10) f_{\max}$
    \item 或采样周期 $T = (0.1 \sim 0.5) T_s$($T_s$ 为系统时间常数)
\end{itemize}
\end{minipage}\hfill
\begin{minipage}[t]{0.45\textwidth}
\vspace{0pt}
\textbf{离散控制系统结构:}

离散控制系统包含以下环节:
\begin{itemize}
    \item 采样器(周期 $T$)
    \item 数字控制器
    \item 保持器(通常为零阶保持器)
    \item 被控对象(连续系统)
\end{itemize}

采样器将连续信号在离散时刻采样,数字控制器对离散信号进行处理,保持器将离散信号还原为阶梯波形式的连续信号,最后作用于被控对象。

\vspace{0.3cm}
\textbf{采样过程:}

采样过程将连续信号 $r(t)$ 在时刻 $t = kT$($k=0,1,2,\ldots$)进行采样,得到离散序列 $r(kT)$。
\end{minipage}

\subsection{Z变换}

Z变换是分析离散系统的数学工具,类似于连续系统中的拉普拉斯变换。

\subsubsection{Z变换的定义}

对于离散序列 $\{f(kT)\}$($k = 0, 1, 2, \ldots$),其Z变换定义为:
\begin{align*}
F(z) = \mathcal{Z}\{f(kT)\} = \sum_{k=0}^{\infty} f(kT) z^{-k}
\end{align*}

其中 $z$ 是复变量。

\subsubsection{Z变换与拉氏变换的关系}

对于采样信号 $f^*(t) = \sum_{k=0}^{\infty} f(kT)\delta(t-kT)$,其拉氏变换为:
\begin{align*}
F^*(s) = \sum_{k=0}^{\infty} f(kT) e^{-skT}
\end{align*}

令 $z = e^{sT}$,则:
\begin{align*}
F(z) = F^*(s)\Big|_{z=e^{sT}}
\end{align*}

\textbf{$s$ 平面与 $z$ 平面的映射:}

\begin{center}
\begin{tikzpicture}[scale=0.7]
% s平面
\begin{scope}[shift={(0,0)}]
\draw[->] (-2,0) -- (2,0) node[right] {$\text{Re}(s)$};
\draw[->] (0,-2) -- (0,2) node[above] {$\text{Im}(s)$};
\draw[blue, thick] (-2,-2) -- (-2,2);
\draw[red, thick] (0,-2) -- (0,2);
\node[blue, left] at (-2,1.5) {不稳定};
\node[red, right] at (0,1.5) {稳定};
\node[below] at (0,-2.5) {$s$ 平面};
\end{scope}

% 箭头
\draw[->, thick] (2.5,0) -- (3.5,0) node[midway, above] {\small $z=e^{sT}$};

% z平面
\begin{scope}[shift={(6,0)}]
\draw[->] (-2,0) -- (2,0) node[right] {$\text{Re}(z)$};
\draw[->] (0,-2) -- (0,2) node[above] {$\text{Im}(z)$};
\draw[thick] (0,0) circle (1);
\fill[blue, opacity=0.2] (0,0) circle (1);
\node[red] at (1.5,0) {不稳定};
\node[blue] at (0.5,0.5) {稳定};
\node[below] at (0,-2.5) {$z$ 平面};
\end{scope}
\end{tikzpicture}
\end{center}

\textbf{关键对应关系:}
\begin{itemize}
    \item $s$ 平面左半平面 $\leftrightarrow$ $z$ 平面单位圆内
    \item $s$ 平面虚轴 $\leftrightarrow$ $z$ 平面单位圆上
    \item $s$ 平面右半平面 $\leftrightarrow$ $z$ 平面单位圆外
\end{itemize}

\subsubsection{常用序列的Z变换}

\begin{center}
\begin{tabular}{|c|c|c|}
\hline
\textbf{序列} $f(kT)$ & \textbf{Z变换} $F(z)$ & \textbf{收敛域} \\
\hline
$\delta(kT)$(单位脉冲) & $1$ & 全平面 \\
\hline
$1(kT)$(单位阶跃) & $\frac{z}{z-1}$ & $|z| > 1$ \\
\hline
$kT$ & $\frac{Tz}{(z-1)^2}$ & $|z| > 1$ \\
\hline
$e^{-akT}$ & $\frac{z}{z-e^{-aT}}$ & $|z| > e^{-aT}$ \\
\hline
$\sin(\omega kT)$ & $\frac{z\sin(\omega T)}{z^2 - 2z\cos(\omega T) + 1}$ & $|z| > 1$ \\
\hline
$\cos(\omega kT)$ & $\frac{z[z-\cos(\omega T)]}{z^2 - 2z\cos(\omega T) + 1}$ & $|z| > 1$ \\
\hline
$a^k$ & $\frac{z}{z-a}$ & $|z| > |a|$ \\
\hline
$ka^k$ & $\frac{az}{(z-a)^2}$ & $|z| > |a|$ \\
\hline
\end{tabular}
\end{center}

\subsubsection{Z变换的性质}

\begin{center}
\begin{tabular}{|l|c|c|}
\hline
\textbf{性质} & \textbf{时域} & \textbf{Z域} \\
\hline
线性 & $a f_1(kT) + b f_2(kT)$ & $a F_1(z) + b F_2(z)$ \\
\hline
右移 & $f(kT - nT)$ & $z^{-n} F(z)$ \\
\hline
左移 & $f(kT + nT)$ & $z^n [F(z) - \sum_{k=0}^{n-1} f(kT) z^{-k}]$ \\
\hline
初值定理 & $f(0)$ & $\lim_{z\to\infty} F(z)$ \\
\hline
终值定理 & $\lim_{k\to\infty} f(kT)$ & $\lim_{z\to 1} (z-1)F(z)$ \\
\hline
卷积 & $\sum_{i=0}^k f_1(iT) f_2(kT-iT)$ & $F_1(z) \cdot F_2(z)$ \\
\hline
\end{tabular}
\end{center}

\subsection{脉冲传递函数}

\subsubsection{定义}

脉冲传递函数是离散系统的输出Z变换与输入Z变换之比(零初始条件):
\begin{align*}
G(z) = \frac{Y(z)}{X(z)}
\end{align*}

\subsubsection{零阶保持器(ZOH)}

零阶保持器在采样周期内保持采样值不变。

\begin{minipage}[t]{0.52\textwidth}
\textbf{零阶保持器的传递函数:}
\begin{align*}
G_h(s) = \frac{1 - e^{-Ts}}{s}
\end{align*}

\textbf{含零阶保持器的系统:}

采样器 + ZOH + 连续对象 $G(s)$

等效脉冲传递函数:
\begin{align*}
G(z) = \mathcal{Z}\left\{\frac{1-e^{-Ts}}{s} G(s)\right\} = (1-z^{-1})\mathcal{Z}\left\{\frac{G(s)}{s}\right\}
\end{align*}

\vspace{0.3cm}
\textbf{常用公式:}

对于 $G(s) = \frac{1}{s+a}$:
\begin{align*}
G(z) = (1-z^{-1})\mathcal{Z}\left\{\frac{1}{s(s+a)}\right\} = \frac{1-e^{-aT}}{z-e^{-aT}}
\end{align*}

对于 $G(s) = \frac{1}{s(s+a)}$:
\begin{align*}
G(z) = (1-z^{-1})\mathcal{Z}\left\{\frac{1}{s^2(s+a)}\right\} = \frac{T - \frac{1-e^{-aT}}{a}}{(z-1)(z-e^{-aT})}
\end{align*}
\end{minipage}\hfill
\begin{minipage}[t]{0.45\textwidth}
\vspace{0pt}
\textbf{零阶保持器输出示意:}

零阶保持器(Zero-Order Hold, ZOH)在每个采样周期内保持采样值不变,将离散信号转换为阶梯波形式的连续信号。

输出特点:
\begin{itemize}
    \item 在每个采样周期 $[kT, (k+1)T)$ 内,输出保持为 $y(kT)$
    \item 在采样时刻发生跳变
    \item 形成阶梯波形
\end{itemize}

\vspace{0.5cm}
\textbf{例题:}求含ZOH和对象 $G(s) = \frac{1}{s+2}$ 的脉冲传递函数,$T = 0.1$ s。

\textit{解:}

\textbf{方法1:查表法}
\begin{align*}
G(z) &= (1-z^{-1})\mathcal{Z}\left\{\frac{1}{s(s+2)}\right\} \\
&= \frac{1-e^{-2T}}{z-e^{-2T}} \\
&= \frac{1-e^{-0.2}}{z-e^{-0.2}} \\
&= \frac{0.1813}{z-0.8187}
\end{align*}

\textbf{方法2:部分分式法}
\begin{align*}
\frac{G(s)}{s} &= \frac{1}{s(s+2)} = \frac{0.5}{s} - \frac{0.5}{s+2} \\
\mathcal{Z}\left\{\frac{1}{s(s+2)}\right\} &= 0.5\frac{z}{z-1} - 0.5\frac{z}{z-e^{-2T}} \\
G(z) &= (1-z^{-1}) \times \text{上式} \\
&= \frac{0.1813}{z-0.8187}
\end{align*}
\end{minipage}

\subsection{离散系统的稳定性}

\subsubsection{稳定性判据}

离散系统稳定的充要条件:闭环脉冲传递函数 $\Phi(z)$ 的所有极点都在 $z$ 平面单位圆内,即:
\begin{align*}
|z_i| < 1, \quad i = 1, 2, \ldots, n
\end{align*}

\subsubsection{劳斯判据的应用(双线性变换)}

通过双线性变换 $z = \frac{1+w}{1-w}$,将 $z$ 平面单位圆内映射到 $w$ 平面左半平面,然后应用劳斯判据。

\textbf{步骤:}

1. 求闭环特征方程 $D(z) = 0$

2. 进行双线性变换:$z = \frac{1+w}{1-w}$,得 $D(w) = 0$

3. 对 $D(w) = 0$ 应用劳斯判据

\vspace{0.3cm}
\textbf{例题:}判断系统 $\Phi(z) = \frac{K}{z^2 - 1.5z + 0.5}$ 的稳定性,$K = 1$。

\textit{解:}

\textbf{1. 特征方程}
\begin{align*}
D(z) = z^2 - 1.5z + 0.5 + K = z^2 - 1.5z + 1.5 = 0
\end{align*}

\textbf{2. 双线性变换}

令 $z = \frac{1+w}{1-w}$:
\begin{align*}
\left(\frac{1+w}{1-w}\right)^2 - 1.5\frac{1+w}{1-w} + 1.5 &= 0 \\
(1+w)^2 - 1.5(1+w)(1-w) + 1.5(1-w)^2 &= 0 \\
1 + 2w + w^2 - 1.5(1-w^2) + 1.5(1-2w+w^2) &= 0 \\
1 + 2w + w^2 - 1.5 + 1.5w^2 + 1.5 - 3w + 1.5w^2 &= 0 \\
4w^2 - w + 1 &= 0
\end{align*}

\textbf{3. 劳斯表}
\begin{center}
\begin{tabular}{c|cc}
$w^2$ & 4 & 1 \\
$w^1$ & -1 & 0 \\
$w^0$ & 1 &  \\
\end{tabular}
\end{center}

第一列有符号变化(+,-,+)$\implies$ \textbf{不稳定}

\textbf{验证:}直接求根:$z = \frac{1.5 \pm \sqrt{2.25-6}}{2} = 0.75 \pm 0.968j$,$|z| = 1.22 > 1$

\subsubsection{Jury稳定性判据}

Jury判据直接在 $z$ 域判断稳定性,无需变换。

设特征方程:
\begin{align*}
D(z) = a_0 z^n + a_1 z^{n-1} + \cdots + a_{n-1} z + a_n = 0
\end{align*}

\textbf{稳定的必要条件:}
\begin{align*}
D(1) > 0, \quad (-1)^n D(-1) > 0
\end{align*}

\textbf{充要条件:}构造Jury表,所有行的首元素满足特定符号条件。

(详细方法略,工程中常用数值计算求根判断)

\subsection{离散系统的动态性能}

\subsubsection{稳态误差}

\textbf{终值定理:}
\begin{align*}
e_{ss} = \lim_{k\to\infty} e(kT) = \lim_{z\to 1} (z-1)E(z)
\end{align*}

\textbf{单位阶跃输入:}$R(z) = \frac{z}{z-1}$
\begin{align*}
e_{ss} = \lim_{z\to 1} (z-1) \frac{R(z)}{1+G(z)} = \frac{1}{1 + \lim_{z\to 1} G(z)}
\end{align*}

\textbf{单位斜坡输入:}$R(z) = \frac{Tz}{(z-1)^2}$
\begin{align*}
e_{ss} = \lim_{z\to 1} (z-1) \frac{Tz}{(z-1)^2[1+G(z)]} = \frac{T}{\lim_{z\to 1} (z-1)G(z)/T}
\end{align*}

\subsubsection{瞬态性能指标}

离散系统的时域响应可以通过Z反变换求得:
\begin{align*}
y(kT) = \mathcal{Z}^{-1}\{Y(z)\}
\end{align*}

\textbf{常用方法:}
\begin{itemize}
    \item \textbf{长除法}:直接展开为序列
    \item \textbf{部分分式法}:分解后查表反变换
    \item \textbf{留数法}:复变函数理论
\end{itemize}

\textbf{性能指标:}
\begin{itemize}
    \item 上升时间 $t_r$
    \item 超调量 $\sigma\%$
    \item 调节时间 $t_s$
    \item 振荡次数
\end{itemize}

\subsection{数字PID控制}

\subsubsection{连续PID的离散化}

连续PID控制器:
\begin{align*}
u(t) = K_p e(t) + K_i \int_0^t e(\tau) d\tau + K_d \frac{de(t)}{dt}
\end{align*}

\textbf{位置式数字PID:}
\begin{align*}
u(k) = K_p e(k) + K_i T \sum_{j=0}^k e(j) + K_d \frac{e(k) - e(k-1)}{T}
\end{align*}

\textbf{速度式数字PID:}
\begin{align*}
\Delta u(k) = u(k) - u(k-1) &= K_p [e(k) - e(k-1)] \\
&\quad + K_i T e(k) \\
&\quad + K_d \frac{e(k) - 2e(k-1) + e(k-2)}{T}
\end{align*}

\subsubsection{数字PID的改进}

\textbf{1. 积分分离}:误差大时不积分,避免积分饱和
\begin{align*}
u(k) = K_p e(k) + K_i T \sum_{j=k_0}^k e(j) + K_d \frac{e(k) - e(k-1)}{T}
\end{align*}

\textbf{2. 微分先行}:仅对输出微分,避免对设定值突变响应过大
\begin{align*}
u(k) = K_p e(k) + K_i T \sum_{j=0}^k e(j) - K_d \frac{y(k) - y(k-1)}{T}
\end{align*}

\textbf{3. 不完全微分}:加入滤波环节,减小噪声影响
\begin{align*}
u_d(k) = \alpha u_d(k-1) + K_d (1-\alpha) \frac{e(k) - e(k-1)}{T}
\end{align*}

其中 $\alpha = \frac{T}{T + \tau_d}$,$\tau_d$ 为滤波时间常数。

\subsection{离散系统设计}

\subsubsection{最少拍控制}

\textbf{设计目标:}在最少的采样周期内使系统输出达到并保持在期望值。

\textbf{设计步骤:}

1. 确定期望闭环传递函数 $\Phi^*(z)$

2. 计算控制器传递函数:
\begin{align*}
D(z) = \frac{\Phi^*(z)}{G(z)[1-\Phi^*(z)]}
\end{align*}

3. 验证物理可实现性和稳定性

\textbf{常用期望响应:}
\begin{itemize}
    \item 单拍系统:$\Phi^*(z) = z^{-1}$
    \item 有纹波最少拍:$\Phi^*(z) = \frac{(1-a)z^{-1}}{1-az^{-1}}$
    \item 无纹波最少拍(I型系统):$\Phi^*(z) = z^{-1}(2-z^{-1})$
\end{itemize}

\subsubsection{数字控制器的实现}

\textbf{直接编程法:}

控制器传递函数 $D(z) = \frac{U(z)}{E(z)}$ 转换为差分方程,直接编程实现。

\textbf{串行校正:}

类似连续系统的超前/滞后校正,设计数字校正器:
\begin{align*}
D(z) = K \frac{z - a}{z - b}
\end{align*}

\textbf{并行校正:}

采用状态反馈等现代控制方法。

\subsection{总结}

\begin{center}
\begin{tabular}{|l|c|c|}
\hline
\textbf{特性} & \textbf{连续系统} & \textbf{离散系统} \\
\hline
数学描述 & 微分方程 & 差分方程 \\
\hline
变换工具 & 拉普拉斯变换 & Z变换 \\
\hline
频域函数 & 传递函数 $G(s)$ & 脉冲传递函数 $G(z)$ \\
\hline
稳定域 & $s$ 平面左半平面 & $z$ 平面单位圆内 \\
\hline
稳定判据 & 劳斯、奈奎斯特 & 双线性变换+劳斯、Jury \\
\hline
控制器 & 模拟PID & 数字PID、最少拍 \\
\hline
\end{tabular}
\end{center}

\textbf{离散系统设计要点:}
\begin{enumerate}
    \item 选择合适的采样周期(满足采样定理,通常 $T = 0.1 \sim 0.5$ 时间常数)
    \item 考虑零阶保持器的影响
    \item 数字控制器设计(数字PID、最少拍等)
    \item 验证闭环稳定性(极点在单位圆内)
    \item 评估动态性能(超调、调节时间等)
\end{enumerate}
