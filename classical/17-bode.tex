\section{伯德图绘制}

\subsection{伯德图的定义}
伯德图(Bode Plot)是由亨德里克·韦德·伯德(Hendrik Wade Bode)提出的频率响应图示方法,由两个图组成:

\begin{itemize}
    \item \textbf{幅频特性图(Magnitude Plot)}:$L(\omega) = 20\lg|G(\jw)|$ dB vs $\lg\omega$
    \begin{itemize}
        \item 纵轴:对数幅值(dB)
        \item 横轴:对数频率($\lg\omega$)
    \end{itemize}
    \item \textbf{相频特性图(Phase Plot)}:$\phi(\omega) = \angle G(\jw)$ vs $\lg\omega$
    \begin{itemize}
        \item 纵轴:相位角(度或弧度)
        \item 横轴:对数频率($\lg\omega$)
    \end{itemize}
\end{itemize}

\textbf{伯德图的优点:}
\begin{enumerate}
    \item 频率范围广,可表示从极低频到极高频的特性
    \item 不同环节的伯德图可以直接相加(叠加原理)
    \item 可用渐近线逼近,绘制简便
    \item 便于分析系统的稳定性和性能指标
\end{enumerate}

\subsection{典型环节的伯德图}

\subsubsection{比例环节 \texorpdfstring{$K$}{K}}
传递函数:$G(s) = K$

频率响应:$G(\jw) = K$

\textbf{幅频特性:}
\begin{align*}
L(\omega) &= 20\lg K \text{ dB(水平线)}
\end{align*}

\textbf{相频特性:}
\begin{align*}
\phi(\omega) &= \begin{cases}
0° & K > 0 \\
180° & K < 0
\end{cases}
\end{align*}

\textbf{伯德图示例:}$G(s) = 2$
\begin{center}
\BodeTF{num/{2},den/{1}}{0.1}{1000}
\end{center}

\subsubsection{积分环节 $\frac{1}{s}$}
传递函数:$G(s) = \frac{1}{s}$

频率响应:$G(\jw) = \frac{1}{\jw}$

\textbf{幅频特性:}
\begin{align*}
L(\omega) &= 20\lg\frac{1}{\omega} = -20\lg\omega \text{ dB}
\end{align*}
\begin{itemize}
    \item 斜率:$-20$ dB/十倍频(decade)
    \item 当 $\omega = 1$ rad/s 时,$L(\omega) = 0$ dB
    \item 当 $\omega$ 增大10倍时,$L(\omega)$ 下降20 dB
\end{itemize}

\textbf{相频特性:}
\begin{align*}
\phi(\omega) &= -90°\text{(恒定)}
\end{align*}

\textbf{伯德图示例:}$G(s) = \frac{1}{s}$
\begin{center}
\BodeTF{num/{1},den/{1,0}}{0.1}{1000}
\end{center}

\subsubsection{微分环节 $s$}
与积分环节对称:
\begin{itemize}
    \item $L(\omega) = 20\lg\omega$ dB(斜率 $+20$ dB/十倍频)
    \item $\phi(\omega) = 90°$
\end{itemize}

\textbf{伯德图示例:}$G(s) = s$
\begin{center}
\BodeTF{num/{1,0},den/{1}}{0.1}{1000}
\end{center}

\subsubsection{惯性环节 \texorpdfstring{$\frac{1}{1+Ts}$}{1/(1+Ts)}}
传递函数:$G(s) = \frac{1}{1+Ts}$

频率响应:$G(\jw) = \frac{1}{1+\jw T}$

\textbf{转折频率(Corner Frequency):}$\omega_c = \frac{1}{T}$ rad/s

\textbf{幅频特性:}
\begin{align*}
L(\omega) &= 20\lg|G(\jw)| = -20\lg\sqrt{1 + \omega^2T^2}
\end{align*}

渐近线:
\begin{itemize}
    \item 当 $\omega \ll \omega_c$:$L(\omega) \approx 0$ dB
    \item 当 $\omega \gg \omega_c$:$L(\omega) \approx -20\lg(\omega T)$ dB(斜率 $-20$ dB/十倍频)
    \item 转折点 $\omega = \omega_c$:精确值 $L(\omega_c) = -3$ dB
\end{itemize}

\textbf{相频特性:}
\begin{align*}
\phi(\omega) &= -\arctan(\omega T)
\end{align*}
\begin{itemize}
    \item $\omega = 0.1\omega_c$:$\phi \approx -6°$
    \item $\omega = \omega_c$:$\phi = -45°$
    \item $\omega = 10\omega_c$:$\phi \approx -84°$
    \item $\omega \to \infty$:$\phi \to -90°$
\end{itemize}

\textbf{伯德图示例:}$G(s) = \frac{1}{1+0.1s}$ ($\omega_c = 10$ rad/s)
\begin{center}
\BodeTF{num/{1},den/{0.1,1}}{0.1}{1000}
\end{center}

\subsubsection{一阶微分环节 $1+Ts$}
与惯性环节对称:
\begin{itemize}
    \item 幅频特性:低频0 dB,高频斜率 $+20$ dB/十倍频
    \item 相频特性:$\phi(\omega) = \arctan(\omega T)$,$\phi(\omega_c) = 45°$
\end{itemize}

\textbf{伯德图示例:}$G(s) = 1 + 0.1s$ ($\omega_c = 10$ rad/s)
\begin{center}
\BodeTF{num/{0.1,1},den/{1}}{0.1}{1000}
\end{center}

\subsubsection{振荡环节 $\frac{\omega_n^2}{s^2 + 2\zeta\omega_n s + \omega_n^2}$}
传递函数标准形式:
\begin{align*}
G(s) = \frac{\omega_n^2}{s^2 + 2\zeta\omega_n s + \omega_n^2}
\end{align*}

其中:
\begin{itemize}
    \item $\omega_n$:无阻尼自然频率
    \item $\zeta$:阻尼比($0 < \zeta < 1$)
\end{itemize}

频率响应:
\begin{align*}
G(\jw) = \frac{\omega_n^2}{\omega_n^2 - \omega^2 + 2\jw\zeta\omega_n}
\end{align*}

\textbf{幅频特性:}
\begin{align*}
L(\omega) &= 20\lg\frac{\omega_n^2}{\sqrt{(\omega_n^2-\omega^2)^2 + (2\zeta\omega_n\omega)^2}}
\end{align*}

渐近线:
\begin{itemize}
    \item 当 $\omega \ll \omega_n$:$L(\omega) \approx 0$ dB
    \item 当 $\omega \gg \omega_n$:$L(\omega) \approx -40\lg(\omega/\omega_n)$ dB(斜率 $-40$ dB/十倍频)
    \item 转折频率:$\omega_c = \omega_n$
\end{itemize}

\textbf{谐振峰值(仅当 $\zeta < 0.707$ 时):}
\begin{itemize}
    \item 谐振频率:$\omega_r = \omega_n\sqrt{1-2\zeta^2}$
    \item 谐振峰值:$M_r = \frac{1}{2\zeta\sqrt{1-\zeta^2}}$
    \item 当 $\zeta$ 很小时,谐振峰值很大
\end{itemize}

\textbf{转折频率处的精确值:}
\begin{itemize}
    \item $L(\omega_n) = -20\lg(2\zeta)$ dB
    \item 当 $\zeta = 0.707$ 时,$L(\omega_n) = -3$ dB(无谐振)
    \item 当 $\zeta < 0.707$ 时,$L(\omega_n) > -3$ dB(有谐振)
    \item 当 $\zeta > 0.707$ 时,$L(\omega_n) < -3$ dB(过阻尼)
\end{itemize}

\textbf{相频特性:}
\begin{align*}
\phi(\omega) &= -\arctan\frac{2\zeta\omega_n\omega}{\omega_n^2-\omega^2}
\end{align*}
\begin{itemize}
    \item $\omega = \omega_n$:$\phi = -90°$(与 $\zeta$ 无关)
    \item $\omega \to 0$:$\phi \to 0°$
    \item $\omega \to \infty$:$\phi \to -180°$
    \item $\zeta$ 越小,相位变化越快
\end{itemize}

\textbf{不同阻尼比的伯德图对比:}

\begin{center}
\begin{tikzpicture}
\begin{groupplot}[
    group style={group size=1 by 2, vertical sep=0.5cm},
    width=12cm, height=4cm,
    xmode=log,
    grid=both,
    xlabel={频率 $\omega$ (rad/s)},
    xmin=0.1, xmax=1000
]
\nextgroupplot[ylabel={幅度 (dB)}, legend pos=south west]
\addplot[blue, thick, samples=100, domain=0.1:1000] {20*log10(100/sqrt((100-x^2)^2 + (2*0.1*10*x)^2))};
\addlegendentry{$\zeta=0.1$}
\addplot[red, thick, samples=100, domain=0.1:1000] {20*log10(100/sqrt((100-x^2)^2 + (2*0.5*10*x)^2))};
\addlegendentry{$\zeta=0.5$}
\addplot[green!60!black, thick, samples=100, domain=0.1:1000] {20*log10(100/sqrt((100-x^2)^2 + (2*1.0*10*x)^2))};
\addlegendentry{$\zeta=1.0$}

\nextgroupplot[ylabel={相位 (度)}, legend pos=south west]
\addplot[blue, thick, samples=100, domain=0.1:1000] {deg(-atan2(2*0.1*10*x, 100-x^2))};
\addlegendentry{$\zeta=0.1$}
\addplot[red, thick, samples=100, domain=0.1:1000] {deg(-atan2(2*0.5*10*x, 100-x^2))};
\addlegendentry{$\zeta=0.5$}
\addplot[green!60!black, thick, samples=100, domain=0.1:1000] {deg(-atan2(2*1.0*10*x, 100-x^2))};
\addlegendentry{$\zeta=1.0$}
\end{groupplot}
\end{tikzpicture}
\end{center}

图中所有曲线对应 $\omega_n = 10$ rad/s,不同阻尼比 $\zeta$。

\textbf{伯德图示例(欠阻尼,有谐振):}$G(s) = \frac{100}{s^2 + 2s + 100}$ ($\omega_n = 10$ rad/s,$\zeta = 0.1$)
\begin{center}
\BodeTF{num/{100},den/{1,2,100}}{0.1}{1000}
\end{center}

\textbf{伯德图示例(临界阻尼):}$G(s) = \frac{100}{s^2 + 14.14s + 100}$ ($\omega_n = 10$ rad/s,$\zeta = 0.707$)
\begin{center}
\BodeTF{num/{100},den/{1,14.14,100}}{0.1}{1000}
\end{center}

\textbf{伯德图示例(过阻尼):}$G(s) = \frac{100}{s^2 + 20s + 100}$ ($\omega_n = 10$ rad/s,$\zeta = 1.0$)
\begin{center}
\BodeTF{num/{100},den/{1,20,100}}{0.1}{1000}
\end{center}

\subsection{伯德图的基本概念}

伯德图由两个图组成,横坐标都是对数刻度的频率 $\omega$:
\begin{itemize}
    \item \textbf{对数幅频图}:纵坐标是系统幅值的对数 $L(\omega) = 20\log_{10}A(\omega)$,单位为分贝(dB)
    \item \textbf{对数相频图}:纵坐标是系统相角 $\phi(\omega)$,单位为度(°)
\end{itemize}

\subsubsection{对数幅频图的转折频率}

\begin{center}
\begin{tabular}{|c|c|c|}
\hline
\textbf{典型环节类别} & \textbf{典型环节传递函数} & \textbf{转折频率及斜率变化} \\
\hline
\multirow{2}{*}{一阶环节} & $\displaystyle\frac{1}{Ts+1}$ & $\omega_c = \frac{1}{T}$,斜率 $-20$ dB/dec \\
\cline{2-3}
& $Ts+1$ & $\omega_c = \frac{1}{T}$,斜率 $+20$ dB/dec \\
\hline
\multirow{2}{*}{二阶环节} & $\displaystyle\frac{1}{(s/\omega_n)^2+2\zeta(s/\omega_n)+1}$ & $\omega_c = \omega_n$,斜率 $-40$ dB/dec \\
\cline{2-3}
& $(s/\omega_n)^2+2\zeta(s/\omega_n)+1$ & $\omega_c = \omega_n$,斜率 $+40$ dB/dec \\
\hline
\end{tabular}
\end{center}

\subsubsection{低频段渐近线的确定}

对于标准型 $G(s) = K \frac{\prod(1+T_i s)}{\prod(1+T_j s)}$(以及其高阶形式):
\begin{itemize}
    \item 低频段的斜率由系统型别 $v$(积分环节 $s^v$ 的个数)决定,为 $-20v$ dB/dec
    \item 这条渐近线(或其延长线)必定经过点 $(1, 20\log_{10}K)$
    \item 也可以通过点 $(\omega_0, 20\lg K - 20v\lg\omega_0)$ 来确定直线
\end{itemize}

\subsubsection{例题:伯德图绘制示例}

\textbf{例1:绘制 $G(s) = \frac{2}{(2s+1)(8s+1)}$ 的开环对数幅频特性曲线。}

\textit{解:}
\begin{enumerate}
    \item \textbf{标准型}:$G(s) = \frac{2}{(2s+1)(8s+1)}$。开环增益 $K=2$,系统为0型 ($v=0$)
    
    \item \textbf{转折频率}:
    \begin{align*}
    T_1=2 &\implies \omega_1=\frac{1}{2}=0.5 \text{ rad/s} \\
    T_2=8 &\implies \omega_2=\frac{1}{8}=0.125 \text{ rad/s}
    \end{align*}
    按从小到大排列:$\omega_1 = 0.125$ rad/s,$\omega_2 = 0.5$ rad/s
    
    \item \textbf{低频段} ($\omega < 0.125$):
    \begin{itemize}
        \item 斜率为 $-20 \times 0 = 0$ dB/dec(水平线)
        \item 幅值为 $L(\omega) = 20\lg(2) \approx 6$ dB
    \end{itemize}
    
    \item \textbf{中频段1} ($0.125 < \omega < 0.5$):
    \begin{itemize}
        \item 经过第一个转折频率 $\omega_1 = 0.125$(一阶极点)
        \item 斜率变为 $0 - 20 = -20$ dB/dec
    \end{itemize}
    
    \item \textbf{高频段} ($\omega > 0.5$):
    \begin{itemize}
        \item 经过第二个转折频率 $\omega_2 = 0.5$(一阶极点)
        \item 斜率变为 $-20 - 20 = -40$ dB/dec
    \end{itemize}
\end{enumerate}

\subsection{伯德图的绘制方法}

\subsubsection{伯德图绘制规则速查表}

以下表格是伯德图绘制中各种典型环节的**快速参考**。对于更详细的分析、图示和计算示例,请参考前面"典型环节的伯德图"部分。

{\renewcommand{\arraystretch}{2.0}
\small
\begin{center}
\begin{tabular}{|c|c|c|p{4.8cm}|c|}
\hline
\rowcolor{blue!30}
\multicolumn{5}{|c|}{\Large\textbf{伯德图典型环节速查表}} \\
\hline
\rowcolor{blue!15}
\textbf{环节} & \textbf{传递函数} & \textbf{转折频率} & \textbf{幅频斜率} & \textbf{相频} \\
\hline

\rowcolor{gray!8}
\textbf{比例} & $K$ & — & $0$ dB/dec & $0°/180°$ \\
\hline

\rowcolor{white}
\textbf{积分} & $\frac{1}{s}$ & — & $-20$ dB/dec & $-90°$ \\
\hline

\rowcolor{gray!8}
\textbf{微分} & $s$ & — & $+20$ dB/dec & $+90°$ \\
\hline

\rowcolor{white}
\textbf{一阶极点} & $\frac{1}{1+Ts}$ & $\omega_c = \frac{1}{T}$ & 
低频:$0$;高频:$-20$ dB/dec & 
$-45°$ @ $\omega_c$ \\
\hline

\rowcolor{gray!8}
\textbf{一阶零点} & $1+Ts$ & $\omega_c = \frac{1}{T}$ & 
低频:$0$;高频:$+20$ dB/dec & 
$+45°$ @ $\omega_c$ \\
\hline

\rowcolor{white}
\textbf{二阶极点} & $\frac{\omega_n^2}{(s/\omega_n)^2+2\zeta(s/\omega_n)+1}$ & $\omega_c = \omega_n$ & 
低频:$0$;高频:$-40$ dB/dec & 
$-90°$ @ $\omega_n$ \\
\hline

\rowcolor{gray!8}
\textbf{二阶零点} & $(s/\omega_n)^2+2\zeta(s/\omega_n)+1$ & $\omega_c = \omega_n$ & 
低频:$0$;高频:$+40$ dB/dec & 
$+90°$ @ $\omega_n$ \\
\hline

\end{tabular}
\end{center}
}

\textbf{快速参考提示:}
\begin{itemize}
    \item \textbf{转折点修正}:一阶环节修正 $\pm 3$ dB,二阶环节修正 $\pm 6$ dB
    \item \textbf{极点}:幅频向下(负斜率),相频向下(变负)
    \item \textbf{零点}:幅频向上(正斜率),相频向上(变正)
    \item \textbf{积分/微分}:无转折频率,斜率固定
\end{itemize}

\subsubsection{关键修正值详表}

注意:极点与零点总是"互为镜像"的,掌握以下规律能加快绘图速度。

{\renewcommand{\arraystretch}{1.6}
\begin{center}
\small
\begin{tabular}{|c|c|c|}
\hline
\rowcolor{blue!20}
\textbf{特征} & \textbf{一阶/二阶极点} & \textbf{一阶/二阶零点} \\
\hline
\rowcolor{gray!5}
\textbf{斜率变化} & 负值(衰减) & 正值(增强) \\
\hline
\textbf{相位变化方向} & 向下(越来越负) & 向上(越来越正) \\
\hline
\rowcolor{gray!5}
\textbf{在转折点修正} & 减少 $\pm 3$ dB / $\pm 6$ dB & 增加 $\pm 3$ dB / $\pm 6$ dB \\
\hline
\textbf{谐振特征} & 可能出现谐振峰 & 可能出现谷值 \\
\hline
\end{tabular}
\end{center}
}

\subsubsection{修正值参考表 — 用于精化渐近线}

用于将渐近线逼近精确曲线。在考试手绘时,可根据精度要求选择性使用。

\textbf{一阶环节(极点和零点)}

\begin{center}
\small
\begin{tabular}{|c|c|c|c|}
\hline
\textbf{环节} & \textbf{频率点} & \textbf{修正值} & \textbf{说明} \\
\hline
\multirow{3}{*}{\shortstack{一阶极点 \\ $\frac{1}{1+Ts}$}} & $0.5\omega_c$ & $-1$ dB & 稍低于渐近线 \\
\cline{2-4}
& $\omega_c$ & $-3$ dB & 精确值 \\
\cline{2-4}
& $2\omega_c$ & $-1$ dB & 稍低于渐近线 \\
\hline
\multirow{3}{*}{\shortstack{一阶零点 \\ $1+Ts$}} & $0.5\omega_c$ & $+1$ dB & 稍高于渐近线 \\
\cline{2-4}
& $\omega_c$ & $+3$ dB & 精确值 \\
\cline{2-4}
& $2\omega_c$ & $+1$ dB & 稍高于渐近线 \\
\hline
\end{tabular}
\end{center}

\textbf{二阶极点 — 不同阻尼比下的谐振特性}

\begin{center}
\small
\begin{tabular}{|c|c|c|c|}
\hline
\textbf{阻尼比 $\zeta$} & \textbf{在 $\omega=\omega_n$ 处} & \textbf{幅值(dB)} & \textbf{特征分类} \\
\hline
$0.1 \sim 0.3$ & 明显谐振峰 & $+3$ 到 $+14$ dB & 强谐振 \\
\hline
$0.5$ & 轻微谐振 & $\approx -1$ dB & 中等阻尼 \\
\hline
$0.707$ & 临界点 & $-3$ dB & \textbf{临界阻尼} \\
\hline
$1.0$ 以上 & 过阻尼 & $-6$ dB 以下 & 无谐振 \\
\hline
\end{tabular}
\end{center}

\vspace{0.3cm}
\textit{提示:当 $\zeta > 0.707$ 时,二阶极点的行为接近两个一阶极点,可用叠加法求解。}

\subsubsection{伯德图绘制六步法}

这是一套标准化、模板化的绘制流程,特别适合考试手绘:

\textbf{第一步:化为"尾1"标准型}

将传递函数中所有的一阶和二阶环节都化为 $(Ts+1)$ 或 $((s/\omega_n)^2 + 2\zeta(s/\omega_n) + 1)$ 的形式。这样可以方便地读出开环增益 $K$ 和各转折频率。

标准形式为:
\begin{align*}
G(s) = K \frac{\prod(1+T_{zi}s)\prod((s/\omega_{ni})^2+2\zeta_i(s/\omega_{ni})+1)}{s^v\prod(1+T_{pi}s)\prod((s/\omega_{nj})^2+2\zeta_j(s/\omega_{nj})+1)}
\end{align*}

\textbf{第二步:列出系统的转折频率}

转折频率(交接频率)是渐近线斜率发生改变的点:
\begin{itemize}
    \item 一阶环节 $(Ts \pm 1)$:转折频率为 $\omega_c = \frac{1}{T}$
    \item 二阶环节 $((s/\omega_n)^2 + ...)$:转折频率为 $\omega_c = \omega_n$
\end{itemize}

\textbf{将所有转折频率从小到大排列}。

\textbf{第三步:确定开环增益 $K$}

从"尾1"标准型中直接读出比例项 $K$。

\textbf{第四步:求与横轴的交点(剪切频率 $\omega_{gc}$)}

横轴(0dB 线)代表 $|G(j\omega)| = 1$。需要求解方程 $|G(j\omega_{gc})| = 1$,利用下面的"幅值近似原则"。

\textbf{第五步:绘制低频段渐近线}

低频段渐近线由以下三个性质唯一确定:
\begin{enumerate}
    \item \textbf{斜率}:由系统型别 $v$(积分环节 $s^v$ 的个数)决定,斜率为 $-20v$ dB/dec
    \item \textbf{定位点1}:低频段渐近线(或其延长线)\textbf{必过点 $(\omega=1, 20\log_{10}K)$}
    \item \textbf{定位点2($v \geq 1$ 时)}:低频段渐近线(或其延长线)\textbf{与0dB横轴相交于点 $(\omega = \sqrt[v]{K}, 0\text{ dB})$}
\end{enumerate}

\textbf{第六步:依次绘制后续曲线}

从最低的转折频率开始,每经过一个转折频率,渐近线的斜率发生一次改变:

\begin{center}
\begin{tabular}{|c|c|c|}
\hline
\textbf{典型环节} & \textbf{位置} & \textbf{斜率变化} \\
\hline
\multirow{2}{*}{一阶环节} & 分母 & $-20$ dB/dec \\
\cline{2-3}
& 分子 & $+20$ dB/dec \\
\hline
\multirow{2}{*}{二阶环节} & 分母 & $-40$ dB/dec \\
\cline{2-3}
& 分子 & $+40$ dB/dec \\
\hline
\end{tabular}
\end{center}

\textbf{【最终验证】}:绘制完成后,检查最后一个频段的斜率是否等于 $-20(n-m)$ dB/dec,其中 $n$ 是分母阶次,$m$ 是分子阶次。

\subsubsection{幅值近似原则(关键技巧)}

在手绘伯德图和计算剪切频率时,使用的是近似幅值而非精确值。这是加快计算的关键:

\textbf{一阶环节 $(Ts+1)$ 的近似}:
\begin{itemize}
    \item 在转折频率前($\omega < 1/T$):虚部 $T\omega < 1$,\textbf{保留常数项1}。环节幅值近似为 $1$
    \item 在转折频率后($\omega > 1/T$):虚部 $T\omega > 1$,\textbf{保留虚部项 $T\omega$}。环节幅值近似为 $T\omega$
\end{itemize}

\textbf{二阶环节 $((s/\omega_n)^2 + ...)$ 的近似}:
\begin{itemize}
    \item 在转折频率前($\omega < \omega_n$):\textbf{保留常数项1}。环节幅值近似为 $1$
    \item 在转折频率后($\omega > \omega_n$):\textbf{保留平方项 $(s/\omega_n)^2$}。环节幅值近似为 $(\omega/\omega_n)^2$
\end{itemize}

\subsubsection{基本步骤(原有内容)}
\begin{enumerate}
    \item \textbf{将传递函数化为标准形式}
    \begin{align*}
    G(s) = \frac{K\prod(1+T_i s)\prod(\frac{s^2}{\omega_{ni}^2}+\frac{2\zeta_i}{\omega_{ni}}s+1)}{s^v\prod(1+T_j s)\prod(\frac{s^2}{\omega_{nj}^2}+\frac{2\zeta_j}{\omega_{nj}}s+1)}
    \end{align*}
    
    \item \textbf{识别各个典型环节}
    \begin{itemize}
        \item 比例环节 $K$
        \item 积分/微分环节 $s^{\pm v}$
        \item 一阶环节 $(1+Ts)^{\pm 1}$
        \item 二阶环节 $(\frac{s^2}{\omega_n^2}+\frac{2\zeta s}{\omega_n}+1)^{\pm 1}$
    \end{itemize}
    
    \item \textbf{确定转折频率}
    \begin{itemize}
        \item 一阶环节:$\omega_c = \frac{1}{T}$
        \item 二阶环节:$\omega_c = \omega_n$
        \item 按从小到大排列所有转折频率
    \end{itemize}
    
    \item \textbf{绘制低频渐近线}
    \begin{itemize}
        \item 起始点:选择 $\omega = 1$ rad/s 或最小转折频率的 $\frac{1}{10}$
        \item 起始斜率:由积分环节数 $v$ 决定($-20v$ dB/十倍频)
        \item 起始幅值:$L(\omega_0) = 20\lg K - 20v\lg\omega_0$
    \end{itemize}
    
    \item \textbf{绘制各段渐近线}
    \begin{itemize}
        \item 每经过一个转折频率,斜率改变
        \item 零点(分子):斜率增加
        \begin{itemize}
            \item 一阶零点:$+20$ dB/十倍频
            \item 二阶零点:$+40$ dB/十倍频
        \end{itemize}
        \item 极点(分母):斜率减少
        \begin{itemize}
            \item 一阶极点:$-20$ dB/十倍频
            \item 二阶极点:$-40$ dB/十倍频
        \end{itemize}
    \end{itemize}
    
    \item \textbf{修正转折频率附近的误差}
    \begin{itemize}
        \item 一阶环节在 $\omega = \omega_c$ 处:$-3$ dB(极点)或 $+3$ dB(零点)
        \item 二阶环节:根据阻尼比 $\zeta$ 修正
    \end{itemize}
    
    \item \textbf{绘制相频特性}
    \begin{itemize}
        \item 每个环节的相位直接相加
        \item 注意各环节的相位贡献
    \end{itemize}
\end{enumerate}

\subsubsection{转折频率处的修正}

\textbf{一阶环节 $\frac{1}{1+Ts}$ 的修正:}

\begin{center}
\begin{tabular}{c|c|c}
\hline
频率 & 渐近线误差 & 精确值修正 \\
\hline
$0.5\omega_c$ & $0$ dB & $-1$ dB \\
$\omega_c$ & $0$ dB & $-3$ dB \\
$2\omega_c$ & $0$ dB & $-1$ dB \\
\hline
\end{tabular}
\end{center}

\textbf{二阶环节的修正}

取决于阻尼比 $\zeta$,在 $\omega = \omega_n$ 处:
\begin{itemize}
    \item $\zeta = 0.1$:谐振峰值约 $+14$ dB
    \item $\zeta = 0.2$:谐振峰值约 $+7$ dB
    \item $\zeta = 0.3$:谐振峰值约 $+3$ dB
    \item $\zeta = 0.5$:误差约 $-1$ dB
    \item $\zeta = 0.707$:误差 $-3$ dB(临界阻尼)
    \item $\zeta = 1.0$:误差 $-6$ dB(过阻尼)
\end{itemize}

\subsubsection{六步法综合例题}

\textbf{例1:绘制 $G(s) = \frac{8(\frac{s}{0.1}+1)}{s(s^2+s+1)(\frac{s}{2}+1)}$ 的开环对数幅频特性曲线。}

\textit{解:}
\begin{enumerate}
    \item \textbf{化为标准型}:
    \[G(s) = \frac{8(10s+1)}{s(s^2+s+1)(0.5s+1)}\]
    
    \item \textbf{转折频率}:
    \begin{itemize}
        \item 分子一阶:$T_z=10 \implies \omega_1 = \frac{1}{10} = 0.1$ rad/s(零点,+20 dB/dec)
        \item 分母二阶:由 $s^2+s+1 = 0$ 得 $2\zeta\omega_n = 1, \omega_n^2 = 1$,所以 $\omega_n=1$ rad/s(极点,-40 dB/dec)
        \item 分母一阶:$T_p=0.5 \implies \omega_3 = \frac{1}{0.5} = 2$ rad/s(极点,-20 dB/dec)
        \item 排序:$0.1, 1, 2$
    \end{itemize}
    
    \item \textbf{开环增益}:$K = 8$
    
    \item \textbf{低频段} ($\omega < 0.1$):
    \begin{itemize}
        \item 系统为I型 ($v=1$),斜率为 $-20$ dB/dec
        \item 与0dB轴交于 $\omega = \sqrt[1]{K} = 8$ rad/s
        \item 但 $8 > 0.1$,说明交点不在低频段
        \item 可用定位点:在 $\omega=1$ 处,幅值为 $L(1) = 20\lg 8 - 20 \times 1 = 18.06$ dB
    \end{itemize}
    
    \item \textbf{各频段渐近线}:
    \begin{itemize}
        \item $0.1 < \omega < 1$:经过 $\omega_1=0.1$(分子一阶零点),斜率变为 $-20+20=0$ dB/dec
        \item $1 < \omega < 2$:经过 $\omega_2=1$(分母二阶极点),斜率变为 $0-40=-40$ dB/dec
        \item $\omega > 2$:经过 $\omega_3=2$(分母一阶极点),斜率变为 $-40-20=-60$ dB/dec
    \end{itemize}
    
    \item \textbf{最终验证}:
    \begin{itemize}
        \item 分子阶次 $m=1$(一阶零点)
        \item 分母阶次 $n=4$(1个一阶 + 1个二阶 + 1个积分 = 4)
        \item 最终斜率应为 $-20(n-m) = -20(4-1) = -60$ dB/dec,正确✓
    \end{itemize}
\end{enumerate}

\subsubsection{剪切频率的计算}

剪切频率(增益穿越频率 $\omega_{gc}$)是幅值等于1(0 dB)的频率,通过求解 $|G(j\omega_{gc})| = 1$ 得到。

\textbf{计算方法}:利用幅值近似原则
\begin{enumerate}
    \item 在不同频段,根据主导环节使用近似幅值
    \item 在所有转折频率处分别计算幅值,判断穿越点在哪一段
    \item 在该频段内使用简化的近似幅值公式求解
\end{enumerate}

\textbf{典型例子}:$G(s) = \frac{2}{(2s+1)(8s+1)}$

设 $\omega_1 = 0.125, \omega_2 = 0.5$,则:
\begin{itemize}
    \item 低频 $\omega < 0.125$:$|G(j\omega)| \approx 2$(>1),不包含穿越点
    \item 中频 $0.125 < \omega < 0.5$:$|G(j\omega)| \approx \frac{2}{8\omega}$
    \begin{itemize}
        \item 令 $\frac{2}{8\omega} = 1 \implies \omega = 0.25$
        \item 由于 $0.125 < 0.25 < 0.5$,假设成立,所以 $\omega_{gc} = 0.25$
    \end{itemize}
    \item 高频验证:若需更精确,可用精确公式验证
\end{itemize}

\subsubsection{绘图示例}

\textbf{例1:}绘制 $G(s) = \frac{10}{s(1+0.1s)}$ 的伯德图

\begin{tcolorbox}[colback=blue!5!white,colframe=blue!75!black,title=例1求解过程与伯德图]

\begin{minipage}[t]{0.47\textwidth}
\textbf{解:}
\begin{enumerate}
    \item 标准形式:$G(s) = \frac{10}{s(1+0.1s)}$,$K = 10$
    \item 环节识别:
    \begin{itemize}
        \item 积分环节:$\frac{1}{s}$($v=1$)
        \item 惯性环节:$\frac{1}{1+0.1s}$,$T=0.1$
    \end{itemize}
    \item 转折频率:$\omega_c = \frac{1}{T} = 10$ rad/s
    \item 幅频特性:
    \begin{itemize}
        \item 起点:$L(1) = 20$ dB
        \item $\omega < 10$:$-20$ dB/十倍频
        \item $\omega > 10$:$-40$ dB/十倍频
        \item 修正:$-3$ dB @ $\omega=10$
    \end{itemize}
    \item 相频特性:
    \begin{itemize}
        \item $\phi(\omega) = -90° - \arctan(0.1\omega)$
        \item @ $\omega = 10$:$\phi = -135°$
    \end{itemize}
\end{enumerate}
\end{minipage}\hfill
\begin{minipage}[t]{0.47\textwidth}
% G(s) = 10/(s(1+0.1s)) -> num=10, den=0.1 s^2 + 1 s + 0
\BodeTF{num/{10},den/{0.1,1,0}}{0.1}{1000}
\end{minipage}

\end{tcolorbox}

\textbf{例2:}绘制 $G(s) = \frac{100(s+1)}{s(s+10)}$ 的伯德图

\begin{tcolorbox}[colback=blue!5!white,colframe=blue!75!black,title=例2求解过程与伯德图]

\begin{minipage}[t]{0.47\textwidth}
\textbf{解:}
\begin{enumerate}
    \item 标准形式:$G(s) = \frac{10(1+s)}{s(1+0.1s)}$,$K = 10$
    \item 环节识别:
    \begin{itemize}
        \item 积分环节:$\frac{1}{s}$
        \item 一阶零点:$(1+s)$
        \item $\omega_{c1} = 1$ rad/s
        \item 一阶极点:$\frac{1}{1+0.1s}$
        \item $\omega_{c2} = 10$ rad/s
    \end{itemize}
    \item 转折频率:$1$ rad/s(零点),$10$ rad/s(极点)
    \item 幅频特性:
    \begin{itemize}
        \item 低频 $\omega<1$:$-20$ dB/十倍频
        \item 中频 $1<\omega<10$:$0$ dB/十倍频
        \item 高频 $\omega>10$:$-20$ dB/十倍频
    \end{itemize}
    \item 相频特性:
    \begin{itemize}
        \item $\phi(\omega) = -90° + \arctan(\omega)$ 
        \item $\quad - \arctan(0.1\omega)$
    \end{itemize}
\end{enumerate}
\end{minipage}\hfill
\begin{minipage}[t]{0.47\textwidth}
% G(s) = 100(s+1)/(s(s+10)) -> num=100 s + 100, den = s^2 + 10 s + 0
\BodeTF{num/{100,100},den/{1,10,0}}{0.1}{1000}
\end{minipage}

\end{tcolorbox}

\textbf{例3:}二阶系统 $G(s) = \frac{100}{s^2 + 2s + 100}$

\begin{tcolorbox}[colback=blue!5!white,colframe=blue!75!black,title=例3求解过程与伯德图]

\begin{minipage}[t]{0.47\textwidth}
\textbf{解:}
\begin{enumerate}
    \item 标准形式:
    $$G(s) = \frac{\omega_n^2}{s^2 + 2\zeta\omega_n s + \omega_n^2}$$
    
    \item 参数识别:
    \begin{itemize}
        \item $\omega_n = 10$ rad/s
        \item $\zeta = 0.1$
        \item 极点:$-1 \pm 9.95j$
    \end{itemize}
    
    \item 特性分析:
    \begin{itemize}
        \item 转折频率:$10$ rad/s
        \item $\zeta < 0.707$,有谐振
        \item 谐振频率:$\approx 9.9$ rad/s
        \item 谐振峰值:$\approx 5.03$(14 dB)
    \end{itemize}
\end{enumerate}
\end{minipage}\hfill
\begin{minipage}[t]{0.47\textwidth}
\BodeTF{num/{100},den/{1,2,100}}{0.1}{1000}
\end{minipage}

\end{tcolorbox}

\subsection{伯德图的应用}

\subsubsection{由伯德图确定传递函数}
根据伯德图的幅频特性,可以反推传递函数:
\begin{enumerate}
    \item 从低频渐近线斜率确定积分环节数 $v$
    \item 从转折频率确定各环节的时间常数
    \item 从低频幅值确定增益 $K$
    \item 从斜率变化确定零极点类型
\end{enumerate}

\subsubsection{稳定性分析}
\textbf{增益裕度(Gain Margin,GM):}
\begin{align*}
\text{GM} = -L(\omega_g) \text{ dB}
\end{align*}
其中 $\omega_g$ 为相位穿越频率($\phi(\omega_g) = -180°$)

\textbf{相位裕度(Phase Margin,PM):}
\begin{align*}
\text{PM} = 180° + \phi(\omega_c)
\end{align*}
其中 $\omega_c$ 为幅值穿越频率($L(\omega_c) = 0$ dB)

\textbf{稳定性判据:}
\begin{itemize}
    \item 稳定条件:GM $> 0$ dB 且 PM $> 0°$
    \item 一般要求:GM $\geq 6$ dB,PM $\geq 30°$
    \item 良好性能:GM $\geq 10$ dB,PM $\geq 45°$
\end{itemize}

\subsubsection{性能指标估算}
\begin{itemize}
    \item \textbf{带宽频率} $\omega_b$:$L(\omega_b) = -3$ dB 对应的频率
    \begin{itemize}
        \item 带宽越大,系统响应越快
    \end{itemize}
    \item \textbf{谐振峰值} $M_r$:幅频特性的最大值
    \begin{itemize}
        \item $M_r$ 越大,超调量越大
        \item 一般要求 $M_r < 1.3$(即 $< 2.3$ dB)
    \end{itemize}
    \item \textbf{低频增益}:反映稳态精度
    \begin{itemize}
        \item 型别越高,低频增益越大,稳态误差越小
    \end{itemize}
    \item \textbf{中频段斜率}:影响稳定性
    \begin{itemize}
        \item 斜率为 $-20$ dB/十倍频时系统一般稳定
        \item 斜率为 $-40$ dB/十倍频时需检查相位裕度
    \end{itemize}
\end{itemize}
