\section{离散系统}
\label{sec:discrete-systems}

\subsection*{引言:从连续到数字的控制革命}

\textbf{一个真实的世界:}你的智能手机、工业机器人、飞机自动驾驶仪、医疗呼吸机——它们都在进行\textbf{数字控制}。没有一个现代控制系统运行在"连续"的世界里;所有的控制决策都是在\textbf{离散的时刻}(通常每毫秒或每微秒)计算一次。这引发了一个根本性的问题:

\textbf{为什么要从连续系统转向离散系统?}

\begin{enumerate}
    \item \textbf{现实:}所有实际的控制器都是数字计算机(CPU执行代码是离散的)
    \item \textbf{好处:}
    \begin{itemize}
        \item 便宜:通用芯片(ARM、DSP)比模拟电路便宜得多
        \item 灵活:可编程,容易修改控制算法
        \item 精确:不受模拟元器件偏差影响
        \item 集成:易于实现复杂的多变量控制
        \item 鲁棒:数字信号处理的抗干扰能力强
    \end{itemize}
    \item \textbf{代价:}离散化带来新的问题(采样失真、量化误差、计算延迟)
\end{enumerate}

\textbf{关键概念的转变:}
\begin{itemize}
    \item 连续→离散:函数 $f(t)$ → 序列 $f(kT)$(采样)
    \item 拉普拉斯变换 → Z变换(分析工具)
    \item 传递函数 $G(s)$ → 脉冲传递函数 $G(z)$(系统模型)
    \item 极点在左半平面 → 极点在单位圆内(稳定域)
    \item 连续PID → 数字PID(实现方式)
\end{itemize}

\textbf{本章的重点:}

这一章不是简单地把连续理论"翻译"成离散版本——那样太无聊了。而是要理解:
\begin{itemize}
    \item \textbf{采样过程}如何影响系统行为(采样定理、信息丢失)
    \item \textbf{Z变换}如何像拉普拉斯变换一样工作(从信号处理视角)
    \item \textbf{零阶保持器}如何改变系统特性(离散→连续的桥梁)
    \item \textbf{稳定性判据}为何改变(z平面的特殊形状)
    \item \textbf{数字控制器}如何实现(从传递函数到差分方程到代码)
    \item \textbf{采样周期的选择}如何影响整个设计(实时性vs精度的权衡)
\end{itemize}

\textbf{实际应用预告:}
\begin{itemize}
    \item \textbf{无人机姿态控制}:每10ms更新一次控制命令
    \item \textbf{电动车动力管理}:每1ms采样一次电流和温度
    \item \textbf{工业机器人}:伺服循环周期通常1-10ms
    \item \textbf{电力系统频率调整}:采样周期从毫秒到秒级
\end{itemize}

\textbf{警告:常见误区}
\begin{itemize}
    \item ✗ 认为离散系统就是"连续系统采样一下"(事实:有本质区别)
    \item ✗ 忽视采样定理(后果:信号失真、控制失效)
    \item ✗ 采用太小的采样周期(后果:计算负担重、成本高)
    \item ✗ 采用太大的采样周期(后果:控制性能差、可能不稳定)
\end{itemize}

\subsection{离散系统概述

\subsubsection{离散系统的定义}

\textbf{离散系统}(数字控制系统)是指信号在时间上离散,通常涉及采样、数字处理和保持等环节。

\begin{minipage}[t]{0.52\textwidth}
\textbf{连续系统 vs 离散系统:}

\textbf{连续系统:}
\begin{itemize}
    \item 信号连续变化
    \item 微分方程描述
    \item 拉普拉斯变换分析
\end{itemize}

\textbf{离散系统:}
\begin{itemize}
    \item 信号在采样时刻定义
    \item 差分方程描述
    \item Z变换分析
\end{itemize}

\vspace{0.3cm}
\textbf{离散系统的组成:}

\begin{enumerate}
    \item \textbf{采样器}:将连续信号转换为离散序列
    \item \textbf{数字控制器}:对离散信号进行数字处理
    \item \textbf{保持器}:将离散信号还原为连续信号
    \item \textbf{被控对象}:连续系统
\end{enumerate}

\vspace{0.3cm}
\textbf{采样定理(Nyquist-Shannon定理):}

为无失真恢复连续信号,采样频率必须:
\begin{align*}
f_s \geq 2f_{\max}
\end{align*}

或采样周期:
\begin{align*}
T \leq \frac{\pi}{\omega_{\max}}
\end{align*}

其中 $f_{\max}$ 是信号的最高频率分量。

\textbf{工程实践:}
\begin{itemize}
    \item 一般取 $f_s = (6 \sim 10) f_{\max}$
    \item 或采样周期 $T = (0.1 \sim 0.5) T_s$($T_s$ 为系统时间常数)
\end{itemize}
\end{minipage}\hfill
\begin{minipage}[t]{0.45\textwidth}
\vspace{0pt}
\textbf{离散控制系统结构:}

离散控制系统包含以下环节:
\begin{itemize}
    \item 采样器(周期 $T$)
    \item 数字控制器
    \item 保持器(通常为零阶保持器)
    \item 被控对象(连续系统)
\end{itemize}

采样器将连续信号在离散时刻采样,数字控制器对离散信号进行处理,保持器将离散信号还原为阶梯波形式的连续信号,最后作用于被控对象。

\vspace{0.3cm}
\textbf{采样过程:}

采样过程将连续信号 $r(t)$ 在时刻 $t = kT$($k=0,1,2,\ldots$)进行采样,得到离散序列 $r(kT)$。
\end{minipage}

\subsection{Z变换}

Z变换是分析离散系统的数学工具,类似于连续系统中的拉普拉斯变换。

\subsubsection{Z变换的定义}

对于离散序列 $\{f(kT)\}$($k = 0, 1, 2, \ldots$),其Z变换定义为:
\begin{align*}
F(z) = \mathcal{Z}\{f(kT)\} = \sum_{k=0}^{\infty} f(kT) z^{-k}
\end{align*}

其中 $z$ 是复变量。

\subsubsection{Z变换与拉氏变换的关系}

对于采样信号 $f^*(t) = \sum_{k=0}^{\infty} f(kT)\delta(t-kT)$,其拉氏变换为:
\begin{align*}
F^*(s) = \sum_{k=0}^{\infty} f(kT) e^{-skT}
\end{align*}

令 $z = e^{sT}$,则:
\begin{align*}
F(z) = F^*(s)\Big|_{z=e^{sT}}
\end{align*}

\textbf{$s$ 平面与 $z$ 平面的映射:}

\begin{center}
\begin{tikzpicture}[scale=0.7]
% s平面
\begin{scope}[shift={(0,0)}]
\draw[->] (-2,0) -- (2,0) node[right] {$\text{Re}(s)$};
\draw[->] (0,-2) -- (0,2) node[above] {$\text{Im}(s)$};
\draw[blue, thick] (-2,-2) -- (-2,2);
\draw[red, thick] (0,-2) -- (0,2);
\node[blue, left] at (-2,1.5) {不稳定};
\node[red, right] at (0,1.5) {稳定};
\node[below] at (0,-2.5) {$s$ 平面};
\end{scope}

% 箭头
\draw[->, thick] (2.5,0) -- (3.5,0) node[midway, above] {\small $z=e^{sT}$};

% z平面
\begin{scope}[shift={(6,0)}]
\draw[->] (-2,0) -- (2,0) node[right] {$\text{Re}(z)$};
\draw[->] (0,-2) -- (0,2) node[above] {$\text{Im}(z)$};
\draw[thick] (0,0) circle (1);
\fill[blue, opacity=0.2] (0,0) circle (1);
\node[red] at (1.5,0) {不稳定};
\node[blue] at (0.5,0.5) {稳定};
\node[below] at (0,-2.5) {$z$ 平面};
\end{scope}
\end{tikzpicture}
\end{center}

\textbf{关键对应关系:}
\begin{itemize}
    \item $s$ 平面左半平面 $\leftrightarrow$ $z$ 平面单位圆内
    \item $s$ 平面虚轴 $\leftrightarrow$ $z$ 平面单位圆上
    \item $s$ 平面右半平面 $\leftrightarrow$ $z$ 平面单位圆外
\end{itemize}

\subsubsection{常用序列的Z变换}

\begin{center}
\begin{tabular}{|c|c|c|}
\hline
\textbf{序列} $f(kT)$ & \textbf{Z变换} $F(z)$ & \textbf{收敛域} \\
\hline
$\delta(kT)$(单位脉冲) & $1$ & 全平面 \\
\hline
$1(kT)$(单位阶跃) & $\frac{z}{z-1}$ & $|z| > 1$ \\
\hline
$kT$ & $\frac{Tz}{(z-1)^2}$ & $|z| > 1$ \\
\hline
$e^{-akT}$ & $\frac{z}{z-e^{-aT}}$ & $|z| > e^{-aT}$ \\
\hline
$\sin(\omega kT)$ & $\frac{z\sin(\omega T)}{z^2 - 2z\cos(\omega T) + 1}$ & $|z| > 1$ \\
\hline
$\cos(\omega kT)$ & $\frac{z[z-\cos(\omega T)]}{z^2 - 2z\cos(\omega T) + 1}$ & $|z| > 1$ \\
\hline
$a^k$ & $\frac{z}{z-a}$ & $|z| > |a|$ \\
\hline
$ka^k$ & $\frac{az}{(z-a)^2}$ & $|z| > |a|$ \\
\hline
\end{tabular}
\end{center}

\subsubsection{Z变换的性质}

\begin{center}
\begin{tabular}{|l|c|c|}
\hline
\textbf{性质} & \textbf{时域} & \textbf{Z域} \\
\hline
线性 & $a f_1(kT) + b f_2(kT)$ & $a F_1(z) + b F_2(z)$ \\
\hline
右移 & $f(kT - nT)$ & $z^{-n} F(z)$ \\
\hline
左移 & $f(kT + nT)$ & $z^n [F(z) - \sum_{k=0}^{n-1} f(kT) z^{-k}]$ \\
\hline
初值定理 & $f(0)$ & $\lim_{z\to\infty} F(z)$ \\
\hline
终值定理 & $\lim_{k\to\infty} f(kT)$ & $\lim_{z\to 1} (z-1)F(z)$ \\
\hline
卷积 & $\sum_{i=0}^k f_1(iT) f_2(kT-iT)$ & $F_1(z) \cdot F_2(z)$ \\
\hline
\end{tabular}
\end{center}

\subsection{脉冲传递函数}

\subsubsection{定义}

脉冲传递函数是离散系统的输出Z变换与输入Z变换之比(零初始条件):
\begin{align*}
G(z) = \frac{Y(z)}{X(z)}
\end{align*}

\subsubsection{零阶保持器(ZOH)}

零阶保持器在采样周期内保持采样值不变。

\begin{minipage}[t]{0.52\textwidth}
\textbf{零阶保持器的传递函数:}
\begin{align*}
G_h(s) = \frac{1 - e^{-Ts}}{s}
\end{align*}

\textbf{含零阶保持器的系统:}

采样器 + ZOH + 连续对象 $G(s)$

等效脉冲传递函数:
\begin{align*}
G(z) = \mathcal{Z}\left\{\frac{1-e^{-Ts}}{s} G(s)\right\} = (1-z^{-1})\mathcal{Z}\left\{\frac{G(s)}{s}\right\}
\end{align*}

\vspace{0.3cm}
\textbf{常用公式:}

对于 $G(s) = \frac{1}{s+a}$:
\begin{align*}
G(z) = (1-z^{-1})\mathcal{Z}\left\{\frac{1}{s(s+a)}\right\} = \frac{1-e^{-aT}}{z-e^{-aT}}
\end{align*}

对于 $G(s) = \frac{1}{s(s+a)}$:
\begin{align*}
G(z) = (1-z^{-1})\mathcal{Z}\left\{\frac{1}{s^2(s+a)}\right\} = \frac{T - \frac{1-e^{-aT}}{a}}{(z-1)(z-e^{-aT})}
\end{align*}
\end{minipage}\hfill
\begin{minipage}[t]{0.45\textwidth}
\vspace{0pt}
\textbf{零阶保持器输出示意:}

零阶保持器(Zero-Order Hold, ZOH)在每个采样周期内保持采样值不变,将离散信号转换为阶梯波形式的连续信号。

输出特点:
\begin{itemize}
    \item 在每个采样周期 $[kT, (k+1)T)$ 内,输出保持为 $y(kT)$
    \item 在采样时刻发生跳变
    \item 形成阶梯波形
\end{itemize}

\vspace{0.5cm}
\textbf{例题:}求含ZOH和对象 $G(s) = \frac{1}{s+2}$ 的脉冲传递函数,$T = 0.1$ s。

\textit{解:}

\textbf{方法1:查表法}
\begin{align*}
G(z) &= (1-z^{-1})\mathcal{Z}\left\{\frac{1}{s(s+2)}\right\} \\
&= \frac{1-e^{-2T}}{z-e^{-2T}} \\
&= \frac{1-e^{-0.2}}{z-e^{-0.2}} \\
&= \frac{0.1813}{z-0.8187}
\end{align*}

\textbf{方法2:部分分式法}
\begin{align*}
\frac{G(s)}{s} &= \frac{1}{s(s+2)} = \frac{0.5}{s} - \frac{0.5}{s+2} \\
\mathcal{Z}\left\{\frac{1}{s(s+2)}\right\} &= 0.5\frac{z}{z-1} - 0.5\frac{z}{z-e^{-2T}} \\
G(z) &= (1-z^{-1}) \times \text{上式} \\
&= \frac{0.1813}{z-0.8187}
\end{align*}
\end{minipage}

\subsection{离散系统的稳定性}

\subsubsection{稳定性判据}

离散系统稳定的充要条件:闭环脉冲传递函数 $\Phi(z)$ 的所有极点都在 $z$ 平面单位圆内,即:
\begin{align*}
|z_i| < 1, \quad i = 1, 2, \ldots, n
\end{align*}

\subsubsection{劳斯判据的应用(双线性变换)}

通过双线性变换 $z = \frac{1+w}{1-w}$,将 $z$ 平面单位圆内映射到 $w$ 平面左半平面,然后应用劳斯判据。

\textbf{步骤:}

1. 求闭环特征方程 $D(z) = 0$

2. 进行双线性变换:$z = \frac{1+w}{1-w}$,得 $D(w) = 0$

3. 对 $D(w) = 0$ 应用劳斯判据

\vspace{0.3cm}
\textbf{例题:}判断系统 $\Phi(z) = \frac{K}{z^2 - 1.5z + 0.5}$ 的稳定性,$K = 1$。

\textit{解:}

\textbf{1. 特征方程}
\begin{align*}
D(z) = z^2 - 1.5z + 0.5 + K = z^2 - 1.5z + 1.5 = 0
\end{align*}

\textbf{2. 双线性变换}

令 $z = \frac{1+w}{1-w}$:
\begin{align*}
\left(\frac{1+w}{1-w}\right)^2 - 1.5\frac{1+w}{1-w} + 1.5 &= 0 \\
(1+w)^2 - 1.5(1+w)(1-w) + 1.5(1-w)^2 &= 0 \\
1 + 2w + w^2 - 1.5(1-w^2) + 1.5(1-2w+w^2) &= 0 \\
1 + 2w + w^2 - 1.5 + 1.5w^2 + 1.5 - 3w + 1.5w^2 &= 0 \\
4w^2 - w + 1 &= 0
\end{align*}

\textbf{3. 劳斯表}
\begin{center}
\begin{tabular}{c|cc}
$w^2$ & 4 & 1 \\
$w^1$ & -1 & 0 \\
$w^0$ & 1 &  \\
\end{tabular}
\end{center}

第一列有符号变化(+,-,+)$\implies$ \textbf{不稳定}

\textbf{验证:}直接求根:$z = \frac{1.5 \pm \sqrt{2.25-6}}{2} = 0.75 \pm 0.968j$,$|z| = 1.22 > 1$

\subsubsection{Jury稳定性判据}

Jury判据直接在 $z$ 域判断稳定性,无需变换。

设特征方程:
\begin{align*}
D(z) = a_0 z^n + a_1 z^{n-1} + \cdots + a_{n-1} z + a_n = 0
\end{align*}

\textbf{稳定的必要条件:}
\begin{align*}
D(1) > 0, \quad (-1)^n D(-1) > 0
\end{align*}

\textbf{充要条件:}构造Jury表,所有行的首元素满足特定符号条件。

(详细方法略,工程中常用数值计算求根判断)

\subsection{离散系统的动态性能}

\subsubsection{稳态误差}

\textbf{终值定理:}
\begin{align*}
e_{ss} = \lim_{k\to\infty} e(kT) = \lim_{z\to 1} (z-1)E(z)
\end{align*}

\textbf{单位阶跃输入:}$R(z) = \frac{z}{z-1}$
\begin{align*}
e_{ss} = \lim_{z\to 1} (z-1) \frac{R(z)}{1+G(z)} = \frac{1}{1 + \lim_{z\to 1} G(z)}
\end{align*}

\textbf{单位斜坡输入:}$R(z) = \frac{Tz}{(z-1)^2}$
\begin{align*}
e_{ss} = \lim_{z\to 1} (z-1) \frac{Tz}{(z-1)^2[1+G(z)]} = \frac{T}{\lim_{z\to 1} (z-1)G(z)/T}
\end{align*}

\subsubsection{瞬态性能指标}

离散系统的时域响应可以通过Z反变换求得:
\begin{align*}
y(kT) = \mathcal{Z}^{-1}\{Y(z)\}
\end{align*}

\textbf{常用方法:}
\begin{itemize}
    \item \textbf{长除法}:直接展开为序列
    \item \textbf{部分分式法}:分解后查表反变换
    \item \textbf{留数法}:复变函数理论
\end{itemize}

\textbf{性能指标:}
\begin{itemize}
    \item 上升时间 $t_r$
    \item 超调量 $\sigma\%$
    \item 调节时间 $t_s$
    \item 振荡次数
\end{itemize}

\subsection{数字PID控制}

\subsubsection{连续PID的离散化}

连续PID控制器:
\begin{align*}
u(t) = K_p e(t) + K_i \int_0^t e(\tau) d\tau + K_d \frac{de(t)}{dt}
\end{align*}

\textbf{位置式数字PID:}
\begin{align*}
u(k) = K_p e(k) + K_i T \sum_{j=0}^k e(j) + K_d \frac{e(k) - e(k-1)}{T}
\end{align*}

\textbf{速度式数字PID:}
\begin{align*}
\Delta u(k) = u(k) - u(k-1) &= K_p [e(k) - e(k-1)] \\
&\quad + K_i T e(k) \\
&\quad + K_d \frac{e(k) - 2e(k-1) + e(k-2)}{T}
\end{align*}

\subsubsection{数字PID的改进}

\textbf{1. 积分分离}:误差大时不积分,避免积分饱和
\begin{align*}
u(k) = K_p e(k) + K_i T \sum_{j=k_0}^k e(j) + K_d \frac{e(k) - e(k-1)}{T}
\end{align*}

\textbf{2. 微分先行}:仅对输出微分,避免对设定值突变响应过大
\begin{align*}
u(k) = K_p e(k) + K_i T \sum_{j=0}^k e(j) - K_d \frac{y(k) - y(k-1)}{T}
\end{align*}

\textbf{3. 不完全微分}:加入滤波环节,减小噪声影响
\begin{align*}
u_d(k) = \alpha u_d(k-1) + K_d (1-\alpha) \frac{e(k) - e(k-1)}{T}
\end{align*}

其中 $\alpha = \frac{T}{T + \tau_d}$,$\tau_d$ 为滤波时间常数。

\subsection{离散系统设计}

\subsubsection{最少拍控制}

\textbf{设计目标:}在最少的采样周期内使系统输出达到并保持在期望值。

\textbf{设计步骤:}

1. 确定期望闭环传递函数 $\Phi^*(z)$

2. 计算控制器传递函数:
\begin{align*}
D(z) = \frac{\Phi^*(z)}{G(z)[1-\Phi^*(z)]}
\end{align*}

3. 验证物理可实现性和稳定性

\textbf{常用期望响应:}
\begin{itemize}
    \item 单拍系统:$\Phi^*(z) = z^{-1}$
    \item 有纹波最少拍:$\Phi^*(z) = \frac{(1-a)z^{-1}}{1-az^{-1}}$
    \item 无纹波最少拍(I型系统):$\Phi^*(z) = z^{-1}(2-z^{-1})$
\end{itemize}

\subsubsection{数字控制器的实现}

\textbf{直接编程法:}

控制器传递函数 $D(z) = \frac{U(z)}{E(z)}$ 转换为差分方程,直接编程实现。

\textbf{串行校正:}

类似连续系统的超前/滞后校正,设计数字校正器:
\begin{align*}
D(z) = K \frac{z - a}{z - b}
\end{align*}

\textbf{并行校正:}

采用状态反馈等现代控制方法。

\subsection*{本章总结}

\subsubsection*{核心主题重申(Restated Thesis)}

离散系统理论不是连续系统理论的\textquotedblleft简单翻译\textquotedblright,而是一次\textbf{范式转变}:从物理世界的连续变化到数字计算机的离散采样。这一转变带来了工程实现的革命性优势(低成本、灵活、可编程),但也引入了新的理论挑战(采样失真、稳定域变化、量化误差)。

\textbf{本章的核心洞察:}
\begin{quote}
\textit{理解离散系统 = 理解三个映射关系}
\begin{itemize}
    \item \textbf{时域映射}:连续信号 $f(t)$ → 离散序列 $f(kT)$(采样定理约束)
    \item \textbf{变换映射}:拉普拉斯$s$域 → Z变换$z$域($z=e^{sT}$,非线性!)
    \item \textbf{稳定域映射}:左半平面 → 单位圆内(极点位置判据改变)
\end{itemize}
\end{quote}

掌握这三个映射,就掌握了连续理论到离散实现的\textbf{桥梁}。

\subsubsection*{要点总结(Summary of Key Concepts)}

\textbf{1. 理论基础(Foundation)}

\begin{table}[htbp]
\centering
\caption{连续vs离散:核心对应关系}
\begin{tabular}{|l|p{5.5cm}|p{5.5cm}|}
\hline
\textbf{特性} & \textbf{连续系统} & \textbf{离散系统} \\
\hline
\textbf{时间描述} & 
$t \in \mathbb{R}^+$(连续)
\newline 信号在所有时刻定义
& 
$k \in \mathbb{N}$(离散)
\newline 信号仅在$t=kT$时刻定义
\\
\hline
\textbf{数学模型} & 
微分方程
\newline $\dot{x} = Ax + Bu$
& 
差分方程
\newline $x(k+1) = Ax(k) + Bu(k)$
\\
\hline
\textbf{变换工具} & 
拉普拉斯变换
\newline $F(s) = \int_0^\infty f(t)e^{-st}dt$
& 
Z变换
\newline $F(z) = \sum_{k=0}^\infty f(k)z^{-k}$
\\
\hline
\textbf{传递函数} & 
$G(s)$(s域)
\newline 极点零点在s平面
& 
$G(z)$(z域)
\newline 极点零点在z平面
\\
\hline
\textbf{稳定域} & 
左半平面 $\text{Re}(s) < 0$
& 
单位圆内 $|z| < 1$
\\
\hline
\textbf{稳定判据} & 
劳斯、奈奎斯特
& 
双线性+劳斯、Jury
\\
\hline
\textbf{控制器实现} & 
模拟电路(RC、运放)
& 
数字代码(差分方程)
\\
\hline
\end{tabular}
\end{table}

\textbf{2. 关键技术要素(Key Technical Elements)}

\paragraph{采样定理(Nyquist-Shannon Theorem)}
\begin{itemize}
    \item \textbf{理论要求}:$f_s \geq 2f_{\max}$(防止走样aliasing)
    \item \textbf{工程实践}:$f_s = (6 \sim 10) f_{\max}$(留有余量)
    \item \textbf{采样周期选择}:$T = (0.1 \sim 0.5) \tau$($\tau$为系统时间常数)
    \item \textbf{物理意义}:采样频率过低 → 信息永久丢失;过高 → 计算负担、成本增加
\end{itemize}

\paragraph{零阶保持器(Zero-Order Hold, ZOH)}
\begin{itemize}
    \item \textbf{作用}:将离散控制信号转换为阶梯波连续信号
    \item \textbf{传递函数}:$G_h(s) = \frac{1-e^{-Ts}}{s}$
    \item \textbf{影响}:引入相位滞后,改变系统动态特性
    \item \textbf{设计意义}:计算离散模型时\textbf{必须}考虑ZOH(不能直接采样!)
\end{itemize}

\paragraph{Z变换与s-z映射}
\begin{itemize}
    \item \textbf{映射公式}:$z = e^{sT}$(非线性映射)
    \item \textbf{极点映射}:$s = \sigma + j\omega \to z = e^{\sigma T}e^{j\omega T}$
    \item \textbf{稳定性映射}:左半平面 → 单位圆内(几何形状完全改变)
    \item \textbf{频率混叠}:$\omega$和$\omega + 2\pi/T$映射到同一个$z$(采样定理的根源)
\end{itemize}

\textbf{3. 设计方法论(Design Methodology)}

\textbf{数字控制器设计流程(五步法):}
\begin{enumerate}
    \item \textbf{采样周期选择}
    \begin{itemize}
        \item 估计系统时间常数$\tau$(从阶跃响应或传递函数)
        \item 选择$T = 0.1\tau \sim 0.5\tau$(快速系统用小值,慢速系统用大值)
        \item 验证:$T < \frac{\pi}{\omega_{max}}$(采样定理)
    \end{itemize}
    
    \item \textbf{离散模型获取}
    \begin{itemize}
        \item 连续模型:已知$G_p(s)$
        \item 加ZOH:$G(s) = \frac{1-e^{-Ts}}{s} G_p(s)$
        \item 离散化:$G(z) = \mathcal{Z}\{G(s)\}$或用双线性变换
    \end{itemize}
    
    \item \textbf{控制器设计}
    \begin{itemize}
        \item 数字PID:位置式或速度式(最常用)
        \item 最少拍:性能最优但设计复杂
        \item 根轨迹/频域:类似连续系统方法
    \end{itemize}
    
    \item \textbf{稳定性验证}
    \begin{itemize}
        \item 计算闭环极点(求根或Jury判据)
        \item 确认所有极点$|z_i| < 1$
        \item 检查稳态误差(终值定理)
    \end{itemize}
    
    \item \textbf{实现与调试}
    \begin{itemize}
        \item 写差分方程(递推关系)
        \item 编程(C/Python):通常2-5行代码
        \item 硬件测试:考虑量化、延迟、中断
    \end{itemize}
\end{enumerate}

\subsubsection*{深层分析(Critical Analysis)}

\textbf{为什么稳定域变成单位圆?(几何直觉)}

观察映射$z = e^{sT} = e^{\sigma T} \cdot e^{j\omega T}$:
\begin{itemize}
    \item \textbf{模长}:$|z| = e^{\sigma T}$
    \begin{itemize}
        \item 若$\sigma < 0$(稳定)→ $|z| < 1$(单位圆内)
        \item 若$\sigma = 0$(边界)→ $|z| = 1$(单位圆上)
        \item 若$\sigma > 0$(不稳定)→ $|z| > 1$(单位圆外)
    \end{itemize}
    \item \textbf{相位}:$\angle z = \omega T$(频率信息)
\end{itemize}

\textbf{结论}:单位圆是左半平面的\textquotedblleft自然像\textquotedblright——稳定性判据的形式变了,但\textbf{物理意义不变}(衰减 vs 发散)。

\textbf{离散化为何改变系统特性?}

\textbf{例子}:连续系统$G(s) = \frac{1}{s+1}$,时间常数$\tau=1$s

\begin{itemize}
    \item \textbf{采样周期$T=0.1$s}:$G(z) = \frac{0.0952}{z-0.9048}$,极点$z=0.9048$(接近单位圆,响应快)
    \item \textbf{采样周期$T=1$s}:$G(z) = \frac{0.632}{z-0.368}$,极点$z=0.368$(远离单位圆,响应慢)
\end{itemize}

\textbf{洞察}:$T$越大,离散系统与连续系统差异越大!采样周期选择\textbf{直接影响}控制性能。

\textbf{数字PID vs 模拟PID:本质区别}

\begin{center}
\begin{tabular}{|l|p{5cm}|p{5cm}|}
\hline
 & \textbf{模拟PID} & \textbf{数字PID} \\
\hline
\textbf{积分} & 连续积分$\int e(t)dt$ & 累加求和$T\sum e(k)$ \\
\hline
\textbf{微分} & 瞬时导数$\frac{de}{dt}$ & 后向差分$\frac{e(k)-e(k-1)}{T}$ \\
\hline
\textbf{参数} & $K_p, K_i, K_d$独立 & $K_i, K_d$显式依赖$T$ \\
\hline
\textbf{调参} & 物理旋钮(难) & 软件修改(易) \\
\hline
\textbf{问题} & 器件漂移、噪声 & 量化、延迟、数值误差 \\
\hline
\end{tabular}
\end{center}

\textbf{关键}:数字PID的$K_i, K_d$\textbf{不是}模拟PID参数的直接复制,需要根据$T$重新调整!

\subsubsection*{常见误区与陷阱(Common Pitfalls)}

\textbf{概念误区:}
\begin{itemize}
    \item ✗ \textbf{误区1}:\textquotedblleft离散系统=连续系统采样一下\textquotedblright
    \item ✓ \textbf{正确}:离散系统在采样间隔内的行为由\textbf{零阶保持器}决定(阶梯波),与连续系统本质不同
    
    \item ✗ \textbf{误区2}:\textquotedblleft 采样定理只是理论,实践中可以违反\textquotedblright
    \item ✓ \textbf{正确}:违反采样定理导致\textbf{不可恢复}的频谱混叠(aliasing),控制系统会看到错误的频率信息
    
    \item ✗ \textbf{误区3}:\textquotedblleft 提高采样频率总是更好\textquotedblright
    \item ✓ \textbf{正确}:过高采样频率增加计算成本、功耗、电磁干扰,且收益递减($T < 0.01\tau$几乎无意义)
    
    \item ✗ \textbf{误区4}:\textquotedblleft Z变换和拉普拉斯变换完全一样\textquotedblright
    \item ✓ \textbf{正确}:形式类似但映射$z=e^{sT}$是\textbf{非线性}的,导致稳定域、频率响应都不同
\end{itemize}

\textbf{计算易错点:}
\begin{itemize}
    \item ✗ 直接用s平面极点判断离散系统稳定性
    \item ✓ 正确:必须通过$z=e^{sT}$映射到z平面,或用Jury判据
    
    \item ✗ 计算$G(z)$时忽视零阶保持器$\frac{1-e^{-Ts}}{s}$
    \item ✓ 正确:$G(z) = \mathcal{Z}\left\{\frac{1-e^{-Ts}}{s}G_p(s)\right\} = (1-z^{-1})\mathcal{Z}\left\{\frac{G_p(s)}{s}\right\}$
    
    \item ✗ 用欧拉法离散化连续PID后不重新调参
    \item ✓ 正确:积分增益要乘$T$,微分增益要除$T$(查看差分方程)
    
    \item ✗ 忘记检查数字控制器的\textbf{物理可实现性}(最少拍设计常见)
    \item ✓ 正确:$D(z)$的分子阶数不能高于分母(非因果系统无法实现)
\end{itemize}

\textbf{设计陷阱:}
\begin{itemize}
    \item ✗ 根据连续系统带宽选择采样周期(忽视系统时间常数)
    \item ✓ 正确:应根据\textbf{最慢}系统模态的时间常数选择($T \sim 0.1\tau_{max}$)
    
    \item ✗ 在z平面设计时忘记考虑输入信号的频率
    \item ✓ 正确:输入频率接近Nyquist频率$\pi/T$时性能严重下降
    
    \item ✗ 忽视计算延迟和量化误差的累积效应
    \item ✓ 正确:这是理论计算与实际系统性能差异的\textbf{主要来源}
\end{itemize}

\subsubsection*{学习检查清单(Self-Assessment Checklist)}

\textbf{理论理解(Understanding):}
\begin{itemize}
    \item[$\square$] 能用自己的话解释为什么需要离散系统(计算机的本质)
    \item[$\square$] 理解采样定理的物理意义(信息保存vs丢失的边界)
    \item[$\square$] 理解$z=e^{sT}$映射如何改变稳定域(左半平面→单位圆)
    \item[$\square$] 知道零阶保持器的作用(不是简单的\textquotedblleft 采样\textquotedblright )
    \item[$\square$] 理解为什么离散化会改变系统特性(采样周期的影响)
\end{itemize}

\textbf{计算能力(Skills):}
\begin{itemize}
    \item[$\square$] 会计算简单序列的Z变换(查表或定义)
    \item[$\square$] 会从连续模型$G_p(s)$计算离散模型$G(z)$(考虑ZOH)
    \item[$\square$] 会用Jury判据或求根判定离散系统稳定性
    \item[$\square$] 会设计位置式/速度式数字PID控制器
    \item[$\square$] 会将传递函数$D(z)$转换为差分方程(递推式)
\end{itemize}

\textbf{工程应用(Application):}
\begin{itemize}
    \item[$\square$] 能根据系统时间常数选择合适的采样周期
    \item[$\square$] 能从差分方程写出可执行的C/Python代码
    \item[$\square$] 能分析采样周期对控制性能的影响
    \item[$\square$] 理解实际产品中的数字控制如何工作(嵌入式系统)
    \item[$\square$] 能解释连续设计在离散实现中为何失效
\end{itemize}

\textbf{深入思考(Critical Thinking):}
\begin{itemize}
    \item[$\square$] 为什么模拟控制器在20世纪90年代几乎完全被数字控制器取代?
    \item[$\square$] 采样和量化,哪个对控制性能影响更大?为什么?
    \item[$\square$] 当采样频率趋于无穷大时,离散系统会怎样趋近连续系统?
    \item[$\square$] 为什么无人机、机器人通常用1-10ms采样周期?(平衡什么?)
\end{itemize}

\subsubsection*{与课程其他章节的衔接(Course Integration)}

\textbf{向后回顾(Prerequisites):}
\begin{itemize}
    \item \textbf{第\ref{sec:transfer-function}章(传递函数)}:$G(z)$是$G(s)$的离散类比,物理意义相同
    \item \textbf{第\ref{sec:state-space}章(状态空间)}:离散状态方程$x(k+1)=Ax(k)+Bu(k)$结构相同
    \item \textbf{第\ref{sec:stability}章(稳定性)}:稳定性判据从左半平面改为单位圆内,但\textbf{本质}(衰减vs发散)不变
    \item \textbf{第\ref{sec:controllability-observability}章}:能控能观判据在离散系统中\textbf{完全适用}(只需替换$A,B,C$)
\end{itemize}

\textbf{前向展望(Applications):}
\begin{itemize}
    \item 这是现代控制理论在实际系统中的\textbf{最终形态}
    \item 所有实际控制系统(无人机、机器人、自动驾驶)都是离散实现
    \item 下一步深化方向:
    \begin{itemize}
        \item 自适应控制(参数在线调整)
        \item 预测控制(MPC,利用模型预测未来)
        \item 强化学习控制(数据驱动,无需模型)
    \end{itemize}
\end{itemize}

\subsubsection*{实践建议与工程智慧(Practical Wisdom)}

\textbf{采样周期选择的工程经验:}
\begin{itemize}
    \item \textbf{快速系统}($\tau < 0.1$s,如伺服电机):$T = 1$-$5$ms
    \item \textbf{中速系统}($\tau \sim 1$s,如温度控制):$T = 0.1$-$0.5$s
    \item \textbf{慢速系统}($\tau > 10$s,如大型化工过程):$T = 1$-$10$s
    \item \textbf{原则}:\textquotedblleft 不要为了理论完美牺牲实用性\textquotedblright——$T$够用即可
\end{itemize}

\textbf{数字PID调参技巧:}
\begin{enumerate}
    \item 先在连续域设计PID(用Ziegler-Nichols或频域方法)
    \item 离散化:$K_i^{digital} = K_i^{continuous} \times T$,$K_d^{digital} = K_d^{continuous} / T$
    \item 实际测试:通常需要微调($\pm 20\%$)
    \item 警惕积分饱和:离散PID更容易出现(加积分限幅)
\end{enumerate}

\textbf{调试常见问题与解决:}
\begin{itemize}
    \item \textbf{问题}:理论稳定但实际振荡
    \item \textbf{原因}:计算延迟(1-2个采样周期)未考虑
    \item \textbf{解决}:在$z$域模型中加$z^{-1}$或$z^{-2}$延迟项,重新设计
\end{itemize}

\begin{itemize}
    \item \textbf{问题}:数字PID性能远不如模拟PID
    \item \textbf{原因}:采样周期过大,或参数未重新调整
    \item \textbf{解决}:减小$T$至$0.1\tau$,或使用速度式PID减小计算量
\end{itemize}

\subsubsection*{结束语(Concluding Thoughts)}

离散系统理论是控制工程从\textquotedblleft 黑板\textquotedblright 走向\textquotedblleft 芯片\textquotedblright 的关键桥梁。它不仅仅是数学变换的游戏,而是深刻反映了\textbf{数字计算的本质}:

\begin{quote}
\textit{\textquotedblleft 所有的连续都是理想化的抽象;真实世界的计算都是离散的。\textquotedblright}
\end{quote}

当你理解了为什么\textbf{稳定域}从左半平面变成单位圆,为什么\textbf{采样定理}不可违反,为什么\textbf{零阶保持器}改变系统动态——你就真正理解了数字控制的精髓。

\textbf{最后的提醒}:
\begin{itemize}
    \item 离散化\textbf{不是}连续系统的简单替代品,而是一个\textbf{新的设计空间}
    \item 采样周期$T$是离散系统设计中\textbf{最关键的参数}——选错了,其他都白费
    \item 传递函数只是起点;差分方程才是终点(可执行的代码)
    \item \textbf{理论计算永远比实际乐观}——量化、延迟、干扰是不可避免的
\end{itemize}

掌握离散系统,你就掌握了现代控制工程的\textbf{实践钥匙}。从算法到芯片,从理论到产品,离散系统理论是\textbf{最后一公里}。

\textbf{记住}:每一个工作的控制系统,背后都是一行行差分方程在\textbf{离散的时刻}默默执行——这就是离散系统的力量!
