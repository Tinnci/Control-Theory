\section{奈奎斯特图绘制}

\subsection{奈奎斯特图的定义}
奈奎斯特图是以 $G(\jw)$ 的实部为横坐标,虚部为纵坐标,$\omega$ 从 $-\infty$ 到 $+\infty$ 变化时 $G(\jw)$ 在复平面上的轨迹。

\subsection{典型环节的奈奎斯特图}
\begin{itemize}
    \item \textbf{比例环节}: $G(\jw) = K$。奈奎斯特图是实轴上的一个点 $(K, 0)$。
    \item \textbf{二阶振荡环节}: 曲线形状取决于阻尼比 $\zeta$。
\end{itemize}

\subsubsection{积分环节}
积分环节的传递函数为 $G(s) = \frac{1}{s}$,其频率特性为 $G(\jw) = \frac{1}{\jw} = -\frac{j}{\omega}$。
当 $\omega$ 从 $0^+$ 变化到 $+\infty$ 时,奈奎斯特图是沿着负虚轴从 $-\infty$ 到原点的一条直线。

\begin{center}
\begin{tikzpicture}
\begin{axis}[
    width=8cm, height=8cm,
    xlabel={实部 Re},
    ylabel={虚部 Im},
    grid=both,
    unit vector ratio=1 1 1,
    title={积分环节奈奎斯特图},
    xmin=-1.5, xmax=1.5,
    ymin=-2, ymax=1,
    axis lines=middle,
    xtick={-1,0,1},
    ytick={-2,-1,0,1},
]
% Nyquist plot for 1/s: G(jω) = -j/ω
\addplot[blue, thick, domain=0.1:20, samples=200, ->] 
    ({0}, {-1/x});
\node at (axis cs:0.5, -1) {$\omega: 0^+ \to +\infty$};
\end{axis}
\end{tikzpicture}
\end{center}

\subsubsection{惯性环节}
惯性环节的传递函数为 $G(s) = \frac{1}{Ts+1}$,其频率特性为 $G(\jw) = \frac{1}{1+j\omega T}$。
奈奎斯特图是一个起点为 $(1,0)$ ($\omega=0$),终点为原点 ($\omega=\infty$) 的半圆。

\begin{center}
\begin{tikzpicture}
\begin{axis}[
    width=8cm, height=6cm,
    xlabel={实部 Re},
    ylabel={虚部 Im},
    grid=both,
    unit vector ratio=1 1 1,
    title={惯性环节奈奎斯特图 ($T=1$)},
    xmin=-0.2, xmax=1.2,
    ymin=-0.7, ymax=0.2,
    axis lines=middle,
]
% Nyquist plot for 1/(1+jωT): Re = 1/(1+ω²T²), Im = -ωT/(1+ω²T²)
\addplot[blue, thick, domain=0:15, samples=200, ->] 
    ({1/(1+x^2)}, {-x/(1+x^2)});
\node at (axis cs:0.5, -0.4) {$\omega: 0 \to \infty$};
\node at (axis cs:1, 0.1) {$(1,0)$};
\node at (axis cs:0.05, 0.1) {$(0,0)$};
\end{axis}
\end{tikzpicture}
\end{center}

\subsubsection{微分环节}
微分环节的传递函数为 $G(s) = s$,其频率特性为 $G(\jw) = j\omega$。
奈奎斯特图是沿着正虚轴从原点到 $+\infty$ 的一条直线。

\begin{center}
\begin{tikzpicture}
\begin{axis}[
    width=8cm, height=8cm,
    xlabel={实部 Re},
    ylabel={虚部 Im},
    grid=both,
    unit vector ratio=1 1 1,
    title={微分环节奈奎斯特图},
    xmin=-1.5, xmax=1.5,
    ymin=-0.5, ymax=2.5,
    axis lines=middle,
]
% Nyquist plot for jω
\addplot[blue, thick, domain=0:20, samples=200, ->] 
    ({0}, {x});
\node at (axis cs:0.5, 1) {$\omega: 0 \to +\infty$};
\end{axis}
\end{tikzpicture}
\end{center}

\subsubsection{比例环节}
比例环节的传递函数为 $G(s) = K$($K > 0$),其频率特性为 $G(\jw) = K$。
奈奎斯特图是实轴上从原点到 $K$ 的一个点。

\begin{center}
\begin{tikzpicture}
\begin{axis}[
    width=8cm, height=6cm,
    xlabel={实部 Re},
    ylabel={虚部 Im},
    grid=both,
    unit vector ratio=1 1 1,
    title={比例环节奈奎斯特图 ($K=2$)},
    xmin=-0.5, xmax=2.5,
    ymin=-1, ymax=1,
    axis lines=middle,
]
\addplot[blue, thick, only marks, mark=*] coordinates {(2, 0)};
\node at (axis cs:2, 0.3) {$(K, 0)$};
\end{axis}
\end{tikzpicture}
\end{center}

\subsubsection{二阶振荡环节}
二阶振荡环节的传递函数为 $G(s) = \frac{\omega_n^2}{s^2 + 2\zeta\omega_n s + \omega_n^2}$,其频率特性为:
$$G(\jw) = \frac{\omega_n^2}{\omega_n^2 - \omega^2 + 2\jw\zeta\omega_n\omega}$$

奈奎斯特图为一条圆弧或更复杂的曲线,形状随阻尼比 $\zeta$ 变化。

\begin{center}
\begin{tikzpicture}
\begin{axis}[
    width=10cm, height=8cm,
    xlabel={实部 Re},
    ylabel={虚部 Im},
    grid=both,
    title={二阶振荡环节奈奎斯特图 ($\omega_n=10$ rad/s)},
    xmin=-0.5, xmax=1.5,
    ymin=-1.5, ymax=1.5,
    axis lines=middle,
]
% ζ = 0.1 (欠阻尼,有谐振)
\addplot[blue, thick, domain=0:25, samples=300] 
    ({(100-x^2)/((100-x^2)^2 + (2*0.1*10*x)^2)}, 
     {-(2*0.1*10*x)/((100-x^2)^2 + (2*0.1*10*x)^2)});
\addlegendentry{$\zeta=0.1$}

% ζ = 0.5 (欠阻尼)
\addplot[red, thick, domain=0:25, samples=300] 
    ({(100-x^2)/((100-x^2)^2 + (2*0.5*10*x)^2)}, 
     {-(2*0.5*10*x)/((100-x^2)^2 + (2*0.5*10*x)^2)});
\addlegendentry{$\zeta=0.5$}

% ζ = 1.0 (临界阻尼)
\addplot[green!60!black, thick, domain=0:25, samples=300] 
    ({(100-x^2)/((100-x^2)^2 + (2*1.0*10*x)^2)}, 
     {-(2*1.0*10*x)/((100-x^2)^2 + (2*1.0*10*x)^2)});
\addlegendentry{$\zeta=1.0$}

\node at (axis cs:1, 0.1) {$\omega=0$};
\node at (axis cs:-0.2, -1.2) {$\omega=\omega_n$};
\end{axis}
\end{tikzpicture}
\end{center}

\subsection{奈奎斯特稳定性判据}

\subsubsection{基本概念}
对于开环传递函数 $G(s)H(s)$,其中 $H(s)$ 通常为反馈传感器传递函数:
\begin{itemize}
    \item \textbf{临界点}:奈奎斯特图中的点 $(-1, 0)$
    \item \textbf{包围}:奈奎斯特曲线绕临界点的圈数
\end{itemize}

\subsubsection{奈奎斯特稳定性判据}
对于闭环系统,设开环传递函数为 $G(s)H(s)$,其:
\begin{itemize}
    \item 右半平面极点数为 $P$
    \item 奈奎斯特曲线逆时针绕 $(-1, 0)$ 点的圈数为 $N$
\end{itemize}

则闭环系统右半平面极点数为:
$$Z = P + N$$

\textbf{稳定条件}:系统稳定 $\Leftrightarrow$ $Z = 0$,即 $N = -P$

特别地,当开环系统稳定($P = 0$)时:
\begin{itemize}
    \item 稳定条件:$N = 0$,即奈奎斯特曲线不包围 $(-1, 0)$ 点
    \item 边界稳定:曲线经过 $(-1, 0)$ 点
    \item 不稳定:曲线包围 $(-1, 0)$ 点
\end{itemize}

\subsubsection{稳定裕度}

\textbf{增益裕度(Gain Margin, GM)}
从 $(-1, 0)$ 点到奈奎斯特曲线与负实轴的交点距离的倒数。
若交点坐标为 $(-a, 0)$,则 $\text{GM} = \frac{1}{a}$ 或 $\text{GM}(\text{dB}) = 20\lg\frac{1}{a}$

\textbf{相位裕度(Phase Margin, PM)}
奈奎斯特曲线与单位圆的交点对应的相位角与 $-180°$ 的夹角。

\subsubsection{绘制方法}
\begin{enumerate}
    \item 建立 $G(\jw)$ 的实部和虚部表达式
    \item 选择足够多的频率点 $\omega$(从 $0$ 到 $+\infty$)
    \item 计算每个频率点对应的 $\text{Re}[G(\jw)]$ 和 $\text{Im}[G(\jw)]$
    \item 在复平面上绘制这些点形成的轨迹
    \item 检查轨迹是否包围 $(-1, 0)$ 点
\end{enumerate}
