\section{系统的结构分解——能控、能观性分解}
\label{sec:structural-decomposition}

\subsection*{引言:找到系统中的\textquotedblleft 盲区\textquotedblright}

\textbf{一个令人不安的发现:}在上一章(第\ref{sec:controllability-observability}章)我们学习了能控性和能观测性的判据。当你计算出某个系统$\text{rank}(W_c) = 2 < 3$时,你知道系统\textbf{不完全能控}——但这个结论太粗糙了!你不知道:
\begin{itemize}
    \item \textbf{哪个状态}是不能控的?(第一个?第二个?还是它们的某种组合?)
    \item 不能控状态对系统有什么影响?(会影响传递函数吗?会导致不稳定吗?)
    \item 如果我只想控制能控部分,该怎么设计控制器?
\end{itemize}

\textbf{类比:探测盲区}

想象你在一个黑暗的房间里拿着手电筒。手电筒能照亮一部分空间(\textbf{能控子空间}),但总有一些角落你照不到(\textbf{不能控子空间})。如果你不知道盲区在哪里,你可能:
\begin{itemize}
    \item 浪费精力试图控制根本无法控制的状态
    \item 忽略盲区中的不稳定动态,导致系统崩溃
    \item 设计过于复杂的控制器,却没有实际效果
\end{itemize}

\textbf{结构分解的目的:}\textbf{精确定位}系统中的能控/不能控、能观/不能观部分,将复杂系统\textbf{拆解}成四个清晰的子系统。

\textbf{为什么要分解系统?}
\begin{enumerate}
    \item \textbf{识别隐藏动态}:不能控且不能观的状态在传递函数中\textbf{不出现},但可能\textbf{影响稳定性}!
    \item \textbf{简化控制器设计}:只需关注能控且能观的部分(\textbf{最小实现})
    \item \textbf{理论分析}:解释为什么有些系统降阶后传递函数不变
    \item \textbf{故障诊断}:某个传感器或执行器失效时,快速判断影响范围
\end{enumerate}

\textbf{不可控/不可观部分的危害:}
\begin{itemize}
    \item \textbf{不能控但能观}:能看到问题,但无能为力(如看着火箭偏离轨道却无法纠正)
    \item \textbf{能控但不能观}:瞎子开车,输入在起作用但不知道效果
    \item \textbf{既不能控也不能观}:\textbf{最危险}!完全隐藏,可能包含不稳定极点
\end{itemize}

\textbf{本章路线图:}
\begin{itemize}
    \item \textbf{Kalman分解}:将系统分解为四个子系统(可控可观、可控不可观、不可控可观、不可控不可观)
    \item \textbf{能控性分解 vs 能观性分解}:两种分解方法的对偶关系
    \item \textbf{实际范例}:手把手分解一个三阶系统,看清楚\textquotedblleft 盲区\textquotedblright 在哪里
    \item \textbf{最小实现}:如何提取传递函数的最简模型
\end{itemize}

\subsection{术语问题:能观/能控 vs 可观/可控}

\subsubsection{术语等价性}
\textbf{能控性 (Controllability)} 和 \textbf{可控性} 是同义词;\textbf{能观性 (Observability)} 和 \textbf{可观性} 是同义词。在国内的控制理论教材和学术文献中,这两种说法都在广泛使用,可以互换。在严谨的定义和数学含义上,它们没有任何区别。

\subsection{系统的结构分解}
一般线性系统可分解为四个子系统:
\begin{itemize}
    \item 能控且能观测部分
    \item 能控但不能观测部分
    \item 不能控但能观测部分
    \item 不能控且不能观测部分
\end{itemize}

\subsection{卡尔曼分解}

\subsubsection{标准分解形式}

通过适当的线性变换,可将系统分解为:
\[\begin{bmatrix} \dot{x}_1 \\ \dot{x}_2 \\ \dot{x}_3 \\ \dot{x}_4 \end{bmatrix} = 
\begin{bmatrix}
A_{11} & 0 & A_{13} & 0 \\
A_{21} & A_{22} & A_{23} & A_{24} \\
0 & 0 & A_{33} & 0 \\
0 & 0 & A_{43} & A_{44}
\end{bmatrix}
\begin{bmatrix} x_1 \\ x_2 \\ x_3 \\ x_4 \end{bmatrix} +
\begin{bmatrix} B_1 \\ B_2 \\ 0 \\ 0 \end{bmatrix} u\]

\[y = \begin{bmatrix} C_1 & C_2 & 0 & 0 \end{bmatrix} \begin{bmatrix} x_1 \\ x_2 \\ x_3 \\ x_4 \end{bmatrix}\]

其中四个子系统对应:
\begin{itemize}
    \item $x_1$:\textbf{能控且能观}($B_1 \neq 0, C_1 \neq 0$)
    \item $x_2$:\textbf{能控但不能观}($B_2 \neq 0, C_2 = 0$)
    \item $x_3$:\textbf{不能控但能观}($B_3 = 0, C_3 \neq 0$)
    \item $x_4$:\textbf{不能控且不能观}($B_4 = 0, C_4 = 0$)
\end{itemize}

\subsubsection{物理意义解读}

\textbf{矩阵结构的含义:}观察上述分解中的\textbf{零块}(用粗体0表示),它们不是偶然的,而是有深刻物理意义的:

\paragraph{1. $B$矩阵的零块($B_3 = 0, B_4 = 0$)}
\begin{itemize}
    \item \textbf{含义}:输入$u$无法直接作用于$x_3$和$x_4$(不能控状态)
    \item \textbf{推论}:无论如何设计控制律,都无法改变这些状态的初值
\end{itemize}

\paragraph{2. $C$矩阵的零块($C_2 = 0, C_4 = 0$)}
\begin{itemize}
    \item \textbf{含义}:$x_2$和$x_4$对输出$y$无贡献(不能观状态)
    \item \textbf{推论}:这些状态的变化在输出中\textbf{完全隐藏},无法从测量中反推
\end{itemize}

\paragraph{3. $A$矩阵的零块(特定位置)}
\begin{itemize}
    \item \textbf{$A_{12} = 0$}:不能观部分$x_2$\textbf{不影响}能观部分$x_1$
    \item \textbf{$A_{14} = 0$}:完全隐藏部分$x_4$\textbf{不影响}能观部分$x_1$
    \item \textbf{$A_{31} = 0, A_{32} = 0$}:不能控部分$x_3$\textbf{不受}能控部分影响(自己演化)
    \item \textbf{$A_{34} = 0$}:不能控但能观$x_3$与完全隐藏$x_4$\textbf{解耦}
    \item \textbf{$A_{41} = 0, A_{42} = 0$}:完全隐藏部分$x_4$\textbf{不受}能控部分影响
\end{itemize}

\textbf{关键洞察:}
\begin{itemize}
    \item \textbf{传递函数$G(s) = C(sI-A)^{-1}B$只与$x_1$(能控且能观)有关!}
    \item $x_2$(能控不能观):输入能影响它,但输出看不到$\Rightarrow$不出现在$G(s)$
    \item $x_3$(不能控但能观):输出能看到它,但输入无法改变$\Rightarrow$不出现在$G(s)$
    \item $x_4$(既不能控也不能观):完全隐藏,\textbf{最危险}!可能包含不稳定极点却无法从$G(s)$中发现
\end{itemize}

\subsubsection{四子系统之间的信息流}

下图展示了四个子系统之间的动态耦合关系(箭头表示\textquotedblleft 影响\textquotedblright):

\begin{center}
\begin{tikzpicture}[node distance=3cm, auto, 
    block/.style={rectangle, draw, fill=blue!20, text width=2.5cm, text centered, rounded corners, minimum height=1.5cm},
    input/.style={coordinate},
    output/.style={coordinate}]

% Nodes
\node [block, fill=green!30] (x1) {$x_1$ \\ 能控能观};
\node [block, fill=yellow!30, right of=x1] (x2) {$x_2$ \\ 能控不能观};
\node [block, fill=orange!30, below of=x1] (x3) {$x_3$ \\ 不能控能观};
\node [block, fill=red!30, below of=x2] (x4) {$x_4$ \\ 不能控不能观};

% Input and output
\node [input, left of=x1, node distance=2cm] (input) {};
\node [output, above of=x1, node distance=2cm] (output) {};

% Arrows showing coupling
\draw[->, thick] (input) -- node {$u$} (x1);
\draw[->, thick] (input) -- (x2);
\draw[->, thick] (x1) -- node {$y$} (output);
\draw[->, thick] (x3) -- (output);

\draw[->, thick] (x2) -- (x1) node[midway, above] {$A_{21}$};
\draw[->, thick] (x3) -- (x1) node[midway, right] {$A_{13}$};
\draw[->, thick] (x2) -- (x3) node[midway, right] {$A_{23}$};
\draw[->, thick] (x4) -- (x2) node[midway, above] {$A_{24}$};
\draw[->, thick] (x4) -- (x3) node[midway, above] {$A_{43}$};

\draw[->, thick, dashed, red] (x2) edge[loop above] node {$A_{22}$} (x2);
\draw[->, thick, dashed, red] (x3) edge[loop left] node {$A_{33}$} (x3);
\draw[->, thick, dashed, red] (x4) edge[loop right] node {$A_{44}$} (x4);

\end{tikzpicture}
\end{center}

\textbf{图解说明:}
\begin{itemize}
    \item \textbf{实线箭头}:存在耦合(矩阵元素非零)
    \item \textbf{虚线箭头}:子系统内部动态(对角块$A_{ii}$)
    \item \textbf{缺失箭头}:零块,表示解耦
    \item \textbf{关键观察}:只有$x_1$同时接收输入$u$和贡献输出$y$
\end{itemize}

\textbf{例子:三阶系统分解}
假设某飞机模型分解后:
\begin{itemize}
    \item $x_1$(升降舵→俯仰角,可测可控):可以通过升降舵控制,且能从传感器测量$\Rightarrow$\textbf{正常工作}
    \item $x_2$(副翼→滚转角,可控但传感器失效):能控制但看不到效果$\Rightarrow$\textbf{瞎子控制}
    \item $x_3$(发动机温度,传感器正常但无控制输入):能监测但无法主动调节$\Rightarrow$\textbf{旁观者}
    \item $x_4$(某内部振动模态,无传感器无执行器):\textbf{完全隐藏},可能不稳定但从外部无法察觉!
\end{itemize}

\subsection{能观性分解与能控性分解对比}

两大分解题型在思想上是\textbf{对偶}的,解题步骤高度相似,但目标和细节有所不同。下表进行并排对比:

\begin{center}
\renewcommand{\arraystretch}{1.8}
\begin{tabular}{|l|p{7cm}|p{7cm}|}
\hline
\rowcolor[gray]{0.9}
\textbf{对比维度} & \textbf{能控性分解} & \textbf{能观性分解} \\
\hline
\textbf{核心目的} & 将状态空间分为「能被输入影响」和「完全不受输入影响」两部分 & 将状态空间分为「能从输出观测」和「完全无法观测」两部分 \\
\hline
\textbf{前提条件} & $\text{rank}(Q_c) < n$(不完全能控) & $\text{rank}(Q_o) < n$(不完全能观) \\
\hline
\textbf{关键矩阵} & 
能控性矩阵 $Q_c = \begin{bmatrix} B & AB & \cdots & A^{n-1}B \end{bmatrix}$ & 
能观性矩阵 $Q_o = \begin{bmatrix} C \\ CA \\ \vdots \\ CA^{n-1} \end{bmatrix}$ \\
\hline
\textbf{子空间} & 
能控:$Q_c$ 的列空间 \newline 不能控:$Q_c^T$ 的零空间 & 
能观:$Q_o^T$ 的列空间 \newline 不能观:$Q_o$ 的零空间 \\
\hline
\textbf{变换矩阵 $P^{-1}$} & 
列向量顺序:[能控基向量...不能控基向量] & 
列向量顺序:[能观基向量...不能观基向量] \\
\hline
\textbf{分解标准形} & 
$\bar{A} = \begin{bmatrix} A_c & A_{12} \\ \mathbf{0} & A_{uc} \end{bmatrix}$ \newline $\bar{B} = \begin{bmatrix} B_c \\ \mathbf{0} \end{bmatrix}$ \newline $\bar{C} = \begin{bmatrix} C_c & C_{uc} \end{bmatrix}$ &
$\bar{A} = \begin{bmatrix} A_o & \mathbf{0} \\ A_{21} & A_{uo} \end{bmatrix}$ \newline $\bar{B} = \begin{bmatrix} B_o \\ B_{uo} \end{bmatrix}$ \newline $\bar{C} = \begin{bmatrix} C_o & \mathbf{0} \end{bmatrix}$ \\
\hline
\textbf{零块含义} & 
$\bar{B}$ 零块:输入无法作用于不能控状态 \newline $\bar{A}$ 零块:不能控动态不受能控状态影响 & 
$\bar{C}$ 零块:不能观状态对输出无贡献 \newline $\bar{A}$ 零块:能观动态不受不能观状态影响 \\
\hline
\end{tabular}
\end{center}

\subsubsection{统一的解题流程(以能控分解为例)}

\begin{enumerate}
    \item \textbf{判断性质:}
    \begin{itemize}
        \item 计算能控性矩阵 $Q_c$
        \item 计算 $Q_c$ 的秩 $r_c = \text{rank}(Q_c)$
        \item 若 $r_c < n$,则系统不完全能控,需要进行分解
    \end{itemize}
    
    \item \textbf{寻找基向量:}
    \begin{itemize}
        \item \textbf{能控子空间}:求出 $Q_c$ 的列空间的一组基(通常取 $Q_c$ 中 $r_c$ 个线性无关的列向量)
        \item \textbf{不能控子空间}:求出 $Q_c^T$ 的零空间(解方程 $Q_c^T v = 0$)的一组基(共 $n-r_c$ 个向量)
    \end{itemize}
    
    \item \textbf{构造变换矩阵:}
    \begin{itemize}
        \item 构造 $P^{-1}$ 矩阵,将上一步求出的基向量作为其\textbf{列向量}
        \item \textbf{顺序至关重要}:必须把\textbf{能控}子空间的基放在\textbf{前面},\textbf{不能控}子空间的基放在\textbf{后面}
        \item 通过求逆得到 $P = (P^{-1})^{-1}$
    \end{itemize}
    
    \item \textbf{进行坐标变换:}
    \begin{itemize}
        \item $\bar{A} = PAP^{-1}$
        \item $\bar{B} = PB$
        \item $\bar{C} = CP^{-1}$
    \end{itemize}
    
    \item \textbf{验证与解读:}
    \begin{itemize}
        \item 检查变换后的 $(\bar{A}, \bar{B}, \bar{C})$ 是否呈现出标准分解形式(特别是关键的零块位置)
        \item 根据分解后的形式,写出能控子系统和不能控子系统的状态方程,并进行解释
    \end{itemize}
\end{enumerate}

\textbf{核心记忆点:}
\begin{itemize}
    \item \textbf{能控分解看 $Q_c$,能观分解看 $Q_o$}
    \item \textbf{构造变换矩阵 $P^{-1}$ 时,列向量的顺序是「好的」部分在前(能控/能观),「坏的」部分在后(不能控/不能观)}
    \item \textbf{记住分解后的标准形式,特别是零块的位置,这是最终的检验标准}
\end{itemize}

\subsection{综合范例 1:能观性结构分解}

\subsubsection{题目}

已知系统:
\begin{align*}
\dot{x} &= \begin{bmatrix} -4 & 2 \\ 2 & -4 \end{bmatrix} x + \begin{bmatrix} 12 \\ 0 \end{bmatrix} u \\
y &= \begin{bmatrix} 2 & 2 \end{bmatrix} x
\end{align*}

\textbf{求:}
\begin{enumerate}
    \item 判断系统的能观性
    \item 进行能观性分解
\end{enumerate}

\subsubsection{解答}

\paragraph{(1) 判断系统的能观性}

\begin{enumerate}
    \item \textbf{识别矩阵:}
    \[A = \begin{bmatrix} -4 & 2 \\ 2 & -4 \end{bmatrix}, \quad C = \begin{bmatrix} 2 & 2 \end{bmatrix}\]
    系统阶数 $n=2$。
    
    \item \textbf{构造能观性矩阵 $Q_o$:}
    \[Q_o = \begin{bmatrix} C \\ CA \end{bmatrix} = \begin{bmatrix} 2 & 2 \\ -4 & -4 \end{bmatrix}\]
    
    \item \textbf{判断秩:}
    第二行为第一行的 $-2$ 倍,故 $\text{rank}(Q_o) = 1$。
    
    \item \textbf{结论:}
    由于 $\text{rank}(Q_o) = 1 < n = 2$,该系统\textbf{不是完全能观的}。
\end{enumerate}

\paragraph{(2) 进行能观性分解}

\begin{enumerate}
    \item \textbf{寻找能观与不能观子空间:}
    \begin{itemize}
        \item \textbf{不能观子空间 ($N_o$):}求解 $Q_o v = 0$
        \[\begin{bmatrix} 2 & 2 \\ -4 & -4 \end{bmatrix} v = 0 \implies 2v_1 + 2v_2 = 0 \implies v_1 = -v_2\]
        基向量:$q_{uo} = \begin{bmatrix} 1 \\ -1 \end{bmatrix}$
        
        \item \textbf{能观子空间 ($R_o$):}$Q_o^T$ 的列空间
        \[Q_o^T = \begin{bmatrix} 2 & -4 \\ 2 & -4 \end{bmatrix}\]
        基向量:$q_o = \begin{bmatrix} 1 \\ 1 \end{bmatrix}$
    \end{itemize}
    
    \item \textbf{构造变换矩阵:}
    将能观基向量放前面,不能观基向量放后面:
    \[P^{-1} = \begin{bmatrix} 1 & 1 \\ 1 & -1 \end{bmatrix}\]
    求逆得:
    \[P = \begin{bmatrix} 1/2 & 1/2 \\ 1/2 & -1/2 \end{bmatrix}\]
    
    \item \textbf{应用坐标变换:}
    \[\bar{A} = PAP^{-1} = \begin{bmatrix} -2 & 0 \\ 0 & -6 \end{bmatrix}\]
    \[\bar{B} = PB = \begin{bmatrix} 6 \\ 6 \end{bmatrix}\]
    \[\bar{C} = CP^{-1} = \begin{bmatrix} 4 & 0 \end{bmatrix}\]
    
    \item \textbf{分解结果:}
    系统的能观性分解为:
    \begin{align*}
    \dot{\bar{x}} &= \begin{bmatrix} -2 & 0 \\ 0 & -6 \end{bmatrix} \bar{x} + \begin{bmatrix} 6 \\ 6 \end{bmatrix} u \\
    y &= \begin{bmatrix} 4 & 0 \end{bmatrix} \bar{x}
    \end{align*}
    
    其中 $\bar{x} = \begin{bmatrix} \bar{x}_o \\ \bar{x}_{uo} \end{bmatrix}$。$\bar{x}_o$ 是能观状态,$\bar{x}_{uo}$ 是不能观状态。$\bar{C}$ 矩阵中对应 $\bar{x}_{uo}$ 的元素为 0,表明该状态对输出没有贡献,故不可见。
\end{enumerate}

\subsection{综合范例 2:能控性结构分解}

\subsubsection{题目}

已知系统状态方程:
\begin{align*}
\dot{x} &= \begin{bmatrix} 0 & 0 & 1 \\ 1 & 0 & 3 \\ 0 & 1 & 1 \end{bmatrix} x + \begin{bmatrix} 1 \\ 1 \\ 0 \end{bmatrix} u \\
y &= \begin{bmatrix} 0 & 0 & 1 \end{bmatrix} x
\end{align*}

\textbf{求:}判断能控性,进行能控性分解。

\subsubsection{解答}

\paragraph{(1) 判断系统的能控性}

\begin{enumerate}
    \item \textbf{识别矩阵:}
    \[A = \begin{bmatrix} 0 & 0 & 1 \\ 1 & 0 & 3 \\ 0 & 1 & 1 \end{bmatrix}, \quad B = \begin{bmatrix} 1 \\ 1 \\ 0 \end{bmatrix}\]
    系统阶数 $n=3$。
    
    \item \textbf{构造能控性矩阵 $Q_c$:}
    计算 $AB$ 和 $A^2B$:
    \[AB = \begin{bmatrix} 0 & 0 & 1 \\ 1 & 0 & 3 \\ 0 & 1 & 1 \end{bmatrix} \begin{bmatrix} 1 \\ 1 \\ 0 \end{bmatrix} = \begin{bmatrix} 0 \\ 1 \\ 1 \end{bmatrix}\]
    \[A^2B = \begin{bmatrix} 0 & 0 & 1 \\ 1 & 0 & 3 \\ 0 & 1 & 1 \end{bmatrix} \begin{bmatrix} 0 \\ 1 \\ 1 \end{bmatrix} = \begin{bmatrix} 1 \\ 3 \\ 2 \end{bmatrix}\]
    \[Q_c = \begin{bmatrix} 1 & 0 & 1 \\ 1 & 1 & 3 \\ 0 & 1 & 2 \end{bmatrix}\]
    
    \item \textbf{判断秩:}
    对 $Q_c$ 进行行初等变换:
    \[\begin{bmatrix} 1 & 0 & 1 \\ 1 & 1 & 3 \\ 0 & 1 & 2 \end{bmatrix} \xrightarrow{R_2-R_1} \begin{bmatrix} 1 & 0 & 1 \\ 0 & 1 & 2 \\ 0 & 1 & 2 \end{bmatrix} \xrightarrow{R_3-R_2} \begin{bmatrix} 1 & 0 & 1 \\ 0 & 1 & 2 \\ 0 & 0 & 0 \end{bmatrix}\]
    因此 $\text{rank}(Q_c) = 2$。
    
    \item \textbf{结论:}
    由于 $\text{rank}(Q_c) = 2 < n = 3$,该系统\textbf{不是完全能控的}。
\end{enumerate}

\paragraph{(2) 进行能控性分解}

\begin{enumerate}
    \item \textbf{寻找能控与不能控子空间:}
    \begin{itemize}
        \item \textbf{能控子空间 ($R_c$):}$Q_c$ 的列空间。秩为 2,可选前两列作为基:
        \[q_{c1} = \begin{bmatrix} 1 \\ 1 \\ 0 \end{bmatrix}, \quad q_{c2} = \begin{bmatrix} 0 \\ 1 \\ 1 \end{bmatrix}\]
        
        \item \textbf{不能控子空间:}求解 $Q_c^T v = 0$
        \[\begin{bmatrix} 1 & 1 & 0 \\ 0 & 1 & 1 \\ 1 & 3 & 2 \end{bmatrix} v = 0\]
        从第一个方程:$v_1 + v_2 = 0 \implies v_1 = -v_2$
        
        从第二个方程:$v_2 + v_3 = 0 \implies v_3 = -v_2$
        
        基向量:$q_{uc} = \begin{bmatrix} 1 \\ -1 \\ 1 \end{bmatrix}$
    \end{itemize}
    
    \item \textbf{构造变换矩阵:}
    将能控基向量放前面,不能控基向量放后面:
    \[P^{-1} = \begin{bmatrix} 1 & 0 & 1 \\ 1 & 1 & -1 \\ 0 & 1 & 1 \end{bmatrix}\]
    
    求逆(使用行列式和伴随矩阵):
    \[\det(P^{-1}) = 1(1+1) - 0 + 1(1-0) = 3\]
    \[P = \frac{1}{3} \begin{bmatrix} 2 & 1 & -1 \\ -1 & 1 & 2 \\ 1 & -1 & 1 \end{bmatrix}\]
    
    \item \textbf{应用坐标变换:}
    \[\bar{A} = PAP^{-1} = \begin{bmatrix} 0 & 1 & 2 \\ 1 & 2 & -1 \\ 0 & 0 & -1 \end{bmatrix}\]
    
    \[\bar{B} = PB = \frac{1}{3}\begin{bmatrix} 2 & 1 & -1 \\ -1 & 1 & 2 \\ 1 & -1 & 1 \end{bmatrix} \begin{bmatrix} 1 \\ 1 \\ 0 \end{bmatrix} = \begin{bmatrix} 1 \\ 0 \\ 0 \end{bmatrix}\]
    
    \[\bar{C} = CP^{-1} = \begin{bmatrix} 0 & 0 & 1 \end{bmatrix} \begin{bmatrix} 1 & 0 & 1 \\ 1 & 1 & -1 \\ 0 & 1 & 1 \end{bmatrix} = \begin{bmatrix} 0 & 1 & 1 \end{bmatrix}\]
    
    \item \textbf{分解结果:}
    系统的能控性分解为:
    \begin{align*}
    \dot{\bar{x}} &= \begin{bmatrix} 0 & 1 & 2 \\ 1 & 2 & -1 \\ 0 & 0 & -1 \end{bmatrix} \bar{x} + \begin{bmatrix} 1 \\ 0 \\ 0 \end{bmatrix} u \\
    y &= \begin{bmatrix} 0 & 1 & 1 \end{bmatrix} \bar{x}
    \end{align*}
    
    其中 $\bar{x} = \begin{bmatrix} \bar{x}_{c} \\ \bar{x}_{uc} \end{bmatrix}$。前 2 个状态 $\bar{x}_c$ 是能控状态,第 3 个状态 $\bar{x}_{uc}$ 是不能控状态。$\bar{B}$ 矩阵中对应 $\bar{x}_{uc}$ 的元素为 0,表明输入 $u$ 无法影响该状态。同时,$\bar{A}$ 的左下角为零块,说明不能控状态的演化不受能控状态的影响(但能控状态可能受其影响)。
\end{enumerate}

\subsection*{本章总结}

\subsubsection*{四子系统特性对比表}

Kalman分解将任意系统分解为四个子系统,下表总结它们的特性和设计意义:

\begin{table}[htbp]
\centering
\caption{四子系统特性与设计意义对比}
\small
\begin{tabular}{|l|c|c|c|c|}
\hline
\textbf{特性} & \textbf{$x_1$ 能控能观} & \textbf{$x_2$ 能控不能观} & \textbf{$x_3$ 不能控能观} & \textbf{$x_4$ 不能控不能观} \\
\hline
\textbf{输入影响} & $\checkmark$ 能控制 & $\checkmark$ 能控制 & $\times$ 无法控制 & $\times$ 无法控制 \\
\hline
\textbf{输出可见} & $\checkmark$ 可观测 & $\times$ 隐藏 & $\checkmark$ 可观测 & $\times$ 隐藏 \\
\hline
\textbf{$B$矩阵} & $B_1 \neq 0$ & $B_2 \neq 0$ & $B_3 = 0$ & $B_4 = 0$ \\
\hline
\textbf{$C$矩阵} & $C_1 \neq 0$ & $C_2 = 0$ & $C_3 \neq 0$ & $C_4 = 0$ \\
\hline
\textbf{传函贡献} & $\checkmark$ \textbf{唯一} & $\times$ 不出现 & $\times$ 不出现 & $\times$ 不出现 \\
\hline
\textbf{极点配置} & $\checkmark$ 可任意配置 & $\checkmark$ 可配置 & $\times$ 无法配置 & $\times$ 无法配置 \\
\hline
\textbf{状态估计} & $\checkmark$ 可设计观测器 & $\times$ 无法估计 & $\checkmark$ 可估计 & $\times$ 无法估计 \\
\hline
\textbf{稳定性} & 
需保证稳定 
& 
需保证稳定
(虽隐藏)
& 
\textbf{必须}稳定
(无法控制!)
& 
\textbf{必须}稳定
(完全隐藏!)
\\
\hline
\textbf{设计策略} & 
正常设计:
控制器+观测器
& 
开环控制
(无反馈)
& 
被动监测
(无法干预)
& 
\textbf{系统重设计}
(增加执行器/传感器)
\\
\hline
\textbf{类比} & 
可见可控:
驾驶员控方向
& 
瞎子控制:
踩油门看不到
& 
旁观者:
看温度无法调
& 
盲区炸弹:
隐藏不稳定
\\
\hline
\textbf{危险等级} & \textcolor{green}{安全} & \textcolor{yellow}{中等} & \textcolor{orange}{中高} & \textcolor{red}{\textbf{极高}} \\
\hline
\end{tabular}
\end{table}

\subsubsection*{核心要点回顾}

\textbf{1. 结构分解的目的}
\begin{itemize}
    \item 精确定位系统中的能控/不能控、能观/不能观部分
    \item 识别隐藏动态(不能控不能观部分可能包含不稳定极点)
    \item 简化控制器设计(最小实现理论基础)
\end{itemize}

\textbf{2. Kalman分解的关键结论}
\begin{itemize}
    \item 传递函数$G(s) = C(sI-A)^{-1}B$\textbf{只}与$x_1$(能控且能观)有关
    \item $x_2, x_3, x_4$的极点在$G(s)$中\textbf{不出现}(零极点相消)
    \item $x_3, x_4$(不能控部分)的稳定性\textbf{无法}通过控制改善
\end{itemize}

\textbf{3. 分解方法对偶性}
\begin{itemize}
    \item 能控性分解:$(A, B) \to (\bar{A}, \bar{B})$,$\bar{B}$出现零块
    \item 能观性分解:$(A, C) \to (\bar{A}, \bar{C})$,$\bar{C}$出现零块
    \item 数学结构完全对称,掌握一种即可类推另一种
\end{itemize}

\textbf{4. 实际设计意义}
\begin{itemize}
    \item \textbf{最小实现}:提取$x_1$子系统,得到最简状态空间模型
    \item \textbf{执行器/传感器配置}:保证系统\textbf{至少}完全能控或完全能观
    \item \textbf{故障分析}:某个执行器/传感器失效$\Rightarrow$快速判断影响范围
    \item \textbf{降阶设计}:对高阶系统,只设计$x_1$子系统的控制器
\end{itemize}

\subsubsection*{常见误区与易错点}

\textbf{概念误区:}
\begin{itemize}
    \item ✗ \textbf{误区1}:认为不能控/不能观部分可以忽略
    \item ✓ \textbf{正确}:$x_3, x_4$\textbf{必须}稳定!否则系统会崩溃(即使$G(s)$看起来稳定)
    
    \item ✗ \textbf{误区2}:认为传递函数能完全描述系统
    \item ✓ \textbf{正确}:传递函数只反映$x_1$,\textbf{丢失}了$x_2, x_3, x_4$的信息
    
    \item ✗ \textbf{误区3}:认为能控但不能观的状态无关紧要
    \item ✓ \textbf{正确}:$x_2$虽然隐藏,但其极点必须稳定(可能导致不稳定但无法观测)
\end{itemize}

\textbf{计算易错点:}
\begin{itemize}
    \item ✗ 构造$P^{-1}$时搞混顺序(\textquotedblleft 好的在前还是在后?\textquotedblright)
    \item ✓ 正确:能控/能观基向量\textbf{在前},不能控/不能观基向量\textbf{在后}
    
    \item ✗ 误以为$\bar{A}$一定是块对角矩阵
    \item ✓ 正确:$\bar{A}$有\textbf{特定零块},但\textbf{不是}完全对角($A_{21}, A_{13}$等可能非零)
    
    \item ✗ 忘记验证分解结果(零块位置错误但未发现)
    \item ✓ 正确:检查$\bar{B}, \bar{C}$的零块位置,以及$\bar{A}$的特定零块(如$A_{31} = A_{32} = 0$)
\end{itemize}

\textbf{物理理解易错点:}
\begin{itemize}
    \item ✗ 认为$x_4$(不能控不能观)\textquotedblleft 无害\textquotedblright 因为传递函数中看不到
    \item ✓ 正确:$x_4$是\textbf{最危险}的!如果包含不稳定极点,系统会崩溃但无法从$G(s)$预测

    \item ✗ 把\textquotedblleft 不能观\textquotedblright 理解成\textquotedblleft 测量误差大\textquotedblright
    \item ✓ 正确:\textquotedblleft 不能观\textquotedblright 是\textbf{结构性}问题,无论测量多精确都无法推断这些状态
\end{itemize}

\subsubsection*{学习清单(掌握程度自检)}

\textbf{基础理解:}
\begin{itemize}
    \item[$\square$] 理解结构分解的动机(为什么要拆成四部分?)
    \item[$\square$] 能画出Kalman分解的矩阵结构(哪些是零块?)
    \item[$\square$] 理解四子系统的物理意义(可见性+可控性组合)
    \item[$\square$] 知道传递函数只与$x_1$有关
\end{itemize}

\textbf{计算能力:}
\begin{itemize}
    \item[$\square$] 给定$(A, B)$,能完整执行能控性分解(找基、构造$P^{-1}$、变换)
    \item[$\square$] 给定$(A, C)$,能完整执行能观性分解
    \item[$\square$] 会求$Q_c$的列空间基向量和$Q_c^T$的零空间基向量
    \item[$\square$] 能验证分解结果(检查零块位置)
\end{itemize}

\textbf{深入应用:}
\begin{itemize}
    \item[$\square$] 能从分解结果判断系统的传递函数形式(极零点相消)
    \item[$\square$] 理解最小实现的含义(提取$x_1$子系统)
    \item[$\square$] 能分析不能控/不能观部分对稳定性的影响
    \item[$\square$] 能设计执行器/传感器配置避免完全不可控/不可观子系统
\end{itemize}

\subsubsection*{后续章节预告与衔接}

\textbf{与前章(第\ref{sec:controllability-observability}章)的关系:}
\begin{itemize}
    \item 前章:判断系统\textbf{是否}完全能控/能观($\text{rank}(W_c) = n?$)
    \item 本章:\textbf{精确定位}不能控/不能观的部分(\textbf{哪些}状态有问题?)
    \item 前章是\textbf{诊断},本章是\textbf{解剖}
\end{itemize}

\textbf{与后章的关系:}
\begin{itemize}
    \item \textbf{第\ref{sec:lyapunov-stability}章(Lyapunov稳定性)}:分析$x_3, x_4$(不能控部分)的稳定性是否满足要求,这是\textbf{系统可用的前提}
    \item \textbf{第\ref{sec:pole-placement}章(极点配置)}:只能配置$x_1, x_2$(能控部分)的极点,$x_3, x_4$无法改变
    \item \textbf{第\ref{sec:state-observer}章(状态观测器)}:只能估计$x_1, x_3$(能观部分),$x_2, x_4$无法观测
\end{itemize}

\textbf{实践建议:}
\begin{itemize}
    \item 在系统建模阶段,检查是否存在$x_4$(完全隐藏)子系统
    \item 如果$x_3$或$x_4$不稳定,\textbf{必须}重新设计系统(增加执行器/传感器)
    \item 对于高阶系统,优先进行结构分解,然后只对$x_1$子系统设计控制器(降阶设计)
    \item 理解\textquotedblleft 传递函数不是全部\textquotedblright——状态空间描述包含更多信息
\end{itemize}

\textbf{理论价值:}
\begin{itemize}
    \item \textbf{最小实现理论}:任何传递函数的最简状态空间实现就是$x_1$子系统
    \item \textbf{系统等价性}:不同状态空间模型可能有相同传递函数($x_2, x_3, x_4$不同)
    \item \textbf{极零点相消}:数学解释了为什么某些极点在传递函数中\textquotedblleft 消失\textquotedblright
\end{itemize}

结构分解是现代控制理论的\textbf{透视工具}——它让你看透系统内部的\textbf{盲区},避免设计出表面稳定但实际危险的控制系统。记住:\textbf{传递函数只是冰山一角,状态空间才是全貌!}
