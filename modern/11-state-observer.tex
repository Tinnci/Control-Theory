\section{状态观测器}
\label{sec:state-observer}

\subsection*{引言:看不见的状态,如何控制?}

想象你站在一个密闭房间外,房间里有一个人在活动。你\textbf{看不见}房间内部(状态不可测),但你可以:
\begin{itemize}
    \item \textbf{听见}声音从门缝传出(输出 $y$)
    \item \textbf{敲门}与里面的人互动(输入 $u$)
\end{itemize}

你能推断出房间里那个人的位置和动作吗?答案是:\textbf{可以!}通过声音的方向、强度、时间,结合你的敲门动作,你可以\textbf{估计}他的状态。

这正是\textbf{状态观测器}(State Observer)要做的事情:

\begin{quote}
\textbf{当系统的内部状态无法直接测量时,利用可测的输入和输出信息,动态地估计出系统的状态。}
\end{quote}

\subsubsection*{为什么需要状态观测器?}

在上一章(极点配置),我们学会了通过状态反馈 $u = -Kx$ 任意配置极点。但有一个\textbf{关键前提}:
\begin{center}
\textit{所有状态变量 $x$ 都必须是可测的!}
\end{center}

\textbf{实际问题:}
\begin{itemize}
    \item \textbf{传感器成本高}:为每个状态变量安装传感器不现实
    \item \textbf{物理上不可测}:某些状态本质上无法直接测量
    \begin{itemize}
        \item 电机内部的磁通量
        \item 飞行器的侧滑角
        \item 化学反应的中间浓度
    \end{itemize}
    \item \textbf{测量噪声大}:某些传感器信号质量差
    \item \textbf{维护困难}:传感器可能失效或精度下降
\end{itemize}

\textbf{具体例子:}

考虑一个二阶机械系统(如倒立摆):
\[x = \begin{bmatrix} \theta \\ \dot{\theta} \end{bmatrix}\]

\begin{itemize}
    \item $\theta$(角度):容易测量(编码器)
    \item $\dot{\theta}$(角速度):直接测量需要昂贵的陀螺仪,或者对角度\textbf{数值微分}(噪声放大严重)
\end{itemize}

\textbf{观测器的解决方案:}只测量 $\theta$,通过观测器\textbf{估计} $\dot{\theta}$!

\subsubsection*{观测器的核心思想}

状态观测器是一个\textbf{动态系统},它:
\begin{enumerate}
    \item \textbf{模拟}真实系统的动态:$\dot{\hat{x}} = A\hat{x} + Bu$(与真实系统相同的模型)
    \item \textbf{校正}估计误差:利用输出误差 $y - \hat{y} = y - C\hat{x}$ 进行反馈校正
    \item \textbf{收敛}到真实状态:通过设计增益 $L$,使估计误差 $e = x - \hat{x} \to 0$
\end{enumerate}

\textbf{直观类比:}
\begin{itemize}
    \item 真实系统:房间里的人(看不见)
    \item 观测器:你的大脑中的\textit{心理模型}(模拟他的行为)
    \item 输出误差:你预测的声音 vs 实际听到的声音
    \item 校正增益 $L$:你根据误差调整心理模型的速度
\end{itemize}

\subsubsection*{观测器与极点配置的对偶性}

\textbf{极点配置}(上一章):
\begin{itemize}
    \item 控制律:$u = -Kx$
    \item 目标:通过反馈改变系统矩阵 $A \to A - BK$
    \item 前提:系统\textbf{能控}
    \item 设计:配置闭环极点
\end{itemize}

\textbf{状态观测器}(本章):
\begin{itemize}
    \item 观测器:$\dot{\hat{x}} = A\hat{x} + Bu + L(y - C\hat{x})$
    \item 目标:通过校正改变误差动态 $A \to A - LC$
    \item 前提:系统\textbf{能观测}
    \item 设计:配置观测器极点
\end{itemize}

两者是\textbf{对偶}的!能控性 $\leftrightarrow$ 能观测性,$K \leftrightarrow L$,$B \leftrightarrow C^T$。

\subsubsection*{实际应用场景}

\begin{itemize}
    \item \textbf{航天器姿态控制}:只测量姿态角,估计角速度和角加速度
    \item \textbf{电机控制}:测量转速,估计负载转矩和磁通量
    \item \textbf{机器人}:测量关节位置,估计速度和加速度
    \item \textbf{自动驾驶}:传感器融合,估计车辆的侧向速度
    \item \textbf{电力系统}:测量电压电流,估计系统内部状态
\end{itemize}

\subsubsection*{本章内容路线图}

\begin{enumerate}
    \item \textbf{观测器的概念}:观测器的数学形式和工作原理
    \item \textbf{全维状态观测器设计}:如何选择增益 $L$ 配置观测器极点
    \item \textbf{观测器极点选择}:多快才合适?平衡收敛速度与噪声敏感性
    \item \textbf{分离定理}:控制器和观测器可以独立设计!
    \item \textbf{基于观测器的控制}:$u = -K\hat{x}$ 的完整设计
    \item \textbf{设计范例}:从头到尾的实际例子
\end{enumerate}

\textbf{学习目标:}
\begin{itemize}
    \item 理解观测器的必要性和工作原理
    \item 掌握观测器增益矩阵 $L$ 的设计方法
    \item 理解分离定理,学会设计完整的控制系统
    \item 能够解决实际的状态估计问题
\end{itemize}

让我们开始探索\textbf{状态估计}的奇妙世界!

\subsection{状态观测器的概念}

\subsubsection*{本节目的}
建立观测器的数学模型,理解观测器的基本结构和工作原理。

\subsubsection{问题的提出}

考虑线性时不变系统:
\begin{align*}
\dot{x} &= Ax + Bu \\
y &= Cx
\end{align*}

\textbf{已知信息:}
\begin{itemize}
    \item 系统矩阵 $A, B, C$(系统模型)
    \item 输入 $u(t)$(我们施加的控制)
    \item 输出 $y(t)$(传感器测量值)
\end{itemize}

\textbf{未知信息:}
\begin{itemize}
    \item 状态 $x(t)$(部分或全部不可测)
\end{itemize}

\textbf{目标:}构造一个\textbf{估计器},输出状态的估计值 $\hat{x}(t)$,使得:
\[\hat{x}(t) \to x(t) \quad \text{as } t \to \infty\]

\subsubsection{观测器的基本结构}

\textbf{朴素想法(开环观测器):}

既然我们知道系统模型,为什么不直接模拟它?
\[\dot{\hat{x}} = A\hat{x} + Bu\]

\textbf{问题:}
\begin{itemize}
    \item 初值误差:$\hat{x}(0) \neq x(0)$ 会永远存在
    \item 模型误差:$A, B$ 不准确会导致估计偏差累积
    \item 没有利用输出信息 $y$!
\end{itemize}

\textbf{改进想法(闭环观测器):}

利用输出误差 $y - \hat{y}$ 进行\textbf{校正}:
\[\dot{\hat{x}} = A\hat{x} + Bu + L(y - \hat{y})\]

其中 $\hat{y} = C\hat{x}$ 是估计输出。

\textbf{最终形式:}
\[\boxed{\dot{\hat{x}} = A\hat{x} + Bu + L(y - C\hat{x})}\]

\textbf{各项的物理意义:}
\begin{itemize}
    \item $A\hat{x} + Bu$:\textbf{预测项}(根据模型预测下一状态)
    \item $y - C\hat{x}$:\textbf{创新}(innovation)或\textbf{输出误差}(实际与预测的差异)
    \item $L$:\textbf{观测器增益矩阵}(校正力度,类似控制器的 $K$)
\end{itemize}

\textbf{直观理解:}
\begin{itemize}
    \item 如果 $y > C\hat{x}$(实际输出大于估计):说明我们低估了状态,应该\textbf{上调} $\hat{x}$
    \item $L$ 决定调整的速度:$L$ 大 $\to$ 快速校正,$L$ 小 $\to$ 缓慢校正
\end{itemize}

\subsubsection{观测器 vs 传感器}

\textbf{观测器不是传感器!}

\begin{center}
\renewcommand{\arraystretch}{1.6}
\begin{tabular}{|l|p{5cm}|p{5cm}|}
\hline
\rowcolor[gray]{0.9}
& \textbf{传感器} & \textbf{观测器} \\
\hline
\textbf{本质} & 物理设备 & 算法/软件 \\
\hline
\textbf{输入} & 物理信号(温度、位移等) & $u, y$(数据) \\
\hline
\textbf{输出} & 测量值(可能有噪声) & 状态估计 $\hat{x}$ \\
\hline
\textbf{成本} & 硬件成本 & 计算成本 \\
\hline
\textbf{优势} & 直接测量 & 减少传感器数量,滤波效果 \\
\hline
\textbf{劣势} & 昂贵,可能失效 & 依赖模型准确性 \\
\hline
\end{tabular}
\end{center}

\textbf{实际应用中:}
\begin{itemize}
    \item 少量传感器(测量部分状态)+ 观测器(估计其余状态)
    \item 传感器提供 $y$,观测器利用 $y$ 和模型重构完整的 $x$
\end{itemize}

\subsection{全维状态观测器}

\subsubsection*{本节目的}
深入理解观测器的数学原理:误差动态、收敛条件、极点配置方法。

\subsubsection{观测器方程}

\textbf{全维状态观测器的标准形式:}
\[\boxed{\dot{\hat{x}} = A\hat{x} + Bu + L(y - C\hat{x})}\]

改写为:
\[\dot{\hat{x}} = (A - LC)\hat{x} + Bu + Ly\]

\textbf{方程结构分析:}
\begin{itemize}
    \item 输入:$u$(控制输入)和 $y$(系统输出)
    \item 输出:$\hat{x}$(状态估计)
    \item 观测器矩阵:$A - LC$(类比闭环控制中的 $A - BK$)
    \item 待设计参数:$L \in \mathbb{R}^{n \times p}$
\end{itemize}

\textbf{为什么叫\textbf{全维}?}
\begin{itemize}
    \item 全维观测器:估计\textbf{所有} $n$ 个状态变量
    \item 降维观测器:只估计\textbf{不可测}的状态(利用 $y$ 已包含部分状态信息)
    \item 全维观测器更简单,是入门的首选
\end{itemize}

\subsubsection{误差动态分析}

定义\textbf{估计误差}:
\[e(t) = x(t) - \hat{x}(t)\]

\textbf{目标:}设计 $L$ 使 $e(t) \to 0$。

\textbf{推导误差动态方程:}

真实系统:
\[\dot{x} = Ax + Bu\]

观测器:
\[\dot{\hat{x}} = A\hat{x} + Bu + L(y - C\hat{x}) = A\hat{x} + Bu + LC(x - \hat{x})\]

两式相减:
\begin{align*}
\dot{e} &= \dot{x} - \dot{\hat{x}} \\
&= Ax + Bu - [A\hat{x} + Bu + LC(x - \hat{x})] \\
&= Ax - A\hat{x} - LCx + LC\hat{x} \\
&= A(x - \hat{x}) - LC(x - \hat{x}) \\
&= (A - LC)e
\end{align*}

\textbf{关键结果:}
\[\boxed{\dot{e} = (A - LC)e}\]

\textbf{重要观察:}
\begin{itemize}
    \item 误差动态与输入 $u$ \textbf{无关}!(控制不影响估计误差)
    \item 误差是\textbf{齐次方程},只依赖于初始误差 $e(0) = x(0) - \hat{x}(0)$
    \item 收敛性完全由 $A - LC$ 的特征值决定
\end{itemize}

\subsubsection{观测器收敛条件}

要使 $e(t) \to 0$,需要 $A - LC$ 的所有特征值在\textbf{左半平面}。

\textbf{如何选择 $L$?}

这与极点配置\textbf{完全类似}:
\begin{itemize}
    \item 极点配置:选择 $K$ 使 $A - BK$ 的特征值在期望位置
    \item 观测器设计:选择 $L$ 使 $A - LC$ 的特征值在期望位置
\end{itemize}

\textbf{对偶性(Duality):}

\begin{center}
\renewcommand{\arraystretch}{1.6}
\begin{tabular}{|l|c|c|}
\hline
\rowcolor[gray]{0.9}
& \textbf{控制器设计} & \textbf{观测器设计} \\
\hline
\textbf{目标} & 配置 $A - BK$ 的极点 & 配置 $A - LC$ 的极点 \\
\hline
\textbf{前提条件} & $(A, B)$ 能控 & $(A, C)$ 能观测 \\
\hline
\textbf{对偶变换} & --- & $(A, C) \leftrightarrow (A^T, C^T)$ \\
\hline
\textbf{增益计算} & $K$ 通过阿克曼公式 & $L^T$ 通过对偶系统 \\
\hline
\end{tabular}
\end{center}

\textbf{利用对偶性设计 $L$:}

\begin{enumerate}
    \item 检查 $(A, C)$ 的能观测性
    \item 将问题转化为对偶系统 $(A^T, C^T)$ 的能控性问题
    \item 为对偶系统设计 $K^T$(使 $A^T - C^T K^T$ 有期望极点)
    \item 取 $L = K^T$(或直接 $L^T = K$)
\end{enumerate}

\textbf{实际计算(MATLAB):}
\begin{verbatim}
% 方法1:直接使用place函数
L = place(A', C', desired_poles)';

% 方法2:利用对偶性
K_dual = acker(A', C', desired_poles);
L = K_dual';
\end{verbatim}

\subsubsection{观测器增益的物理意义}

$L$ 是 $n \times p$ 矩阵,其中:
\begin{itemize}
    \item $n$:状态变量个数
    \item $p$:输出变量个数
\end{itemize}

\textbf{增益元素的含义:}

$L_{ij}$:当第 $j$ 个输出有单位误差时,对第 $i$ 个状态估计的校正强度。

\textbf{极端情况分析:}

\begin{itemize}
    \item \textbf{$L = 0$}:开环观测器
    \begin{itemize}
        \item 完全忽略输出误差
        \item 误差动态:$\dot{e} = Ae$(不稳定或缓慢收敛)
        \item 适用场景:模型极其准确,无需校正
    \end{itemize}
    
    \item \textbf{$L$ 很大}:高增益观测器
    \begin{itemize}
        \item 强烈依赖输出信息
        \item 误差快速收敛
        \item 代价:对测量噪声\textbf{极度敏感}
    \end{itemize}
    
    \item \textbf{$L$ 适中}:平衡设计
    \begin{itemize}
        \item 收敛速度与噪声抑制的折中
        \item 实际工程的典型选择
    \end{itemize}
\end{itemize}

\subsubsection{观测器设计定理}

\textbf{全维状态观测器设计定理:}

若系统 $(A, C)$ \textbf{完全能观测},则对于任意给定的 $n$ 个复数 $\mu_1, \mu_2, \ldots, \mu_n$(复数成对共轭),\textbf{存在}观测器增益矩阵 $L$,使得误差矩阵 $A - LC$ 的特征值恰好为 $\mu_1, \mu_2, \ldots, \mu_n$。

\textbf{定理的意义:}
\begin{itemize}
    \item 能观测性是任意配置观测器极点的\textbf{充要条件}
    \item 这与极点配置定理\textbf{完全对偶}
    \item 设计方法可以直接套用极点配置的算法
\end{itemize}

\textbf{与极点配置的对比:}

\begin{center}
\begin{tabular}{|l|c|c|}
\hline
\rowcolor[gray]{0.9}
& \textbf{极点配置} & \textbf{观测器设计} \\
\hline
系统 & $\dot{x} = Ax + Bu$ & $\dot{e} = (A - LC)e$ \\
\hline
配置目标 & $A - BK$ 的极点 & $A - LC$ 的极点 \\
\hline
条件 & $(A, B)$ 能控 & $(A, C)$ 能观测 \\
\hline
增益 & $K$ & $L$ \\
\hline
对偶关系 & $B \leftrightarrow C^T$ & $K \leftrightarrow L^T$ \\
\hline
\end{tabular}
\end{center}

\subsection{观测器的设计}

\subsubsection*{本节目的}
掌握观测器增益 $L$ 的具体计算步骤,以及观测器极点的选择原则。

\subsubsection{设计步骤}

\textbf{观测器设计的标准流程:}

\begin{enumerate}
    \item \textbf{验证能观测性}
    \begin{itemize}
        \item 计算能观测性矩阵:$W_o = \begin{bmatrix} C \\ CA \\ CA^2 \\ \vdots \\ CA^{n-1} \end{bmatrix}$
        \item 检查:$\text{rank}(W_o) = n$
        \item 若不能观测,则\textbf{无法}任意配置观测器极点
    \end{itemize}
    
    \item \textbf{选择观测器极点} $\mu_1, \mu_2, \ldots, \mu_n$
    \begin{itemize}
        \item 必须在左半平面(保证误差收敛)
        \item 通常比控制器极点\textbf{更靠左}(收敛更快)
        \item 经验法则:观测器极点实部 = 控制器极点实部的 $2 \sim 5$ 倍
    \end{itemize}
    
    \item \textbf{计算观测器增益 $L$}
    \begin{itemize}
        \item 方法1:直接使用 MATLAB \texttt{place(A', C', poles)'}
        \item 方法2:利用对偶性,设计 $(A^T, C^T)$ 的控制器增益
        \item 方法3:手算低阶系统($n \leq 3$)
    \end{itemize}
    
    \item \textbf{验证设计}
    \begin{itemize}
        \item 检查 $A - LC$ 的特征值是否为期望值
        \item 仿真误差动态 $\dot{e} = (A - LC)e$
        \item 评估对测量噪声的敏感性
    \end{itemize}
\end{enumerate}

\subsubsection{观测器极点选择准则}

\textbf{1. 为什么观测器极点要比控制器极点快?}

\textbf{原因:}
\begin{itemize}
    \item 控制器需要\textbf{真实状态} $x$ 才能发挥作用
    \item 如果估计误差 $e$ 收敛太慢,控制器会\textit{误用}错误的状态信息
    \item 让观测器先收敛,控制器才能基于准确的 $\hat{x}$ 工作
\end{itemize}

\textbf{经验法则:}

设控制器主导极点为 $s_c = -\sigma_c \pm j\omega_c$,则观测器极点选择为:
\[s_o = -(2\sim 5)\sigma_c \pm j\omega_o\]

\textbf{典型选择:}
\begin{itemize}
    \item \textbf{保守设计}:观测器极点 = 2倍控制器极点
    \item \textbf{中等设计}:3-4倍
    \item \textbf{激进设计}:5倍(快速收敛,但噪声敏感)
\end{itemize}

\textbf{2. 观测器极点配置的实际限制}

\begin{itemize}
    \item \textbf{测量噪声}
    \begin{itemize}
        \item 观测器极点越靠左,增益 $L$ 越大
        \item $L$ 大 $\to$ 对 $y$ 的噪声放大严重
        \item 平衡:收敛速度 vs 噪声抑制
    \end{itemize}
    
    \item \textbf{计算负担}
    \begin{itemize}
        \item 观测器需要实时运行
        \item 极点太快 $\to$ 需要更小的采样周期(数字实现时)
        \item 嵌入式系统可能无法承受
    \end{itemize}
    
    \item \textbf{模型误差}
    \begin{itemize}
        \item 观测器依赖准确的 $A, B, C$ 矩阵
        \item 高增益观测器对模型误差\textbf{不鲁棒}
        \item 实际系统建模总有偏差
    \end{itemize}
    
    \item \textbf{执行器饱和}
    \begin{itemize}
        \item 观测器初始误差大时,可能导致 $\hat{x}$ 剧烈变化
        \item 基于 $\hat{x}$ 的控制律 $u = -K\hat{x}$ 可能饱和
        \item 需要考虑\textit{抗饱和}设计
    \end{itemize}
\end{itemize}

\textbf{3. 观测器极点的典型配置模式}

\textbf{实极点配置(无超调):}
\begin{itemize}
    \item 适用于噪声较大的系统
    \item 避免振荡放大噪声
    \item 例:控制器极点 $-3, -4$,观测器极点 $-10, -12$
\end{itemize}

\textbf{Bessel型配置(最优延迟):}
\begin{itemize}
    \item 最小化估计延迟
    \item 适合快速跟踪应用
\end{itemize}

\textbf{经验公式(二阶系统):}

控制器:$s_{c} = -\zeta_c\omega_{nc} \pm j\omega_{nc}\sqrt{1-\zeta_c^2}$

观测器:$s_o = -\alpha\zeta_c\omega_{nc} \pm j\alpha\omega_{nc}\sqrt{1-\zeta_c^2}$

其中 $\alpha = 3 \sim 5$(加速因子)。

\subsubsection{对偶性的具体应用}

\textbf{利用极点配置工具设计观测器:}

\textbf{步骤:}
\begin{enumerate}
    \item 构造对偶系统:$(A^T, C^T, B^T)$
    \item 用极点配置方法(如阿克曼公式)为对偶系统设计控制器增益:
    \[K_{\text{dual}} = [0 \quad \cdots \quad 0 \quad 1] W_o^{-1} \alpha_o(A^T)\]
    其中 $W_o = [C^T \quad A^TC^T \quad \cdots \quad (A^T)^{n-1}C^T]$
    \item 转换回观测器增益:
    \[L = K_{\text{dual}}^T\]
\end{enumerate}

\textbf{MATLAB实现:}
\begin{verbatim}
% 方法1:直接设计观测器
L = place(A', C', observer_poles)';

% 方法2:利用对偶性
K_dual = acker(A', C', observer_poles);
L = K_dual';

% 验证
eig(A - L*C)  % 应等于 observer_poles
\end{verbatim}

\textbf{为什么对偶性如此重要?}
\begin{itemize}
    \item 所有极点配置的算法(直接法、变换法、阿克曼公式)都可以\textbf{直接用于}观测器设计
    \item 理论统一:能控性理论 $\leftrightarrow$ 能观测性理论
    \item 工具复用:同一套代码,只需转置矩阵
\end{itemize}

\subsubsection{观测器增益计算范例:倒立摆系统}

\paragraph*{范例说明}
通过一个经典的倒立摆系统,详细展示观测器增益 $L$ 的计算步骤,并深入分析其物理意义及对实际问题的考量。

\paragraph{1. 问题背景与系统模型}

考虑一个线性化的倒立摆系统:
\begin{itemize}
    \item 状态变量:$x_1 = \theta$(摆角),$x_2 = \dot{\theta}$(角速度)
    \item 控制输入:$u$(施加在底座上的水平力)
    \item 测量输出:$y = \theta$(\textbf{只能}通过编码器测量摆角)
\end{itemize}

系统模型如下:
\begin{align*}
\dot{x} &= \begin{bmatrix} 0 & 1 \\ 2 & 0 \end{bmatrix} x + \begin{bmatrix} 0 \\ 1 \end{bmatrix} u \\
y &= \begin{bmatrix} 1 & 0 \end{bmatrix} x
\end{align*}

\textbf{系统特性分析:}

计算开环极点:
\[\det(sI - A) = \det\begin{bmatrix} s & -1 \\ -2 & s \end{bmatrix} = s^2 - 2 = 0\]

极点:$s = \pm\sqrt{2} \approx \pm 1.414$

\textbf{关键观察:}
\begin{itemize}
    \item 存在一个\textbf{正实部极点} $+\sqrt{2}$,说明系统开环\textbf{不稳定}
    \item 物理意义:倒立摆在没有控制时会倾倒
    \item 角速度 $\dot{\theta}$ 无法直接测量(需要昂贵的陀螺仪或噪声敏感的数值微分)
\end{itemize}

\paragraph{2. 设计任务与系统分析}

\textbf{任务:}设计一个状态观测器,用于估计不可直接测量的角速度 $x_2 = \dot{\theta}$。

\textbf{能观测性检查:}
\[W_o = \begin{bmatrix} C \\ CA \end{bmatrix} = \begin{bmatrix} 1 & 0 \\ 0 & 1 \end{bmatrix}\]

$\det(W_o) = 1 \neq 0$,系统\textbf{完全能观测},因此可以任意配置观测器极点。

\textbf{能观测性的物理意义:}
\begin{itemize}
    \item 从摆角 $\theta$ 的测量值可以推断角速度 $\dot{\theta}$
    \item 第一个状态直接可测:$y = x_1$
    \item 第二个状态可从 $y$ 的变化率推断:$\dot{y} = Cx_2$
\end{itemize}

\paragraph{3. 选择观测器极点}

\textbf{设计考虑:}
\begin{itemize}
    \item 由于原系统\textbf{不稳定}(极点 $+\sqrt{2}$),估计误差必须\textbf{快速收敛}
    \item 观测器要为控制器提供可靠的状态估计
    \item 同时要避免增益过大,以抑制测量噪声
\end{itemize}

\textbf{极点选择策略:}

我们选择一对位于实轴上且比系统不稳定极点快得多的极点:
\[\mu_1 = -6, \quad \mu_2 = -8\]

\textbf{选择理由:}
\begin{itemize}
    \item \textbf{实极点}:避免振荡,减少噪声放大
    \item \textbf{足够快}:$|-6| \gg |\sqrt{2}|$,保证估计误差快速收敛
    \item \textbf{不过度激进}:不是 -20, -30 这样的极端值,平衡收敛速度与噪声抑制
\end{itemize}

\textbf{收敛时间估算:}
\[t_s \approx \frac{4}{|\text{Re}(\mu_{\min})|} = \frac{4}{6} \approx 0.67 \text{ 秒}\]

误差在不到 1 秒内收敛到 2\% 以内。

\textbf{期望特征多项式:}
\[\alpha_o(s) = (s + 6)(s + 8) = s^2 + 14s + 48\]

\paragraph{4. 计算观测器增益 L}

\textbf{方法:直接法}

设 $L = \begin{bmatrix} l_1 \\ l_2 \end{bmatrix}$,计算误差动态矩阵 $A - LC$:
\begin{align*}
A - LC &= \begin{bmatrix} 0 & 1 \\ 2 & 0 \end{bmatrix} - \begin{bmatrix} l_1 \\ l_2 \end{bmatrix} \begin{bmatrix} 1 & 0 \end{bmatrix} \\
&= \begin{bmatrix} 0 & 1 \\ 2 & 0 \end{bmatrix} - \begin{bmatrix} l_1 & 0 \\ l_2 & 0 \end{bmatrix} \\
&= \begin{bmatrix} -l_1 & 1 \\ 2 - l_2 & 0 \end{bmatrix}
\end{align*}

其特征方程为:
\begin{align*}
\det(sI - (A - LC)) &= \det\begin{bmatrix} s + l_1 & -1 \\ l_2 - 2 & s \end{bmatrix} \\
&= (s + l_1) \cdot s - (-1) \cdot (l_2 - 2) \\
&= s^2 + l_1 s + l_2 - 2
\end{align*}

将此多项式与期望的 $\alpha_o(s)$ 进行系数匹配:
\[s^2 + l_1 s + l_2 - 2 = s^2 + 14s + 48\]

比较系数可得:
\begin{align*}
l_1 &= 14 \\
l_2 - 2 &= 48 \quad \Rightarrow \quad l_2 = 50
\end{align*}

\textbf{结果:}
\[\boxed{L = \begin{bmatrix} 14 \\ 50 \end{bmatrix}}\]

\paragraph{5. 验证设计}

\textbf{检查闭环极点:}
\[A - LC = \begin{bmatrix} -14 & 1 \\ -48 & 0 \end{bmatrix}\]

特征值计算:
\[\det(sI - (A-LC)) = s^2 + 14s + 48 = (s+6)(s+8)\]

极点:$s = -6, -8$ ✓(与设计目标一致)

\paragraph{6. 结果分析与讨论}

\textbf{观测器增益的物理意义:}

观测器方程为:
\[\dot{\hat{x}} = A\hat{x} + Bu + L(y - C\hat{x})\]

展开为:
\begin{align*}
\dot{\hat{x}}_1 &= \hat{x}_2 + 14(y - \hat{x}_1) \\
\dot{\hat{x}}_2 &= 2\hat{x}_1 + u + 50(y - \hat{x}_1)
\end{align*}

增益 $L = [14, 50]^T$ 的含义:
\begin{itemize}
    \item $l_1 = 14$:当测量角度 $y$ 与估计角度 $\hat{x}_1$ 存在误差时,该误差会以 \textbf{14 倍}的增益来校正角度估计的变化率 $\dot{\hat{x}}_1$
    \item $l_2 = 50$:同时,该角度误差会以 \textbf{50 倍}的更大增益来校正角速度估计的变化率 $\dot{\hat{x}}_2$
    \item \textbf{角速度估计更依赖校正}:因为 $x_2$ 不可测,需要更强的校正力度
\end{itemize}

\textbf{噪声敏感性分析:}

假设测量值 $y$ 存在一个幅值为 $w$ 的高频噪声:$y = \theta + w$。

这个噪声会通过增益 $L$ 被放大并影响状态估计:
\begin{itemize}
    \item 对 $\dot{\hat{x}}_1$ 的影响:$14w$
    \item 对 $\dot{\hat{x}}_2$ 的影响:$50w$(\textbf{放大 50 倍}!)
\end{itemize}

\textbf{实际权衡:}

如果测量噪声 $w$ 的标准差为 0.01 弧度(约 0.57°),则:
\begin{itemize}
    \item 角速度估计的噪声:$50 \times 0.01 = 0.5$ rad/s
    \item 这可能导致控制输入抖动
\end{itemize}

\textbf{调整策略:}

如果实际噪声过大,可以考虑:
\begin{enumerate}
    \item \textbf{放慢观测器极点}:如选择 $-4, -5$(代价:收敛变慢)
    \item \textbf{增加滤波器}:对测量信号 $y$ 预先滤波
    \item \textbf{使用卡尔曼滤波器}:在随机噪声存在时的最优观测器
\end{enumerate}

\textbf{对比不同极点选择:}

\begin{center}
\renewcommand{\arraystretch}{1.6}
\begin{tabular}{|c|c|c|c|}
\hline
\rowcolor[gray]{0.9}
\textbf{观测器极点} & \textbf{$L$} & \textbf{收敛时间} & \textbf{噪声放大} \\
\hline
$-4, -5$ & $[9, 17]^T$ & 1.0 s & 低 \\
\hline
$-6, -8$ & $[14, 50]^T$ & 0.67 s & 中等 \\
\hline
$-10, -12$ & $[22, 118]^T$ & 0.4 s & 高 \\
\hline
\end{tabular}
\end{center}

\textbf{范例总结:}

这个倒立摆例子展示了:
\begin{itemize}
    \item 观测器设计的\textbf{完整流程}:能观测性检查 → 极点选择 → 增益计算 → 验证
    \item 增益矩阵的\textbf{物理意义}:校正力度与状态的可测性相关
    \item \textbf{实际权衡}:收敛速度 vs 噪声敏感性
    \item 不同极点选择的\textbf{对比分析}
\end{itemize}

\subsection{分离定理}

\subsubsection*{本节目的}
理解现代控制理论的\textbf{核心定理}之一:控制器和观测器可以独立设计,且联合系统的性能可以预测。

\subsubsection{基于观测器的状态反馈}

\textbf{问题背景:}

\begin{itemize}
    \item 极点配置需要:$u = -Kx$(需要完整状态 $x$)
    \item 实际情况:$x$ 不可测,只有估计值 $\hat{x}$
\end{itemize}

\textbf{实际控制律:}
\[\boxed{u = -K\hat{x} + v}\]

用\textbf{估计状态}代替\textbf{真实状态}!

\textbf{直观担忧:}
\begin{itemize}
    \item 用 $\hat{x}$ 代替 $x$ 会破坏闭环极点吗?
    \item 控制器和观测器的极点会相互干扰吗?
    \item 我们需要重新设计 $K$ 和 $L$ 吗?
\end{itemize}

\textbf{分离定理的答案:}
\begin{quote}
\textit{不用担心!控制器和观测器可以\textbf{完全独立}设计,就像 $x$ 真的可测一样。}
\end{quote}

\subsubsection{分离定理(Separation Principle)}

\textbf{定理陈述:}

对于系统:
\begin{align*}
\dot{x} &= Ax + Bu \\
y &= Cx
\end{align*}

采用基于观测器的状态反馈:
\begin{align*}
u &= -K\hat{x} + v \\
\dot{\hat{x}} &= A\hat{x} + Bu + L(y - C\hat{x})
\end{align*}

闭环系统的特征多项式等于:
\[\boxed{\det(sI - A + BK) \cdot \det(sI - A + LC)}\]

即:\textbf{控制器特征多项式 × 观测器特征多项式}

\textbf{定理的深刻含义:}

\begin{enumerate}
    \item \textbf{极点分离}:闭环系统的极点 = 控制器极点 + 观测器极点
    \begin{itemize}
        \item 控制器极点:由 $K$ 单独决定(如上一章设计)
        \item 观测器极点:由 $L$ 单独决定(本章设计)
        \item 两者\textbf{不相互影响}
    \end{itemize}
    
    \item \textbf{独立设计}:可以分两步设计
    \begin{itemize}
        \item 步骤1:假设 $x$ 可测,设计 $K$(极点配置)
        \item 步骤2:设计观测器,配置 $L$(观测器设计)
        \item 无需迭代或联合优化
    \end{itemize}
    
    \item \textbf{性能保证}:闭环稳定性由两者的\textit{最慢}极点决定
    \begin{itemize}
        \item 若控制器和观测器都稳定 $\to$ 整体系统稳定
        \item 响应速度由两者的\textit{瓶颈}决定
    \end{itemize}
\end{enumerate}

\subsubsection{分离定理的证明(思路)}

\textbf{定义增广状态向量:}
\[\xi = \begin{bmatrix} x \\ e \end{bmatrix}, \quad e = x - \hat{x}\]

\textbf{推导增广系统:}

真实系统:
\[\dot{x} = Ax + Bu = Ax + B(-K\hat{x}) = Ax - BK(x - e) = (A - BK)x + BKe\]

误差动态(与 $u$ 无关):
\[\dot{e} = (A - LC)e\]

\textbf{增广形式:}
\[\begin{bmatrix} \dot{x} \\ \dot{e} \end{bmatrix} = \begin{bmatrix} A - BK & BK \\ 0 & A - LC \end{bmatrix} \begin{bmatrix} x \\ e \end{bmatrix}\]

\textbf{关键观察:}矩阵是\textbf{块上三角}形式!

特征多项式:
\[\det(sI - \begin{bmatrix} A - BK & BK \\ 0 & A - LC \end{bmatrix}) = \det(sI - A + BK) \cdot \det(sI - A + LC)\]

$\square$(证毕)

\textbf{块上三角的意义:}
\begin{itemize}
    \item $e$ 的动态\textbf{不依赖} $x$(误差自己演化)
    \item $x$ 的动态受 $e$ 影响(但不改变极点位置)
    \item 极点 = 对角块的极点之并集
\end{itemize}

\subsubsection{分离定理的实际意义}

\textbf{1. 简化设计流程}

\textbf{无需分离定理的情况:}
\begin{itemize}
    \item 联合设计 $K$ 和 $L$
    \item 求解耦合的非线性方程
    \item 计算量巨大,难以直观理解
\end{itemize}

\textbf{有分离定理:}
\begin{itemize}
    \item 步骤1:设计 $K$(假设状态可测)
    \item 步骤2:设计 $L$(配置观测器极点)
    \item 两步独立,清晰明了
\end{itemize}

\textbf{2. 设计自由度}

\begin{itemize}
    \item $K$:根据\textbf{性能要求}选择(响应速度、超调量)
    \item $L$:根据\textbf{估计需求}选择(收敛速度、噪声抑制)
    \item 两者有不同的设计准则,互不干扰
\end{itemize}

\textbf{3. 鲁棒性考虑}

\textbf{理想情况:}
\begin{itemize}
    \item 模型准确:$A, B, C$ 完全已知
    \item 无噪声:测量 $y$ 无误差
    \item 分离定理\textbf{严格成立}
\end{itemize}

\textbf{实际情况:}
\begin{itemize}
    \item 模型误差:$\Delta A, \Delta B$
    \item 测量噪声:$y = Cx + w$
    \item 分离定理\textbf{近似成立}(鲁棒性分析需要额外工具)
\end{itemize}

\textbf{启示:}
\begin{itemize}
    \item 观测器极点不宜过快(对模型误差敏感)
    \item 需要在收敛速度与鲁棒性间折中
    \item 可能需要\textit{鲁棒观测器}设计(如 $H_\infty$ 观测器)
\end{itemize}

\subsubsection{完整系统的设计流程}

\textbf{标准设计程序:}

\begin{enumerate}
    \item \textbf{系统分析}
    \begin{itemize}
        \item 建立状态空间模型 $(A, B, C)$
        \item 验证能控性:$\text{rank}([B \quad AB \quad \cdots \quad A^{n-1}B]) = n$
        \item 验证能观测性:$\text{rank}(\begin{bmatrix} C \\ CA \\ \vdots \\ CA^{n-1} \end{bmatrix}) = n$
    \end{itemize}
    
    \item \textbf{控制器设计}
    \begin{itemize}
        \item 根据性能要求选择期望闭环极点 $\lambda_1, \ldots, \lambda_n$
        \item 使用极点配置方法计算 $K$
        \item 验证:$\det(sI - A + BK) = \prod (s - \lambda_i)$
    \end{itemize}
    
    \item \textbf{观测器设计}
    \begin{itemize}
        \item 选择观测器极点 $\mu_1, \ldots, \mu_n$(通常比 $\lambda_i$ 快2-5倍)
        \item 利用对偶性计算 $L$
        \item 验证:$\det(sI - A + LC) = \prod (s - \mu_i)$
    \end{itemize}
    
    \item \textbf{实现控制律}
    \begin{itemize}
        \item 观测器:$\dot{\hat{x}} = A\hat{x} + Bu + L(y - C\hat{x})$
        \item 控制律:$u = -K\hat{x} + v$
        \item 数字实现时需要离散化
    \end{itemize}
    
    \item \textbf{验证与调试}
    \begin{itemize}
        \item 仿真增广系统:$2n$ 个状态($x$ 和 $e$)
        \item 检查瞬态响应(初始误差、阶跃输入)
        \item 评估对噪声和扰动的敏感性
        \item 必要时调整极点位置
    \end{itemize}
\end{enumerate}

\subsubsection{分离定理的局限性}

\textbf{定理成立的前提:}
\begin{itemize}
    \item \textbf{线性系统}(非线性系统不适用)
    \item \textbf{确定性系统}(无随机干扰)
    \item \textbf{模型准确}($A, B, C$ 已知且精确)
\end{itemize}

\textbf{实际中的挑战:}
\begin{itemize}
    \item 模型误差导致性能下降
    \item 测量噪声被高增益观测器放大
    \item 非线性和饱和效应
    \item 计算延迟(数字实现)
\end{itemize}

\textbf{扩展方向:}
\begin{itemize}
    \item \textbf{卡尔曼滤波器}(随机噪声情况)
    \item \textbf{鲁棒观测器}(模型不确定性)
    \item \textbf{非线性观测器}(扩展卡尔曼滤波、UKF)
    \item \textbf{自适应观测器}(在线参数估计)
\end{itemize}

\subsection{综合设计范例:直流电机位置控制}

\subsubsection*{本节目的}
将本章所学的所有概念——\textbf{控制器设计、观测器设计、分离定理}——应用于一个具体的物理系统,完整地走一遍从问题分析到系统验证的全过程。

\subsubsection{问题描述与系统建模}

\textbf{场景:}为一个直流电机设计高精度位置控制器。

\textbf{控制目标:}
\begin{itemize}
    \item 使电机角度 $\theta(t)$ 快速、准确地跟踪参考角度 $\theta_{\text{ref}}$
    \item 调节时间 $T_s \approx 1.5$ 秒
    \item 超调量 $M_p \leq 5\%$
\end{itemize}

\textbf{系统变量:}
\begin{itemize}
    \item \textbf{输入} $u(t)$:电枢电压 (V)
    \item \textbf{状态变量} $x(t)$:$x_1 = \theta(t)$ (角度, rad),$x_2 = \dot{\theta}(t)$ (角速度, rad/s)
    \item \textbf{输出} $y(t)$:$y = \theta(t)$ (仅能测量角度)
\end{itemize}

\textbf{关键约束:}
\begin{center}
\textit{角速度 $\dot{\theta}(t)$ 无法直接测量(传感器成本高或不可用)}
\end{center}

\textbf{状态空间模型}(参数已简化):
\[A = \begin{bmatrix} 0 & 1 \\ 0 & -1 \end{bmatrix}, \quad B = \begin{bmatrix} 0 \\ 1 \end{bmatrix}, \quad C = \begin{bmatrix} 1 & 0 \end{bmatrix}\]

\textbf{物理意义:}
\begin{itemize}
    \item $\dot{x}_1 = x_2$:角度的变化率是角速度(运动学)
    \item $\dot{x}_2 = -x_2 + u$:角速度受摩擦阻尼($-x_2$)和电压驱动($u$)影响
    \item $y = x_1$:只测量角度
\end{itemize}

\subsubsection{系统分析与性能指标}

\textbf{开环特性分析:}

特征方程:
\[\det(sI - A) = \det\begin{bmatrix} s & -1 \\ 0 & s+1 \end{bmatrix} = s(s+1) = 0\]

开环极点:$s = 0, -1$

\textbf{系统特性:}
\begin{itemize}
    \item 极点 $s=0$:\textbf{临界稳定}(积分器特性)
    \item 极点 $s=-1$:稳定的阻尼项
    \item 开环系统无法自动回到期望位置,需要闭环控制
\end{itemize}

\textbf{能控性检查:}
\[W_c = [B \quad AB] = \begin{bmatrix} 0 & 1 \\ 1 & -1 \end{bmatrix}\]

$\det(W_c) = -1 \neq 0$,系统\textbf{完全能控} ✓

\textbf{能观测性检查:}
\[W_o = \begin{bmatrix} C \\ CA \end{bmatrix} = \begin{bmatrix} 1 & 0 \\ 0 & 1 \end{bmatrix}\]

$\det(W_o) = 1 \neq 0$,系统\textbf{完全能观测} ✓

\textbf{结论:}可以独立设计控制器和观测器!

\textbf{性能指标转化为极点位置:}

根据二阶系统理论,要求 $T_s \approx 1.5$s,$M_p \leq 5\%$:
\begin{itemize}
    \item 超调量 $M_p = 5\%$ $\Rightarrow$ 阻尼比 $\zeta \approx 0.69$
    \item 调节时间 $T_s = \frac{4}{\zeta\omega_n} = 1.5$ $\Rightarrow$ $\omega_n \approx 3.86$ rad/s
\end{itemize}

期望的\textbf{控制器主导极点}:
\[\lambda_{1,2} = -\zeta\omega_n \pm j\omega_n\sqrt{1-\zeta^2} \approx -2.67 \pm j2.77\]

为了计算简便,我们选择接近的极点:
\[\boxed{\lambda_{1,2} = -2.5 \pm j2.5}\]

\subsubsection{控制器设计(基于分离定理的第一步)}

我们首先\textbf{假设所有状态均可测量},设计状态反馈增益 $K$。

\textbf{控制律:}$u = -Kx + v = -[k_1 \quad k_2]x + v$

\textbf{目标:}使 $A-BK$ 的特征值为 $-2.5 \pm j2.5$。

\textbf{期望特征多项式:}
\[\alpha_c(s) = (s+2.5-j2.5)(s+2.5+j2.5) = s^2 + 5s + 12.5\]

\textbf{计算 $K$(直接法):}

\[A - BK = \begin{bmatrix} 0 & 1 \\ 0 & -1 \end{bmatrix} - \begin{bmatrix} 0 \\ 1 \end{bmatrix} [k_1 \quad k_2] = \begin{bmatrix} 0 & 1 \\ -k_1 & -1-k_2 \end{bmatrix}\]

特征方程:
\[\det(sI - (A-BK)) = s(s+1+k_2) + k_1 = s^2 + (1+k_2)s + k_1\]

令其等于 $s^2 + 5s + 12.5$:
\begin{align*}
1 + k_2 &= 5 \quad \Rightarrow \quad k_2 = 4 \\
k_1 &= 12.5
\end{align*}

\textbf{结果:}
\[\boxed{K = \begin{bmatrix} 12.5 & 4 \end{bmatrix}}\]

\textbf{理想控制律:}$u = -12.5\theta - 4\dot{\theta} + v$

\textbf{问题:}$\dot{\theta}$ 是未知的!这就是为什么我们需要观测器。

\subsubsection{观测器设计(基于分离定理的第二步)}

为解决 $\dot{\theta}$ 不可测的问题,我们设计一个观测器来提供状态估计 $\hat{x}$。

\textbf{选择观测器极点:}

\textbf{原则:}观测器的收敛速度必须比控制器快,以保证估计误差 $e(t)$ 能迅速衰减。

\textbf{经验法则:}观测器极点的实部应为控制器极点实部的 2~5 倍。

控制器极点实部为 $-2.5$,我们选择 \textbf{4 倍}:$-10$。

为避免振荡(减少噪声放大),选择两个\textbf{实极点}:
\[\mu_1 = -10, \quad \mu_2 = -12\]

\textbf{收敛时间:}
\[t_s \approx \frac{4}{10} = 0.4 \text{ 秒}\]

观测器误差在 0.4 秒内收敛,远快于控制器的 1.5 秒响应时间 ✓

\textbf{期望误差动态特征多项式:}
\[\alpha_o(s) = (s+10)(s+12) = s^2 + 22s + 120\]

\textbf{计算观测器增益 $L$(直接法):}

设 $L = \begin{bmatrix} l_1 \\ l_2 \end{bmatrix}$:

\[A - LC = \begin{bmatrix} 0 & 1 \\ 0 & -1 \end{bmatrix} - \begin{bmatrix} l_1 \\ l_2 \end{bmatrix} [1 \quad 0] = \begin{bmatrix} -l_1 & 1 \\ -l_2 & -1 \end{bmatrix}\]

特征方程:
\[\det(sI - (A-LC)) = (s+l_1)(s+1) + l_2 = s^2 + (1+l_1)s + (l_1+l_2)\]

令其等于 $s^2 + 22s + 120$:
\begin{align*}
1 + l_1 &= 22 \quad \Rightarrow \quad l_1 = 21 \\
l_1 + l_2 &= 120 \quad \Rightarrow \quad l_2 = 120 - 21 = 99
\end{align*}

\textbf{结果:}
\[\boxed{L = \begin{bmatrix} 21 \\ 99 \end{bmatrix}}\]

\subsubsection{组合系统与最终实现}

根据\textbf{分离定理},我们可以将上述独立设计的控制器和观测器安全地组合起来。

\textbf{观测器方程(实现):}
\[\dot{\hat{x}} = A\hat{x} + Bu + L(y - C\hat{x})\]

展开为:
\begin{align*}
\dot{\hat{x}}_1 &= \hat{x}_2 + 21(y - \hat{x}_1) \\
\dot{\hat{x}}_2 &= -\hat{x}_2 + u + 99(y - \hat{x}_1)
\end{align*}

\textbf{实际控制律(实现):}
\[u = -K\hat{x} + v = -12.5\hat{x}_1 - 4\hat{x}_2 + v\]

其中 $v = 12.5\theta_{\text{ref}}$ 是参考输入(使稳态误差为零)。

\textbf{完整控制系统框图:}

\begin{center}
\begin{tikzpicture}[auto, node distance=2cm, >=stealth]
    \node [draw, circle] (sum1) {$+$};
    \node [draw, rectangle, right of=sum1, node distance=2cm] (K) {$-K$};
    \node [draw, rectangle, right of=K, node distance=2.5cm] (plant) {系统 $(A,B,C)$};
    \node [draw, rectangle, below of=plant, node distance=2cm] (obs) {观测器};
    \node [coordinate, right of=plant, node distance=2cm] (output) {};
    \node [coordinate, left of=sum1, node distance=2cm] (input) {};
    
    \draw [->] (input) -- node {$v$} (sum1);
    \draw [->] (sum1) -- node {$u$} (K);
    \draw [->] (K) -- (plant);
    \draw [->] (plant) -- node [name=y] {$y$} (output);
    \draw [->] (y) |- (obs);
    \draw [->] (obs) -| node [pos=0.9] {$\hat{x}$} (K);
    \draw [->] (K) |- (obs);
\end{tikzpicture}
\end{center}

\subsubsection{系统整体特性分析}

\textbf{系统阶数:}

最终的闭环系统是一个 \textbf{4 阶系统}(2个物理状态 + 2个观测器状态)。

\textbf{系统总极点}(根据分离定理):
\[\text{极点} = \{ \underbrace{-2.5 \pm j2.5}_{\text{控制器极点,决定系统响应}}, \underbrace{-10, -12}_{\text{观测器极点,决定估计误差收敛}} \}\]

\textbf{定性分析:}

\begin{enumerate}
    \item \textbf{初始阶段}($t < 0.4$s):
    \begin{itemize}
        \item 观测器快速收敛(由 $-10, -12$ 决定)
        \item 估计状态 $\hat{x}$ 迅速逼近真实状态 $x$
        \item 此时控制效果还未完全发挥
    \end{itemize}
    
    \item \textbf{主导阶段}($t > 0.4$s):
    \begin{itemize}
        \item 观测器误差 $e \approx 0$,即 $\hat{x} \approx x$
        \item 系统动态主要由控制器极点($-2.5 \pm j2.5$)主导
        \item 展现设计的响应性能:$T_s \approx 1.5$s,$M_p \leq 5\%$
    \end{itemize}
\end{enumerate}

\textbf{稳定性分析:}

所有极点均在左半平面 $\Rightarrow$ 系统\textbf{渐近稳定}!

\textbf{响应速度:}

\begin{itemize}
    \item 观测器误差收敛:$t_s^{(o)} \approx 0.4$s(非常快)
    \item 系统输出响应:$t_s^{(c)} \approx 1.5$s(符合设计目标)
\end{itemize}

由于观测器远快于控制器,系统整体响应由控制器主导 ✓

\subsubsection{设计验证与性能评估}

\textbf{验证步骤:}

\begin{enumerate}
    \item \textbf{极点验证}
    \begin{itemize}
        \item 控制器闭环极点:$\det(sI - A + BK) = s^2 + 5s + 12.5$ ✓
        \item 观测器极点:$\det(sI - A + LC) = s^2 + 22s + 120$ ✓
    \end{itemize}
    
    \item \textbf{仿真测试}
    \begin{itemize}
        \item 初始条件:$x(0) = [0, 0]^T$,$\hat{x}(0) = [0.5, 0]^T$(初始估计误差)
        \item 参考输入:$\theta_{\text{ref}} = 1$ rad(阶跃信号)
        \item 观察:估计误差 $e(t)$、系统输出 $\theta(t)$、控制输入 $u(t)$
    \end{itemize}
\end{enumerate}

\textbf{预期结果:}

\begin{itemize}
    \item 估计误差 $e(t)$ 在 0.4s 内衰减到零
    \item 系统输出 $\theta(t)$ 在 1.5s 内到达 1 rad,超调量 $< 5\%$
    \item 控制输入 $u(t)$ 无剧烈抖动(观测器增益适中)
\end{itemize}

\textbf{MATLAB实现代码:}
\begin{verbatim}
% 系统矩阵
A = [0 1; 0 -1]; B = [0; 1]; C = [1 0];

% 控制器设计
K = [12.5 4];
eig(A - B*K)  % 验证: -2.5±2.5j

% 观测器设计
L = [21; 99];
eig(A - L*C)  % 验证: -10, -12

% 增广系统仿真(4阶)
A_aug = [A-B*K, B*K; zeros(2), A-L*C];
B_aug = [B; zeros(2,1)];
C_aug = [C, zeros(1,2)];

sys = ss(A_aug, B_aug, C_aug, 0);
step(sys * 12.5);  % 阶跃响应
\end{verbatim}

\subsubsection{实际考虑与改进方向}

\textbf{实际问题:}

\begin{enumerate}
    \item \textbf{测量噪声}
    \begin{itemize}
        \item 编码器可能有噪声($\pm 0.01$ rad)
        \item 观测器增益 $L = [21, 99]^T$ 会放大噪声
        \item 角速度估计可能抖动
    \end{itemize}
    \textbf{解决:}增加低通滤波器或使用卡尔曼滤波器
    
    \item \textbf{执行器饱和}
    \begin{itemize}
        \item 电压限制:$|u| \leq u_{\max}$
        \item 初始阶段控制输入可能饱和
    \end{itemize}
    \textbf{解决:}抗饱和设计,限制参考输入变化率
    
    \item \textbf{模型误差}
    \begin{itemize}
        \item 实际摩擦系数可能不是精确的 $-1$
        \item 可能存在未建模动态
    \end{itemize}
    \textbf{解决:}鲁棒控制方法,增加积分项消除稳态误差
\end{enumerate}

\textbf{改进方向:}

\begin{itemize}
    \item \textbf{LQR控制}:用最优控制理论自动选择 $K$
    \item \textbf{卡尔曼滤波器}:在噪声环境下的最优观测器
    \item \textbf{降维观测器}:只估计 $\dot{\theta}$,减少计算量
    \item \textbf{自适应控制}:在线估计参数变化
\end{itemize}

\subsubsection*{范例总结}

这个直流电机例子完美展示了\textbf{分离定理}的威力:

\begin{enumerate}
    \item \textbf{模块化设计}:
    \begin{itemize}
        \item 步骤1:独立设计控制器 $K$(假设状态可测)
        \item 步骤2:独立设计观测器 $L$(配置快速极点)
        \item 步骤3:直接组合,无需重新设计
    \end{itemize}
    
    \item \textbf{性能可预测}:
    \begin{itemize}
        \item 系统极点 = 控制器极点 + 观测器极点
        \item 响应特性由控制器主导(观测器足够快)
    \end{itemize}
    
    \item \textbf{实际可行}:
    \begin{itemize}
        \item 减少了传感器需求(只测角度)
        \item 降低了硬件成本
        \item 通过软件实现状态估计
    \end{itemize}
\end{enumerate}

\textbf{关键经验:}
\begin{itemize}
    \item 观测器极点选择为控制器的 3-5 倍,平衡收敛与噪声
    \item 验证能控性和能观测性是设计的前提
    \item 实际系统需要考虑噪声、饱和等非理想因素
    \item 分离定理简化了复杂控制系统的设计流程
\end{itemize}

这个例子从\textbf{问题分析}到\textbf{系统建模},从\textbf{独立设计}到\textbf{系统集成},从\textbf{理论验证}到\textbf{实际考虑},完整展示了现代控制理论在实际工程中的应用!

\subsection*{本章总结}

\subsubsection*{核心要点回顾}

\textbf{1. 状态观测器的本质}

\begin{itemize}
    \item 观测器是一个\textbf{动态系统},通过模拟真实系统并利用输出误差校正来估计状态
    \item 观测器方程:$\dot{\hat{x}} = A\hat{x} + Bu + L(y - C\hat{x})$
    \item 关键创新:利用\textbf{创新}(innovation)$y - C\hat{x}$ 进行闭环校正
    \item 与传感器的区别:算法/软件 vs 物理设备
\end{itemize}

\textbf{2. 误差动态与收敛}

\begin{itemize}
    \item 估计误差:$e = x - \hat{x}$
    \item 误差动态:$\dot{e} = (A - LC)e$(\textbf{与控制输入 $u$ 无关})
    \item 收敛条件:$A - LC$ 的所有特征值在左半平面
    \item 设计目标:选择 $L$ 配置观测器极点
\end{itemize}

\textbf{3. 对偶性(Duality)}

\begin{center}
\renewcommand{\arraystretch}{1.6}
\begin{tabular}{|l|c|c|}
\hline
\rowcolor[gray]{0.9}
& \textbf{极点配置} & \textbf{观测器设计} \\
\hline
目标矩阵 & $A - BK$ & $A - LC$ \\
\hline
前提条件 & $(A, B)$ 能控 & $(A, C)$ 能观测 \\
\hline
增益设计 & $K$ & $L$ \\
\hline
对偶关系 & --- & $(A, C) \leftrightarrow (A^T, C^T)$ \\
\hline
\end{tabular}
\end{center}

\textbf{利用对偶性:}所有极点配置算法(直接法、变换法、阿克曼公式)都可直接用于观测器设计!

\textbf{4. 分离定理(Separation Principle)}

\textbf{定理陈述:}

基于观测器的状态反馈系统特征多项式 = 控制器特征多项式 × 观测器特征多项式

\[\det(sI - A + BK) \cdot \det(sI - A + LC)\]

\textbf{定理意义:}
\begin{itemize}
    \item 控制器和观测器可以\textbf{完全独立}设计
    \item 闭环极点 = 控制器极点 + 观测器极点
    \item 无需联合优化,大大简化设计流程
\end{itemize}

\subsubsection*{设计流程总结}

\textbf{完整的基于观测器的控制系统设计:}

\begin{enumerate}
    \item \textbf{系统分析}
    \begin{itemize}
        \item 建立状态空间模型 $(A, B, C)$
        \item 验证能控性:$\text{rank}(W_c) = n$
        \item 验证能观测性:$\text{rank}(W_o) = n$
    \end{itemize}
    
    \item \textbf{控制器设计}(假设状态可测)
    \begin{itemize}
        \item 根据性能要求选择期望极点 $\lambda_1, \ldots, \lambda_n$
        \item 计算反馈增益 $K$(如阿克曼公式)
        \item 验证闭环极点
    \end{itemize}
    
    \item \textbf{观测器设计}
    \begin{itemize}
        \item 选择观测器极点 $\mu_1, \ldots, \mu_n$(通常比控制器快 2-5 倍)
        \item 利用对偶性计算 $L$
        \item 验证观测器极点
    \end{itemize}
    
    \item \textbf{系统实现}
    \begin{itemize}
        \item 观测器:$\dot{\hat{x}} = A\hat{x} + Bu + L(y - C\hat{x})$
        \item 控制律:$u = -K\hat{x} + v$
        \item 数字实现时需要离散化
    \end{itemize}
    
    \item \textbf{验证与调试}
    \begin{itemize}
        \item 仿真增广系统($2n$ 阶)
        \item 检查瞬态响应和稳态性能
        \item 评估噪声敏感性
        \item 必要时调整极点位置
    \end{itemize}
\end{enumerate}

\subsubsection*{观测器极点选择准则}

\textbf{经验法则:}

\[\text{观测器极点实部} = (2 \sim 5) \times \text{控制器极点实部}\]

\textbf{选择考虑:}

\begin{center}
\renewcommand{\arraystretch}{1.6}
\begin{tabular}{|l|p{5cm}|p{5cm}|}
\hline
\rowcolor[gray]{0.9}
\textbf{因素} & \textbf{快速极点(5倍)} & \textbf{保守极点(2倍)} \\
\hline
收敛速度 & 非常快 & 较慢 \\
\hline
噪声敏感性 & 高(大增益) & 低(小增益) \\
\hline
计算负担 & 高(小采样周期) & 低 \\
\hline
鲁棒性 & 对模型误差敏感 & 较鲁棒 \\
\hline
适用场景 & 低噪声、准确模型 & 高噪声、模型不确定 \\
\hline
\end{tabular}
\end{center}

\textbf{实用建议:}
\begin{itemize}
    \item 初次设计:选择 3-4 倍(中庸之道)
    \item 实际调试:根据噪声水平和响应速度权衡调整
    \item 实极点:适合噪声环境(避免振荡)
    \item 复数极点:适合快速跟踪应用
\end{itemize}

\subsubsection*{重要概念辨析}

\textbf{1. 观测器 vs 滤波器}

\begin{itemize}
    \item \textbf{观测器}:基于系统模型的状态估计(需要知道 $A, B, C$)
    \item \textbf{卡尔曼滤波器}:在随机噪声环境下的最优观测器(需要噪声统计特性)
    \item \textbf{低通滤波器}:简单的信号平滑(不利用系统模型)
\end{itemize}

\textbf{2. 全维观测器 vs 降维观测器}

\begin{itemize}
    \item \textbf{全维}:估计所有 $n$ 个状态(简单,本章重点)
    \item \textbf{降维}:只估计不可测的 $(n-p)$ 个状态(计算量更小)
    \item 当 $y = Cx$ 中部分状态直接可测时,降维观测器更高效
\end{itemize}

\textbf{3. 确定性观测器 vs 随机观测器}

\begin{itemize}
    \item \textbf{确定性}(本章):无随机噪声,极点配置方法
    \item \textbf{随机}(卡尔曼滤波):考虑过程噪声和测量噪声,最小化估计方差
\end{itemize}

\subsubsection*{常见误区与正确做法}

\begin{itemize}
    \item ✗ \textbf{观测器极点配置过快}
    \begin{itemize}
        \item 后果:对噪声极度敏感,控制输入抖动
        \item ✓ 正确:平衡收敛速度与噪声抑制,选择 2-5 倍控制器极点
    \end{itemize}
    
    \item ✗ \textbf{忽略能观测性检查}
    \begin{itemize}
        \item 后果:无法任意配置极点,观测器可能发散
        \item ✓ 正确:设计前必须验证 $\text{rank}(W_o) = n$
    \end{itemize}
    
    \item ✗ \textbf{认为分离定理对非线性系统成立}
    \begin{itemize}
        \item 后果:非线性系统中分离定理\textbf{不成立},需要联合设计
        \item ✓ 正确:线性系统才能独立设计控制器和观测器
    \end{itemize}
    
    \item ✗ \textbf{用数值微分代替观测器}
    \begin{itemize}
        \item 后果:噪声严重放大(如 $\dot{\theta}$ 从 $\theta$ 微分)
        \item ✓ 正确:使用观测器估计导数,自带滤波效果
    \end{itemize}
    
    \item ✗ \textbf{观测器初值随意设置}
    \begin{itemize}
        \item 后果:初始误差过大可能导致控制饱和
        \item ✓ 正确:尽量根据先验知识设置接近真实值的初值
    \end{itemize}
\end{itemize}

\subsubsection*{MATLAB工具箱}

\textbf{能观测性检查:}
\begin{verbatim}
Wo = obsv(A, C);
rank(Wo)  % 应等于 n
\end{verbatim}

\textbf{观测器设计:}
\begin{verbatim}
% 方法1:直接设计
L = place(A', C', poles)';

% 方法2:利用对偶性
L = acker(A', C', poles)';

% 验证
eig(A - L*C)  % 应等于期望极点
\end{verbatim}

\textbf{基于观测器的控制系统仿真:}
\begin{verbatim}
% 增广系统 [x; e]
A_aug = [A-B*K, B*K; zeros(n), A-L*C];
B_aug = [B; zeros(n,1)];
C_aug = [C, zeros(1,n)];

sys = ss(A_aug, B_aug, C_aug, 0);
step(sys);
\end{verbatim}

\subsubsection*{与前后章节的联系}

\textbf{与第\ref{sec:controllability-observability}章(能控性和能观测性):}
\begin{itemize}
    \item 第\ref{sec:controllability-observability}章:定义能观测性,判别准则
    \item 本章:\textbf{应用}能观测性——观测器设计
    \item 能观测性是任意配置观测器极点的\textbf{充要条件}
\end{itemize}

\textbf{与第\ref{sec:pole-placement}章(极点配置):}
\begin{itemize}
    \item 第\ref{sec:pole-placement}章:状态反馈 $u = -Kx$(假设状态可测)
    \item 本章:解决状态不可测问题——用 $\hat{x}$ 代替 $x$
    \item 分离定理:两者可以独立设计
\end{itemize}

\textbf{与后续章节(最优控制和卡尔曼滤波):}
\begin{itemize}
    \item 本章:极点配置方法设计观测器(设计者指定极点)
    \item 最优控制:LQR + 卡尔曼滤波器(自动优化极点)
    \item 线性二次高斯(LQG):随机环境下的最优设计
\end{itemize}

\subsubsection*{实际应用场景}

\textbf{1. 航空航天}
\begin{itemize}
    \item 飞行器姿态控制:测量姿态角,估计角速度
    \item 火箭导航:GPS + 惯性测量单元(IMU)融合
\end{itemize}

\textbf{2. 机器人}
\begin{itemize}
    \item 关节控制:编码器测位置,观测器估计速度和加速度
    \item 减少陀螺仪和加速度计数量
\end{itemize}

\textbf{3. 电机驱动}
\begin{itemize}
    \item 无传感器控制:只测电流和电压,估计转速和转矩
    \item 降低成本,提高可靠性
\end{itemize}

\textbf{4. 自动驾驶}
\begin{itemize}
    \item 车辆状态估计:GPS + 轮速传感器 + 观测器
    \item 估计侧向速度、横摆角速度等难以直接测量的状态
\end{itemize}

\subsubsection*{学习检查清单}

\textbf{理论理解:}
\begin{itemize}
    \item ☐ 能解释观测器的工作原理和校正机制
    \item ☐ 能推导误差动态方程 $\dot{e} = (A-LC)e$
    \item ☐ 理解对偶性:$(A, C)$ 能观测 $\leftrightarrow$ $(A^T, C^T)$ 能控
    \item ☐ 理解分离定理及其适用条件
\end{itemize}

\textbf{设计能力:}
\begin{itemize}
    \item ☐ 能验证系统的能观测性
    \item ☐ 能根据性能要求选择观测器极点
    \item ☐ 能计算观测器增益矩阵 $L$
    \item ☐ 能设计完整的控制器+观测器系统
\end{itemize}

\textbf{实践技能:}
\begin{itemize}
    \item ☐ 会使用 MATLAB 的 \texttt{place}, \texttt{acker} 设计观测器
    \item ☐ 能仿真和验证观测器性能
    \item ☐ 能分析噪声对观测器的影响
    \item ☐ 能调试和优化实际系统
\end{itemize}

\subsubsection*{延伸阅读}

\begin{itemize}
    \item \textbf{卡尔曼滤波}:随机环境下的最优观测器
    \item \textbf{滑模观测器}:对模型不确定性鲁棒的观测器
    \item \textbf{高增益观测器}:非线性系统的观测器设计
    \item \textbf{扩展卡尔曼滤波(EKF)}:非线性系统的状态估计
    \item \textbf{无迹卡尔曼滤波(UKF)}:更精确的非线性状态估计
\end{itemize}

\subsubsection*{本章核心公式}

\begin{tcolorbox}[colback=blue!5!white, colframe=blue!75!black, title=状态观测器核心公式]

\textbf{观测器方程:}
\[\dot{\hat{x}} = A\hat{x} + Bu + L(y - C\hat{x})\]

\textbf{误差动态:}
\[\dot{e} = (A - LC)e, \quad e = x - \hat{x}\]

\textbf{观测器设计定理:}

$(A, C)$ 完全能观测 $\Leftrightarrow$ 可任意配置 $A-LC$ 的极点

\textbf{分离定理:}

闭环特征多项式 = $\det(sI - A + BK) \cdot \det(sI - A + LC)$

\textbf{基于观测器的控制:}
\[u = -K\hat{x} + v\]

\end{tcolorbox}

\textbf{结语:}

状态观测器是现代控制理论的\textbf{基石之一}。它将第\ref{sec:controllability-observability}章的能观测性理论转化为实际的估计算法,与第\ref{sec:pole-placement}章的极点配置方法结合,通过分离定理构成了完整的状态反馈控制系统。

观测器的核心思想——\textbf{利用输出误差动态校正模型估计}——不仅在控制理论中至关重要,在信号处理、机器学习、导航定位等众多领域也有广泛应用。

掌握观测器设计,就掌握了从\textbf{部分信息重构完整状态}的强大工具!
