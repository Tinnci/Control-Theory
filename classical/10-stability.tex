\section{稳定性分析及劳斯稳定判据}
\label{sec:stability-analysis}

\subsection{系统稳定性的定义}
线性系统稳定的充分必要条件是:系统特征方程的所有根都具有负实部,即所有特征根都位于 $s$ 平面的左半部分。

\subsection{劳斯稳定判据}
对于特征方程:
\[a_n s^n + a_{n-1}s^{n-1} + \cdots + a_1 s + a_0 = 0\]

构造劳斯表:
\begin{center}
\begin{tabular}{c|cccc}
$s^n$ & $a_n$ & $a_{n-2}$ & $a_{n-4}$ & $\cdots$ \\
$s^{n-1}$ & $a_{n-1}$ & $a_{n-3}$ & $a_{n-5}$ & $\cdots$ \\
$s^{n-2}$ & $b_1$ & $b_2$ & $b_3$ & $\cdots$ \\
$\vdots$ & $\vdots$ & $\vdots$ & $\vdots$ & $\ddots$ \\
\end{tabular}
\end{center}

其中:$b_1 = \frac{a_{n-1}a_{n-2} - a_n a_{n-3}}{a_{n-1}}$

\textbf{劳斯稳定判据}:系统稳定的充分必要条件是劳斯表第一列的元素全部为正。

\subsection{特殊情况的处理}
\begin{itemize}
    \item 第一列出现零元素:用小正数 $\varepsilon$ 代替
    \item 某一行全为零:用前一行的辅助方程的导数代替
\end{itemize}

\subsubsection{情况一:第一列出现零元素}
当劳斯表第一列的某个元素为零,但该行其他元素不全为零时,用一个很小的正数 $\varepsilon$ 代替该零元素,然后继续计算。最后根据 $\varepsilon \to 0^+$ 的趋势来判断符号。

\textbf{示例}:系统的特征方程为 $s^4 + s^3 + 2s^2 + 2s + 3 = 0$。

劳斯表:
\begin{center}
\begin{tabular}{c|ccc}
$s^4$ & 1 & 2 & 3 \\
$s^3$ & 1 & 2 & \\
$s^2$ & $0 \to \varepsilon$ & 3 & \\
$s^1$ & $\frac{2\varepsilon - 3}{\varepsilon}$ & & \\
$s^0$ & 3 & &
\end{tabular}
\end{center}

当 $\varepsilon \to 0^+$ 时,第一列的元素为 $1, 1, \varepsilon, \frac{2\varepsilon - 3}{\varepsilon} (\approx -\infty), 3$。

由于第一列出现了两次符号变化(从 $\varepsilon$ 到负无穷,再从负无穷到 3),因此系统不稳定,且在右半 $s$ 平面有两个根。

\subsubsection{情况二:某一行全为零}
当劳斯表中出现某一行所有元素都为零时,表明系统存在关于原点对称的根(如纯虚根、大小相等符号相反的实根等)。

处理方法是:
\begin{enumerate}
    \item 利用全零行的上一行构造辅助多项式 $A(s)$。
    \item 对辅助多项式求导,$\frac{dA(s)}{ds}$。
    \item 用求导后多项式的系数替换全零行,继续计算。
\end{enumerate}

\textbf{示例}:系统的特征方程为 $s^3 + s^2 + s + 1 = 0$。

劳斯表:
\begin{center}
\begin{tabular}{c|cc}
$s^3$ & 1 & 1 \\
$s^2$ & 1 & 1 \\
$s^1$ & 0 & 0
\end{tabular}
\end{center}

$s^1$ 行为全零行。利用其上一行($s^2$ 行)构造辅助多项式:
\[A(s) = 1 \cdot s^2 + 1 \cdot s^0 = s^2 + 1\]

求导:
\[\frac{dA(s)}{ds} = 2s\]

用导数的系数 [2, 0] 替换 $s^1$ 行,得到新的劳斯表:
\begin{center}
\begin{tabular}{c|cc}
$s^3$ & 1 & 1 \\
$s^2$ & 1 & 1 \\
$s^1$ & 2 & 0 \\
$s^0$ & 1 & 
\end{tabular}
\end{center}

第一列元素 $[1, 1, 2, 1]$ 全部为正,说明在劳斯表剩下的部分没有符号变化,系统没有位于右半平面的根。全零行的出现说明系统存在关于原点对称的根,这些根由辅助方程 $A(s) = s^2 + 1 = 0$ 给出,即 $s = \pm j$。因此,系统是临界稳定的。