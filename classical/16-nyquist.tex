\section{奈奎斯特图绘制}

\subsection{奈奎斯特图的定义}
奈奎斯特图是以 $G(\jw)$ 的实部为横坐标,虚部为纵坐标,$\omega$ 从 $-\infty$ 到 $+\infty$ 变化时 $G(\jw)$ 在复平面上的轨迹。

\subsection{典型环节的奈奎斯特图}
\begin{itemize}
    \item \textbf{比例环节}: $G(\jw) = K$。奈奎斯特图是实轴上的一个点 $(K, 0)$。
    \item \textbf{二阶振荡环节}: 曲线形状取决于阻尼比 $\zeta$。
\end{itemize}

\subsubsection{积分环节}
积分环节的传递函数为 $G(s) = \frac{1}{s}$,其频率特性为 $G(\jw) = \frac{1}{\jw} = -\frac{j}{\omega}$。
当 $\omega$ 从 $0^+$ 变化到 $+\infty$ 时,奈奎斯特图是沿着负虚轴从 $-\infty$ 到原点的一条直线。

\begin{center}
\begin{tikzpicture}
\begin{axis}[
    width=8cm, height=8cm,
    xlabel={实部 Re},
    ylabel={虚部 Im},
    grid=both,
    unit vector ratio=1 1 1,
    title={积分环节奈奎斯特图},
    xmin=-1.5, xmax=1.5,
    ymin=-2, ymax=1,
    axis lines=middle,
    xtick={-1,0,1},
    ytick={-2,-1,0,1},
]
% Nyquist plot for 1/s
\addplot[blue, thick, ->] coordinates {(0, -0.1) (0, -2)};
\node at (axis cs:0.5, -1) {$\omega: 0^+ \to +\infty$};
\end{axis}
\end{tikzpicture}
\end{center}

\subsubsection{惯性环节}
惯性环节的传递函数为 $G(s) = \frac{1}{Ts+1}$,其频率特性为 $G(\jw) = \frac{1}{1+j\omega T}$。
奈奎斯特图是一个起点为 $(1,0)$ ($\omega=0$),终点为原点 ($\omega=\infty$) 的半圆。

\begin{center}
\begin{tikzpicture}
\begin{axis}[
    width=8cm, height=6cm,
    xlabel={实部 Re},
    ylabel={虚部 Im},
    grid=both,
    unit vector ratio=1 1 1,
    title={惯性环节奈奎斯特图},
    xmin=-0.2, xmax=1.2,
    ymin=-0.7, ymax=0.7,
    axis lines=middle,
]
% Nyquist plot for 1/(Ts+1)
\addplot[blue, thick, domain=0:10, samples=100, ->] 
    ({1/(1+x^2)}, {-x/(1+x^2)});
\node at (axis cs:0.5, -0.4) {$\omega: 0 \to \infty$};
\node at (axis cs:1, 0.1) {$\omega=0$};
\node at (axis cs:0, 0.1) {$\omega=\infty$};
\end{axis}
\end{tikzpicture}
\end{center}

\subsubsection{微分环节}
微分环节的传递函数为 $G(s) = s$,其频率特性为 $G(\jw) = j\omega$。
奈奎斯特图是沿着正虚轴从原点到 $+\infty$ 的一条直线。

\begin{center}
\begin{tikzpicture}
\begin{axis}[
    width=8cm, height=8cm,
    xlabel={实部 Re},
    ylabel={虚部 Im},
    grid=both,
    unit vector ratio=1 1 1,
    title={微分环节奈奎斯特图},
    xmin=-1.5, xmax=1.5,
    ymin=-1, ymax=2,
    axis lines=middle,
]
% Nyquist plot for s
\addplot[blue, thick, ->] coordinates {(0, 0) (0, 2)};
\node at (axis cs:0.5, 1) {$\omega: 0 \to +\infty$};
\end{axis}
\end{tikzpicture}
\end{center}
