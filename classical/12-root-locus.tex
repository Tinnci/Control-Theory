\section{根轨迹基本概念及其绘制(180°)}

\subsection{根轨迹的定义}
当系统中某一参数从零变化到无穷大时,闭环系统特征方程的根在 $s$ 平面上的运动轨迹称为根轨迹。

\subsection{根轨迹方程}
闭环特征方程:$1 + KG(s)H(s) = 0$

根轨迹方程的两个条件:
\begin{itemize}
    \item \textbf{幅值条件}:$|G(s)H(s)| = \frac{1}{K}$
    \item \textbf{相角条件}:$\angle G(s)H(s) = \pm 180°(2k+1)$,$k = 0, 1, 2, \cdots$
\end{itemize}

\subsection{绘制根轨迹的基本法则}
\begin{enumerate}
    \item 根轨迹的分支数等于开环极点数 $n$ 和开环零点数 $m$ 中的较大者
    \item 根轨迹起始于开环极点,终止于开环零点(有限零点或无限远零点)
    \item 根轨迹关于实轴对称
    \item 实轴上的根轨迹:实轴上某点右侧开环实零点和实极点总数为奇数
    \item 根轨迹的渐近线:
    \begin{align}
    \sigma_a &= \frac{\sum_{i=1}^n p_i - \sum_{j=1}^m z_j}{n-m} \\
    \phi_a &= \frac{(2k+1)180°}{n-m}, \quad k = 0, 1, \cdots, n-m-1
    \end{align}
    \item 分离点的计算:$\frac{d}{ds}[G(s)H(s)] = 0$
    \item 与虚轴的交点:利用劳斯判据
\end{enumerate}

\subsection{使用 LaTeX 绘制根轨迹}
使用 \texttt{rootlocus} 包绘制根轨迹:

\begin{center}
\begin{tikzpicture}
\begin{axis}[
    width=10cm,height=8cm,
    xlabel={实部 $\sigma$},
    ylabel={虚部 $j\omega$},
    grid=both,
    unit vector ratio=1 1 1,
    title={根轨迹图($K$ 从 0 到 $\infty$)},
    xmin=-4, xmax=1,
    ymin=-3, ymax=3
]

% 极点位置
\addplot[mark=x, mark size=6pt, mark options={red,thick}, only marks] 
    coordinates {(-1,0) (-3,0)};

% 根轨迹 - 实轴上的部分
\addplot[blue, thick, domain=-3:-1] {0};

% 根轨迹 - 圆形部分
\addplot[blue, thick, domain=-180:180, samples=100] 
    ({-2 + cos(x)}, {sin(x)});

% 渐近线
\addplot[dashed, gray] coordinates {(-2,-3) (-2,3)};

% 标注
\node at (axis cs:-1,-0.3) {$p_1=-1$};
\node at (axis cs:-3,-0.3) {$p_2=-3$};
\node at (axis cs:-2,2.5) {渐近线};

\end{axis}
\end{tikzpicture}
\end{center}

\textbf{示例说明}:上图显示了传递函数 $G(s) = \frac{K}{s(s+1)(s+3)}$ 的根轨迹。
