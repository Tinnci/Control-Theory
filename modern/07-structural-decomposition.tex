\section{系统的结构分解——能控、能观性分解}
\label{sec:structural-decomposition}

\subsection{术语问题:能观/能控 vs 可观/可控}

\subsubsection{术语等价性}
\textbf{能控性 (Controllability)} 和 \textbf{可控性} 是同义词;\textbf{能观性 (Observability)} 和 \textbf{可观性} 是同义词。在国内的控制理论教材和学术文献中,这两种说法都在广泛使用,可以互换。在严谨的定义和数学含义上,它们没有任何区别。

\subsection{系统的结构分解}
一般线性系统可分解为四个子系统:
\begin{itemize}
    \item 能控且能观测部分
    \item 能控但不能观测部分
    \item 不能控但能观测部分
    \item 不能控且不能观测部分
\end{itemize}

\subsection{卡尔曼分解}
通过适当的线性变换,可将系统分解为:
\[\begin{bmatrix} \dot{x}_1 \\ \dot{x}_2 \\ \dot{x}_3 \\ \dot{x}_4 \end{bmatrix} = 
\begin{bmatrix}
A_{11} & 0 & A_{13} & 0 \\
A_{21} & A_{22} & A_{23} & A_{24} \\
0 & 0 & A_{33} & 0 \\
0 & 0 & A_{43} & A_{44}
\end{bmatrix}
\begin{bmatrix} x_1 \\ x_2 \\ x_3 \\ x_4 \end{bmatrix} +
\begin{bmatrix} B_1 \\ B_2 \\ 0 \\ 0 \end{bmatrix} u\]

\[y = \begin{bmatrix} C_1 & C_2 & 0 & 0 \end{bmatrix} \begin{bmatrix} x_1 \\ x_2 \\ x_3 \\ x_4 \end{bmatrix}\]

\subsection{能观性分解与能控性分解对比}

两大分解题型在思想上是\textbf{对偶}的,解题步骤高度相似,但目标和细节有所不同。下表进行并排对比:

\begin{center}
\renewcommand{\arraystretch}{1.8}
\begin{tabular}{|l|p{7cm}|p{7cm}|}
\hline
\rowcolor[gray]{0.9}
\textbf{对比维度} & \textbf{能控性分解} & \textbf{能观性分解} \\
\hline
\textbf{核心目的} & 将状态空间分为「能被输入影响」和「完全不受输入影响」两部分 & 将状态空间分为「能从输出观测」和「完全无法观测」两部分 \\
\hline
\textbf{前提条件} & $\text{rank}(Q_c) < n$(不完全能控) & $\text{rank}(Q_o) < n$(不完全能观) \\
\hline
\textbf{关键矩阵} & 
能控性矩阵 $Q_c = \begin{bmatrix} B & AB & \cdots & A^{n-1}B \end{bmatrix}$ & 
能观性矩阵 $Q_o = \begin{bmatrix} C \\ CA \\ \vdots \\ CA^{n-1} \end{bmatrix}$ \\
\hline
\textbf{子空间} & 
能控:$Q_c$ 的列空间 \newline 不能控:$Q_c^T$ 的零空间 & 
能观:$Q_o^T$ 的列空间 \newline 不能观:$Q_o$ 的零空间 \\
\hline
\textbf{变换矩阵 $P^{-1}$} & 
列向量顺序:[能控基向量...不能控基向量] & 
列向量顺序:[能观基向量...不能观基向量] \\
\hline
\textbf{分解标准形} & 
$\bar{A} = \begin{bmatrix} A_c & A_{12} \\ \mathbf{0} & A_{uc} \end{bmatrix}$ \newline $\bar{B} = \begin{bmatrix} B_c \\ \mathbf{0} \end{bmatrix}$ \newline $\bar{C} = \begin{bmatrix} C_c & C_{uc} \end{bmatrix}$ &
$\bar{A} = \begin{bmatrix} A_o & \mathbf{0} \\ A_{21} & A_{uo} \end{bmatrix}$ \newline $\bar{B} = \begin{bmatrix} B_o \\ B_{uo} \end{bmatrix}$ \newline $\bar{C} = \begin{bmatrix} C_o & \mathbf{0} \end{bmatrix}$ \\
\hline
\textbf{零块含义} & 
$\bar{B}$ 零块:输入无法作用于不能控状态 \newline $\bar{A}$ 零块:不能控动态不受能控状态影响 & 
$\bar{C}$ 零块:不能观状态对输出无贡献 \newline $\bar{A}$ 零块:能观动态不受不能观状态影响 \\
\hline
\end{tabular}
\end{center}

\subsubsection{统一的解题流程(以能控分解为例)}

\begin{enumerate}
    \item \textbf{判断性质:}
    \begin{itemize}
        \item 计算能控性矩阵 $Q_c$
        \item 计算 $Q_c$ 的秩 $r_c = \text{rank}(Q_c)$
        \item 若 $r_c < n$,则系统不完全能控,需要进行分解
    \end{itemize}
    
    \item \textbf{寻找基向量:}
    \begin{itemize}
        \item \textbf{能控子空间}:求出 $Q_c$ 的列空间的一组基(通常取 $Q_c$ 中 $r_c$ 个线性无关的列向量)
        \item \textbf{不能控子空间}:求出 $Q_c^T$ 的零空间(解方程 $Q_c^T v = 0$)的一组基(共 $n-r_c$ 个向量)
    \end{itemize}
    
    \item \textbf{构造变换矩阵:}
    \begin{itemize}
        \item 构造 $P^{-1}$ 矩阵,将上一步求出的基向量作为其\textbf{列向量}
        \item \textbf{顺序至关重要}:必须把\textbf{能控}子空间的基放在\textbf{前面},\textbf{不能控}子空间的基放在\textbf{后面}
        \item 通过求逆得到 $P = (P^{-1})^{-1}$
    \end{itemize}
    
    \item \textbf{进行坐标变换:}
    \begin{itemize}
        \item $\bar{A} = PAP^{-1}$
        \item $\bar{B} = PB$
        \item $\bar{C} = CP^{-1}$
    \end{itemize}
    
    \item \textbf{验证与解读:}
    \begin{itemize}
        \item 检查变换后的 $(\bar{A}, \bar{B}, \bar{C})$ 是否呈现出标准分解形式(特别是关键的零块位置)
        \item 根据分解后的形式,写出能控子系统和不能控子系统的状态方程,并进行解释
    \end{itemize}
\end{enumerate}

\textbf{核心记忆点:}
\begin{itemize}
    \item \textbf{能控分解看 $Q_c$,能观分解看 $Q_o$}
    \item \textbf{构造变换矩阵 $P^{-1}$ 时,列向量的顺序是「好的」部分在前(能控/能观),「坏的」部分在后(不能控/不能观)}
    \item \textbf{记住分解后的标准形式,特别是零块的位置,这是最终的检验标准}
\end{itemize}

\subsection{综合范例 1:能观性结构分解}

\subsubsection{题目}

已知系统:
\begin{align*}
\dot{x} &= \begin{bmatrix} -4 & 2 \\ 2 & -4 \end{bmatrix} x + \begin{bmatrix} 12 \\ 0 \end{bmatrix} u \\
y &= \begin{bmatrix} 2 & 2 \end{bmatrix} x
\end{align*}

\textbf{求:}
\begin{enumerate}
    \item 判断系统的能观性
    \item 进行能观性分解
\end{enumerate}

\subsubsection{解答}

\paragraph{(1) 判断系统的能观性}

\begin{enumerate}
    \item \textbf{识别矩阵:}
    \[A = \begin{bmatrix} -4 & 2 \\ 2 & -4 \end{bmatrix}, \quad C = \begin{bmatrix} 2 & 2 \end{bmatrix}\]
    系统阶数 $n=2$。
    
    \item \textbf{构造能观性矩阵 $Q_o$:}
    \[Q_o = \begin{bmatrix} C \\ CA \end{bmatrix} = \begin{bmatrix} 2 & 2 \\ -4 & -4 \end{bmatrix}\]
    
    \item \textbf{判断秩:}
    第二行为第一行的 $-2$ 倍,故 $\text{rank}(Q_o) = 1$。
    
    \item \textbf{结论:}
    由于 $\text{rank}(Q_o) = 1 < n = 2$,该系统\textbf{不是完全能观的}。
\end{enumerate}

\paragraph{(2) 进行能观性分解}

\begin{enumerate}
    \item \textbf{寻找能观与不能观子空间:}
    \begin{itemize}
        \item \textbf{不能观子空间 ($N_o$):}求解 $Q_o v = 0$
        \[\begin{bmatrix} 2 & 2 \\ -4 & -4 \end{bmatrix} v = 0 \implies 2v_1 + 2v_2 = 0 \implies v_1 = -v_2\]
        基向量:$q_{uo} = \begin{bmatrix} 1 \\ -1 \end{bmatrix}$
        
        \item \textbf{能观子空间 ($R_o$):}$Q_o^T$ 的列空间
        \[Q_o^T = \begin{bmatrix} 2 & -4 \\ 2 & -4 \end{bmatrix}\]
        基向量:$q_o = \begin{bmatrix} 1 \\ 1 \end{bmatrix}$
    \end{itemize}
    
    \item \textbf{构造变换矩阵:}
    将能观基向量放前面,不能观基向量放后面:
    \[P^{-1} = \begin{bmatrix} 1 & 1 \\ 1 & -1 \end{bmatrix}\]
    求逆得:
    \[P = \begin{bmatrix} 1/2 & 1/2 \\ 1/2 & -1/2 \end{bmatrix}\]
    
    \item \textbf{应用坐标变换:}
    \[\bar{A} = PAP^{-1} = \begin{bmatrix} -2 & 0 \\ 0 & -6 \end{bmatrix}\]
    \[\bar{B} = PB = \begin{bmatrix} 6 \\ 6 \end{bmatrix}\]
    \[\bar{C} = CP^{-1} = \begin{bmatrix} 4 & 0 \end{bmatrix}\]
    
    \item \textbf{分解结果:}
    系统的能观性分解为:
    \begin{align*}
    \dot{\bar{x}} &= \begin{bmatrix} -2 & 0 \\ 0 & -6 \end{bmatrix} \bar{x} + \begin{bmatrix} 6 \\ 6 \end{bmatrix} u \\
    y &= \begin{bmatrix} 4 & 0 \end{bmatrix} \bar{x}
    \end{align*}
    
    其中 $\bar{x} = \begin{bmatrix} \bar{x}_o \\ \bar{x}_{uo} \end{bmatrix}$。$\bar{x}_o$ 是能观状态,$\bar{x}_{uo}$ 是不能观状态。$\bar{C}$ 矩阵中对应 $\bar{x}_{uo}$ 的元素为 0,表明该状态对输出没有贡献,故不可见。
\end{enumerate}

\subsection{综合范例 2:能控性结构分解}

\subsubsection{题目}

已知系统状态方程:
\begin{align*}
\dot{x} &= \begin{bmatrix} 0 & 0 & 1 \\ 1 & 0 & 3 \\ 0 & 1 & 1 \end{bmatrix} x + \begin{bmatrix} 1 \\ 1 \\ 0 \end{bmatrix} u \\
y &= \begin{bmatrix} 0 & 0 & 1 \end{bmatrix} x
\end{align*}

\textbf{求:}判断能控性,进行能控性分解。

\subsubsection{解答}

\paragraph{(1) 判断系统的能控性}

\begin{enumerate}
    \item \textbf{识别矩阵:}
    \[A = \begin{bmatrix} 0 & 0 & 1 \\ 1 & 0 & 3 \\ 0 & 1 & 1 \end{bmatrix}, \quad B = \begin{bmatrix} 1 \\ 1 \\ 0 \end{bmatrix}\]
    系统阶数 $n=3$。
    
    \item \textbf{构造能控性矩阵 $Q_c$:}
    计算 $AB$ 和 $A^2B$:
    \[AB = \begin{bmatrix} 0 & 0 & 1 \\ 1 & 0 & 3 \\ 0 & 1 & 1 \end{bmatrix} \begin{bmatrix} 1 \\ 1 \\ 0 \end{bmatrix} = \begin{bmatrix} 0 \\ 1 \\ 1 \end{bmatrix}\]
    \[A^2B = \begin{bmatrix} 0 & 0 & 1 \\ 1 & 0 & 3 \\ 0 & 1 & 1 \end{bmatrix} \begin{bmatrix} 0 \\ 1 \\ 1 \end{bmatrix} = \begin{bmatrix} 1 \\ 3 \\ 2 \end{bmatrix}\]
    \[Q_c = \begin{bmatrix} 1 & 0 & 1 \\ 1 & 1 & 3 \\ 0 & 1 & 2 \end{bmatrix}\]
    
    \item \textbf{判断秩:}
    对 $Q_c$ 进行行初等变换:
    \[\begin{bmatrix} 1 & 0 & 1 \\ 1 & 1 & 3 \\ 0 & 1 & 2 \end{bmatrix} \xrightarrow{R_2-R_1} \begin{bmatrix} 1 & 0 & 1 \\ 0 & 1 & 2 \\ 0 & 1 & 2 \end{bmatrix} \xrightarrow{R_3-R_2} \begin{bmatrix} 1 & 0 & 1 \\ 0 & 1 & 2 \\ 0 & 0 & 0 \end{bmatrix}\]
    因此 $\text{rank}(Q_c) = 2$。
    
    \item \textbf{结论:}
    由于 $\text{rank}(Q_c) = 2 < n = 3$,该系统\textbf{不是完全能控的}。
\end{enumerate}

\paragraph{(2) 进行能控性分解}

\begin{enumerate}
    \item \textbf{寻找能控与不能控子空间:}
    \begin{itemize}
        \item \textbf{能控子空间 ($R_c$):}$Q_c$ 的列空间。秩为 2,可选前两列作为基:
        \[q_{c1} = \begin{bmatrix} 1 \\ 1 \\ 0 \end{bmatrix}, \quad q_{c2} = \begin{bmatrix} 0 \\ 1 \\ 1 \end{bmatrix}\]
        
        \item \textbf{不能控子空间:}求解 $Q_c^T v = 0$
        \[\begin{bmatrix} 1 & 1 & 0 \\ 0 & 1 & 1 \\ 1 & 3 & 2 \end{bmatrix} v = 0\]
        从第一个方程:$v_1 + v_2 = 0 \implies v_1 = -v_2$
        
        从第二个方程:$v_2 + v_3 = 0 \implies v_3 = -v_2$
        
        基向量:$q_{uc} = \begin{bmatrix} 1 \\ -1 \\ 1 \end{bmatrix}$
    \end{itemize}
    
    \item \textbf{构造变换矩阵:}
    将能控基向量放前面,不能控基向量放后面:
    \[P^{-1} = \begin{bmatrix} 1 & 0 & 1 \\ 1 & 1 & -1 \\ 0 & 1 & 1 \end{bmatrix}\]
    
    求逆(使用行列式和伴随矩阵):
    \[\det(P^{-1}) = 1(1+1) - 0 + 1(1-0) = 3\]
    \[P = \frac{1}{3} \begin{bmatrix} 2 & 1 & -1 \\ -1 & 1 & 2 \\ 1 & -1 & 1 \end{bmatrix}\]
    
    \item \textbf{应用坐标变换:}
    \[\bar{A} = PAP^{-1} = \begin{bmatrix} 0 & 1 & 2 \\ 1 & 2 & -1 \\ 0 & 0 & -1 \end{bmatrix}\]
    
    \[\bar{B} = PB = \frac{1}{3}\begin{bmatrix} 2 & 1 & -1 \\ -1 & 1 & 2 \\ 1 & -1 & 1 \end{bmatrix} \begin{bmatrix} 1 \\ 1 \\ 0 \end{bmatrix} = \begin{bmatrix} 1 \\ 0 \\ 0 \end{bmatrix}\]
    
    \[\bar{C} = CP^{-1} = \begin{bmatrix} 0 & 0 & 1 \end{bmatrix} \begin{bmatrix} 1 & 0 & 1 \\ 1 & 1 & -1 \\ 0 & 1 & 1 \end{bmatrix} = \begin{bmatrix} 0 & 1 & 1 \end{bmatrix}\]
    
    \item \textbf{分解结果:}
    系统的能控性分解为:
    \begin{align*}
    \dot{\bar{x}} &= \begin{bmatrix} 0 & 1 & 2 \\ 1 & 2 & -1 \\ 0 & 0 & -1 \end{bmatrix} \bar{x} + \begin{bmatrix} 1 \\ 0 \\ 0 \end{bmatrix} u \\
    y &= \begin{bmatrix} 0 & 1 & 1 \end{bmatrix} \bar{x}
    \end{align*}
    
    其中 $\bar{x} = \begin{bmatrix} \bar{x}_{c} \\ \bar{x}_{uc} \end{bmatrix}$。前 2 个状态 $\bar{x}_c$ 是能控状态,第 3 个状态 $\bar{x}_{uc}$ 是不能控状态。$\bar{B}$ 矩阵中对应 $\bar{x}_{uc}$ 的元素为 0,表明输入 $u$ 无法影响该状态。同时,$\bar{A}$ 的左下角为零块,说明不能控状态的演化不受能控状态的影响(但能控状态可能受其影响)。
\end{enumerate}
