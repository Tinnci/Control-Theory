\section{奈奎斯特图绘制}

\subsection{奈奎斯特图的基本概念}

\subsubsection{什么是奈奎斯特图}
奈奎斯特图(Nyquist Plot)是开环频率响应 $G(j\omega)$ 在复平面上的图形表示。

\begin{itemize}
    \item \textbf{定义}:以 $G(j\omega)$ 的实部为横坐标,虚部为纵坐标,当 $\omega$ 从 $-\infty$ 到 $+\infty$ 变化时 $G(j\omega)$ 在复平面上的轨迹
    \item \textbf{向量形式}:$G(j\omega) = A(\omega)\angle\phi(\omega)$,其中 $A(\omega)$ 是幅值,$\phi(\omega)$ 是相角
    \item \textbf{复数形式}:$G(j\omega) = P(\omega) + jQ(\omega)$,其中 $P(\omega)$ 是实部,$Q(\omega)$ 是虚部
\end{itemize}

\subsubsection{奈奎斯特图的对称性}
\begin{itemize}
    \item 当 $\omega$ 从 $0$ 变化到 $+\infty$ 时,得到的曲线为\textbf{实线部分}
    \item 当 $\omega$ 从 $0$ 变化到 $-\infty$ 时,对应的轨迹与实线部分\textbf{关于实轴对称}
    \item 通常只需绘制 $\omega: 0 \to +\infty$ 的部分,然后关于实轴做镜像
\end{itemize}

\subsection{奈奎斯特图的绘制步骤}

\subsubsection{最小相位系统的绘制流程}

最小相位系统是指开环传递函数在s平面右半部分没有零点和极点的系统。

\textbf{通用绘制步骤:}
\begin{enumerate}
    \item 将 $G(s)$ 中的 $s$ 全部替换为 $j\omega$,得到 $G(j\omega)$
    \item 写出幅值表达式 $A(\omega) = |G(j\omega)|$ 和相角表达式 $\phi(\omega) = \angle G(j\omega)$
    \item 分析关键频率点:
    \begin{itemize}
        \item $\omega = 0$:起点幅值和相角
        \item $\omega \to \infty$:终点幅值和相角
    \end{itemize}
    \item 计算坐标轴交点(可选但有助于精确绘图):
    \begin{itemize}
        \item 令虚部 $\text{Im}[G(j\omega)] = 0$,求与实轴交点
        \item 令实部 $\text{Re}[G(j\omega)] = 0$,求与虚轴交点
        \item 或用相角法:$\phi(\omega) = k \cdot 90°$($k$ 为整数)
    \end{itemize}
    \item 根据幅值和相角的变化趋势,绘制曲线
    \item 对于I型及更高型系统,补全虚线部分(无穷大圆弧)
    \item 最后做实轴对称,得到完整的奈奎斯特图
\end{enumerate}

\subsubsection{系统类型对起始点的影响}

\begin{itemize}
    \item \textbf{0型系统}:
    \begin{itemize}
        \item 起点:$(K, 0)$(正实轴)
        \item 终点:原点 $(0, 0)$
        \item 无渐近线
    \end{itemize}
    
    \item \textbf{I型系统}:
    \begin{itemize}
        \item $\omega = 0$:幅值 $\infty$,相角 $0°$(仅看非积分项)
        \item $\omega = 0^+$:幅值 $\infty$,相角 $-90°$
        \item 有垂直的低频渐近线
        \item 终点:原点 $(0, 0)$
    \end{itemize}
    
    \item \textbf{II型系统}:
    \begin{itemize}
        \item 起点:负实轴无穷远处,相角 $-180°$
        \item 终点:原点 $(0, 0)$
        \item 有无穷大圆弧补全虚线部分
    \end{itemize}
\end{itemize}

\subsubsection{I型系统的渐近线计算}

对于标准形式 $G(s) = K \frac{\prod(T_{zi}s+1)}{\prod(T_{pi}s+1) \cdot s}$,当 $\omega \to 0^+$ 时,对应的垂直渐近线为:
$$x = K\left(\sum T_{zi} - \sum T_{pi}\right)$$

其中:
\begin{itemize}
    \item $\sum T_{zi}$:分子所有一阶项的时间常数之和
    \item $\sum T_{pi}$:分母所有一阶项的时间常数之和(不含积分项)
\end{itemize}

\subsection{典型环节的奈奎斯特图}

\subsubsection{积分环节}
积分环节的传递函数为 $G(s) = \frac{1}{s}$,其频率特性为 $G(\jw) = \frac{1}{\jw} = -\frac{j}{\omega}$。
当 $\omega$ 从 $0^+$ 变化到 $+\infty$ 时,奈奎斯特图是沿着负虚轴从 $-\infty$ 到原点的一条直线。

\begin{center}
\begin{tikzpicture}
\begin{axis}[
    width=8cm, height=8cm,
    xlabel={实部 Re},
    ylabel={虚部 Im},
    grid=both,
    unit vector ratio=1 1 1,
    title={积分环节奈奎斯特图},
    xmin=-1.5, xmax=1.5,
    ymin=-2, ymax=1,
    axis lines=middle,
    xtick={-1,0,1},
    ytick={-2,-1,0,1},
]
% Nyquist plot for 1/s: G(jω) = -j/ω
\addplot[blue, thick, domain=0.1:20, samples=100, ->] 
    ({0}, {-1/x});
\node at (axis cs:0.5, -1) {$\omega: 0^+ \to +\infty$};
\end{axis}
\end{tikzpicture}
\end{center}

\subsubsection{惯性环节}
惯性环节的传递函数为 $G(s) = \frac{1}{Ts+1}$,其频率特性为 $G(\jw) = \frac{1}{1+j\omega T}$。
奈奎斯特图是一个起点为 $(1,0)$ ($\omega=0$),终点为原点 ($\omega=\infty$) 的半圆。

\begin{center}
\begin{tikzpicture}
\begin{axis}[
    width=8cm, height=6cm,
    xlabel={实部 Re},
    ylabel={虚部 Im},
    grid=both,
    unit vector ratio=1 1 1,
    title={惯性环节奈奎斯特图 ($T=1$)},
    xmin=-0.2, xmax=1.2,
    ymin=-0.7, ymax=0.2,
    axis lines=middle,
]
% Nyquist plot for 1/(1+jωT): Re = 1/(1+ω²T²), Im = -ωT/(1+ω²T²)
\addplot[blue, thick, domain=0:15, samples=100, ->] 
    ({1/(1+x^2)}, {-x/(1+x^2)});
\node at (axis cs:0.5, -0.4) {$\omega: 0 \to \infty$};
\node at (axis cs:1, 0.1) {$(1,0)$};
\node at (axis cs:0.05, 0.1) {$(0,0)$};
\end{axis}
\end{tikzpicture}
\end{center}

\subsubsection{微分环节}
微分环节的传递函数为 $G(s) = s$,其频率特性为 $G(\jw) = j\omega$。
奈奎斯特图是沿着正虚轴从原点到 $+\infty$ 的一条直线。

\begin{center}
\begin{tikzpicture}
\begin{axis}[
    width=8cm, height=8cm,
    xlabel={实部 Re},
    ylabel={虚部 Im},
    grid=both,
    unit vector ratio=1 1 1,
    title={微分环节奈奎斯特图},
    xmin=-1.5, xmax=1.5,
    ymin=-0.5, ymax=2.5,
    axis lines=middle,
]
% Nyquist plot for jω
\addplot[blue, thick, domain=0:20, samples=100, ->] 
    ({0}, {x});
\node at (axis cs:0.5, 1) {$\omega: 0 \to +\infty$};
\end{axis}
\end{tikzpicture}
\end{center}

\subsubsection{比例环节}
比例环节的传递函数为 $G(s) = K$($K > 0$),其频率特性为 $G(\jw) = K$。
奈奎斯特图是实轴上从原点到 $K$ 的一个点。

\begin{center}
\begin{tikzpicture}
\begin{axis}[
    width=8cm, height=6cm,
    xlabel={实部 Re},
    ylabel={虚部 Im},
    grid=both,
    unit vector ratio=1 1 1,
    title={比例环节奈奎斯特图 ($K=2$)},
    xmin=-0.5, xmax=2.5,
    ymin=-1, ymax=1,
    axis lines=middle,
]
\addplot[blue, thick, only marks, mark=*] coordinates {(2, 0)};
\node at (axis cs:2, 0.3) {$(K, 0)$};
\end{axis}
\end{tikzpicture}
\end{center}

\subsubsection{二阶振荡环节}
二阶振荡环节的传递函数为 $G(s) = \frac{\omega_n^2}{s^2 + 2\zeta\omega_n s + \omega_n^2}$,其频率特性为:
$$G(\jw) = \frac{\omega_n^2}{\omega_n^2 - \omega^2 + 2\jw\zeta\omega_n\omega}$$

奈奎斯特图为一条圆弧或更复杂的曲线,形状随阻尼比 $\zeta$ 变化。

\begin{center}
\begin{tikzpicture}
\begin{axis}[
    width=10cm, height=8cm,
    xlabel={实部 Re},
    ylabel={虚部 Im},
    grid=both,
    title={二阶振荡环节奈奎斯特图 ($\omega_n=10$ rad/s)},
    xmin=-0.5, xmax=1.5,
    ymin=-1.5, ymax=1.5,
    axis lines=middle,
]
% ζ = 0.1 (欠阻尼,有谐振)
\addplot[blue, thick, domain=0:25, samples=80] 
    ({(100-x^2)/((100-x^2)^2 + (2*0.1*10*x)^2)}, 
     {-(2*0.1*10*x)/((100-x^2)^2 + (2*0.1*10*x)^2)});
\addlegendentry{$\zeta=0.1$}

% ζ = 0.5 (欠阻尼)
\addplot[red, thick, domain=0:25, samples=80] 
    ({(100-x^2)/((100-x^2)^2 + (2*0.5*10*x)^2)}, 
     {-(2*0.5*10*x)/((100-x^2)^2 + (2*0.5*10*x)^2)});
\addlegendentry{$\zeta=0.5$}

% ζ = 1.0 (临界阻尼)
\addplot[green!60!black, thick, domain=0:25, samples=80] 
    ({(100-x^2)/((100-x^2)^2 + (2*1.0*10*x)^2)}, 
     {-(2*1.0*10*x)/((100-x^2)^2 + (2*1.0*10*x)^2)});
\addlegendentry{$\zeta=1.0$}

\node at (axis cs:1, 0.1) {$\omega=0$};
\node at (axis cs:-0.2, -1.2) {$\omega=\omega_n$};
\end{axis}
\end{tikzpicture}
\end{center}

\subsection{0型系统的奈奎斯特图}

\subsubsection{绘制流程}
\begin{enumerate}
    \item 将 $G(s)$ 中的 $s$ 换成 $j\omega$
    \item 写出幅值 $A(\omega)$ 和相角 $\phi(\omega)$ 表达式
    \item 分析起点($\omega=0$)和终点($\omega=\infty$)
    \item 根据相角变化趋势绘制曲线
\end{enumerate}

\subsubsection{例题}

\textbf{例:绘制 $G(s)H(s) = \frac{6}{s^2+3s+2}$ 的幅相特性曲线。}

\textit{解:}
\begin{enumerate}
    \item \textbf{变换与分解}:
    \[G(s)H(s) = \frac{6}{(s+1)(s+2)} \implies G(j\omega)H(j\omega) = \frac{6}{(1+j\omega)(2+j\omega)}\]
    
    \item \textbf{幅相表达式}:
    \begin{align*}
    A(\omega) &= \frac{6}{\sqrt{1+\omega^2}\sqrt{4+\omega^2}} \\
    \phi(\omega) &= -\arctan(\omega) - \arctan(\frac{\omega}{2})
    \end{align*}
    
    \item \textbf{起点/终点分析}:
    \begin{align*}
    \text{起点 } (\omega=0): \quad & A(0) = \frac{6}{1 \cdot 2} = 3, \quad \phi(0) = 0° \\
    \text{终点 } (\omega=\infty): \quad & A(\infty) = 0, \quad \phi(\infty) = -90° - 90° = -180°
    \end{align*}
    
    \item \textbf{趋势与绘制}:
    \begin{itemize}
        \item 起点为 $(3, 0)$,终点为原点 $(0, 0)$
        \item 相角从 $0°$ 单调减小到 $-180°$
        \item 曲线从起点出发,顺时针旋转,经第四象限,从负实轴方向趋近于原点
    \end{itemize}
\end{enumerate}

\subsection{I型系统的奈奎斯特图}

\subsubsection{绘制流程}
\begin{enumerate}
    \item 将 $G(s)$ 中的 $s$ 替换为 $j\omega$
    \item 写出幅值 $A(\omega)$ 和相角 $\phi(\omega)$ 表达式
    \item \textbf{【特殊】}分析 $\omega=0$ 和 $\omega=0^+$ 时的幅值和相角
    \item \textbf{【特殊】}计算垂直渐近线(仅I型系统有)
    \item 根据趋势绘制实线部分
    \item 补全虚线部分(从 $\omega=0$ 的方向经过无穷大圆弧连到 $\omega=0^+$ 的方向)
\end{enumerate}

\subsubsection{例题}

\textbf{例:绘制 $G(s)H(s) = \frac{250}{s(s+5)(s+15)}$ 的幅相特性曲线。}

\textit{解:}
\begin{enumerate}
    \item \textbf{变换}:
    \[G(j\omega)H(j\omega) = \frac{250}{j\omega(j\omega+5)(j\omega+15)}\]
    
    \item \textbf{幅相表达式}:
    \begin{align*}
    A(\omega) &= \frac{250}{\omega\sqrt{\omega^2+25}\sqrt{\omega^2+225}} \\
    \phi(\omega) &= -90° - \arctan(\frac{\omega}{5}) - \arctan(\frac{\omega}{15})
    \end{align*}
    
    \item \textbf{关键频率点}:
    \begin{align*}
    \text{$\omega=0$:} \quad & A(0)=\infty, \quad \phi(0)=0° \text{ (虚线用)} \\
    \text{$\omega=0^+$:} \quad & A(0^+)=\infty, \quad \phi(0^+)=-90° \text{ (实线起点)} \\
    \text{$\omega=\infty$:} \quad & A(\infty)=0, \quad \phi(\infty)=-270° \text{ (或 }+90°\text{)}
    \end{align*}
    
    \item \textbf{渐近线计算}:
    \begin{itemize}
        \item 化为标准型:$G(s) = \frac{250}{s \cdot 5(0.2s+1) \cdot 15(s/15+1)}$
        \item 提取参数:$K=\frac{10}{3}$,$\sum T_z = 0$,$\sum T_p = 0.2 + \frac{1}{15}$
        \item 渐近线位置:$x = \frac{10}{3}(0 - 0.2 - 1/15) = -\frac{8}{9} \approx -0.89$
    \end{itemize}
    
    \item \textbf{趋势与绘制}:
    \begin{itemize}
        \item 实线部分从负虚轴无穷远处开始($\phi=-90°$)
        \item 逼近左侧垂直渐近线 $x \approx -0.89$
        \item 相角持续减小至 $-270°$,因此会穿越负实轴
        \item 终点为原点
        \item 虚线部分是从 $\phi=0°$ 到 $\phi=-90°$ 的无穷大顺时针圆弧,连接 $\omega=0$ 和 $\omega=0^+$
    \end{itemize}
\end{enumerate}

\subsection{非最小相位系统的奈奎斯特图}

\subsubsection{非最小相位系统的特点}

非最小相位系统在右半平面有零点或极点(如 $(s-a)$ 或 $(T_zs-1)$ 的项),这会深刻影响相角计算。

\subsubsection{相角的特殊处理}

对于右半平面的项,例如 $(s-a)$($a > 0$),其相角为:
\[\angle(j\omega-a) = \angle(-a+j\omega) = 180° - \arctan(\frac{\omega}{a})\]

\textbf{结论}:
\begin{itemize}
    \item 传递函数中每出现一个分母的非最小相位项,总相角就要减去 $(180° - \arctan(...))$
    \item 每出现一个分子的非最小相位项,总相角就要加上 $(180° - \arctan(...))$
\end{itemize}

\subsubsection{例题}

\textbf{例:绘制非最小相位I型系统 $G(s)H(s) = \frac{10}{s(s-1)}$ 的幅相特性曲线。}

\textit{解:}
\begin{enumerate}
    \item \textbf{变换}:
    \[G(j\omega)H(j\omega) = \frac{10}{j\omega(j\omega-1)}\]
    
    \item \textbf{幅相表达式}:
    \begin{align*}
    A(\omega) &= \frac{10}{\omega\sqrt{\omega^2+1}} \\
    \phi(\omega) &= -\angle(j\omega) - \angle(j\omega-1) \\
    &= -90° - (180° - \arctan(\omega)) \\
    &= \arctan(\omega) - 270°
    \end{align*}
    
    \item \textbf{关键频率点}:
    \begin{align*}
    \text{$\omega=0$:} \quad & A(0)=\infty, \quad \phi(0)=-180° \text{ (分母 }s(s-1) \to s(-1)\text{)} \\
    \text{$\omega=0^+$:} \quad & A(0^+)=\infty, \quad \phi(0^+)=\arctan(0)-270°=-270° \text{ (或 }+90°\text{)} \\
    \text{$\omega=\infty$:} \quad & A(\infty)=0, \quad \phi(\infty)=90°-270°=-180°
    \end{align*}
    
    \item \textbf{渐近线计算}:
    \begin{itemize}
        \item $G(s)=\frac{-10}{s(1-s)}$,参数为 $K=-10$,$T_p=-1$
        \item 渐近线:$x = (-10)(0-(-1)) = -10$
    \end{itemize}
    
    \item \textbf{趋势与绘制}:
    \begin{itemize}
        \item 实线起点在正虚轴无穷远处($\phi=-270°$),不同于最小相位系统
        \item 相角从 $-270°$ 增大到 $-180°$
        \item 终点为原点
        \item 虚线部分是从 $\phi=-180°$ 到 $\phi=-270°$ 的无穷大顺时针圆弧
    \end{itemize}
\end{enumerate}

\subsection{奈奎斯特稳定性判据}

\subsubsection{基本概念}
对于开环传递函数 $G(s)H(s)$,其中 $H(s)$ 通常为反馈传感器传递函数:
\begin{itemize}
    \item \textbf{临界点}:奈奎斯特图中的点 $(-1, 0)$
    \item \textbf{包围}:奈奎斯特曲线绕临界点的圈数
\end{itemize}

\subsubsection{奈奎斯特稳定性判据}
对于闭环系统,设开环传递函数为 $G(s)H(s)$,其:
\begin{itemize}
    \item 右半平面极点数为 $P$
    \item 奈奎斯特曲线逆时针绕 $(-1, 0)$ 点的圈数为 $N$
\end{itemize}

则闭环系统右半平面极点数为:
$$Z = P + N$$

\textbf{稳定条件}:系统稳定 $\Leftrightarrow$ $Z = 0$,即 $N = -P$

特别地,当开环系统稳定($P = 0$)时:
\begin{itemize}
    \item 稳定条件:$N = 0$,即奈奎斯特曲线不包围 $(-1, 0)$ 点
    \item 边界稳定:曲线经过 $(-1, 0)$ 点
    \item 不稳定:曲线包围 $(-1, 0)$ 点
\end{itemize}

\subsubsection{稳定裕度}

\textbf{增益裕度(Gain Margin, GM)}
从 $(-1, 0)$ 点到奈奎斯特曲线与负实轴的交点距离的倒数。
若交点坐标为 $(-a, 0)$,则 $\text{GM} = \frac{1}{a}$ 或 $\text{GM}(\text{dB}) = 20\lg\frac{1}{a}$

\textbf{相位裕度(Phase Margin, PM)}
奈奎斯特曲线与单位圆的交点对应的相位角与 $-180°$ 的夹角。

\subsubsection{绘制方法}
\begin{enumerate}
    \item 建立 $G(\jw)$ 的实部和虚部表达式
    \item 选择足够多的频率点 $\omega$(从 $0$ 到 $+\infty$)
    \item 计算每个频率点对应的 $\text{Re}[G(\jw)]$ 和 $\text{Im}[G(\jw)]$
    \item 在复平面上绘制这些点形成的轨迹
    \item 检查轨迹是否包围 $(-1, 0)$ 点
\end{enumerate}
