\section{稳定性分析及劳斯稳定判据}

\subsection{系统稳定性的定义}
线性系统稳定的充分必要条件是:系统特征方程的所有根都具有负实部,即所有特征根都位于 $s$ 平面的左半部分。

\subsection{劳斯稳定判据}
对于特征方程:
\[a_n s^n + a_{n-1}s^{n-1} + \cdots + a_1 s + a_0 = 0\]

构造劳斯表:
\begin{center}
\begin{tabular}{c|cccc}
$s^n$ & $a_n$ & $a_{n-2}$ & $a_{n-4}$ & $\cdots$ \\
$s^{n-1}$ & $a_{n-1}$ & $a_{n-3}$ & $a_{n-5}$ & $\cdots$ \\
$s^{n-2}$ & $b_1$ & $b_2$ & $b_3$ & $\cdots$ \\
$\vdots$ & $\vdots$ & $\vdots$ & $\vdots$ & $\ddots$ \\
\end{tabular}
\end{center}

其中:$b_1 = \frac{a_{n-1}a_{n-2} - a_n a_{n-3}}{a_{n-1}}$

\textbf{劳斯稳定判据}:系统稳定的充分必要条件是劳斯表第一列的元素全部为正。

\subsection{特殊情况的处理}
\begin{itemize}
    \item 第一列出现零元素:用小正数 $\varepsilon$ 代替
    \item 某一行全为零:用前一行的辅助方程的导数代替
\end{itemize}