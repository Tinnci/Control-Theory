\section{极点配置——状态反馈}
\label{sec:pole-placement}

\subsection*{引言:为什么我们需要极点配置?}

想象你正在驾驶一辆汽车。如果车辆的动力学特性很差——转向反应迟钝、加速太慢、或者容易失控,你会怎么办?理想情况下,你希望\textbf{改造}这辆车,让它响应更快、更稳定、更好操控。

在控制系统中,我们面临同样的问题:
\begin{itemize}
    \item 原始系统可能\textbf{不稳定}(极点在右半平面)
    \item 即使稳定,响应可能\textbf{太慢}(极点太靠近虚轴)
    \item 或者\textbf{振荡太大}(极点的阻尼比太小)
\end{itemize}

我们能改变系统的动态特性吗?答案是:\textbf{能!}通过\textbf{状态反馈},我们可以\textbf{任意配置闭环系统的极点}(前提是系统能控)。

\textbf{极点配置的核心思想:}
\begin{quote}
通过反馈控制,将不满意的开环极点\textit{搬到}我们期望的位置,从而获得理想的动态性能。
\end{quote}

\textbf{这有多强大?}
\begin{itemize}
    \item 不稳定系统 $\to$ 可以变成稳定的
    \item 慢系统 $\to$ 可以变得快速响应
    \item 振荡系统 $\to$ 可以变得平稳
    \item 你可以\textbf{精确指定}系统的超调量、调节时间等性能指标
\end{itemize}

\textbf{但有一个重要前提:}系统必须\textbf{完全能控}!这就是为什么我们在上一章花大量篇幅讨论能控性——它是极点配置的\textbf{充要条件}。

\textbf{实际应用场景:}
\begin{itemize}
    \item \textbf{航天器姿态控制}:配置极点以获得快速无超调的姿态调整
    \item \textbf{机器人关节控制}:通过极点配置实现精确的轨迹跟踪
    \item \textbf{电力系统稳定器}:将不稳定的电力系统稳定化
    \item \textbf{汽车主动悬架}:改善舒适性和操控性
\end{itemize}

本章将介绍:
\begin{itemize}
    \item 状态反馈的基本原理
    \item 极点配置定理(为什么能控性是关键)
    \item 三种极点配置方法(直接法、变换法、阿克曼公式)
    \item 实际设计范例
    \item 极点选择的实用准则
\end{itemize}

\subsection{状态反馈}

\subsubsection*{本节目的}
在讨论如何配置极点之前,我们需要理解\textbf{状态反馈}的概念——这是极点配置的基本工具。

\subsubsection{状态反馈的基本思想}

\textbf{经典反馈 vs 状态反馈对比:}

\begin{center}
\begin{tikzpicture}[>=stealth, node distance=2.5cm, thick]
    % 经典输出反馈
    \node[draw, rectangle, minimum width=1.5cm, minimum height=1cm] (sys1) at (0,3) {系统};
    \node[draw, rectangle, minimum width=1cm, minimum height=0.8cm] (ctrl1) at (-3,3) {$K_p$};
    \node[circle, draw, inner sep=2pt] (sum1) at (-1.5,3) {$+$};
    
    \draw[->] (ctrl1) -- (sum1);
    \draw[->] (sum1) -- node[above] {$u$} (sys1);
    \draw[->] (sys1) -- ++(1.5,0) node[above] {$y$} coordinate (out1);
    \draw[->] (out1) -- ++(0,-1.5) -| node[near start, below] {\textcolor{blue}{输出反馈}} (ctrl1);
    
    \node[above] at (0,4.2) {{\small 经典输出反馈(PID)}};
    \node[right, text width=3cm, align=left] at (2,3) {{\small 只用输出$y$\\不能任意配置极点}};
    
    % 状态反馈
    \node[draw, rectangle, minimum width=1.5cm, minimum height=1cm] (sys2) at (0,0) {系统};
    \node[draw, rectangle, minimum width=1cm, minimum height=0.8cm] (ctrl2) at (-3,0) {$K$};
    \node[circle, draw, inner sep=2pt] (sum2) at (-1.5,0) {$+$};
    
    \draw[->] (ctrl2) -- (sum2);
    \draw[->] (sum2) -- node[above] {$u$} (sys2);
    \draw[->] (sys2) -- ++(1.5,0) coordinate (out2);
    \draw[->] (out2) -- ++(0,-1.5) -| node[near start, below] {\textcolor{red!80!black}{\textbf{状态反馈}}} (ctrl2);
    
    % 状态向量指示
    \draw[->, thick, red!80!black] (sys2.south) -- ++(0,-0.8) node[below] {$x_1, x_2, \ldots, x_n$};
    
    \node[above] at (0,1.2) {{\small 状态反馈}};
    \node[right, text width=3cm, align=left, red!80!black] at (2,0) {{\small \textbf{用所有状态$\mathbf{x}$}\\能任意配置极点}};
\end{tikzpicture}
\end{center}

\begin{tcolorbox}[colback=blue!5!white, colframe=blue!75!black, title=\textbf{为什么需要状态反馈?}]
\begin{itemize}
    \item 经典输出反馈:$u = -K_p y$ \quad $\to$ 只利用\textit{部分信息}($y = Cx$)
    \item \textcolor{red!80!black}{\textbf{状态反馈}}:$u = -Kx$ \quad $\to$ 利用\textit{完整系统信息}
    \item \textbf{关键优势}:\emph{只有状态反馈才能任意配置极点}(前提:系统能控)
\end{itemize}
\end{tcolorbox}

\subsubsection{状态反馈控制律}

\begin{tcolorbox}[colback=yellow!10!white, colframe=orange!75!black, title=\textbf{状态反馈控制律——基本形式}]
\begin{center}
\Large
$\boxed{u = -Kx + v}$
\end{center}

\textbf{参数含义:}
\begin{itemize}
    \item $x \in \mathbb{R}^n$ \quad 状态向量(需全部可测或通过\textit{观测器}估计)
    \item $K \in \mathbb{R}^{p \times n}$ \quad \textcolor{red!80!black}{\textbf{反馈增益矩阵}}(\emph{待设计的关键参数})
    \item $v \in \mathbb{R}^p$ \quad 参考输入(期望的目标值)
    \item 负号 \quad 表示\textbf{负反馈}(稳定控制的标准配置)
\end{itemize}

\textbf{直观理解:}
\begin{itemize}
    \item $-Kx$ \quad \textit{校正项}:根据当前状态偏差进行调整
    \item $v$ \quad \textit{目标项}:驱动系统到期望状态
    \item $K$ 的每个元素 \quad 决定对应状态变量的\textit{权重}
\end{itemize}
\end{tcolorbox}

\vspace{0.5cm}

\begin{center}
\begin{tikzpicture}[>=stealth, node distance=2cm, thick]
    % 参考输入
    \node (ref) at (-6,0) {$v$};
    \node[circle, draw, minimum size=0.8cm] (sum1) at (-4.5,0) {$\Sigma$};
    
    % 系统
    \node[draw, rectangle, minimum width=2cm, minimum height=1.2cm] (sys) at (-1,0) {
        \begin{tabular}{c}
            \textbf{系统} \\
            $\dot{x} = Ax + Bu$
        \end{tabular}
    };
    
    % 输出
    \draw[->] (ref) -- node[above] {} (sum1);
    \draw[->] (sum1) -- node[above] {$u$} (sys);
    \draw[->] (sys) -- ++(2.5,0) node[right] {$x$};
    
    % 反馈路径
    \coordinate (fb) at (1.5,0);
    \node[draw, rectangle, minimum width=1.5cm, minimum height=0.8cm, fill=red!10] (K) at (0,-2.5) {\textcolor{red!80!black}{\textbf{$K$}}};
    
    \draw[->] (fb) -- ++(0,-1.5) -| (K);
    \draw[->] (K) -| node[near start, right] {\textcolor{red!80!black}{$-Kx$}} (sum1);
    
    % 标注
    \node[below, text width=2.5cm, align=center] at (0,-3.5) {\small \textcolor{red!80!black}{\textbf{状态反馈增益}}\\{\small(设计目标)}};
    \node[above, text width=2cm, align=center] at (-1,1.2) {\small 开环系统\\{\small 矩阵:$A$}};
\end{tikzpicture}
\end{center}

\subsubsection{闭环系统}

\textbf{闭环系统的推导:}

将 $u = -Kx + v$ 代入开环系统 $\dot{x} = Ax + Bu$:
\begin{align*}
\dot{x} &= Ax + B(\textcolor{blue}{-Kx + v}) \\
&= Ax - BKx + Bv \\
&= \textcolor{red!80!black}{(A - BK)}x + Bv
\end{align*}

\begin{tcolorbox}[colback=green!5!white, colframe=green!60!black, title=\textbf{闭环系统方程}]
\begin{center}
\Large
$\boxed{\dot{x} = \underbrace{(A - BK)}_{\textcolor{red!80!black}{A_{\text{cl}}}}x + Bv}$
\end{center}
\end{tcolorbox}

\vspace{0.5cm}

\begin{center}
\begin{tikzpicture}[>=stealth, node distance=3cm, thick]
    % 开环到闭环的变化
    \node[draw, rounded corners, fill=blue!10, minimum width=3cm, minimum height=1.5cm] (open) at (0,0) {
        \begin{tabular}{c}
            \textbf{开环系统} \\[0.2cm]
            $\dot{x} = \textcolor{blue}{A}x + Bu$ \\[0.1cm]
            极点:$\det(\lambda I - A) = 0$
        \end{tabular}
    };
    
    \node[draw, rounded corners, fill=red!10, minimum width=3cm, minimum height=1.5cm] (closed) at (7,0) {
        \begin{tabular}{c}
            \textbf{闭环系统} \\[0.2cm]
            $\dot{x} = \textcolor{red!80!black}{(A-BK)}x + Bv$ \\[0.1cm]
            极点:$\det(\lambda I - A + BK) = 0$
        \end{tabular}
    };
    
    \draw[->, very thick, blue!50!red] (open) -- node[above, text width=2.5cm, align=center] {\small \textbf{状态反馈}\\$u = -Kx + v$} (closed);
    
    % 关键观察
    \node[below, text width=3cm, align=center] at (0,-2) {\small 系统矩阵:\textcolor{blue}{\textbf{$A$}}};
    \node[below, text width=3cm, align=center] at (7,-2) {\small 系统矩阵:\textcolor{red!80!black}{\textbf{$A-BK$}}};
    
    \node[below, text width=7cm, align=center, fill=yellow!20, draw=orange, thick, rounded corners] at (3.5,-3.5) {
        \textbf{核心思想}:通过选择 $K$,改变系统矩阵,从而\emph{改变极点位置}!
    };
\end{tikzpicture}
\end{center}

\begin{tcolorbox}[colback=yellow!5!white, colframe=orange!75!black, title=\textbf{极点的变化规律}]
\begin{itemize}
    \item \textbf{开环极点}:$\det(\lambda I - \textcolor{blue}{A}) = 0$ 的根 \quad (\textit{原始系统固有特性})
    \item \textcolor{red!80!black}{\textbf{闭环极点}}:$\det(\lambda I - \textcolor{red!80!black}{A + BK}) = 0$ 的根 \quad (\emph{通过$K$可调节})
    \item $K$ 是我们的\textit{"旋钮"},可以\textbf{精确控制}极点位置
\end{itemize}
\end{tcolorbox}

\begin{tcolorbox}[colback=red!5!white, colframe=red!75!black, title=\textbf{! 重要提醒}]
\begin{itemize}
    \item 状态反馈\textbf{不改变}系统的能控性和能观测性
    \item 状态反馈\textbf{不改变}系统的零点(\emph{只改变极点})
    \item 状态反馈要求状态\textbf{可测}或\textbf{可估计}(需要观测器辅助)
\end{itemize}
\end{tcolorbox}

\subsection{极点配置定理}

\subsubsection*{本节目的}
这是本章的\textbf{核心定理},它回答了一个关键问题:我们能把极点配置到任意位置吗?答案是:\textbf{能控就能配!}

\subsubsection{定理陈述}

\begin{tcolorbox}[colback=blue!5!white, colframe=blue!75!black, title=\textbf{$\bigstar$ 极点配置定理(Pole Placement Theorem)}]
对于单输入系统 $(A, B)$,若系统\textcolor{red!80!black}{\textbf{完全能控}},则对于\emph{任意给定}的 $n$ 个复数 
\[\lambda_1, \lambda_2, \ldots, \lambda_n\]
(复数必须成对共轭出现),\textbf{存在}反馈增益矩阵 $K$,使得闭环系统矩阵 $A - BK$ 的特征值\emph{恰好}为 $\lambda_1, \lambda_2, \ldots, \lambda_n$。
\end{tcolorbox}

\vspace{0.5cm}

\begin{center}
\begin{tikzpicture}[>=stealth, thick]
    % 左侧:能控性
    \node[draw, rounded corners, fill=green!10, minimum width=3.5cm, minimum height=1.5cm] (ctrl) at (0,0) {
        \begin{tabular}{c}
            \textbf{前提条件} \\[0.2cm]
            系统\textcolor{green!60!black}{\textbf{完全能控}} \\[0.1cm]
            $\text{rank}(W_c) = n$
        \end{tabular}
    };
    
    % 中间:箭头
    \draw[->, ultra thick, green!60!black] (3.5,0) -- node[above] {\Large \textbf{保证}} (5.5,0);
    
    % 右侧:极点配置
    \node[draw, rounded corners, fill=yellow!10, minimum width=3.5cm, minimum height=1.5cm] (pole) at (9,0) {
        \begin{tabular}{c}
            \textbf{结论} \\[0.2cm]
            可\textcolor{orange!80!black}{\textbf{任意配置}}极点 \\[0.1cm]
            $\lambda_1, \ldots, \lambda_n$
        \end{tabular}
    };
    
    % 双向箭头(充要条件)
    \draw[<->, ultra thick, red!80!black] (ctrl.south) -- ++(0,-1) -| node[near start, below] {\textcolor{red!80!black}{\textbf{充要条件}}} (pole.south);
\end{tikzpicture}
\end{center}

\textbf{定理的核心含义:}

\begin{itemize}
    \item \textcolor{green!60!black}{\textbf{充要条件}}:完全能控 $\Leftrightarrow$ 可任意配置极点
    \item \textcolor{orange!80!black}{\textbf{任意性}}:极点位置\emph{完全由你决定}(只要满足共轭对要求)
    \item \textcolor{blue!80!black}{\textbf{构造性}}:不仅说\textit{存在},还有\textit{具体的计算方法}
\end{itemize}

\subsubsection{定理的深刻意义}

\textbf{1. 能控性是充要条件}

\begin{center}
\begin{tikzpicture}[>=stealth, thick, node distance=1.5cm]
    % 能控性保证 - 横向排列更紧凑
    \node[draw, rectangle, rounded corners, fill=blue!10, text width=3.5cm, align=center, minimum height=2cm] (ctrl) at (0,0) {
        \textbf{能控性} \\[0.1cm]
        $W_c = [B\ AB\ \cdots]$ \\
        张成整个状态空间
    };
    
    \node[draw, rectangle, rounded corners, fill=green!10, text width=3.5cm, align=center, minimum height=2cm] (inf) at (4.5,0) {
        \textbf{影响力} \\[0.1cm]
        通过 $BK$ 可在 \\
        所有方向施加影响
    };
    
    \node[draw, rectangle, rounded corners, fill=orange!10, text width=3.5cm, align=center, minimum height=2cm] (pole) at (9,0) {
        \textbf{极点配置} \\[0.1cm]
        可将 $A$ 的特征值 \\
        推到任意位置
    };
    
    \draw[->, very thick, blue!60!black] (ctrl) -- (inf);
    \draw[->, very thick, green!60!black] (inf) -- (pole);
\end{tikzpicture}
\end{center}

\vspace{0.3cm}

\begin{itemize}
    \item \textbf{充分性}:能控 $\Rightarrow$ 可任意配置极点(\emph{定理保证})
    \item \textbf{必要性}:可任意配置极点 $\Rightarrow$ 能控 \\
          {\small (反证:不能控意味着某些模态\textit{不受控制影响},无法移动)}
\end{itemize}

\vspace{0.3cm}

\textbf{2. 复数共轭对要求}

\begin{tcolorbox}[colback=yellow!5!white, colframe=orange!75!black]
由于系统矩阵 $A, B, K$ 都是\textbf{实矩阵},闭环矩阵 $A - BK$ 的特征多项式系数也是\textit{实数}。因此:
\begin{itemize}
    \item \textcolor{blue!80!black}{\textbf{实数极点}}:可以单独指定(如 $s = -3$)
    \item \textcolor{red!80!black}{\textbf{复数极点}}:必须\emph{成对共轭}出现(如 $s = -2 \pm 3j$)
\end{itemize}

\textbf{物理意义}:实系统不能产生单个复数极点,这会导致\textit{非物理的复数状态轨迹}。
\end{tcolorbox}

\vspace{0.3cm}

\textbf{3. 多输入系统的推广}

定理对多输入系统($p > 1$)也成立,但特点是:

\begin{itemize}
    \item $K$ \textbf{不是唯一的}(有更多自由度)
    \item 可利用额外自由度优化其他性能指标:
    \begin{itemize}
        \item[-] 鲁棒性(对参数摄动的敏感度)
        \item[-] 输入能量(控制成本)
        \item[-] 传递函数零点位置
    \end{itemize}
    \item 这是\textcolor{blue!80!black}{\textbf{最优控制}}理论(如LQR)的起点
\end{itemize}

\subsubsection{设计步骤概览}

\begin{tcolorbox}[colback=green!5!white, colframe=green!60!black, title=\textbf{$\triangleright$ 极点配置设计流程}]

\begin{center}
\begin{tikzpicture}[>=stealth, thick, node distance=2.3cm]
    \node[draw, rectangle, rounded corners, fill=blue!15, minimum width=3.2cm, minimum height=1cm] (step1) at (0,0) {
        \begin{tabular}{c}
            \textbf{步骤1:验证能控性} \\
            $\text{rank}(W_c) \stackrel{?}{=} n$
        \end{tabular}
    };
    
    \node[draw, rectangle, rounded corners, fill=green!15, minimum width=3.2cm, minimum height=1cm] (step2) at (4.5,0) {
        \begin{tabular}{c}
            \textbf{步骤2:选择极点} \\
            $\lambda_1, \ldots, \lambda_n$
        \end{tabular}
    };
    
    \node[draw, rectangle, rounded corners, fill=orange!15, minimum width=3.2cm, minimum height=1cm] (step3) at (9,0) {
        \begin{tabular}{c}
            \textbf{步骤3:计算$K$} \\
            Ackermann公式
        \end{tabular}
    };
    
    \node[draw, rectangle, rounded corners, fill=red!15, minimum width=3.2cm, minimum height=1cm] (step4) at (4.5,-2.3) {
        \begin{tabular}{c}
            \textbf{步骤4:验证设计} \\
            检查闭环极点
        \end{tabular}
    };
    
    \draw[->, very thick] (step1) -- (step2);
    \draw[->, very thick] (step2) -- (step3);
    \draw[->, very thick] (step3) -- ++(0,-1.15) -| (step4);
    \draw[->, very thick, dashed] (step4) -- ++(-3.5,0) |- node[near start, left, text width=2cm, align=center] {\small 性能不满足\\重新调整} (step2);
\end{tikzpicture}
\end{center}

\end{tcolorbox}

\begin{tcolorbox}[colback=red!5!white, colframe=red!75!black, title=\textbf{! 定理的局限性}]
\begin{itemize}
    \item 只适用于\textcolor{red!80!black}{\textbf{能控}}系统(这是\emph{硬性要求})
    \item 没有告诉你\textbf{如何选择}期望极点(需要经验和性能分析)
    \item 假设状态\textbf{完全可测}(实际中可能需要\textit{观测器})
    \item 可能导致很大的控制输入(\textbf{饱和问题})
\end{itemize}
\end{tcolorbox}

\subsection{极点配置的方法}

\subsubsection*{本节目的}
定理告诉我们极点\textbf{可以}任意配置,但\textbf{怎么}计算反馈增益矩阵 $K$ 呢?本节介绍三种实用方法。

\subsubsection{方法概览}

\begin{center}
\begin{tikzpicture}[>=stealth, thick]
    % 方法1
    \node[draw, rounded corners, fill=blue!10, text width=3.5cm, minimum height=2.2cm, align=center] (m1) at (0,0) {
        \textbf{方法1:直接法} \\[0.2cm]
        $\det(sI - A + BK) = 0$ \\[0.2cm]
        {\small 适合 $n \leq 3$} \\
        {\small 直观但繁琐}
    };
    
    % 方法2
    \node[draw, rounded corners, fill=green!10, text width=3.5cm, minimum height=2.2cm, align=center] (m2) at (5,0) {
        \textbf{方法2:变换法} \\[0.2cm]
        能控标准型 \\[0.2cm]
        {\small 适合任意阶次} \\
        {\small 需要变换矩阵}
    };
    
    % 方法3
    \node[draw, rounded corners, fill=orange!10, text width=3.5cm, minimum height=2.2cm, align=center] (m3) at (10,0) {
        \textbf{方法3:Ackermann} \\[0.2cm]
        $K = [\cdots 1]W_c^{-1}\alpha_c(A)$ \\[0.2cm]
        {\small \textcolor{red!80!black}{\textbf{最简洁}}} \\
        {\small 单输入系统}
    };
    
    % 推荐标记
    \node[star, star points=5, star point ratio=2.5, draw=red!80!black, fill=yellow, minimum size=1.2cm, thick] at (10,2) {};
    \node[text=red!80!black] at (10,2) {\small \textbf{推荐}};
\end{tikzpicture}
\end{center}

\subsubsection{方法1:直接方法}

\begin{tcolorbox}[colback=blue!5!white, colframe=blue!75!black, title=\textbf{方法1:直接方法}]
\textbf{基本思路}:直接让闭环特征多项式\emph{等于}期望特征多项式

\begin{center}
\begin{tikzpicture}[>=stealth, thick]
    \node[draw, rounded corners, fill=yellow!10, minimum width=3cm, minimum height=1cm] (desire) at (0,0) {
        \begin{tabular}{c}
            \textbf{期望特征多项式} \\
            $\alpha_d(s) = \prod(s-\lambda_i)$
        \end{tabular}
    };
    
    \node[draw, rounded corners, fill=green!10, minimum width=3cm, minimum height=1cm] (actual) at (6.5,0) {
        \begin{tabular}{c}
            \textbf{闭环特征多项式} \\
            $\det(sI - A + BK)$
        \end{tabular}
    };
    
    \draw[<->, ultra thick, red!80!black] (desire) -- node[above] {令相等} (actual);
\end{tikzpicture}
\end{center}

\textbf{计算步骤}:
\begin{enumerate}
    \item 期望多项式:$\alpha_d(s) = (s - \lambda_1)(s - \lambda_2) \cdots (s - \lambda_n) = s^n + a_1 s^{n-1} + \cdots + a_n$
    \item 闭环多项式:$\det(sI - A + BK) = s^n + b_1(K) s^{n-1} + \cdots + b_n(K)$
    \item 系数匹配:$b_i(K) = a_i, \quad i = 1, 2, \ldots, n$
    \item 解方程组得 $K$
\end{enumerate}

\begin{itemize}
    \item[\textcolor{green!60!black}{\checkmark}] \textbf{优点}:概念直接,适合手算低阶系统($n \leq 3$)
    \item[\textcolor{red!80!black}{\times}] \textbf{缺点}:高阶系统计算量大,展开 $\det$ 繁琐
\end{itemize}
\end{tcolorbox}

\subsubsection{方法2:变换方法}

\begin{tcolorbox}[colback=green!5!white, colframe=green!75!black, title=\textbf{方法2:变换方法(能控标准型)}]
\textbf{核心思想}:在\emph{能控标准型}下,极点配置变得\textbf{非常简单}!

\begin{center}
\begin{tikzpicture}[>=stealth, thick, node distance=3cm]
    \node[draw, rounded corners, fill=blue!10, minimum width=2.5cm, minimum height=1cm] (orig) at (0,0) {
        \begin{tabular}{c}
            \textbf{原系统} \\
            $(A, B)$
        \end{tabular}
    };
    
    \node[draw, rounded corners, fill=green!10, minimum width=2.5cm, minimum height=1cm] (std) at (4.5,0) {
        \begin{tabular}{c}
            \textbf{标准型} \\
            $(\bar{A}, \bar{B})$
        \end{tabular}
    };
    
    \node[draw, rounded corners, fill=orange!10, minimum width=2.5cm, minimum height=1cm] (gain) at (9,0) {
        \begin{tabular}{c}
            \textbf{反馈增益} \\
            $K$
        \end{tabular}
    };
    
    \draw[->, very thick] (orig) -- node[above] {\small 变换 $T$} (std);
    \draw[->, very thick] (std) -- node[above] {\small 简单计算} (gain);
    \draw[->, very thick, dashed, bend right=40] (orig) to node[below] {\small 直接很难} (gain);
\end{tikzpicture}
\end{center}

\textbf{能控标准型的\emph{神奇性质}}:

\[\bar{A} = \begin{bmatrix}
0 & 1 & 0 & \cdots & 0 \\
0 & 0 & 1 & \cdots & 0 \\
\vdots & \vdots & \vdots & \ddots & \vdots \\
0 & 0 & 0 & \cdots & 1 \\
\textcolor{red!80!black}{-a_0} & \textcolor{red!80!black}{-a_1} & \textcolor{red!80!black}{-a_2} & \cdots & \textcolor{red!80!black}{-a_{n-1}}
\end{bmatrix}, \quad \bar{B} = \begin{bmatrix} 0 \\ 0 \\ \vdots \\ 0 \\ 1 \end{bmatrix}\]

特征多项式\textbf{直接从最后一行读出}:
\[\boxed{\det(sI - \bar{A}) = s^n + \textcolor{red!80!black}{a_{n-1}}s^{n-1} + \cdots + \textcolor{red!80!black}{a_1}s + \textcolor{red!80!black}{a_0}}\]

\textbf{设计步骤}:
\begin{enumerate}
    \item 找变换矩阵 $T$,使 $\bar{A} = TAT^{-1}, \bar{B} = TB$
    \item 在标准型下设计(\textbf{超级简单}):
    \[\textcolor{red!80!black}{\bar{K} = [\alpha_0 - a_0 \quad \alpha_1 - a_1 \quad \cdots \quad \alpha_{n-1} - a_{n-1}]}\]
    其中 $\alpha_i$ 是\emph{期望}特征多项式的系数
    \item 变换回原坐标:$K = \bar{K}T$
\end{enumerate}

\begin{itemize}
    \item[\textcolor{green!60!black}{\checkmark}] \textbf{优点}:计算规范,适用任意阶次,不易出错
    \item[\textcolor{red!80!black}{\times}] \textbf{缺点}:需要计算 $T$,当系统\textit{接近不能控}时数值不稳定
\end{itemize}
\end{tcolorbox}

\subsubsection{方法3:阿克曼公式(Ackermann's Formula)}

\begin{tcolorbox}[colback=orange!5!white, colframe=orange!75!black, title=\textbf{$\bigstar$ 方法3:Ackermann公式(最简洁!)}]

\textbf{单输入系统的显式公式}:

\begin{center}
\Large
\colorbox{yellow!30}{$\boxed{K = [0 \quad 0 \quad \cdots \quad 0 \quad 1] W_c^{-1} \alpha_c(A)}$}
\end{center}

\textbf{符号说明}:
\begin{itemize}
    \item $W_c = [B \quad AB \quad A^2B \quad \cdots \quad A^{n-1}B]$ \quad 能控性矩阵
    \item $\alpha_c(s) = s^n + \alpha_1 s^{n-1} + \cdots + \alpha_n$ \quad \emph{期望}特征多项式
    \item $\alpha_c(A) = A^n + \alpha_1 A^{n-1} + \cdots + \alpha_n I$ \quad \textbf{矩阵多项式}
\end{itemize}

\begin{center}
\begin{tikzpicture}[>=stealth, thick]
    \node[draw, circle, fill=blue!10, minimum size=1.5cm] (wc) at (0,0) {$W_c^{-1}$};
    \node[draw, circle, fill=green!10, minimum size=1.5cm] (alpha) at (3,0) {$\alpha_c(A)$};
    \node[draw, circle, fill=orange!10, minimum size=1.5cm] (k) at (7,0) {$K$};
    
    \node[above, text width=2cm, align=center] at (0,1.2) {\small 能控性\\矩阵的逆};
    \node[above, text width=2.5cm, align=center] at (3,1.2) {\small 期望多项式\\代入 $A$};
    \node[above, text width=2cm, align=center] at (7,1.2) {\small \textcolor{red!80!black}{\textbf{反馈增益}}};
    
    \draw[->, ultra thick] (wc) -- node[above] {$\times$} (alpha);
    \draw[->, ultra thick] (alpha) -- node[above] {\small 取最后一行} (k);
\end{tikzpicture}
\end{center}

\textbf{关键提醒}:$\alpha_c(A)$ 是将多项式中的 $s$ \emph{替换为矩阵} $A$!

\begin{itemize}
    \item[\textcolor{green!60!black}{\checkmark}] \textbf{优点}:
    \begin{itemize}
        \item 最简洁——\textit{一个公式搞定}
        \item 易于编程实现(MATLAB: \texttt{acker(A,B,poles)})
        \item 不需要坐标变换
    \end{itemize}
    \item[\textcolor{red!80!black}{\times}] \textbf{缺点}:
    \begin{itemize}
        \item 仅适用于\textbf{单输入}系统
        \item 需要求逆 $W_c^{-1}$(能控时必可逆)
        \item 数值稳定性依赖于 $W_c$ 的条件数
    \end{itemize}
\end{itemize}

\textbf{MATLAB实现}:
\begin{verbatim}
K = acker(A, B, desired_poles)
\end{verbatim}

\end{tcolorbox}

\subsubsection{三种方法的对比}

\begin{center}
\renewcommand{\arraystretch}{1.6}
\begin{tabular}{@{}l c c c@{}}
\toprule
\textbf{特性} & \textbf{直接方法} & \textbf{变换方法} & \textbf{Ackermann公式} \\
\midrule
\textbf{适用范围} & 低阶($n \leq 3$) & 任意阶次 & 单输入系统 \\[0.2em]
\textbf{计算复杂度} & 中等 & 较高 & \textcolor{green!60!black}{\textbf{低}} \\[0.2em]
\textbf{直观性} & \textcolor{blue!80!black}{\textbf{高}} & 中等 & 低 \\[0.2em]
\textbf{数值稳定性} & 好 & 取决于 $T$ & 取决于 $W_c$ \\[0.2em]
\midrule
\textbf{推荐场景} & 手算 & 理论分析 & \textcolor{red!80!black}{\textbf{程序实现}} \\
\bottomrule
\end{tabular}
\end{center}

\vspace{0.3cm}

\begin{center}
\small
\begin{tabular}{@{}l p{11cm}@{}}
\toprule
\multicolumn{2}{c}{\textbf{详细说明}} \\
\midrule
\textbf{直接方法} & 展开行列式 $\det(sI - A + BK)$ 并与期望多项式系数匹配,适合手算低阶系统 \\[0.3em]
\textbf{变换方法} & 先变换到能控标准型,在标准型下设计 $\bar{K}$,再变换回原坐标 $K = \bar{K}T$ \\[0.3em]
\textbf{Ackermann} & 直接公式 $K = [0 \cdots 0 \quad 1] W_c^{-1} \alpha_c(A)$,最简洁实用 \\
\bottomrule
\end{tabular}
\end{center}

\begin{center}
\begin{tcolorbox}[colback=yellow!10, colframe=orange!75!black, width=12cm]
\textbf{实用建议}:
\begin{itemize}
    \item \textbf{课堂作业}:直接方法(便于理解原理)
    \item \textbf{理论推导}:变换方法(揭示结构特性)
    \item \textcolor{red!80!black}{\textbf{工程应用}}:\emph{Ackermann公式}(快速准确)
\end{itemize}
\end{tcolorbox}
\end{center}

\subsection{极点选择的实用准则}

\subsubsection*{本节目的}
理论上可以任意配置极点,但\textbf{实际中如何选择}期望极点?这需要平衡\textit{性能}、\textit{鲁棒性}和\textit{实现成本}。

\subsubsection{性能指标与极点位置的关系}

\textbf{二阶系统的标准形式:}
\[s^2 + 2\zeta\omega_n s + \omega_n^2 = 0\]

极点位置:$s_{1,2} = -\zeta\omega_n \pm j\omega_n\sqrt{1-\zeta^2}$

\begin{center}
\begin{tikzpicture}[>=stealth, scale=1.2]
    % 坐标轴
    \draw[->] (-0.5,0) -- (5,0) node[right] {$\text{Re}(s)$};
    \draw[->] (0,-3) -- (0,3) node[above] {$\text{Im}(s)$};
    
    % 稳定区域
    \fill[blue!10, opacity=0.5] (-0.5,-3) rectangle (0,3);
    \node[text=red!80!black, rotate=90] at (0.3,0) {\textbf{不稳定区域}};
    \node[text=green!60!black, rotate=90] at (-0.3,0) {\textbf{稳定区域}};
    
    % 极点示例
    \node[circle, fill=blue, inner sep=2pt, label=above right:{实极点}] (p1) at (1.5,0) {};
    \node[circle, fill=red, inner sep=2pt, label=above:{复数极点对}] (p2) at (2,1.5) {};
    \node[circle, fill=red, inner sep=2pt] (p3) at (2,-1.5) {};
    
    % 参数标注
    \draw[dashed, thick, blue!50] (2,0) -- (p2);
    \draw[<->, orange, thick] (0,0) -- node[below, sloped] {$\omega_n$} (p2);
    \draw[orange, thick] (2,0) arc (0:36.87:1.5) node[midway, right] {$\theta$};
    \draw[<->, green!60!black, thick] (2,0) -- node[right] {$\omega_d$} (2,1.5);
    \draw[<->, red!80!black, thick] (0,0) -- node[above] {$\zeta\omega_n$} (2,0);
    
    % 性能区域
    \draw[thick, purple, dashed] (3.5,-2.5) -- (3.5,2.5);
    \node[text=purple, text width=2cm, align=center] at (4.2,2) {\small 响应\\更快};
    
    \draw[thick, orange, dashed] (1,-2.8) -- (3.5,0);
    \draw[thick, orange, dashed] (1,2.8) -- (3.5,0);
    \node[text=orange, text width=2cm, align=center] at (0.5,2.5) {\small 超调\\较大};
    
    % 公式标注
    \node[below, text width=7cm, align=left] at (2.5,-3.5) {
        $\zeta = \cos\theta$, \quad $\omega_d = \omega_n\sqrt{1-\zeta^2}$
    };
\end{tikzpicture}
\end{center}

\begin{tcolorbox}[colback=blue!5!white, colframe=blue!75!black, title=\textbf{关键参数}]
\begin{itemize}
    \item $\omega_n$ \quad \textbf{自然频率}——影响\emph{响应速度}(越大越快)
    \item $\zeta$ \quad \textbf{阻尼比}——影响\emph{超调量和振荡}(越大越平稳)
    \item $\omega_d = \omega_n\sqrt{1-\zeta^2}$ \quad 阻尼自然频率(虚部大小)
\end{itemize}
\end{tcolorbox}

\textbf{性能指标与极点位置的关系}:

\begin{center}
\renewcommand{\arraystretch}{1.5}
\begin{tabular}{|l|c|c|}
\hline
\rowcolor{blue!20}
\textbf{性能指标} & \textbf{公式} & \textbf{极点参数} \\
\hline
调节时间 & $t_s \approx \dfrac{4}{\zeta\omega_n}$ & \cellcolor{yellow!20}实部越大越小 \\
\hline
超调量 & $M_p \approx e^{-\pi\zeta/\sqrt{1-\zeta^2}}$ & \cellcolor{green!20}$\zeta$越大越小 \\
\hline
峰值时间 & $t_p = \dfrac{\pi}{\omega_d}$ & 虚部越大越小 \\
\hline
\end{tabular}
\end{center}

\subsubsection{极点配置的实用原则}

\begin{tcolorbox}[colback=green!5!white, colframe=green!60!black, title=\textbf{$\bigcirc$ 原则1:稳定性要求(基本要求)}]
\begin{itemize}
    \item 所有极点必须在\textcolor{green!60!black}{\textbf{左半平面}}($\text{Re}(s_i) < 0$)
    \item 距离虚轴\emph{越远},稳定裕度\emph{越大}
\end{itemize}

\begin{center}
\begin{tikzpicture}[>=stealth, scale=0.8]
    \draw[->] (-4,0) -- (1,0) node[right] {$\text{Re}$};
    \draw[->] (0,-2) -- (0,2) node[above] {$\text{Im}$};
    \fill[green!20, opacity=0.5] (-4,-2) rectangle (0,2);
    \node at (-2,1.5) {\textcolor{green!60!black}{\textbf{稳定}}};
    \fill[red!20, opacity=0.5] (0,-2) rectangle (1,2);
    \node at (0.5,1.5) {\textcolor{red!80!black}{\textbf{不稳定}}};
    
    \node[circle, fill=green!60!black, inner sep=2pt] at (-3,0.5) {};
    \node[circle, fill=green!60!black, inner sep=2pt] at (-3,-0.5) {};
    \node[right] at (-2.8,0) {\small 稳定裕度大};
    
    \node[circle, fill=orange, inner sep=2pt] at (-0.5,1) {};
    \node[circle, fill=orange, inner sep=2pt] at (-0.5,-1) {};
    \node[right] at (-0.3,0) {\small 稳定裕度小};
\end{tikzpicture}
\end{center}
\end{tcolorbox}

\begin{tcolorbox}[colback=blue!5!white, colframe=blue!75!black, title=\textbf{$\bullet$ 原则2:响应速度要求}]
\begin{itemize}
    \item \textbf{主导极点}的实部决定响应速度:$t_s \approx \dfrac{4}{|\text{Re}(s)|}$
    \item 更快响应 $\to$ 极点更靠左(\textcolor{red!80!black}{\textbf{但}}需要更大的控制能量)
\end{itemize}

\begin{center}
\begin{tikzpicture}[>=stealth, scale=0.8]
    \draw[->] (-5,0) -- (1,0) node[right] {$\text{Re}$};
    \draw[->] (0,-2) -- (0,2) node[above] {$\text{Im}$};
    
    \node[circle, fill=red, inner sep=2pt, label=below:{\small 慢}] at (-1,0) {};
    \draw[<->, red] (-1,-1.5) -- node[right] {$t_s$ 大} (-1,-2.5);
    
    \node[circle, fill=orange, inner sep=2pt, label=below:{\small 中}] at (-2.5,0) {};
    \draw[<->, orange] (-2.5,-1.5) -- node[right] {$t_s$ 中} (-2.5,-2.3);
    
    \node[circle, fill=green!60!black, inner sep=2pt, label=below:{\small 快}] at (-4,0) {};
    \draw[<->, green!60!black] (-4,-1.5) -- node[right] {$t_s$ 小} (-4,-2);
\end{tikzpicture}
\end{center}
\end{tcolorbox}

\begin{tcolorbox}[colback=yellow!5!white, colframe=orange!75!black, title=\textbf{$\sim$ 原则3:超调和振荡要求}]
\begin{itemize}
    \item \textcolor{blue!80!black}{\textbf{实极点}}:无超调,单调响应
    \item \textcolor{red!80!black}{\textbf{复数极点}}:有超调和振荡
    \item 常用选择:$\zeta = 0.5 \sim 0.707$(\emph{适度阻尼})
\end{itemize}

\begin{center}
\begin{tikzpicture}[>=stealth, scale=0.8]
    % 三种情况
    \draw[->] (-5,0) -- (0.5,0) node[right] {$\text{Re}$};
    \draw[->] (0,-2.5) -- (0,2.5) node[above] {$\text{Im}$};
    
    % 实极点
    \node[circle, fill=blue, inner sep=2pt, label=left:{\small 实极点}] at (-3,0) {};
    \node[below, text width=2cm, align=center] at (-3,-2.8) {\small \textcolor{blue!80!black}{无超调}};
    
    % 适度阻尼
    \node[circle, fill=green!60!black, inner sep=2pt] at (-2,1.2) {};
    \node[circle, fill=green!60!black, inner sep=2pt, label=right:{\small $\zeta \approx 0.7$}] at (-2,-1.2) {};
    \node[below, text width=2cm, align=center] at (-2,-2.8) {\small \textcolor{green!60!black}{\textbf{推荐}}};
    
    % 欠阻尼
    \node[circle, fill=red, inner sep=2pt] at (-1,2) {};
    \node[circle, fill=red, inner sep=2pt, label=right:{\small $\zeta$ 小}] at (-1,-2) {};
    \node[below, text width=2cm, align=center] at (-1,-2.8) {\small \textcolor{red!80!black}{振荡大}};
\end{tikzpicture}
\end{center}
\end{tcolorbox}

\begin{tcolorbox}[colback=purple!5!white, colframe=purple!75!black, title=\textbf{$\triangleright$ 原则4:高阶系统的极点配置}]
\begin{itemize}
    \item \textcolor{red!80!black}{\textbf{主导极点}}:最靠近虚轴的极点对(\emph{决定主要动态特性})
    \item \textcolor{blue!80!black}{\textbf{非主导极点}}:远离虚轴(衰减快,影响小)
    \item 典型配置:2个主导极点 + $(n-2)$ 个快衰减极点
\end{itemize}

\begin{center}
\begin{tikzpicture}[>=stealth, scale=0.9]
    \draw[->] (-5,0) -- (0.5,0) node[right] {$\text{Re}$};
    \draw[->] (0,-2.5) -- (0,2.5) node[above] {$\text{Im}$};
    
    % 主导极点
    \node[circle, fill=red, inner sep=3pt] (d1) at (-1,1.2) {};
    \node[circle, fill=red, inner sep=3pt] (d2) at (-1,-1.2) {};
    \node[left, text=red!80!black] at (-1,1.5) {\textbf{主导极点}};
    \draw[->, thick, red!80!black] (d1) -- ++(-0.5,0) node[left] {\small 决定性能};
    
    % 非主导极点
    \node[circle, fill=blue, inner sep=2pt] at (-3.5,0.5) {};
    \node[circle, fill=blue, inner sep=2pt] at (-3.5,-0.5) {};
    \node[circle, fill=blue, inner sep=2pt] at (-4.5,0) {};
    \node[left, text=blue!80!black] at (-4,1) {\textbf{非主导极点}};
    \draw[->, thick, blue!80!black] (-4,-0.3) -- ++(-0.5,-0.3) node[left, text width=2cm] {\small 快速衰减\\影响小};
\end{tikzpicture}
\end{center}
\end{tcolorbox}

\begin{tcolorbox}[colback=red!5!white, colframe=red!75!black, title=\textbf{! 原则5:实际限制}]
\begin{itemize}
    \item \textcolor{red!80!black}{\textbf{控制饱和}}:极点太左 $\to$ 控制信号\emph{过大}(执行器限制)
    \item \textcolor{orange!80!black}{\textbf{测量噪声}}:极点太左 $\to$ 对噪声\emph{敏感}(高增益放大噪声)
    \item \textcolor{blue!80!black}{\textbf{模型误差}}:极点太精确 $\to$ \emph{鲁棒性差}(对参数变化敏感)
\end{itemize}

\textbf{设计平衡}:性能 $\leftrightarrow$ 稳定性 $\leftrightarrow$ 实现成本
\end{tcolorbox}

\subsubsection{经验法则}

\textbf{巴特沃斯配置(Butterworth Pattern):}

将极点均匀分布在以原点为中心的圆弧上,角度为:
\[\theta_k = \frac{(2k+1)\pi}{2n}, \quad k = 0, 1, \ldots, n-1\]

半径选择为 $r = \omega_n$(期望的响应速度)。

\textbf{贝塞尔配置(Bessel Pattern):}
\begin{itemize}
    \item 优化阶跃响应的延迟时间
    \item 极点分布更集中
    \item 适合对延迟敏感的应用
\end{itemize}

\textbf{ITAE最优配置(Integral of Time-weighted Absolute Error):}
\begin{itemize}
    \item 最小化 $\int_0^\infty t|e(t)|dt$
    \item 有标准的极点位置表(查表即可)
\end{itemize}

\subsubsection{极点配置的常见错误}

\begin{itemize}
    \item[$\times$] 极点配置过于激进(远离虚轴)$\to$ 控制饱和
    \item[$\times$] 忽略零点的影响(系统零点不变,可能抵消部分极点效果)
    \item[$\times$] 所有极点聚在同一位置 $\to$ 数值不稳定
    \item[$\checkmark$] 平衡性能和实现成本
    \item[$\checkmark$] 验证闭环系统对参数变化的鲁棒性
\end{itemize}

\subsection{实际应用范例}

\subsubsection*{范例说明}
通过具体例子展示:
\begin{itemize}
    \item 如何使用阿克曼公式设计控制器
    \item 如何选择期望极点
    \item 如何验证设计结果
\end{itemize}

\subsubsection{范例:二阶系统的极点配置}

\textbf{题目:}设计状态反馈控制器,使系统具有快速无超调响应

\textbf{系统描述:}
\begin{align*}
\dot{x} &= \begin{bmatrix} 0 & 1 \\ -2 & -3 \end{bmatrix} x + \begin{bmatrix} 0 \\ 1 \end{bmatrix} u \\
y &= \begin{bmatrix} 1 & 0 \end{bmatrix} x
\end{align*}

\textbf{性能要求:}
\begin{itemize}
    \item 调节时间 $t_s < 2$s
    \item 无超调(实极点)
\end{itemize}

\textbf{解答:}

\paragraph{第一步:验证能控性}

\[W_c = [B \quad AB] = \begin{bmatrix} 0 & 1 \\ 1 & -3 \end{bmatrix}\]

$\det(W_c) = -1 \neq 0$,系统\textbf{完全能控},可以任意配置极点。

\paragraph{第二步:选择期望极点}

要求无超调 $\to$ 选择\textbf{实极点}

要求 $t_s < 2$s,由 $t_s \approx \frac{4}{|\text{Re}(s)|}$ 得:
\[|\text{Re}(s)| > \frac{4}{2} = 2\]

选择两个快速实极点:$s_1 = -3, s_2 = -4$

期望特征多项式:
\[\alpha_d(s) = (s+3)(s+4) = s^2 + 7s + 12\]

\paragraph{第三步:使用阿克曼公式}

计算 $\alpha_d(A)$:
\begin{align*}
\alpha_d(A) &= A^2 + 7A + 12I \\
&= \begin{bmatrix} -2 & -3 \\ 6 & 7 \end{bmatrix} + 7\begin{bmatrix} 0 & 1 \\ -2 & -3 \end{bmatrix} + 12\begin{bmatrix} 1 & 0 \\ 0 & 1 \end{bmatrix} \\
&= \begin{bmatrix} -2 & -3 \\ 6 & 7 \end{bmatrix} + \begin{bmatrix} 0 & 7 \\ -14 & -21 \end{bmatrix} + \begin{bmatrix} 12 & 0 \\ 0 & 12 \end{bmatrix} \\
&= \begin{bmatrix} 10 & 4 \\ -8 & -2 \end{bmatrix}
\end{align*}

计算 $W_c^{-1}$:
\[W_c^{-1} = \frac{1}{-1}\begin{bmatrix} -3 & -1 \\ -1 & 0 \end{bmatrix} = \begin{bmatrix} 3 & 1 \\ 1 & 0 \end{bmatrix}\]

应用阿克曼公式:
\begin{align*}
K &= [0 \quad 1] W_c^{-1} \alpha_d(A) \\
&= [0 \quad 1] \begin{bmatrix} 3 & 1 \\ 1 & 0 \end{bmatrix} \begin{bmatrix} 10 & 4 \\ -8 & -2 \end{bmatrix} \\
&= [1 \quad 0] \begin{bmatrix} 10 & 4 \\ -8 & -2 \end{bmatrix} \\
&= [10 \quad 4]
\end{align*}

因此,反馈增益:$\boxed{K = [10 \quad 4]}$

\paragraph{第四步:验证设计}

闭环系统矩阵:
\[A - BK = \begin{bmatrix} 0 & 1 \\ -2 & -3 \end{bmatrix} - \begin{bmatrix} 0 \\ 1 \end{bmatrix} [10 \quad 4] = \begin{bmatrix} 0 & 1 \\ -12 & -7 \end{bmatrix}\]

特征方程:
\[\det(sI - (A-BK)) = s^2 + 7s + 12 = (s+3)(s+4)\]

极点确实为 $-3, -4$,\textbf{设计成功}!

\paragraph{性能分析}

\begin{itemize}
    \item 调节时间:$t_s \approx \frac{4}{3} \approx 1.33$s(满足 $< 2$s)
    \item 无超调(实极点)
    \item 控制律:$u = -10x_1 - 4x_2 + v$
\end{itemize}

\subsubsection*{范例总结}

\textbf{关键步骤回顾:}
\begin{enumerate}
    \item 验证能控性(必须满足)
    \item 根据性能要求选择极点(平衡速度和成本)
    \item 计算反馈增益(阿克曼公式最简洁)
    \item 验证闭环极点(检查计算正确性)
\end{enumerate}

\textbf{实际考虑:}
\begin{itemize}
    \item 如果状态不可测,需要设计观测器(下一章)
    \item 极点位置影响控制能量($K$ 越大,$u$ 越大)
    \item 实际系统需要考虑执行器饱和限制
\end{itemize}

\subsection*{本章总结}

\subsubsection*{核心要点回顾}

\textbf{1. 极点配置的本质}
\begin{itemize}
    \item 通过状态反馈 $u = -Kx + v$ 改变系统矩阵
    \item 开环 $A$ $\to$ 闭环 $A - BK$
    \item 从而改变系统的动态特性(极点位置)
\end{itemize}

\textbf{2. 极点配置定理(核心)}
\begin{itemize}
    \item \textbf{完全能控} $\Leftrightarrow$ \textbf{可任意配置极点}
    \item 这是能控性概念的直接应用
    \item 复数极点必须成对共轭出现
\end{itemize}

\textbf{3. 三种计算方法}
\begin{itemize}
    \item \textbf{直接法}:展开特征方程,适合低阶手算
    \item \textbf{变换法}:转为能控标准型,适合理论分析
    \item \textbf{阿克曼公式}:$K = [0 \cdots 0 \quad 1] W_c^{-1} \alpha_d(A)$,\textbf{最实用}
\end{itemize}

\textbf{4. 极点选择原则}
\begin{itemize}
    \item 所有极点在左半平面(稳定性)
    \item 实部大小决定响应速度($t_s \approx 4/|\text{Re}(s)|$)
    \item 虚部/实部比决定超调量($\zeta$)
    \item 平衡性能、鲁棒性和实现成本
\end{itemize}

\subsubsection*{设计流程总结}

\textbf{标准极点配置流程:}
\begin{enumerate}
    \item \textbf{系统分析}
    \begin{itemize}
        \item 写出系统方程 $\dot{x} = Ax + Bu$
        \item 计算开环极点(了解原始特性)
        \item 验证能控性 $\text{rank}(W_c) = n$
    \end{itemize}
    
    \item \textbf{性能要求转化}
    \begin{itemize}
        \item 调节时间 $t_s$ $\to$ 极点实部
        \item 超调量 $M_p$ $\to$ 阻尼比 $\zeta$
        \item 确定期望极点 $\lambda_1, \ldots, \lambda_n$
    \end{itemize}
    
    \item \textbf{控制器计算}
    \begin{itemize}
        \item 选择计算方法(推荐阿克曼公式)
        \item 计算反馈增益 $K$
        \item 控制律 $u = -Kx + v$
    \end{itemize}
    
    \item \textbf{验证与调整}
    \begin{itemize}
        \item 验证闭环极点位置
        \item 仿真闭环响应
        \item 检查控制信号幅值(是否饱和)
        \item 必要时调整极点位置
    \end{itemize}
\end{enumerate}

\subsubsection*{重要概念辨析}

\textbf{极点配置 vs 最优控制:}
\begin{itemize}
    \item \textbf{极点配置}:直接指定极点位置(设计者经验)
    \item \textbf{最优控制(LQR)}:优化性能指标得到极点(自动平衡)
    \item 极点配置更直观,LQR更系统化
\end{itemize}

\textbf{状态反馈 vs 输出反馈:}
\begin{itemize}
    \item 状态反馈需要所有状态(可能需要观测器)
    \item 输出反馈只用输出(但不能任意配置极点)
    \item 实际常用:状态观测器 + 状态反馈
\end{itemize}

\subsubsection*{常见误区}

\begin{itemize}
    \item[$\times$] 认为能控性不重要(能控性是极点配置的\textbf{充要条件})
    \item[$\times$] 极点配置得越快越好(要考虑控制饱和和噪声)
    \item[$\times$] 忽略系统零点(零点不变,会影响实际响应)
    \item[$\times$] 期望所有极点重合(数值不稳定)
    \item[$\checkmark$] 仔细选择极点位置,平衡性能和鲁棒性
    \item[$\checkmark$] 验证闭环系统对参数摄动的敏感性
\end{itemize}

\subsubsection*{实际应用建议}

\textbf{工程实践:}
\begin{itemize}
    \item 从保守的极点位置开始(不要过于激进)
    \item 逐步调整,观察实际系统响应
    \item 考虑测量噪声对高增益的放大效应
    \item 使用仿真验证非线性因素(饱和、死区等)
\end{itemize}

\textbf{MATLAB工具:}
\begin{itemize}
    \item \texttt{ctrb(A,B)}:计算能控性矩阵
    \item \texttt{acker(A,B,p)}:阿克曼公式(单输入)
    \item \texttt{place(A,B,p)}:极点配置(多输入)
    \item \texttt{step(sys)}:阶跃响应仿真
\end{itemize}

\subsubsection*{后续章节预告}

极点配置解决了\textbf{控制器设计}问题,但实际中常面临:
\begin{itemize}
    \item \textbf{状态不可测} $\to$ 需要状态观测器(第\ref{sec:state-observer}章)
    \item \textbf{性能指标优化} $\to$ 最优控制理论(LQR)
    \item \textbf{鲁棒性要求} $\to$ 鲁棒控制方法
\end{itemize}

极点配置是现代控制理论的\textbf{第一个综合性设计方法}。它将前面学习的能控性概念转化为实际的控制器,体现了从理论到实践的完整链条:

\begin{center}
\textbf{能控性} $\to$ \textbf{极点配置} $\to$ \textbf{性能实现}
\end{center}

掌握极点配置,就掌握了状态反馈控制的核心技术!
