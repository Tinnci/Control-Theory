\section{能控、能观标准型及其实现}
\label{sec:standard-forms}

\subsection*{引言:为系统选择 \textquotedblleft 最佳姿态 \textquotedblright}

想象你要搬运一个大箱子。如果箱子是竖着的,搬起来很困难;但如果把它放平,就容易多了。\textbf{这就是标准型的作用:通过线性变换,将系统矩阵转换为结构简单、便于分析和设计的 \textquotedblleft 标准姿态 \textquotedblright 。}

\subsubsection*{为什么需要标准型?}

在\ref{sec:linear-transformation}节中,我们学会了线性变换 $x = Pz$。但问题是:\textbf{如何选择变换矩阵 $P$,才能让新系统矩阵 $\bar{A}$ 的结构最简单?}标准型就是答案。

\begin{itemize}[leftmargin=2em]
    \item \textbf{分析便利}:标准型矩阵结构清晰,系统性质一目了然
    \item \textbf{设计便利}:极点配置、观测器设计等算法在标准型下最简单
    \item \textbf{计算便利}:标准型可直接从传递函数或能控/能观矩阵获得
    \item \textbf{理论价值}:不同标准型揭示了系统的本质特性
\end{itemize}

\subsubsection*{三种主要标准型}

\begin{table}[h]
\centering
\begin{tabular}{|l|l|l|}
\hline
\rowcolor{gray!20}
\textbf{标准型} & \textbf{关键特征} & \textbf{主要用途} \\
\hline
\textbf{能控标准型} & $\bar{B} = [0, 0, \ldots, 1]^T$ & 极点配置设计 \\
\hline
\textbf{能观标准型} & $\bar{C} = [0, 0, \ldots, 1]$ & 状态观测器设计 \\
\hline
\textbf{对角标准型} & $\bar{A} = \text{diag}(\lambda_1, \ldots, \lambda_n)$ & 系统解耦、稳定性分析 \\
\hline
\end{tabular}
\end{table}

\subsubsection*{本章路线图}

\begin{enumerate}
    \item \textbf{能控标准型}:定义、矩阵结构、变换方法、极点配置应用
    \item \textbf{能观标准型}:定义、对偶关系、观测器设计应用
    \item \textbf{对角标准型}:条件、实现步骤、与\ref{sec:linear-transformation}节的联系
    \item \textbf{标准型对比}:如何根据任务选择合适的标准型
    \item \textbf{通向结构分解}:为\ref{sec:structural-decomposition}节的Kalman分解做准备
\end{enumerate}

\subsection{能控标准型}

\subsubsection*{定义与矩阵结构}

对于\textbf{单输入}系统 $\dot{x} = Ax + Bu$,如果系统\textbf{完全能控},则存在线性变换使其化为\textbf{能控标准型}:

\begin{empheq}[box=\fbox]{equation}
\bar{A} = \begin{bmatrix}
0 & 1 & 0 & \cdots & 0 \\
0 & 0 & 1 & \cdots & 0 \\
\vdots & \vdots & \vdots & \ddots & \vdots \\
0 & 0 & 0 & \cdots & 1 \\
-a_0 & -a_1 & -a_2 & \cdots & -a_{n-1}
\end{bmatrix}, \quad \bar{B} = \begin{bmatrix} 0 \\ 0 \\ \vdots \\ 0 \\ 1 \end{bmatrix}
\end{equation}

\subsubsection*{关键特性}

\begin{enumerate}
    \item \textbf{矩阵结构}:$\bar{A}$ 是伴随矩阵(Companion Matrix),最后一行为特征多项式系数的负值
    \item \textbf{输入矩阵}:$\bar{B}$ 的最后一个元素为1,其余为0(输入直接作用在最后一个状态)
    \item \textbf{能控性}:能控矩阵 $\bar{Q}_c = [\bar{B}, \bar{A}\bar{B}, \ldots, \bar{A}^{n-1}\bar{B}]$ 自动满秩
    \item \textbf{特征多项式}:$\det(\lambda I - \bar{A}) = \lambda^n + a_{n-1}\lambda^{n-1} + \cdots + a_1\lambda + a_0$
\end{enumerate}

\subsubsection*{变换矩阵的构造}

设原系统的能控矩阵为 $Q_c = [B, AB, \ldots, A^{n-1}B]$,则变换矩阵为:

\begin{empheq}[box=\fbox]{equation}
P = Q_c \cdot T
\end{equation}

其中 $T$ 是特定的变换矩阵(与特征多项式系数有关,推导见教材)。

\subsubsection*{物理意义}

能控标准型将系统表示为\textbf{串联积分器链}:

\begin{tcolorbox}[colback=blue!5!white, colframe=blue!75!black, title=能控标准型的物理结构]
输入 $u$ → 积分器1 ($\dot{z}_n$) → 积分器2 ($\dot{z}_{n-1}$) → $\cdots$ → 积分器n ($\dot{z}_1$)

最后一个状态 $z_n$ 受输入和所有状态的反馈控制:
\[z_n = -a_0 z_1 - a_1 z_2 - \cdots - a_{n-1} z_n + u\]
\end{tcolorbox}

\subsection{能观标准型}

\subsubsection*{定义与矩阵结构}

对于\textbf{单输出}系统 $\dot{x} = Ax + Bu$,$y = Cx$,如果系统\textbf{完全能观},则存在线性变换使其化为\textbf{能观标准型}:

\begin{empheq}[box=\fbox]{equation}
\bar{A} = \begin{bmatrix}
0 & 0 & \cdots & 0 & -a_0 \\
1 & 0 & \cdots & 0 & -a_1 \\
0 & 1 & \cdots & 0 & -a_2 \\
\vdots & \vdots & \ddots & \vdots & \vdots \\
0 & 0 & \cdots & 1 & -a_{n-1}
\end{bmatrix}, \quad \bar{C} = \begin{bmatrix} 0 & 0 & \cdots & 0 & 1 \end{bmatrix}
\end{equation}

\subsubsection*{与能控标准型的对偶关系}

能观标准型是能控标准型的\textbf{转置}:

\begin{align*}
\bar{A}_{\text{能观}} &= \bar{A}_{\text{能控}}^T \\
\bar{C}_{\text{能观}} &= \bar{B}_{\text{能控}}^T
\end{align*}

这种对偶性反映了能控性和能观性的对称关系。

\subsection{对角标准型}

\subsubsection*{定义}

当系统矩阵 $A$ 有 $n$ 个\textbf{互不相同的特征值} $\lambda_1, \lambda_2, \ldots, \lambda_n$ 时,可化为\textbf{对角标准型}(即\ref{sec:linear-transformation}节中的对角化):

\begin{empheq}[box=\fbox]{equation}
\bar{A} = P^{-1}AP = \begin{bmatrix}
\lambda_1 & 0 & \cdots & 0 \\
0 & \lambda_2 & \cdots & 0 \\
\vdots & \vdots & \ddots & \vdots \\
0 & 0 & \cdots & \lambda_n
\end{bmatrix} = \Lambda
\end{equation}

其中 $P = [v_1, v_2, \ldots, v_n]$ 是特征向量矩阵。

\subsubsection*{优势}

\begin{itemize}
    \item \textbf{完全解耦}:每个状态独立演化,$\dot{z}_i = \lambda_i z_i$
    \item \textbf{稳定性判定}:直接观察特征值(实部<0则稳定)
    \item \textbf{求解简便}:$e^{\bar{A}t} = \text{diag}(e^{\lambda_1 t}, \ldots, e^{\lambda_n t})$(参考\ref{sec:solving-state-space}节)
\end{itemize}

\subsection{例题1:转换为能控标准型}

\textbf{问题}:将系统转换为能控标准型
\[A = \begin{bmatrix} -2 & -3 \\ 1 & 0 \end{bmatrix}, \quad B = \begin{bmatrix} 1 \\ 0 \end{bmatrix}\]

\textbf{解}:

\textbf{步骤1}:验证能控性
\[Q_c = [B, AB] = \begin{bmatrix} 1 & -2 \\ 0 & 1 \end{bmatrix}, \quad \det(Q_c) = 1 \neq 0 \quad \checkmark\]

系统完全能控。

\textbf{步骤2}:求特征多项式
\[\det(\lambda I - A) = \det \begin{bmatrix} \lambda + 2 & 3 \\ -1 & \lambda \end{bmatrix} = \lambda^2 + 2\lambda + 3\]

因此 $a_0 = 3$,$a_1 = 2$。

\textbf{步骤3}:构造能控标准型
\[\bar{A} = \begin{bmatrix} 0 & 1 \\ -3 & -2 \end{bmatrix}, \quad \bar{B} = \begin{bmatrix} 0 \\ 1 \end{bmatrix}\]

\textbf{步骤4}:求变换矩阵(利用 $Q_c$ 和能控标准型的能控矩阵)
\[\bar{Q}_c = [\bar{B}, \bar{A}\bar{B}] = \begin{bmatrix} 0 & 1 \\ 1 & -2 \end{bmatrix}\]

变换矩阵:
\[P = Q_c \bar{Q}_c^{-1} = \begin{bmatrix} 1 & -2 \\ 0 & 1 \end{bmatrix} \begin{bmatrix} 0 & 1 \\ 1 & -2 \end{bmatrix}^{-1} = \begin{bmatrix} 1 & -2 \\ 0 & 1 \end{bmatrix} \begin{bmatrix} 2 & 1 \\ 1 & 0 \end{bmatrix} = \begin{bmatrix} 0 & 1 \\ 1 & 0 \end{bmatrix}\]

\textbf{步骤5}:验证
\begin{align*}
P^{-1}AP &= \begin{bmatrix} 0 & 1 \\ 1 & 0 \end{bmatrix} \begin{bmatrix} -2 & -3 \\ 1 & 0 \end{bmatrix} \begin{bmatrix} 0 & 1 \\ 1 & 0 \end{bmatrix} = \begin{bmatrix} 0 & 1 \\ -3 & -2 \end{bmatrix} \quad \checkmark \\
P^{-1}B &= \begin{bmatrix} 0 & 1 \\ 1 & 0 \end{bmatrix} \begin{bmatrix} 1 \\ 0 \end{bmatrix} = \begin{bmatrix} 0 \\ 1 \end{bmatrix} \quad \checkmark
\end{align*}

\subsection{例题2:三种标准型的对比}

\textbf{问题}:对于系统 $A = \begin{bmatrix} -3 & 1 \\ -2 & 0 \end{bmatrix}$,$B = \begin{bmatrix} 1 \\ 1 \end{bmatrix}$,$C = \begin{bmatrix} 1 & 0 \end{bmatrix}$,分别求能控标准型、能观标准型和对角标准型。

\textbf{解}:

\textbf{1. 能控标准型}

特征多项式:$\det(\lambda I - A) = \lambda^2 + 3\lambda + 2 = (\lambda+1)(\lambda+2)$

因此:
\[\bar{A}_c = \begin{bmatrix} 0 & 1 \\ -2 & -3 \end{bmatrix}, \quad \bar{B}_c = \begin{bmatrix} 0 \\ 1 \end{bmatrix}\]

\textbf{2. 能观标准型}

\[\bar{A}_o = \begin{bmatrix} 0 & -2 \\ 1 & -3 \end{bmatrix}, \quad \bar{C}_o = \begin{bmatrix} 0 & 1 \end{bmatrix}\]

\textbf{3. 对角标准型}

特征值:$\lambda_1 = -1$,$\lambda_2 = -2$

对于 $\lambda_1 = -1$:$v_1 = \begin{bmatrix} 1 \\ 2 \end{bmatrix}$

对于 $\lambda_2 = -2$:$v_2 = \begin{bmatrix} 1 \\ 1 \end{bmatrix}$

因此:
\[P = \begin{bmatrix} 1 & 1 \\ 2 & 1 \end{bmatrix}, \quad \bar{A}_d = \begin{bmatrix} -1 & 0 \\ 0 & -2 \end{bmatrix}\]

\textbf{对比}:三种标准型的特征值完全相同(-1和-2),这是相似变换的不变性。

\subsection{MATLAB实现}

\begin{lstlisting}[style=Matlab-editor, caption=标准型转换的MATLAB实现]
% 定义原系统
A = [-2 -3; 1 0];
B = [1; 0];
C = [1 0];
D = 0;
sys = ss(A, B, C, D);

% 方法1:使用canon函数转换为能控标准型
sys_ctrbf = canon(sys, 'companion');
[A_c, B_c, C_c, D_c] = ssdata(sys_ctrbf);
fprintf('能控标准型:\n');
A_c, B_c

% 方法2:使用canon函数转换为能观标准型
sys_obsvf = canon(sys, 'modal');  % modal是对角型的近似
[A_o, B_o, C_o, D_o] = ssdata(sys_obsvf);
fprintf('模态标准型(对角化):\n');
A_o, B_o

% 方法3:手动对角化
[V, Lambda] = eig(A);  % V是特征向量矩阵
A_diag = inv(V) * A * V;
B_diag = inv(V) * B;
C_diag = C * V;
fprintf('对角标准型:\n');
Lambda, B_diag

% 验证传递函数不变
[num, den] = ss2tf(A, B, C, D);
[num_c, den_c] = ss2tf(A_c, B_c, C_c, D_c);
fprintf('传递函数对比:\n');
tf_original = tf(num, den)
tf_canonical = tf(num_c, den_c)
\end{lstlisting}

\subsection*{本章小结}

\subsubsection*{核心要点}

\begin{tcolorbox}[colback=green!5!white, colframe=green!75!black, title=三种标准型的核心特征]
\textbf{能控标准型}:
\begin{itemize}
    \item 矩阵:伴随矩阵(最后一行为 $-a_0, -a_1, \ldots, -a_{n-1}$)
    \item 条件:系统完全能控
    \item 应用:极点配置设计(\ref{sec:pole-placement}节)
\end{itemize}

\textbf{能观标准型}:
\begin{itemize}
    \item 矩阵:能控标准型的转置
    \item 条件:系统完全能观
    \item 应用:状态观测器设计(\ref{sec:state-observer}节)
\end{itemize}

\textbf{对角标准型}:
\begin{itemize}
    \item 矩阵:对角矩阵 $\text{diag}(\lambda_1, \ldots, \lambda_n)$
    \item 条件:特征值互不相同
    \item 应用:系统解耦、稳定性分析
\end{itemize}
\end{tcolorbox}

\subsubsection*{标准型选择指南}

\begin{table}[h]
\centering
\caption{标准型选择对比}
\begin{tabular}{|l|l|l|l|}
\hline
\rowcolor{gray!20}
\textbf{任务} & \textbf{推荐标准型} & \textbf{原因} & \textbf{前提条件} \\
\hline
极点配置 & 能控标准型 & 状态反馈增益计算简单 & 完全能控 \\
\hline
观测器设计 & 能观标准型 & 观测器增益计算简单 & 完全能观 \\
\hline
稳定性分析 & 对角标准型 & 直接读取特征值 & 无重特征值 \\
\hline
系统解耦 & 对角标准型 & 状态独立演化 & 无重特征值 \\
\hline
时域响应计算 & 对角标准型 & $e^{\Lambda t}$ 计算简便 & 无重特征值 \\
\hline
重特征值系统 & 约当标准型 & 处理不可对角化情况 & 见\ref{sec:jordan-form}节 \\
\hline
\end{tabular}
\end{table}

\subsubsection*{常见误区}

\begin{itemize}
    \item ❌ \textbf{误区1}:任何系统都能对角化(错!重特征值且线性无关特征向量不足时不能对角化)
    \item ❌ \textbf{误区2}:能控标准型和能观标准型是同一个(错!它们是转置关系)
    \item ❌ \textbf{误区3}:标准型改变了系统的特征值(错!特征值是相似变换的不变量)
\end{itemize}

\subsubsection*{与其他章节的联系}

\begin{itemize}
    \item \textbf{向后链接}:
    \begin{itemize}
        \item \ref{sec:linear-transformation}节:对角标准型就是对角化
        \item \ref{sec:controllability-observability}节:能控/能观性是标准型存在的前提
        \item \ref{sec:transfer-function}节:能控标准型直接对应传递函数形式
    \end{itemize}
    \item \textbf{向前链接}:
    \begin{itemize}
        \item \ref{sec:structural-decomposition}节:标准型是Kalman分解的基础
        \item \ref{sec:jordan-form}节:约当型处理不可对角化情况
        \item \ref{sec:pole-placement}节:极点配置在能控标准型下最简单
        \item \ref{sec:state-observer}节:观测器设计在能观标准型下最简单
    \end{itemize}
\end{itemize}

\subsubsection*{学习检查清单}

\begin{itemize}
    \item[$\square$] 理解三种标准型的矩阵结构(能控、能观、对角)
    \item[$\square$] 掌握能控标准型的构造步骤(能控矩阵、特征多项式)
    \item[$\square$] 理解能控型和能观型的对偶关系(转置)
    \item[$\square$] 会判断系统能否对角化(特征值是否互不相同)
    \item[$\square$] 知道如何根据任务选择合适的标准型
    \item[$\square$] 能够使用MATLAB的 \texttt{canon()} 函数进行标准型转换
\end{itemize}
