\section{结构图与信号流图}

\subsection{结构图的基本元件}
\begin{itemize}
    \item \textbf{方块}:表示系统或环节的传递函数
    \item \textbf{信号线}:表示信号的传输方向
    \item \textbf{相加点}:表示信号的相加或相减
    \item \textbf{分支点}:表示信号的分叉
\end{itemize}

\tikzstyle{block} = [draw, rectangle, minimum height=3em, minimum width=4em]
\tikzstyle{sum} = [draw, circle, node distance=1.5cm]
\tikzstyle{input} = [coordinate]
\tikzstyle{output} = [coordinate]

\begin{figure}[h!]
\centering
\begin{tikzpicture}[auto, node distance=2.5cm, >=Latex]
    \node [input, name=input] {};
    \node [block, right=of input] (controller) {控制器};
    \node [block, right=of controller] (plant) {被控对象};
    \node [output, right=of plant] (output) {};
    \draw [->] (input) -- node[name=r] {$R(s)$} (controller);
    \draw [->] (controller) -- node {$U(s)$} (plant);
    \draw [->] (plant) -- node[name=y] {$Y(s)$} (output);
\end{tikzpicture}
\caption{开环控制系统示例}
\end{figure}

\begin{figure}[h!]
\centering
\begin{tikzpicture}[auto, node distance=2cm, >=Latex]
    \node [input, name=input] {};
    \node [sum, right=of input] (sum) {};
    \node [block, right=of sum] (controller) {控制器};
    \node [block, right=of controller, node distance=3cm] (plant) {被控对象};
    \coordinate [right=of plant, node distance=3cm] (output) {};
    \node [block, below=of plant] (sensor) {传感器};
    
    \draw [->] (input) -- node[pos=0.9] {$R(s)$} node[pos=0.9, above] {$+$} (sum);
    \draw [->] (sum) -- node {$E(s)$} (controller);
    \draw [->] (controller) -- (plant);
    \draw [->] (plant) -- node [name=y, near end] {$Y(s)$} (output);
    \draw [->] (y) |- (sensor);
    \draw [->] (sensor) -| node[pos=0.9, right] {$-$} (sum);
\end{tikzpicture}
\caption{闭环(反馈)控制系统示例}
\end{figure}

\subsection{结构图的等效变换}
\begin{itemize}
    \item 串联:$G(s) = G_1(s)G_2(s)$
    \item 并联:$G(s) = G_1(s) + G_2(s)$
    \item 反馈:$G(s) = \frac{G_1(s)}{1 \pm G_1(s)H(s)}$
\end{itemize}

\subsection{信号流图}
信号流图是用有向线段和节点组成的图形,用来表示系统各变量之间的关系。
