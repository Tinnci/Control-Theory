\section{伯德图绘制}

\subsection{伯德图的定义}
伯德图(Bode Plot)是由亨德里克·韦德·伯德(Hendrik Wade Bode)提出的频率响应图示方法,由两个图组成:

\begin{itemize}
    \item \textbf{幅频特性图(Magnitude Plot)}:$L(\omega) = 20\lg|G(\jw)|$ dB vs $\lg\omega$
    \begin{itemize}
        \item 纵轴:对数幅值(dB)
        \item 横轴:对数频率($\lg\omega$)
    \end{itemize}
    \item \textbf{相频特性图(Phase Plot)}:$\phi(\omega) = \angle G(\jw)$ vs $\lg\omega$
    \begin{itemize}
        \item 纵轴:相位角(度或弧度)
        \item 横轴:对数频率($\lg\omega$)
    \end{itemize}
\end{itemize}

\textbf{伯德图的优点:}
\begin{enumerate}
    \item 频率范围广,可表示从极低频到极高频的特性
    \item 不同环节的伯德图可以直接相加(叠加原理)
    \item 可用渐近线逼近,绘制简便
    \item 便于分析系统的稳定性和性能指标
\end{enumerate}

\subsection{典型环节的伯德图}

\subsubsection{比例环节 \texorpdfstring{$K$}{K}}
传递函数:$G(s) = K$

频率响应:$G(\jw) = K$

\textbf{幅频特性:}
\begin{align*}
L(\omega) &= 20\lg K \text{ dB(水平线)}
\end{align*}

\textbf{相频特性:}
\begin{align*}
\phi(\omega) &= \begin{cases}
0° & K > 0 \\
180° & K < 0
\end{cases}
\end{align*}

\subsubsection{积分环节 $\frac{1}{s}$}
传递函数:$G(s) = \frac{1}{s}$

频率响应:$G(\jw) = \frac{1}{\jw}$

\textbf{幅频特性:}
\begin{align*}
L(\omega) &= 20\lg\frac{1}{\omega} = -20\lg\omega \text{ dB}
\end{align*}
\begin{itemize}
    \item 斜率:$-20$ dB/十倍频(decade)
    \item 当 $\omega = 1$ rad/s 时,$L(\omega) = 0$ dB
    \item 当 $\omega$ 增大10倍时,$L(\omega)$ 下降20 dB
\end{itemize}

\textbf{相频特性:}
\begin{align*}
\phi(\omega) &= -90°\text{(恒定)}
\end{align*}

\subsubsection{微分环节 $s$}
与积分环节对称:
\begin{itemize}
    \item $L(\omega) = 20\lg\omega$ dB(斜率 $+20$ dB/十倍频)
    \item $\phi(\omega) = 90°$
\end{itemize}

\subsubsection{惯性环节 \texorpdfstring{$\frac{1}{1+Ts}$}{1/(1+Ts)}}
传递函数:$G(s) = \frac{1}{1+Ts}$

频率响应:$G(\jw) = \frac{1}{1+\jw T}$

\textbf{转折频率(Corner Frequency):}$\omega_c = \frac{1}{T}$ rad/s

\textbf{幅频特性:}
\begin{align*}
L(\omega) &= 20\lg|G(\jw)| = -20\lg\sqrt{1 + \omega^2T^2}
\end{align*}

渐近线:
\begin{itemize}
    \item 当 $\omega \ll \omega_c$:$L(\omega) \approx 0$ dB
    \item 当 $\omega \gg \omega_c$:$L(\omega) \approx -20\lg(\omega T)$ dB(斜率 $-20$ dB/十倍频)
    \item 转折点 $\omega = \omega_c$:精确值 $L(\omega_c) = -3$ dB
\end{itemize}

\textbf{相频特性:}
\begin{align*}
\phi(\omega) &= -\arctan(\omega T)
\end{align*}
\begin{itemize}
    \item $\omega = 0.1\omega_c$:$\phi \approx -6°$
    \item $\omega = \omega_c$:$\phi = -45°$
    \item $\omega = 10\omega_c$:$\phi \approx -84°$
    \item $\omega \to \infty$:$\phi \to -90°$
\end{itemize}

\textbf{示例伯德图:}$G(s) = \frac{1}{1+0.1s}$ ($\omega_c = 10$ rad/s)
\begin{center}
% Bode plot for G(s) = 1/(1+0.1 s), corner freq ω_c = 10 rad/s
\BodeTF{num/{1},den/{0.1,1}}{0.1}{1000}
\end{center}

\subsubsection{一阶微分环节 $1+Ts$}
与惯性环节对称:
\begin{itemize}
    \item 幅频特性:低频0 dB,高频斜率 $+20$ dB/十倍频
    \item 相频特性:$\phi(\omega) = \arctan(\omega T)$,$\phi(\omega_c) = 45°$
\end{itemize}

\subsubsection{振荡环节 $\frac{\omega_n^2}{s^2 + 2\zeta\omega_n s + \omega_n^2}$}
传递函数标准形式:
\begin{align*}
G(s) = \frac{\omega_n^2}{s^2 + 2\zeta\omega_n s + \omega_n^2}
\end{align*}

其中:
\begin{itemize}
    \item $\omega_n$:无阻尼自然频率
    \item $\zeta$:阻尼比($0 < \zeta < 1$)
\end{itemize}

频率响应:
\begin{align*}
G(\jw) = \frac{\omega_n^2}{\omega_n^2 - \omega^2 + 2\jw\zeta\omega_n}
\end{align*}

\textbf{幅频特性:}
\begin{align*}
L(\omega) &= 20\lg\frac{\omega_n^2}{\sqrt{(\omega_n^2-\omega^2)^2 + (2\zeta\omega_n\omega)^2}}
\end{align*}

渐近线:
\begin{itemize}
    \item 当 $\omega \ll \omega_n$:$L(\omega) \approx 0$ dB
    \item 当 $\omega \gg \omega_n$:$L(\omega) \approx -40\lg(\omega/\omega_n)$ dB(斜率 $-40$ dB/十倍频)
    \item 转折频率:$\omega_c = \omega_n$
\end{itemize}

\textbf{谐振峰值(仅当 $\zeta < 0.707$ 时):}
\begin{itemize}
    \item 谐振频率:$\omega_r = \omega_n\sqrt{1-2\zeta^2}$
    \item 谐振峰值:$M_r = \frac{1}{2\zeta\sqrt{1-\zeta^2}}$
    \item 当 $\zeta$ 很小时,谐振峰值很大
\end{itemize}

\textbf{转折频率处的精确值:}
\begin{itemize}
    \item $L(\omega_n) = -20\lg(2\zeta)$ dB
    \item 当 $\zeta = 0.707$ 时,$L(\omega_n) = -3$ dB(无谐振)
    \item 当 $\zeta < 0.707$ 时,$L(\omega_n) > -3$ dB(有谐振)
    \item 当 $\zeta > 0.707$ 时,$L(\omega_n) < -3$ dB(过阻尼)
\end{itemize}

\textbf{相频特性:}
\begin{align*}
\phi(\omega) &= -\arctan\frac{2\zeta\omega_n\omega}{\omega_n^2-\omega^2}
\end{align*}
\begin{itemize}
    \item $\omega = \omega_n$:$\phi = -90°$(与 $\zeta$ 无关)
    \item $\omega \to 0$:$\phi \to 0°$
    \item $\omega \to \infty$:$\phi \to -180°$
    \item $\zeta$ 越小,相位变化越快
\end{itemize}

\textbf{不同阻尼比的伯德图对比:}

\begin{center}
\begin{tikzpicture}
\begin{groupplot}[
    group style={group size=1 by 2, vertical sep=0.5cm},
    width=12cm, height=4cm,
    xmode=log,
    grid=both,
    xlabel={频率 $\omega$ (rad/s)},
    xmin=0.1, xmax=1000
]
\nextgroupplot[ylabel={幅度 (dB)}, legend pos=south west]
\addplot[blue, thick, samples=200, domain=0.1:1000] {20*log10(100/sqrt((100-x^2)^2 + (2*0.1*10*x)^2))};
\addlegendentry{$\zeta=0.1$}
\addplot[red, thick, samples=200, domain=0.1:1000] {20*log10(100/sqrt((100-x^2)^2 + (2*0.5*10*x)^2))};
\addlegendentry{$\zeta=0.5$}
\addplot[green!60!black, thick, samples=200, domain=0.1:1000] {20*log10(100/sqrt((100-x^2)^2 + (2*1.0*10*x)^2))};
\addlegendentry{$\zeta=1.0$}

\nextgroupplot[ylabel={相位 (度)}, legend pos=south west]
\addplot[blue, thick, samples=200, domain=0.1:1000] {deg(-atan2(2*0.1*10*x, 100-x^2))};
\addlegendentry{$\zeta=0.1$}
\addplot[red, thick, samples=200, domain=0.1:1000] {deg(-atan2(2*0.5*10*x, 100-x^2))};
\addlegendentry{$\zeta=0.5$}
\addplot[green!60!black, thick, samples=200, domain=0.1:1000] {deg(-atan2(2*1.0*10*x, 100-x^2))};
\addlegendentry{$\zeta=1.0$}
\end{groupplot}
\end{tikzpicture}
\end{center}

图中所有曲线对应 $\omega_n = 10$ rad/s,不同阻尼比 $\zeta$。

\subsection{伯德图的绘制方法}

\subsubsection{基本步骤}
\begin{enumerate}
    \item \textbf{将传递函数化为标准形式}
    \begin{align*}
    G(s) = \frac{K\prod(1+T_i s)\prod(\frac{s^2}{\omega_{ni}^2}+\frac{2\zeta_i}{\omega_{ni}}s+1)}{s^v\prod(1+T_j s)\prod(\frac{s^2}{\omega_{nj}^2}+\frac{2\zeta_j}{\omega_{nj}}s+1)}
    \end{align*}
    
    \item \textbf{识别各个典型环节}
    \begin{itemize}
        \item 比例环节 $K$
        \item 积分/微分环节 $s^{\pm v}$
        \item 一阶环节 $(1+Ts)^{\pm 1}$
        \item 二阶环节 $(\frac{s^2}{\omega_n^2}+\frac{2\zeta s}{\omega_n}+1)^{\pm 1}$
    \end{itemize}
    
    \item \textbf{确定转折频率}
    \begin{itemize}
        \item 一阶环节:$\omega_c = \frac{1}{T}$
        \item 二阶环节:$\omega_c = \omega_n$
        \item 按从小到大排列所有转折频率
    \end{itemize}
    
    \item \textbf{绘制低频渐近线}
    \begin{itemize}
        \item 起始点:选择 $\omega = 1$ rad/s 或最小转折频率的 $\frac{1}{10}$
        \item 起始斜率:由积分环节数 $v$ 决定($-20v$ dB/十倍频)
        \item 起始幅值:$L(\omega_0) = 20\lg K - 20v\lg\omega_0$
    \end{itemize}
    
    \item \textbf{绘制各段渐近线}
    \begin{itemize}
        \item 每经过一个转折频率,斜率改变
        \item 零点(分子):斜率增加
        \begin{itemize}
            \item 一阶零点:$+20$ dB/十倍频
            \item 二阶零点:$+40$ dB/十倍频
        \end{itemize}
        \item 极点(分母):斜率减少
        \begin{itemize}
            \item 一阶极点:$-20$ dB/十倍频
            \item 二阶极点:$-40$ dB/十倍频
        \end{itemize}
    \end{itemize}
    
    \item \textbf{修正转折频率附近的误差}
    \begin{itemize}
        \item 一阶环节在 $\omega = \omega_c$ 处:$-3$ dB(极点)或 $+3$ dB(零点)
        \item 二阶环节:根据阻尼比 $\zeta$ 修正
    \end{itemize}
    
    \item \textbf{绘制相频特性}
    \begin{itemize}
        \item 每个环节的相位直接相加
        \item 注意各环节的相位贡献
    \end{itemize}
\end{enumerate}

\subsubsection{转折频率处的修正}

\textbf{一阶环节 $\frac{1}{1+Ts}$ 的修正:}

\begin{center}
\begin{tabular}{c|c|c}
\hline
频率 & 渐近线误差 & 精确值修正 \\
\hline
$0.5\omega_c$ & $0$ dB & $-1$ dB \\
$\omega_c$ & $0$ dB & $-3$ dB \\
$2\omega_c$ & $0$ dB & $-1$ dB \\
\hline
\end{tabular}
\end{center}

\textbf{二阶环节的修正}

取决于阻尼比 $\zeta$,在 $\omega = \omega_n$ 处:
\begin{itemize}
    \item $\zeta = 0.1$:谐振峰值约 $+14$ dB
    \item $\zeta = 0.2$:谐振峰值约 $+7$ dB
    \item $\zeta = 0.3$:谐振峰值约 $+3$ dB
    \item $\zeta = 0.5$:误差约 $-1$ dB
    \item $\zeta = 0.707$:误差 $-3$ dB(临界阻尼)
    \item $\zeta = 1.0$:误差 $-6$ dB(过阻尼)
\end{itemize}

\subsubsection{绘图示例}

\textbf{例1:}绘制 $G(s) = \frac{10}{s(1+0.1s)}$ 的伯德图

\textbf{解:}
\begin{enumerate}
    \item 标准形式:$G(s) = \frac{10}{s(1+0.1s)}$,$K = 10$
    \item 环节识别:
    \begin{itemize}
        \item 积分环节:$\frac{1}{s}$($v=1$)
        \item 惯性环节:$\frac{1}{1+0.1s}$,$T=0.1$
    \end{itemize}
    \item 转折频率:$\omega_c = \frac{1}{T} = 10$ rad/s
    \item 绘制幅频特性:
    \begin{itemize}
        \item 选 $\omega = 1$ rad/s 为起点
        \item $L(1) = 20\lg 10 - 20\lg 1 = 20$ dB
        \item $\omega < 10$:斜率 $-20$ dB/十倍频
        \item $\omega > 10$:斜率 $-20-20 = -40$ dB/十倍频
        \item 在 $\omega = 10$ 处修正 $-3$ dB
    \end{itemize}
    \item 绘制相频特性:
    \begin{itemize}
        \item 积分环节贡献:$-90°$
        \item 惯性环节贡献:$-\arctan(0.1\omega)$
        \item 总相位:$\phi(\omega) = -90° - \arctan(0.1\omega)$
        \item $\omega = 10$:$\phi = -90° - 45° = -135°$
    \end{itemize}
\end{enumerate}

\textbf{伯德图:}
\begin{center}
% G(s) = 10/(s(1+0.1s)) -> num=10, den=0.1 s^2 + 1 s + 0
\BodeTF{num/{10},den/{0.1,1,0}}{0.1}{1000}
\end{center}

\textbf{例2:}绘制 $G(s) = \frac{100(s+1)}{s(s+10)}$ 的伯德图

\textbf{解:}
\begin{enumerate}
    \item 标准形式:$G(s) = \frac{10(1+s)}{s(1+0.1s)}$,$K = 10$
    \item 环节识别:
    \begin{itemize}
        \item 积分环节:$\frac{1}{s}$
        \item 一阶零点:$(1+s)$,$\omega_{c1} = 1$ rad/s
        \item 一阶极点:$\frac{1}{1+0.1s}$,$\omega_{c2} = 10$ rad/s
    \end{itemize}
    \item 转折频率:$\omega_{c1} = 1$ rad/s,$\omega_{c2} = 10$ rad/s
    \item 绘制幅频特性:
    \begin{itemize}
        \item 低频段($\omega < 1$):斜率 $-20$ dB/十倍频
        \item 中频段($1 < \omega < 10$):斜率 $-20+20 = 0$ dB/十倍频
        \item 高频段($\omega > 10$):斜率 $0-20 = -20$ dB/十倍频
    \end{itemize}
    \item 相位特性:
    \begin{itemize}
        \item $\phi(\omega) = -90° + \arctan(\omega) - \arctan(0.1\omega)$
    \end{itemize}
\end{enumerate}

\textbf{伯德图:}
\begin{center}
% G(s) = 100(s+1)/(s(s+10)) -> num=100 s + 100, den = s^2 + 10 s + 0
\BodeTF{num/{100,100},den/{1,10,0}}{0.1}{1000}
\end{center}

\textbf{例3:}二阶系统 $G(s) = \frac{100}{s^2 + 2s + 100}$

\textbf{解:}
\begin{enumerate}
    \item 标准形式:$G(s) = \frac{\omega_n^2}{s^2 + 2\zeta\omega_n s + \omega_n^2}$
    \item 参数识别:
    \begin{itemize}
        \item $\omega_n^2 = 100 \Rightarrow \omega_n = 10$ rad/s
        \item $2\zeta\omega_n = 2 \Rightarrow \zeta = 0.1$
        \item 复数极点:$s = -1 \pm j\sqrt{99} \approx -1 \pm 9.95j$
    \end{itemize}
    \item 特性分析:
    \begin{itemize}
        \item 转折频率:$\omega_c = \omega_n = 10$ rad/s
        \item 由于 $\zeta = 0.1 < 0.707$,有谐振峰值
        \item 谐振频率:$\omega_r = \omega_n\sqrt{1-2\zeta^2} \approx 9.9$ rad/s
        \item 谐振峰值:$M_r = \frac{1}{2\zeta\sqrt{1-\zeta^2}} \approx 5.03$(约14 dB)
    \end{itemize}
\end{enumerate}

\textbf{伯德图:}(使用传递函数系数形式)
\begin{center}
\BodeTF{num/{100},den/{1,2,100}}{0.1}{1000}
\end{center}

\subsection{伯德图的应用}

\subsubsection{由伯德图确定传递函数}
根据伯德图的幅频特性,可以反推传递函数:
\begin{enumerate}
    \item 从低频渐近线斜率确定积分环节数 $v$
    \item 从转折频率确定各环节的时间常数
    \item 从低频幅值确定增益 $K$
    \item 从斜率变化确定零极点类型
\end{enumerate}

\subsubsection{稳定性分析}
\textbf{增益裕度(Gain Margin,GM):}
\begin{align*}
\text{GM} = -L(\omega_g) \text{ dB}
\end{align*}
其中 $\omega_g$ 为相位穿越频率($\phi(\omega_g) = -180°$)

\textbf{相位裕度(Phase Margin,PM):}
\begin{align*}
\text{PM} = 180° + \phi(\omega_c)
\end{align*}
其中 $\omega_c$ 为幅值穿越频率($L(\omega_c) = 0$ dB)

\textbf{稳定性判据:}
\begin{itemize}
    \item 稳定条件:GM $> 0$ dB 且 PM $> 0°$
    \item 一般要求:GM $\geq 6$ dB,PM $\geq 30°$
    \item 良好性能:GM $\geq 10$ dB,PM $\geq 45°$
\end{itemize}

\subsubsection{性能指标估算}
\begin{itemize}
    \item \textbf{带宽频率} $\omega_b$:$L(\omega_b) = -3$ dB 对应的频率
    \begin{itemize}
        \item 带宽越大,系统响应越快
    \end{itemize}
    \item \textbf{谐振峰值} $M_r$:幅频特性的最大值
    \begin{itemize}
        \item $M_r$ 越大,超调量越大
        \item 一般要求 $M_r < 1.3$(即 $< 2.3$ dB)
    \end{itemize}
    \item \textbf{低频增益}:反映稳态精度
    \begin{itemize}
        \item 型别越高,低频增益越大,稳态误差越小
    \end{itemize}
    \item \textbf{中频段斜率}:影响稳定性
    \begin{itemize}
        \item 斜率为 $-20$ dB/十倍频时系统一般稳定
        \item 斜率为 $-40$ dB/十倍频时需检查相位裕度
    \end{itemize}
\end{itemize}
