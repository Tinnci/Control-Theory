\section{线性变换}
\label{sec:linear-transformation}

\subsection*{引言:为什么需要更换 \textquotedblleft 坐标系  \textquotedblright ?}

想象你站在一个房间里观察一个旋转的陀螺。如果你从正面看,陀螺的运动可能非常复杂;但如果你沿着陀螺的旋转轴观察,运动就变得简单——只是一个圆周运动。\textbf{这就是线性变换的核心思想:通过选择合适的\textquotedblleft 观察角度 \textquotedblright(坐标系),让复杂的系统变得简单。}

\subsubsection*{为什么状态不唯一?}

在\ref{sec:state-space}节中,我们知道状态空间表达式的形式为 $\dot{x} = Ax + Bu$。但问题是:\textbf{状态变量的选择不是唯一的!}

\begin{itemize}[leftmargin=2em]
    \item \textbf{物理角度}:描述同一个系统,可以选择不同的物理量作为状态(例如位置-速度 vs 动量-能量)
    \item \textbf{数学角度}:即使选择相同的物理量,也可以用不同的坐标系表示(直角坐标 vs 极坐标)
    \item \textbf{工程角度}:某些坐标系下,系统矩阵 $A$ 的结构更简单,便于分析和设计
\end{itemize}

\subsubsection*{线性变换的作用}

线性变换(也称\textbf{坐标变换}或\textbf{相似变换})允许我们在不改变系统本质特性的前提下,\textbf{切换到更方便的坐标系}。关键性质:

\begin{tcolorbox}[colback=blue!5!white, colframe=blue!75!black, title=线性变换的不变量]
无论如何变换坐标系,以下系统特性保持不变:
\begin{itemize}
    \item \textbf{特征值}(系统的固有频率和稳定性)
    \item \textbf{传递函数}(输入输出关系)
    \item \textbf{能控性和能观性}(系统的结构性质)
\end{itemize}
\end{tcolorbox}

\subsubsection*{本章路线图}

\begin{enumerate}
    \item \textbf{数学框架}:线性变换的定义和推导
    \item \textbf{相似变换性质}:哪些量变、哪些量不变
    \item \textbf{实用例题}:如何选择变换矩阵 $P$
    \item \textbf{几何直观}:坐标变换的可视化理解
    \item \textbf{通向标准型}:为\ref{sec:standard-forms}节的标准型做准备
\end{enumerate}

\subsection{线性变换的定义}

设 $x \in \mathbb{R}^n$ 和 $z \in \mathbb{R}^n$ 是描述同一系统的两组状态变量。如果存在\textbf{非奇异矩阵} $P \in \mathbb{R}^{n \times n}$(即 $\det(P) \neq 0$,$P$ 可逆),使得:
\begin{empheq}[box=\fbox]{equation}
x = Pz
\end{empheq}

则称此变换为\textbf{线性变换}(Linear Transformation)或\textbf{坐标变换}(Coordinate Transformation)。

\subsubsection*{关键理解}

\begin{itemize}
    \item $P$ 称为\textbf{变换矩阵},它的列向量构成新坐标系的基
    \item $x$ 是原坐标系下的状态,$z$ 是新坐标系下的状态
    \item 非奇异条件 $\det(P) \neq 0$ 保证变换可逆:$z = P^{-1}x$
    \item 同一个物理系统,不同坐标下的\textquotedblleft 坐标值 \textquotedblright 不同,但描述的是同一个 \textquotedblleft 点 \textquotedblright
\end{itemize}

\subsection{变换后的状态方程}

原系统的状态空间表达式为:
\begin{align}
\dot{x} &= Ax + Bu \label{eq:original-system} \\
y &= Cx + Du \nonumber
\end{align}

应用线性变换 $x = Pz$,求导得 $\dot{x} = P\dot{z}$,代入方程 \eqref{eq:original-system}:
\begin{align*}
P\dot{z} &= APz + Bu \\
\dot{z} &= P^{-1}APz + P^{-1}Bu
\end{align*}

同理,输出方程变为:
\[y = CPz + Du\]

定义变换后的系统矩阵:
\begin{empheq}[box=\fbox]{align}
\bar{A} &= P^{-1}AP \quad \text{(相似变换)} \label{eq:A-transform} \\
\bar{B} &= P^{-1}B \quad \text{(输入矩阵变换)} \nonumber \\
\bar{C} &= CP \quad \text{(输出矩阵变换)} \nonumber
\end{align}

则变换后的系统为:
\begin{align}
\dot{z} &= \bar{A}z + \bar{B}u \\
y &= \bar{C}z + Du
\end{align}

其中:
\begin{align}
\bar{A} &= P^{-1}AP \\
\bar{B} &= P^{-1}B \\
\bar{C} &= CP
\end{align}

\subsection{相似变换的性质}

式 \eqref{eq:A-transform} 中的 $\bar{A} = P^{-1}AP$ 称为矩阵 $A$ 的\textbf{相似变换}。相似矩阵具有以下重要性质:

\subsubsection*{不变量(Invariants)}

\begin{enumerate}
    \item \textbf{特征值不变}:$\det(\lambda I - \bar{A}) = \det(\lambda I - A)$,因此 $A$ 和 $\bar{A}$ 有相同的特征值
    \item \textbf{迹不变}:$\text{tr}(\bar{A}) = \text{tr}(A)$(对角线元素之和)
    \item \textbf{行列式不变}:$\det(\bar{A}) = \det(A)$
    \item \textbf{秩不变}:$\text{rank}(\bar{A}) = \text{rank}(A)$
\end{enumerate}

\subsubsection*{传递函数不变}

根据\ref{sec:transfer-function}节的公式,传递函数为:
\begin{align*}
G(s) &= C(sI - A)^{-1}B + D \\
\bar{G}(s) &= \bar{C}(sI - \bar{A})^{-1}\bar{B} + D \\
&= CP(sI - P^{-1}AP)^{-1}P^{-1}B + D \\
&= CP \cdot P^{-1}(sI - A)^{-1}P \cdot P^{-1}B + D \\
&= C(sI - A)^{-1}B + D = G(s)
\end{align*}

因此,\textbf{线性变换不改变系统的输入输出关系}。

\subsection{如何选择变换矩阵 $P$?}

变换矩阵 $P$ 的选择取决于我们想要达到的目标:

\begin{table}[h]
\centering
\caption{常见的变换矩阵选择策略}
\begin{tabular}{|l|l|l|}
\hline
\rowcolor{gray!20}
\textbf{目标} & \textbf{变换矩阵 $P$} & \textbf{结果 $\bar{A}$} \\
\hline
对角化(最简形式) & 特征向量矩阵 & 对角矩阵 $\Lambda$ \\
\hline
能控标准型 & 能控性矩阵的变换 & 能控标准形式 \\
\hline
能观标准型 & 能观性矩阵的变换 & 能观标准形式 \\
\hline
约当标准型 & 广义特征向量矩阵 & 约当矩阵 $J$ \\
\hline
\end{tabular}
\end{table}

下面通过两个例题说明具体应用。

\subsection{例题1:对角化变换}

\textbf{问题}:给定系统矩阵
\[A = \begin{bmatrix} 0 & 1 \\ -2 & -3 \end{bmatrix}\]
将其对角化。

\textbf{解}:

\textbf{步骤1}:求特征值
\[\det(\lambda I - A) = \det \begin{bmatrix} \lambda & -1 \\ 2 & \lambda + 3 \end{bmatrix} = \lambda(\lambda + 3) + 2 = \lambda^2 + 3\lambda + 2 = (\lambda + 1)(\lambda + 2) = 0\]

特征值:$\lambda_1 = -1$,$\lambda_2 = -2$

\textbf{步骤2}:求特征向量

对于 $\lambda_1 = -1$:
\[(\lambda_1 I - A)v_1 = \begin{bmatrix} -1 & -1 \\ 2 & 2 \end{bmatrix} \begin{bmatrix} v_{11} \\ v_{12} \end{bmatrix} = 0 \quad \Rightarrow \quad v_1 = \begin{bmatrix} 1 \\ -1 \end{bmatrix}\]

对于 $\lambda_2 = -2$:
\[(\lambda_2 I - A)v_2 = \begin{bmatrix} -2 & -1 \\ 2 & 1 \end{bmatrix} \begin{bmatrix} v_{21} \\ v_{22} \end{bmatrix} = 0 \quad \Rightarrow \quad v_2 = \begin{bmatrix} 1 \\ -2 \end{bmatrix}\]

\textbf{步骤3}:构造变换矩阵
\[P = \begin{bmatrix} v_1 & v_2 \end{bmatrix} = \begin{bmatrix} 1 & 1 \\ -1 & -2 \end{bmatrix}, \quad P^{-1} = \begin{bmatrix} 2 & 1 \\ -1 & -1 \end{bmatrix}\]

\textbf{步骤4}:验证对角化
\[\bar{A} = P^{-1}AP = \begin{bmatrix} 2 & 1 \\ -1 & -1 \end{bmatrix} \begin{bmatrix} 0 & 1 \\ -2 & -3 \end{bmatrix} \begin{bmatrix} 1 & 1 \\ -1 & -2 \end{bmatrix} = \begin{bmatrix} -1 & 0 \\ 0 & -2 \end{bmatrix}\]

\textbf{结果}:在新坐标系 $z$ 下,系统矩阵变为对角矩阵,两个状态完全解耦!

\subsection{例题2:传递函数不变性验证}

\textbf{问题}:对于例题1的系统,设 $B = \begin{bmatrix} 0 \\ 1 \end{bmatrix}$,$C = \begin{bmatrix} 1 & 0 \end{bmatrix}$,验证变换前后传递函数相同。

\textbf{解}:

\textbf{原系统传递函数}:
\begin{align*}
G(s) &= C(sI - A)^{-1}B \\
&= \begin{bmatrix} 1 & 0 \end{bmatrix} \begin{bmatrix} s & -1 \\ 2 & s+3 \end{bmatrix}^{-1} \begin{bmatrix} 0 \\ 1 \end{bmatrix} \\
&= \begin{bmatrix} 1 & 0 \end{bmatrix} \frac{1}{s^2 + 3s + 2} \begin{bmatrix} s+3 & 1 \\ -2 & s \end{bmatrix} \begin{bmatrix} 0 \\ 1 \end{bmatrix} \\
&= \frac{1}{s^2 + 3s + 2}
\end{align*}

\textbf{变换后的系统}:
\[\bar{B} = P^{-1}B = \begin{bmatrix} 2 & 1 \\ -1 & -1 \end{bmatrix} \begin{bmatrix} 0 \\ 1 \end{bmatrix} = \begin{bmatrix} 1 \\ -1 \end{bmatrix}, \quad \bar{C} = CP = \begin{bmatrix} 1 & 0 \end{bmatrix} \begin{bmatrix} 1 & 1 \\ -1 & -2 \end{bmatrix} = \begin{bmatrix} 1 & 1 \end{bmatrix}\]

\textbf{新系统传递函数}(利用对角矩阵的简单形式):
\begin{align*}
\bar{G}(s) &= \bar{C}(sI - \bar{A})^{-1}\bar{B} \\
&= \begin{bmatrix} 1 & 1 \end{bmatrix} \begin{bmatrix} s+1 & 0 \\ 0 & s+2 \end{bmatrix}^{-1} \begin{bmatrix} 1 \\ -1 \end{bmatrix} \\
&= \begin{bmatrix} 1 & 1 \end{bmatrix} \begin{bmatrix} \frac{1}{s+1} & 0 \\ 0 & \frac{1}{s+2} \end{bmatrix} \begin{bmatrix} 1 \\ -1 \end{bmatrix} \\
&= \frac{1}{s+1} - \frac{1}{s+2} = \frac{(s+2) - (s+1)}{(s+1)(s+2)} = \frac{1}{s^2 + 3s + 2}
\end{align*}

\textbf{验证}:$G(s) = \bar{G}(s)$ ✓

\subsection{MATLAB实现}

\begin{lstlisting}[style=Matlab-editor, caption=线性变换的MATLAB实现]
% 定义原系统
A = [0 1; -2 -3];
B = [0; 1];
C = [1 0];
D = 0;

% 方法1:使用特征向量对角化
[P, Lambda] = eig(A);  % P是特征向量矩阵,Lambda是对角矩阵
A_bar = inv(P) * A * P;  % 应该等于Lambda
B_bar = inv(P) * B;
C_bar = C * P;

% 方法2:验证传递函数不变
sys_original = ss(A, B, C, D);
sys_transformed = ss(A_bar, B_bar, C_bar, D);
[num1, den1] = ss2tf(A, B, C, D);
[num2, den2] = ss2tf(A_bar, B_bar, C_bar, D);
fprintf('原系统传递函数: '); tf(num1, den1)
fprintf('变换后传递函数: '); tf(num2, den2)

% 方法3:验证特征值不变
eig_A = eig(A);
eig_A_bar = eig(A_bar);
fprintf('特征值对比:\n原矩阵: %.4f, %.4f\n变换后: %.4f, %.4f\n', ...
    eig_A(1), eig_A(2), eig_A_bar(1), eig_A_bar(2));
\end{lstlisting}

\subsection*{本章小结}

\subsubsection*{核心要点}

\begin{tcolorbox}[colback=green!5!white, colframe=green!75!black, title=线性变换的核心公式]
\textbf{变换关系}:$x = Pz$($P$ 非奇异)

\textbf{系统矩阵变换}:
\begin{align*}
\bar{A} &= P^{-1}AP \quad \text{(相似变换)} \\
\bar{B} &= P^{-1}B, \quad \bar{C} = CP
\end{align*}

\textbf{不变量}:特征值、迹、行列式、传递函数、能控性、能观性
\end{tcolorbox}

\subsubsection*{常见变换类型对比}

\begin{table}[h]
\centering
\caption{常见线性变换及其应用}
\begin{tabular}{|l|l|l|l|}
\hline
\rowcolor{gray!20}
\textbf{变换类型} & \textbf{$P$ 的构成} & \textbf{$\bar{A}$ 的形式} & \textbf{主要用途} \\
\hline
对角化 & 特征向量 & 对角矩阵 & 系统解耦、稳定性分析 \\
\hline
能控标准型 & 能控矩阵变换 & 伴随矩阵 & 极点配置设计 \\
\hline
能观标准型 & 能观矩阵变换 & 转置伴随矩阵 & 状态观测器设计 \\
\hline
约当标准型 & 广义特征向量 & 约当块 & 处理重特征值情况 \\
\hline
\end{tabular}
\end{table}

\subsubsection*{与其他章节的联系}

\begin{itemize}
    \item \textbf{向后链接}:
    \begin{itemize}
        \item \ref{sec:state-space}节:状态空间表达式的基础
        \item \ref{sec:transfer-function}节:传递函数不变性的应用
    \end{itemize}
    \item \textbf{向前链接}:
    \begin{itemize}
        \item \ref{sec:standard-forms}节:利用线性变换得到各种标准型
        \item \ref{sec:jordan-form}节:重特征值情况下的约当标准型
        \item \ref{sec:controllability-observability}节:能控性和能观性在变换下的不变性
    \end{itemize}
\end{itemize}

\subsubsection*{学习检查清单}

\begin{itemize}
    \item[$\square$] 理解线性变换的几何意义(坐标系旋转/缩放)
    \item[$\square$] 掌握相似变换的计算:$\bar{A} = P^{-1}AP$
    \item[$\square$] 能够求解特征向量并构造对角化矩阵
    \item[$\square$] 理解哪些量在变换下不变(特征值、传递函数)
    \item[$\square$] 知道如何根据目标选择合适的变换矩阵
\end{itemize}

