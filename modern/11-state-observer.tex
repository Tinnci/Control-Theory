\section{状态观测器}

\subsection{状态观测器的概念}
当系统的状态不能直接测量时,需要根据系统的输入输出信息来估计状态变量,这种估计装置称为状态观测器。

\subsection{全维状态观测器}
全维状态观测器的方程:
\[\dot{\hat{x}} = A\hat{x} + Bu + L(y - C\hat{x})\]

其中 $\hat{x}$ 为状态估计值,$L$ 为观测器增益矩阵。

\subsection{观测器的设计}
观测误差:$e = x - \hat{x}$

观测误差动态方程:
\[\dot{e} = (A - LC)e\]

观测器设计就是选择 $L$,使得 $A - LC$ 的特征值位于左半平面。

\textbf{观测器设计定理}:若 $(A, C)$ 完全能观测,则可任意配置观测器的极点。

\subsection{分离定理}
状态反馈与状态观测器可以分别独立设计,即:
\begin{itemize}
    \item 先设计状态反馈增益 $K$,配置闭环系统的极点
    \item 再设计观测器增益 $L$,配置观测器的极点
\end{itemize}

基于观测器的状态反馈系统的特征多项式等于控制器特征多项式与观测器特征多项式的乘积。
