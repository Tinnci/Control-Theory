\section{线性控制系统状态空间表达式的求解}
\label{sec:solving-state-space}

\subsection*{引言:从 \textquotedblleft 描述系统 \textquotedblright 到 \textquotedblleft 预测行为 \textquotedblright }

想象你是一名航天工程师,面对火箭发射的状态方程 $\dot{x} = Ax + Bu$。仅仅知道这个方程是不够的——\textbf{你需要知道:给定初始状态 $x(0)$ 和控制输入 $u(t)$,未来任意时刻 $t$ 的状态 $x(t)$ 是什么?}这就是本章的核心任务。

\subsubsection*{为什么要求解状态方程?}

在\ref{sec:state-space}节中,我们学会了\textbf{建立}状态空间模型;现在我们要学会\textbf{求解}它。求解的目的:

\begin{itemize}[leftmargin=2em]
    \item \textbf{时域响应预测}:已知初始状态和输入,计算系统的完整轨迹 $x(t)$
    \item \textbf{稳定性分析}:观察状态是否随时间收敛到零(或有界)
    \item \textbf{性能评估}:计算超调量、调节时间等指标
    \item \textbf{控制器设计验证}:检验设计的控制律是否达到期望效果
\end{itemize}

\subsubsection*{求解的挑战}

状态方程 $\dot{x} = Ax + Bu$ 看似简单,但实际求解有两大难点:

\begin{enumerate}
    \item \textbf{矩阵指数 $e^{At}$}:不是简单的标量指数,而是矩阵的无穷级数
    \item \textbf{时变积分}:输入项 $\int_0^t e^{A(t-\tau)}Bu(\tau)d\tau$ 涉及矩阵和时间的耦合
\end{enumerate}

好在我们有\textbf{三种强大的方法}来应对这些挑战。

\subsubsection*{本章路线图}

\begin{enumerate}
    \item \textbf{状态转移矩阵}:核心概念 $\Phi(t) = e^{At}$ 及其性质
    \item \textbf{求解方法对比}:拉普拉斯变换法 vs 矩阵指数法 vs 对角化法
    \item \textbf{齐次方程求解}:初始状态响应(自由运动)
    \item \textbf{非齐次方程求解}:完全响应(自由运动+强迫运动)
    \item \textbf{MATLAB实现}:三种方法的代码实现和效率对比
\end{enumerate}

\subsection{状态转移矩阵}

\subsubsection*{核心概念}

对于齐次状态方程(无外部输入):
\begin{empheq}[box=\fbox]{equation}
\dot{x} = Ax, \quad x(0) = x_0
\end{equation}

其解为:
\begin{empheq}[box=\fbox]{equation}
x(t) = e^{At}x_0 = \Phi(t)x_0
\end{equation}

其中 $\Phi(t) = e^{At}$ 称为\textbf{状态转移矩阵}(State Transition Matrix)。

\subsubsection*{物理意义}

状态转移矩阵 $\Phi(t)$ 描述了\textbf{系统从 $t=0$ 时刻的状态如何演化到 $t$ 时刻}。可以理解为:

\begin{tcolorbox}[colback=blue!5!white, colframe=blue!75!black, title=状态转移矩阵的直观理解]
$\Phi(t)$ 是一个 \textquotedblleft 时间旅行算子 \textquotedblright:
\begin{itemize}
    \item 输入:初始状态 $x(0)$(现在)
    \item 输出:未来状态 $x(t)$($t$ 秒后)
    \item 作用:$x(t) = \Phi(t) \cdot x(0)$
\end{itemize}

例如,$\Phi(2)$ 告诉你:\textquotedblleft 2秒后的状态是初始状态的什么变换\textquotedblright。
\end{tcolorbox}

\subsubsection*{状态转移矩阵的性质}

$\Phi(t)$ 具有以下重要性质(证明略):

\begin{enumerate}
    \item \textbf{初始条件}:$\Phi(0) = I$(零时刻无变化)
    \item \textbf{半群性质}:$\Phi(t_1 + t_2) = \Phi(t_1)\Phi(t_2)$(时间可叠加)
    \item \textbf{可逆性}:$\Phi^{-1}(t) = \Phi(-t)$(时间可逆转)
    \item \textbf{微分性质}:$\frac{d\Phi(t)}{dt} = A\Phi(t) = \Phi(t)A$(满足原方程)
\end{enumerate}

\subsection{方法1:拉普拉斯变换法}

\subsubsection*{基本思路}

将时域的微分方程转换到复频域($s$ 域),利用代数方法求解,再反变换回时域。

\subsubsection*{求解步骤}

对 $\dot{x} = Ax$ 两边做拉普拉斯变换:
\begin{align*}
sX(s) - x(0) &= AX(s) \\
(sI - A)X(s) &= x(0) \\
X(s) &= (sI - A)^{-1}x(0)
\end{align*}

反变换得时域解:
\begin{empheq}[box=\fbox]{equation}
x(t) = \mathcal{L}^{-1}[(sI - A)^{-1}]x(0)
\end{equation}

因此,状态转移矩阵为:
\begin{empheq}[box=\fbox]{equation}
\Phi(t) = e^{At} = \mathcal{L}^{-1}[(sI - A)^{-1}]
\end{equation}

\subsubsection*{例题1:拉普拉斯法求解2×2系统}

\textbf{问题}:求解系统 $\dot{x} = Ax$,其中
\[A = \begin{bmatrix} 0 & 1 \\ -2 & -3 \end{bmatrix}, \quad x(0) = \begin{bmatrix} 1 \\ 0 \end{bmatrix}\]

\textbf{解}:

\textbf{步骤1}:计算 $(sI - A)$
\[sI - A = \begin{bmatrix} s & -1 \\ 2 & s+3 \end{bmatrix}\]

\textbf{步骤2}:求逆矩阵
\begin{align*}
(sI - A)^{-1} &= \frac{1}{\det(sI - A)} \text{adj}(sI - A) \\
&= \frac{1}{s(s+3) + 2} \begin{bmatrix} s+3 & 1 \\ -2 & s \end{bmatrix} \\
&= \frac{1}{s^2 + 3s + 2} \begin{bmatrix} s+3 & 1 \\ -2 & s \end{bmatrix} \\
&= \frac{1}{(s+1)(s+2)} \begin{bmatrix} s+3 & 1 \\ -2 & s \end{bmatrix}
\end{align*}

\textbf{步骤3}:部分分式展开(以第一行第一列元素为例)
\[\frac{s+3}{(s+1)(s+2)} = \frac{A}{s+1} + \frac{B}{s+2}\]

求解得 $A = 2$,$B = -1$,因此:
\[\frac{s+3}{(s+1)(s+2)} = \frac{2}{s+1} - \frac{1}{s+2}\]

对所有元素类似处理后:
\[(sI - A)^{-1} = \begin{bmatrix} \frac{2}{s+1} - \frac{1}{s+2} & \frac{1}{s+1} - \frac{1}{s+2} \\ \frac{-2}{s+1} + \frac{2}{s+2} & \frac{-1}{s+1} + \frac{2}{s+2} \end{bmatrix}\]

\textbf{步骤4}:反拉普拉斯变换
\[\Phi(t) = e^{At} = \begin{bmatrix} 2e^{-t} - e^{-2t} & e^{-t} - e^{-2t} \\ -2e^{-t} + 2e^{-2t} & -e^{-t} + 2e^{-2t} \end{bmatrix}\]

\textbf{步骤5}:计算状态响应
\begin{align*}
x(t) &= \Phi(t)x(0) = \begin{bmatrix} 2e^{-t} - e^{-2t} & e^{-t} - e^{-2t} \\ -2e^{-t} + 2e^{-2t} & -e^{-t} + 2e^{-2t} \end{bmatrix} \begin{bmatrix} 1 \\ 0 \end{bmatrix} \\
&= \begin{bmatrix} 2e^{-t} - e^{-2t} \\ -2e^{-t} + 2e^{-2t} \end{bmatrix}
\end{align*}

\subsection{方法2:矩阵指数级数展开法}

\subsubsection*{基本思路}

类比标量指数函数的泰勒展开 $e^{at} = 1 + at + \frac{a^2t^2}{2!} + \cdots$,定义矩阵指数:

\begin{empheq}[box=\fbox]{equation}
e^{At} = I + At + \frac{A^2t^2}{2!} + \frac{A^3t^3}{3!} + \cdots = \sum_{k=0}^{\infty} \frac{A^k t^k}{k!}
\end{equation}

\subsubsection*{适用场景}

当 $A$ 的幂次有规律(例如幂零矩阵、$A^n = 0$)时,级数会在有限项截断,计算简便。

\subsubsection*{例题2:幂零矩阵的矩阵指数}

\textbf{问题}:求解 $e^{At}$,其中
\[A = \begin{bmatrix} 0 & 1 & 0 \\ 0 & 0 & 1 \\ 0 & 0 & 0 \end{bmatrix}\]

\textbf{解}:

\textbf{步骤1}:计算 $A$ 的幂次
\[A^2 = \begin{bmatrix} 0 & 0 & 1 \\ 0 & 0 & 0 \\ 0 & 0 & 0 \end{bmatrix}, \quad A^3 = 0\]

\textbf{步骤2}:级数截断
由于 $A^3 = 0$,级数只有前三项:
\begin{align*}
e^{At} &= I + At + \frac{A^2t^2}{2!} \\
&= \begin{bmatrix} 1 & 0 & 0 \\ 0 & 1 & 0 \\ 0 & 0 & 1 \end{bmatrix} + \begin{bmatrix} 0 & t & 0 \\ 0 & 0 & t \\ 0 & 0 & 0 \end{bmatrix} + \begin{bmatrix} 0 & 0 & \frac{t^2}{2} \\ 0 & 0 & 0 \\ 0 & 0 & 0 \end{bmatrix} \\
&= \begin{bmatrix} 1 & t & \frac{t^2}{2} \\ 0 & 1 & t \\ 0 & 0 & 1 \end{bmatrix}
\end{align*}

\textbf{物理意义}:这是一个三重积分器系统(如位置、速度、加速度),状态转移矩阵恰好对应运动学方程!

\subsection{方法3:对角化法(最常用)}

\subsubsection*{基本思路}

利用\ref{sec:linear-transformation}节的线性变换,将 $A$ 对角化为 $\Lambda = P^{-1}AP$,则:

\begin{empheq}[box=\fbox]{equation}
e^{At} = P e^{\Lambda t} P^{-1}
\end{equation}

其中对角矩阵的指数非常简单:
\[e^{\Lambda t} = \begin{bmatrix} e^{\lambda_1 t} & 0 & \cdots & 0 \\ 0 & e^{\lambda_2 t} & \cdots & 0 \\ \vdots & \vdots & \ddots & \vdots \\ 0 & 0 & \cdots & e^{\lambda_n t} \end{bmatrix}\]

\subsubsection*{例题3:对角化法求解(延续例题1)}

\textbf{问题}:用对角化法重新求解例题1的 $e^{At}$。

\textbf{解}:

\textbf{步骤1}:对角化(参考\ref{sec:linear-transformation}节例题1)
\[A = \begin{bmatrix} 0 & 1 \\ -2 & -3 \end{bmatrix} = P\Lambda P^{-1}\]

其中:
\[P = \begin{bmatrix} 1 & 1 \\ -1 & -2 \end{bmatrix}, \quad \Lambda = \begin{bmatrix} -1 & 0 \\ 0 & -2 \end{bmatrix}, \quad P^{-1} = \begin{bmatrix} 2 & 1 \\ -1 & -1 \end{bmatrix}\]

\textbf{步骤2}:计算对角矩阵的指数
\[e^{\Lambda t} = \begin{bmatrix} e^{-t} & 0 \\ 0 & e^{-2t} \end{bmatrix}\]

\textbf{步骤3}:合成状态转移矩阵
\begin{align*}
e^{At} &= P e^{\Lambda t} P^{-1} \\
&= \begin{bmatrix} 1 & 1 \\ -1 & -2 \end{bmatrix} \begin{bmatrix} e^{-t} & 0 \\ 0 & e^{-2t} \end{bmatrix} \begin{bmatrix} 2 & 1 \\ -1 & -1 \end{bmatrix} \\
&= \begin{bmatrix} e^{-t} & e^{-2t} \\ -e^{-t} & -2e^{-2t} \end{bmatrix} \begin{bmatrix} 2 & 1 \\ -1 & -1 \end{bmatrix} \\
&= \begin{bmatrix} 2e^{-t} - e^{-2t} & e^{-t} - e^{-2t} \\ -2e^{-t} + 2e^{-2t} & -e^{-t} + 2e^{-2t} \end{bmatrix}
\end{align*}

\textbf{验证}:与例题1拉普拉斯法的结果完全一致!✓

\subsection{非齐次状态方程的完全解}

对于完整的状态方程(含输入):
\begin{empheq}[box=\fbox]{equation}
\dot{x} = Ax + Bu, \quad x(0) = x_0
\end{equation}

其\textbf{完全解}由两部分组成:

\begin{empheq}[box=\fbox]{equation}
x(t) = \underbrace{e^{At}x_0}_{\text{自由响应}} + \underbrace{\int_0^t e^{A(t-\tau)}Bu(\tau)d\tau}_{\text{强迫响应}}
\end{equation}

\subsubsection*{物理解释}

\begin{itemize}
    \item \textbf{自由响应}(零输入响应):系统在无外部输入时,仅由初始状态引起的运动
    \item \textbf{强迫响应}(零状态响应):系统在零初始状态下,由外部输入 $u(t)$ 驱动的运动
\end{itemize}

\subsubsection*{例题4:单位阶跃输入的完全响应}

\textbf{问题}:对于系统 $A = \begin{bmatrix} 0 & 1 \\ -2 & -3 \end{bmatrix}$,$B = \begin{bmatrix} 0 \\ 1 \end{bmatrix}$,$x(0) = \begin{bmatrix} 0 \\ 0 \end{bmatrix}$,输入 $u(t) = 1$(单位阶跃),求 $x(t)$。

\textbf{解}:

由于 $x(0) = 0$,只需计算强迫响应:
\begin{align*}
x(t) &= \int_0^t e^{A(t-\tau)}B \cdot 1 \, d\tau \\
&= \int_0^t \begin{bmatrix} 2e^{-(t-\tau)} - e^{-2(t-\tau)} & e^{-(t-\tau)} - e^{-2(t-\tau)} \\ -2e^{-(t-\tau)} + 2e^{-2(t-\tau)} & -e^{-(t-\tau)} + 2e^{-2(t-\tau)} \end{bmatrix} \begin{bmatrix} 0 \\ 1 \end{bmatrix} d\tau
\end{align*}

积分后(过程略):
\[x(t) = \begin{bmatrix} \frac{1}{2} - 2e^{-t} + \frac{3}{2}e^{-2t} \\ -1 + 2e^{-t} - e^{-2t} \end{bmatrix}\]

稳态值:$x(\infty) = \begin{bmatrix} 1/2 \\ -1 \end{bmatrix}$

\subsection{三种方法的对比}

\begin{table}[h]
\centering
\caption{状态转移矩阵计算方法对比}
\begin{tabular}{|l|l|l|l|}
\hline
\rowcolor{gray!20}
\textbf{方法} & \textbf{适用条件} & \textbf{计算复杂度} & \textbf{优缺点} \\
\hline
拉普拉斯变换法 & 通用 & 中等(需要部分分式) & 系统化,适合手算 \\
\hline
级数展开法 & $A$ 幂零或低阶 & 低(有限项) & 简单但适用范围窄 \\
\hline
对角化法 & $A$ 可对角化 & 低(利用特征值) & \textbf{最常用},效率高 \\
\hline
约当标准型法 & $A$ 不可对角化 & 高(需求广义特征向量) & 处理重特征值 \\
\hline
\end{tabular}
\end{table}

\subsection{MATLAB实现}

\begin{lstlisting}[style=Matlab-editor, caption=状态转移矩阵的三种计算方法]
% 定义系统矩阵
A = [0 1; -2 -3];
B = [0; 1];
x0 = [1; 0];
t = 0:0.01:5;  % 时间向量

% 方法1:MATLAB内置函数 expm(最精确)
Phi_expm = @(t) expm(A*t);
x_expm = zeros(2, length(t));
for i = 1:length(t)
    x_expm(:,i) = Phi_expm(t(i)) * x0;
end

% 方法2:对角化法(手动实现)
[P, Lambda] = eig(A);
Phi_diag = @(t) P * diag(exp(diag(Lambda)*t)) * inv(P);
x_diag = zeros(2, length(t));
for i = 1:length(t)
    x_diag(:,i) = Phi_diag(t(i)) * x0;
end

% 方法3:使用initial函数(最方便)
sys = ss(A, B, eye(2), zeros(2,1));
[y_initial, t_initial, x_initial] = initial(sys, x0, t);

% 绘图对比
figure;
subplot(2,1,1);
plot(t, x_expm(1,:), 'r-', t, x_diag(1,:), 'b--', t_initial, x_initial(:,1), 'go');
legend('expm法', '对角化法', 'initial函数');
ylabel('x_1(t)');
title('状态响应对比');

subplot(2,1,2);
plot(t, x_expm(2,:), 'r-', t, x_diag(2,:), 'b--', t_initial, x_initial(:,2), 'go');
legend('expm法', '对角化法', 'initial函数');
ylabel('x_2(t)');
xlabel('时间 (s)');

% 验证状态转移矩阵在t=1时刻
fprintf('t=1时的状态转移矩阵:\n');
Phi_1 = expm(A*1)
\end{lstlisting}

\begin{lstlisting}[style=Matlab-editor, caption=非齐次方程求解(阶跃输入)]
% 阶跃响应(零初始状态)
sys = ss(A, B, [1 0], 0);  % 观测x1
[y_step, t_step, x_step] = step(sys, t);

% 绘制阶跃响应
figure;
subplot(2,1,1);
plot(t_step, x_step(:,1), 'b-', 'LineWidth', 1.5);
ylabel('x_1(t)');
title('单位阶跃响应(零初始状态)');
grid on;

subplot(2,1,2);
plot(t_step, x_step(:,2), 'r-', 'LineWidth', 1.5);
ylabel('x_2(t)');
xlabel('时间 (s)');
grid on;

% 输出稳态值
fprintf('稳态值: x1_ss = %.4f, x2_ss = %.4f\n', x_step(end,1), x_step(end,2));
\end{lstlisting}

\subsection*{本章小结}

\subsubsection*{核心公式}

\begin{tcolorbox}[colback=green!5!white, colframe=green!75!black, title=状态方程求解的核心公式]
\textbf{状态转移矩阵}:$\Phi(t) = e^{At}$

\textbf{计算方法}:
\begin{itemize}
    \item 拉普拉斯法:$e^{At} = \mathcal{L}^{-1}[(sI-A)^{-1}]$
    \item 级数法:$e^{At} = I + At + \frac{A^2t^2}{2!} + \cdots$
    \item 对角化法:$e^{At} = P e^{\Lambda t} P^{-1}$(\textbf{最常用})
\end{itemize}

\textbf{完全解}:
\[x(t) = e^{At}x_0 + \int_0^t e^{A(t-\tau)}Bu(\tau)d\tau\]
\end{tcolorbox}

\subsubsection*{方法选择指南}

\begin{itemize}
    \item \textbf{手算简单系统}:拉普拉斯变换法(系统化,步骤清晰)
    \item \textbf{特殊结构矩阵}:级数展开法(如幂零矩阵、上三角矩阵)
    \item \textbf{一般系统}:对角化法(效率高,适合编程)
    \item \textbf{重特征值}:约当标准型法(参考\ref{sec:jordan-form}节)
    \item \textbf{MATLAB仿真}:直接使用 \texttt{expm()}、\texttt{initial()}、\texttt{step()} 函数
\end{itemize}

\subsubsection*{常见误区}

\begin{itemize}
    \item ❌ \textbf{误区1}:$e^{At} \neq \begin{bmatrix} e^{a_{11}t} & e^{a_{12}t} \\ e^{a_{21}t} & e^{a_{22}t} \end{bmatrix}$(矩阵指数不是元素指数)
    \item ❌ \textbf{误区2}:忘记对角化的前提条件($A$ 必须有 $n$ 个线性无关的特征向量)
    \item ❌ \textbf{误区3}:积分上下限错误(强迫响应从0到$t$积分,不是从$-\infty$)
\end{itemize}

\subsubsection*{与其他章节的联系}

\begin{itemize}
    \item \textbf{向后链接}:
    \begin{itemize}
        \item \ref{sec:state-space}节:状态空间表达式的建立
        \item \ref{sec:linear-transformation}节:对角化法的理论基础
        \item \ref{sec:transfer-function}节:$(sI-A)^{-1}$ 的应用
    \end{itemize}
    \item \textbf{向前链接}:
    \begin{itemize}
        \item \ref{sec:lyapunov-stability}节:$e^{At} \to 0$ 判定稳定性
        \item \ref{sec:pole-placement}节:通过 $x(t)$ 验证极点配置效果
        \item \ref{sec:jordan-form}节:不可对角化时的求解方法
    \end{itemize}
\end{itemize}

\subsubsection*{学习检查清单}

\begin{itemize}
    \item[$\square$] 理解状态转移矩阵的物理意义(时间演化算子)
    \item[$\square$] 掌握拉普拉斯变换法的完整步骤($(sI-A)^{-1}$、部分分式、反变换)
    \item[$\square$] 会用对角化法快速计算 $e^{At}$(特征值、特征向量、$Pe^{\Lambda t}P^{-1}$)
    \item[$\square$] 理解完全解的两部分(自由响应+强迫响应)
    \item[$\square$] 能够使用MATLAB的 \texttt{expm()}、\texttt{initial()}、\texttt{step()} 函数
    \item[$\square$] 知道三种方法的适用场景和计算复杂度对比
\end{itemize}