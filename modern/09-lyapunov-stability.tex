\section{稳定性与李雅普诺夫方法}
\label{sec:lyapunov-stability}

\subsection*{引言:为什么研究稳定性?}

在控制系统设计中,我们最关心的问题之一就是:\textbf{系统受到扰动后,能否恢复到期望的工作状态?}

想象一个简单的场景:一个倒立摆系统在竖直位置保持平衡。如果轻轻推它一下,它会倒下还是回到竖直位置?这就是\textbf{稳定性}问题的核心。对于飞机、火箭、机器人等复杂系统,稳定性更是关系到安全和性能的关键指标。

传统的频域方法(如奈奎斯特判据、伯德图)虽然强大,但\textbf{仅适用于线性系统}。面对实际工程中大量存在的\textbf{非线性系统}——如饱和、死区、摩擦等非线性特性,我们需要更通用的工具。

19世纪末,俄国数学家\textbf{李雅普诺夫(Lyapunov)}提出了一套革命性的稳定性分析方法,\textbf{无需求解微分方程},仅通过构造一个「能量函数」就能判断系统稳定性。这套方法不仅适用于线性系统,更能处理复杂的非线性系统,成为现代控制理论的基石。

本章将系统介绍李雅普诺夫稳定性理论,包括:
\begin{itemize}
    \item \textbf{稳定性的严格数学定义}(什么是稳定、渐近稳定、全局稳定)
    \item \textbf{李雅普诺夫第一方法}(线性化方法:简单但有局限)
    \item \textbf{李雅普诺夫第二方法}(直接方法:强大但需要技巧)
    \item \textbf{如何区分局部稳定与全局稳定}(这是非线性系统的关键)
    \item \textbf{实际应用范例}(从理论到实践)
\end{itemize}

\subsection{李雅普诺夫稳定性定义}

\subsubsection*{本节目的}
在开始分析方法之前,我们需要严格定义「稳定」的含义。日常语言中的「稳定」过于模糊,数学上需要精确的量化标准。

考虑自治系统 $\dot{x} = f(x)$,设 $x_e$ 为平衡点:
\begin{itemize}
    \item \textbf{稳定}:对任意 $\varepsilon > 0$,存在 $\delta > 0$,当 $\|x(0) - x_e\| < \delta$ 时,有 $\|x(t) - x_e\| < \varepsilon$,$\forall t \geq 0$
    \item \textbf{渐近稳定}:稳定且 $\lim_{t \to \infty} x(t) = x_e$
    \item \textbf{大范围渐近稳定}:渐近稳定且吸引域为整个状态空间
\end{itemize}

\textbf{直观理解:}
\begin{itemize}
    \item \textbf{稳定}意味着「小的扰动只产生小的偏差」——系统不会失控,但也不一定回到原点
    \item \textbf{渐近稳定}更强,不仅不失控,而且「最终会回到平衡点」
    \item \textbf{大范围渐近稳定}最强,「无论初始扰动多大,都能回到平衡点」
\end{itemize}

这三个层次的区别非常重要,后续分析会频繁用到。

\subsection{李雅普诺夫第一方法(线性化方法)}

\subsubsection*{本节目的}
对于非线性系统,直接分析往往很困难。李雅普诺夫第一方法提供了一个\textbf{捷径}:在平衡点附近用线性系统近似,然后利用我们熟悉的线性系统理论来判断稳定性。

\subsubsection{基本原理}

对于非线性系统 $\dot{\mathbf{x}} = \mathbf{f}(\mathbf{x})$,在平衡点 $\mathbf{x}_e$ 附近,可以用线性系统近似:
\[\dot{\mathbf{z}} = A\mathbf{z}\]

其中 $A$ 是\textbf{雅可比矩阵}(Jacobian Matrix):
\[A = \left.\frac{\partial \mathbf{f}}{\partial \mathbf{x}}\right|_{\mathbf{x}=\mathbf{x}_e}\]

\textbf{稳定性判据:}

\begin{itemize}
    \item 如果 $A$ 的所有特征值实部\textbf{都严格小于零},则平衡点\textbf{局部渐近稳定}
    \item 如果 $A$ 至少有一个特征值实部\textbf{大于零},则平衡点\textbf{不稳定}
    \item 如果 $A$ 有特征值实部\textbf{等于零}(且无实部大于零的特征值),则方法\textbf{失效},无法判断
\end{itemize}

对于线性系统 $\dot{x} = Ax$,系统渐近稳定的充要条件是矩阵 $A$ 的所有特征值都具有负实部。

\textbf{第一方法的优缺点总结:}
\begin{itemize}
    \item \textbf{优点}:计算简单,步骤明确(求雅可比矩阵→算特征值→判断符号)
    \item \textbf{缺点}:只能得到\textbf{局部稳定性}结论;特征值实部为零时方法失效
    \item \textbf{适用场景}:作为初步快速判断,或者局部稳定性分析已经足够的情况
\end{itemize}

\subsection{李雅普诺夫第二方法(直接方法)}

\subsubsection*{本节目的}
李雅普诺夫第二方法是更强大、更通用的工具。它的核心思想是:\textbf{借鉴物理学中的能量概念}——如果能证明系统的「能量」总是递减,系统必然稳定。

\textbf{李雅普诺夫定理}:如果存在标量函数 $V(x)$ 满足:
\begin{enumerate}
    \item $V(x)$ 连续且有连续的一阶偏导数
    \item $V(x_e) = 0$,当 $x \neq x_e$ 时 $V(x) > 0$(正定)
    \item $\dot{V}(x) = \frac{\partial V}{\partial x} f(x) \leq 0$(半负定)
\end{enumerate}
则平衡点 $x_e$ 稳定。

若进一步有 $\dot{V}(x) < 0$(负定),则平衡点渐近稳定。

\textbf{定理的直观意义:}这个定理告诉我们,\textbf{不需要求解微分方程},只要能找到一个满足条件的函数 $V(x)$(称为李雅普诺夫函数),就能判断稳定性。这就像物理学家通过能量守恒定律判断运动趋势,而不必求解复杂的运动方程。

\subsection{李雅普诺夫方法的核心思想}

\subsubsection*{本节目的}
上一节给出了定理,但可能让人感到抽象。本节将深入解释\textbf{为什么}这个方法有效,以及\textbf{如何}在实际问题中应用它。

\subsubsection{直观理解}
李雅普诺夫方法的思想非常直观:\textbf{如果能为系统找到一个类似于「能量」的函数 $V(x)$,并且证明这个「能量」总是随时间减少(或至少不增加),那么系统最终必然会稳定下来。}

\begin{itemize}
    \item 这个函数 $V(x)$ 称为\textbf{李雅普诺夫函数}
    \item $V(x)$ 必须是\textbf{正定的}(类似能量总是正的)
    \item $V(x)$ 的导数 $\dot{V}(x)$ 必须是\textbf{负定的或半负定的}(能量随时间衰减或保持不变)
\end{itemize}

\subsubsection{线性系统的二次型李雅普诺夫函数}

对于线性系统 $\dot{x} = Ax$,我们选择二次型的李雅普诺夫函数 $V(x) = x^T P x$,其中 $P$ 是对称正定矩阵。

对 $V(x)$ 求导:
\begin{align*}
\dot{V}(x) &= \frac{d}{dt}(x^T P x) \\
&= \dot{x}^T P x + x^T P \dot{x} \\
&= (Ax)^T P x + x^T P (Ax) \\
&= x^T A^T P x + x^T P A x \\
&= x^T(A^T P + PA)x
\end{align*}

为了让 $\dot{V}(x)$ 是负定的,通常令 $\dot{V}(x) = -x^T Q x$,其中 $Q$ 是我们选择的对称正定矩阵。因此得到:
\[\boxed{A^T P + PA = -Q}\]

这就是著名的\textbf{李雅普诺夫方程}。

\subsubsection{判据总结}

对于给定的系统矩阵 $A$:
\begin{itemize}
    \item 如果对\textbf{任意}正定矩阵 $Q$,李雅普诺夫方程都有唯一的\textbf{正定解} $P$,则系统\textbf{渐近稳定}
    \item 如果对某个正定矩阵 $Q$,方程无解或解不是正定的,则系统\textbf{不是渐近稳定的}
    \item 如果能找到半正定的 $Q$ 使得存在正定的 $P$,但对所有正定 $Q$ 都无解,则系统可能是\textbf{李雅普诺夫稳定}(临界稳定)
\end{itemize}

\subsection{线性系统的李雅普诺夫方程}

\subsubsection*{本节目的}
对于线性系统,我们有特别简洁的方法:选择\textbf{二次型}李雅普诺夫函数。这不仅计算方便,而且给出了稳定性的\textbf{充要条件}。

对于线性系统 $\dot{x} = Ax$,选择二次型李雅普诺夫函数:
\[V(x) = x^T P x\]

其中 $P$ 为正定矩阵。稳定的充要条件是李雅普诺夫方程:
\[A^T P + PA = -Q\]
对于给定的正定矩阵 $Q$,存在唯一的正定解 $P$。

\textbf{实用价值:}这个方程将稳定性判断转化为\textbf{求解线性矩阵方程}的问题。在MATLAB等软件中,可以直接调用\texttt{lyap}函数求解。如果解 $P$ 是正定的,系统就是渐近稳定的。

\subsection{局部稳定性 vs 全局稳定性}

\subsubsection*{本节目的}
这是非线性系统分析中\textbf{最容易混淆的概念}。很多初学者证明了系统在平衡点附近稳定,就错误地声称系统「稳定」。实际上,\textbf{局部稳定}和\textbf{全局稳定}是完全不同的性质。本节将彻底澄清这个问题。

\subsubsection{核心概念辨析}

在李雅普诺夫理论中,\textbf{局部渐近稳定}和\textbf{全局渐近稳定}(又称\textbf{大范围渐近稳定})是两个截然不同的概念。区分它们对于理解非线性系统的稳定性至关重要。

\paragraph{1. 局部渐近稳定 (Local Asymptotic Stability)}

\textbf{直观比喻:}想象一个\textbf{放在巨大桌子中央的小碗}。

\textbf{含义:}
\begin{itemize}
    \item 只要初始状态在平衡点的\textbf{某个邻域}内(碗里面),系统最终会收敛到平衡点
    \item 但如果初始状态在这个邻域之外(碗外面的桌子上),系统行为无法保证
    \item 这个邻域称为\textbf{吸引域 (Domain of Attraction)} 或\textbf{稳定域}
\end{itemize}

\textbf{数学定义:}存在 $\delta > 0$,当 $\|x(0) - x_e\| < \delta$ 时,有 $\lim_{t \to \infty} x(t) = x_e$。

\paragraph{2. 全局渐近稳定 (Global Asymptotic Stability)}

\textbf{直观比喻:}想象\textbf{整个地球就是一个巨大的碗},只有一个最低点。

\textbf{含义:}
\begin{itemize}
    \item \textbf{无论初始状态在状态空间的任何位置},系统最终都会收敛到平衡点
    \item 吸引域是\textbf{整个状态空间}
    \item 这是一个非常强的性质,在实际系统中并不常见
\end{itemize}

\textbf{数学定义:}对于\textbf{任意}初始状态 $x(0)$,都有 $\lim_{t \to \infty} x(t) = x_e$。

\subsubsection{全局稳定性的判断条件}

要证明全局渐近稳定,李雅普诺夫函数必须满足更严格的条件:

\begin{enumerate}
    \item $V(x)$ 在\textbf{整个状态空间}都是正定的
    \item $\dot{V}(x)$ 在\textbf{整个状态空间}都是负定的(除平衡点外)
    \item \textbf{$V(x)$ 是径向无界的 (Radially Unbounded)}:
    \[\lim_{\|x\| \to \infty} V(x) = \infty\]
\end{enumerate}

\paragraph{为什么需要「径向无界」条件?}

\textbf{比喻:}这个条件确保「碗壁是无限高的」。

\textbf{原因:}
\begin{itemize}
    \item 如果 $V(x)$ 在远处趋向某个有限值,可能存在「平坦的高原」
    \item 系统状态可能停留在这个高原上,永远无法回到平衡点
    \item 无限高的碗壁保证了状态总是处于向平衡点倾斜的「斜坡」上
\end{itemize}

对于常用的二次型函数 $V(x) = x^T P x$($P$ 正定),它天然满足径向无界条件。因此,关键是判断 $\dot{V}(x)$ 是否在\textbf{整个空间}都是负定的。

\subsubsection{判断示例:为什么局部≠全局}

考虑系统:
\begin{align*}
\dot{x}_1 &= -2x_1 + 2x_2^4 \\
\dot{x}_2 &= -x_2
\end{align*}

选择李雅普诺夫函数 $V(x) = \frac{1}{2}x_1^2 + x_2^2$,计算得:
\[\dot{V}(x) = -2x_1^2 + 2x_1x_2^4 - 2x_2^2\]

\paragraph{局部分析(原点附近)}

当 $x_1, x_2$ 很小时,二次项 $-2x_1^2 - 2x_2^2$ 占主导,高阶项 $2x_1x_2^4$ 可忽略,因此 $\dot{V}(x) < 0$。

\textbf{结论:}系统在原点附近是\textbf{局部渐近稳定}的。

\paragraph{全局分析(远离原点)}

取 $x_2 = 2$,$x_1 = 10$:
\[\dot{V}(10, 2) = -2(100) + 2(10)(16) - 2(4) = -200 + 320 - 8 = 112 > 0\]

在点 $(10, 2)$ 处,$\dot{V}(x) > 0$,能量\textbf{正在增加}!

\textbf{结论:}$\dot{V}(x)$ 不是全局负定的,系统\textbf{不是全局渐近稳定}的。

\subsubsection{关键要点总结}

\begin{center}
\renewcommand{\arraystretch}{1.6}
\begin{tabular}{|l|p{6cm}|p{6cm}|}
\hline
\rowcolor[gray]{0.9}
\textbf{对比维度} & \textbf{局部渐近稳定} & \textbf{全局渐近稳定} \\
\hline
\textbf{吸引域} & 平衡点的某个邻域 & 整个状态空间 \\
\hline
\textbf{初始条件要求} & 必须在稳定域内 & 任意初始状态 \\
\hline
\textbf{$V(x)$ 正定性} & 在邻域内正定 & 在整个空间正定 \\
\hline
\textbf{$\dot{V}(x)$ 负定性} & 在邻域内负定 & 在整个空间负定 \\
\hline
\textbf{径向无界条件} & 不要求 & \textbf{必须满足} \\
\hline
\textbf{证明难度} & 相对容易 & 通常很困难 \\
\hline
\textbf{实际意义} & 小扰动下稳定 & 任何扰动下都稳定 \\
\hline
\end{tabular}
\end{center}

\textbf{重要提醒:}
\begin{itemize}
    \item 对于\textbf{线性系统},局部渐近稳定等价于全局渐近稳定
    \item 对于\textbf{非线性系统},局部稳定\textbf{不能}推出全局稳定
    \item 在解题时,如果只证明了 $\dot{V}(x)$ 在邻域内负定,结论应是「局部渐近稳定」
    \item 声称「全局稳定」需要更严格的证明
\end{itemize}

\subsubsection*{本节小结}
局部稳定性与全局稳定性的区别是理解非线性系统的关键。记住:
\begin{itemize}
    \item 局部稳定只保证「小扰动下」系统安全
    \item 全局稳定保证「任意扰动下」系统都安全
    \item 证明全局稳定需要验证 $\dot{V}(x)$ 在\textbf{整个状态空间}都负定,并且 $V(x)$ 径向无界
\end{itemize}

至此,我们已经掌握了李雅普诺夫稳定性理论的核心内容。接下来通过两个典型范例,展示如何将理论应用于实际问题。

\section{李雅普诺夫稳定性判断——范例}

\subsection*{范例说明}
以下两个范例分别展示:
\begin{itemize}
    \item \textbf{范例1}:线性系统的李雅普诺夫方程求解(临界稳定情况)
    \item \textbf{范例2}:非线性系统的线性化分析(第一方法应用)
\end{itemize}

通过这些例子,你将看到理论如何转化为具体的计算步骤。

\subsection{范例1:临界稳定系统的李雅普诺夫分析}

\subsubsection{题目}

判断系统稳定性:
\begin{align*}
\dot{x} &= \begin{bmatrix} 0 & 1 \\ -1 & 0 \end{bmatrix} x + \begin{bmatrix} 1 \\ 0 \end{bmatrix} u \\
y &= \begin{bmatrix} 0 & 1 \end{bmatrix} x
\end{align*}

其中 $A = \begin{bmatrix} 0 & 1 \\ -1 & 0 \end{bmatrix}$

\subsubsection{解答}

判断稳定性时,我们只关心\textbf{零输入响应}(令 $u=0$),因此只分析系统矩阵 $A$。

\paragraph{方法一:特征值法(快速验证)}

这是最直接的方法,用于判断系统的稳定性。

计算特征值,解 $\det(\lambda I - A) = 0$:
\[\det \begin{bmatrix} \lambda & -1 \\ 1 & \lambda \end{bmatrix} = \lambda^2 + 1 = 0\]

解得特征值为 $\lambda = \pm i$。

\textbf{结论:}特征值的实部为 0,因此系统是\textbf{李雅普诺夫稳定}的(也称\textbf{临界稳定}),但\textbf{不是渐进稳定}。系统受扰动后会产生等幅振荡。

\paragraph{方法二:李雅普诺夫第二方法(题目要求)}

\begin{enumerate}
    \item \textbf{选择对称正定矩阵 $Q$:}
    
    最简单的选择是单位矩阵:
    \[Q = \begin{bmatrix} 1 & 0 \\ 0 & 1 \end{bmatrix}\]
    
    \item \textbf{设定待求的对称矩阵 $P$:}
    
    令 $P = \begin{bmatrix} p_{11} & p_{12} \\ p_{12} & p_{22} \end{bmatrix}$
    
    \item \textbf{构建李雅普诺夫方程 $A^T P + PA = -Q$:}
    
    首先计算各项:
    \[A^T = \begin{bmatrix} 0 & -1 \\ 1 & 0 \end{bmatrix}\]
    
    \[A^T P = \begin{bmatrix} 0 & -1 \\ 1 & 0 \end{bmatrix} \begin{bmatrix} p_{11} & p_{12} \\ p_{12} & p_{22} \end{bmatrix} = \begin{bmatrix} -p_{12} & -p_{22} \\ p_{11} & p_{12} \end{bmatrix}\]
    
    \[PA = \begin{bmatrix} p_{11} & p_{12} \\ p_{12} & p_{22} \end{bmatrix} \begin{bmatrix} 0 & 1 \\ -1 & 0 \end{bmatrix} = \begin{bmatrix} -p_{12} & p_{11} \\ -p_{22} & p_{12} \end{bmatrix}\]
    
    \[A^T P + PA = \begin{bmatrix} -2p_{12} & p_{11}-p_{22} \\ p_{11}-p_{22} & 2p_{12} \end{bmatrix}\]
    
    令其等于 $-Q = \begin{bmatrix} -1 & 0 \\ 0 & -1 \end{bmatrix}$:
    \[\begin{bmatrix} -2p_{12} & p_{11}-p_{22} \\ p_{11}-p_{22} & 2p_{12} \end{bmatrix} = \begin{bmatrix} -1 & 0 \\ 0 & -1 \end{bmatrix}\]
    
    \item \textbf{分析方程组:}
    
    由上式可得:
    \begin{align*}
    -2p_{12} &= -1 \implies p_{12} = \frac{1}{2} & \text{(第(1,1)元素)} \\
    p_{11} - p_{22} &= 0 \implies p_{11} = p_{22} & \text{(第(1,2)元素)} \\
    2p_{12} &= -1 \implies p_{12} = -\frac{1}{2} & \text{(第(2,2)元素)}
    \end{align*}
    
    \textbf{出现矛盾!}从第(1,1)元素得 $p_{12} = \frac{1}{2}$,而从第(2,2)元素得 $p_{12} = -\frac{1}{2}$。
    
    这意味着对于正定矩阵 $Q=I$,\textbf{李雅普诺夫方程无解}。
    
    \item \textbf{得出结论:}
    
    根据李雅普诺夫稳定性判据,如果系统是渐进稳定的,那么对于\textbf{任意}正定 $Q$,都\textbf{必须}能解出唯一的正定 $P$。
    
    现在我们发现,对于最简单的正定 $Q=I$,方程都无解,这直接说明了系统\textbf{不是渐进稳定的}。
\end{enumerate}

\subsubsection{结论汇总}

\begin{center}
\begin{tabular}{|l|l|}
\hline
\textbf{判断方法} & \textbf{结论} \\
\hline
特征值法 & 特征值 $\lambda = \pm i$,实部为 0,系统李雅普诺夫稳定但不渐近稳定 \\
\hline
李雅普诺夫第二方法 & 对正定 $Q=I$,方程无正定解 $P$,系统不渐近稳定 \\
\hline
\textbf{最终结论} & \textbf{系统临界稳定(李雅普诺夫稳定),受扰动后产生等幅振荡} \\
\hline
\end{tabular}
\end{center}

\subsubsection*{范例1总结}
这个例子展示了\textbf{临界稳定系统}的特点:
\begin{itemize}
    \item 特征值实部为零,系统处于稳定与不稳定的边界
    \item 李雅普诺夫方程无解,证明系统不是渐近稳定
    \item 实际表现为等幅振荡——既不收敛也不发散
\end{itemize}

\textbf{关键经验:}遇到方程矛盾(无解)时,不要怀疑计算错误,这恰恰说明系统不满足渐近稳定条件!

\subsection{范例2:非线性系统的线性化分析}

\subsubsection{题目}

使用李雅普诺夫第一方法(线性化方法)分析系统的稳定性:
\begin{align*}
\dot{x}_1 &= -6x_1 - x_2 \\
\dot{x}_2 &= -2x_1 - 6x_2 - 2x_2^3
\end{align*}

\subsubsection{解答思路}

对于非线性系统,直接寻找合适的李雅普诺夫函数(第二方法)往往非常困难。\textbf{李雅普诺夫第一方法}(也称\textbf{间接法})提供了一条捷径:在平衡点附近对系统进行\textbf{线性化},然后分析线性化系统的稳定性。

\paragraph{第一步:寻找平衡点}

令 $\dot{x}_1 = 0$ 和 $\dot{x}_2 = 0$:
\begin{align*}
-6x_1 - x_2 &= 0 \implies x_2 = -6x_1 \quad (1) \\
-2x_1 - 6x_2 - 2x_2^3 &= 0 \quad (2)
\end{align*}

将 (1) 式代入 (2) 式:
\begin{align*}
-2x_1 - 6(-6x_1) - 2(-6x_1)^3 &= 0 \\
-2x_1 + 36x_1 - 2(-216x_1^3) &= 0 \\
34x_1 + 432x_1^3 &= 0 \\
x_1(34 + 432x_1^2) &= 0
\end{align*}

由于 $34 + 432x_1^2 > 0$ 对所有实数 $x_1$ 成立,因此 $x_1 = 0$,进而 $x_2 = 0$。

\textbf{结论:}系统有唯一平衡点 $(0, 0)$。

\paragraph{第二步:在平衡点处线性化系统}

对于非线性系统 $\dot{\mathbf{x}} = \mathbf{f}(\mathbf{x})$,在平衡点 $\mathbf{x}_e$ 附近的线性化系统为:
\[\dot{\mathbf{z}} = A \mathbf{z}\]

其中 $A$ 是\textbf{雅可比矩阵 (Jacobian Matrix)}:
\[A = J(\mathbf{x}_e) = \left.\begin{bmatrix} 
\frac{\partial f_1}{\partial x_1} & \frac{\partial f_1}{\partial x_2} \\
\frac{\partial f_2}{\partial x_1} & \frac{\partial f_2}{\partial x_2}
\end{bmatrix}\right|_{\mathbf{x} = \mathbf{x}_e}\]

\textbf{1. 定义非线性函数}
\[\mathbf{f}(\mathbf{x}) = \begin{bmatrix} f_1(x_1, x_2) \\ f_2(x_1, x_2) \end{bmatrix} = \begin{bmatrix} -6x_1 - x_2 \\ -2x_1 - 6x_2 - 2x_2^3 \end{bmatrix}\]

\textbf{2. 计算偏导数}
\begin{align*}
\frac{\partial f_1}{\partial x_1} &= -6, \quad \frac{\partial f_1}{\partial x_2} = -1 \\
\frac{\partial f_2}{\partial x_1} &= -2, \quad \frac{\partial f_2}{\partial x_2} = -6 - 6x_2^2
\end{align*}

\textbf{3. 构造雅可比矩阵}
\[J(\mathbf{x}) = \begin{bmatrix} -6 & -1 \\ -2 & -6 - 6x_2^2 \end{bmatrix}\]

\textbf{4. 在平衡点 $(0, 0)$ 处计算}
\[A = J(0, 0) = \begin{bmatrix} -6 & -1 \\ -2 & -6 \end{bmatrix}\]

\paragraph{第三步:分析线性化系统的特征值}

求解特征方程 $\det(\lambda I - A) = 0$:
\begin{align*}
\det \begin{bmatrix} \lambda + 6 & 1 \\ 2 & \lambda + 6 \end{bmatrix} &= 0 \\
(\lambda + 6)^2 - 2 &= 0 \\
\lambda^2 + 12\lambda + 36 - 2 &= 0 \\
\lambda^2 + 12\lambda + 34 &= 0
\end{align*}

使用求根公式:
\begin{align*}
\lambda &= \frac{-12 \pm \sqrt{144 - 136}}{2} = \frac{-12 \pm \sqrt{8}}{2} \\
&= \frac{-12 \pm 2\sqrt{2}}{2} = -6 \pm \sqrt{2}
\end{align*}

两个特征值为:
\begin{align*}
\lambda_1 &= -6 + \sqrt{2} \approx -4.586 \\
\lambda_2 &= -6 - \sqrt{2} \approx -7.414
\end{align*}

\paragraph{第四步:根据李雅普诺夫第一方法下结论}

\textbf{李雅普诺夫第一方法判据:}

\begin{itemize}
    \item 如果线性化系统的所有特征值\textbf{实部都严格小于零},则原非线性系统在该平衡点\textbf{局部渐近稳定}
    \item 如果至少有一个特征值实部大于零,则原系统\textbf{不稳定}
    \item 如果存在实部为零的特征值,则该方法\textbf{失效},无法判断
\end{itemize}

\textbf{分析:}两个特征值都是负实数,实部均严格小于零。

\textbf{最终结论:}根据李雅普诺夫第一方法,原非线性系统在平衡点 $(0, 0)$ 处是\textbf{局部渐近稳定}的。

\subsubsection{方法对比与选择}

\begin{center}
\renewcommand{\arraystretch}{1.6}
\begin{tabular}{|l|p{6cm}|p{6cm}|}
\hline
\rowcolor[gray]{0.9}
\textbf{方法} & \textbf{李雅普诺夫第一方法} & \textbf{李雅普诺夫第二方法} \\
\hline
\textbf{别名} & 间接法、线性化方法 & 直接法 \\
\hline
\textbf{核心思想} & 在平衡点附近线性化,分析线性系统特征值 & 构造能量函数 $V(x)$,验证其沿轨迹递减 \\
\hline
\textbf{优点} & 
计算简单、步骤明确 \newline 
只需计算雅可比矩阵和特征值 & 
适用于所有非线性系统 \newline 
可能得到全局稳定性结论 \\
\hline
\textbf{缺点} & 
只能判断局部稳定性 \newline 
特征值实部为零时失效 & 
构造李雅普诺夫函数困难 \newline 
没有通用方法 \\
\hline
\textbf{适用场景} & 
平衡点附近稳定性分析 \newline 
作为初步快速判断 & 
需要全局稳定性分析 \newline 
线性化方法失效时 \\
\hline
\end{tabular}
\end{center}

\textbf{解题建议:}
\begin{itemize}
    \item 对于非线性系统,\textbf{优先尝试第一方法}(线性化),因为计算简单
    \item 如果线性化后特征值实部为零,或需要全局稳定性结论,再考虑第二方法
    \item 第二方法需要灵活性和经验,往往需要多次尝试不同的候选函数
\end{itemize}

\subsubsection*{范例2总结}
这个例子展示了李雅普诺夫第一方法的\textbf{标准流程}:
\begin{enumerate}
    \item 求平衡点(令 $\dot{\mathbf{x}} = \mathbf{0}$)
    \item 计算雅可比矩阵(在平衡点处求偏导)
    \item 求特征值(解特征方程)
    \item 根据特征值实部判断稳定性
\end{enumerate}

\textbf{关键经验:}
\begin{itemize}
    \item 第一方法计算量小,适合作为初步分析
    \item 结论仅限于\textbf{局部稳定性}——记住这个限制!
    \item 如果特征值恰好在虚轴上(实部为零),方法失效,需改用第二方法
\end{itemize}

\subsection*{本章总结}

\subsubsection*{核心要点回顾}

\textbf{1. 三个稳定性层次}
\begin{itemize}
    \item 稳定:扰动有界
    \item 渐近稳定:扰动有界且最终消失
    \item 全局渐近稳定:任意大的扰动都能消失
\end{itemize}

\textbf{2. 两种李雅普诺夫方法}
\begin{itemize}
    \item \textbf{第一方法}(线性化):简单快速,但仅得局部结论,特征值为零时失效
    \item \textbf{第二方法}(直接法):强大通用,可得全局结论,但构造李雅普诺夫函数困难
\end{itemize}

\textbf{3. 局部 vs 全局}
\begin{itemize}
    \item 线性系统:局部稳定 ⟺ 全局稳定
    \item 非线性系统:局部稳定 ⇏ 全局稳定
    \item 全局稳定要求:$V(x)$ 径向无界,$\dot{V}(x)$ 在整个空间负定
\end{itemize}

\subsubsection*{学习建议}

\textbf{解题策略:}
\begin{enumerate}
    \item 先判断系统类型(线性/非线性)
    \item 非线性系统优先尝试第一方法(线性化)
    \item 如果第一方法失效或需要全局结论,使用第二方法
    \item 对于线性系统,直接用李雅普诺夫方程或特征值法
\end{enumerate}

\textbf{常见误区:}
\begin{itemize}
    \item ✗ 混淆「稳定」与「渐近稳定」
    \item ✗ 将局部稳定的结论推广为全局稳定
    \item ✗ 遇到李雅普诺夫方程无解就认为计算错误
    \item ✓ 仔细区分稳定性的层次和适用范围
\end{itemize}

李雅普诺夫稳定性理论是现代控制理论的基石,掌握它不仅能分析系统稳定性,更为后续的控制器设计(极点配置、状态观测器等)奠定了基础。
