\section{稳定性与李雅普诺夫方法}

\subsection{李雅普诺夫稳定性定义}
考虑自治系统 $\dot{x} = f(x)$,设 $x_e$ 为平衡点:
\begin{itemize}
    \item \textbf{稳定}:对任意 $\varepsilon > 0$,存在 $\delta > 0$,当 $\|x(0) - x_e\| < \delta$ 时,有 $\|x(t) - x_e\| < \varepsilon$,$\forall t \geq 0$
    \item \textbf{渐近稳定}:稳定且 $\lim_{t \to \infty} x(t) = x_e$
    \item \textbf{大范围渐近稳定}:渐近稳定且吸引域为整个状态空间
\end{itemize}

\subsection{李雅普诺夫第一方法(线性化方法)}
对于线性系统 $\dot{x} = Ax$,系统渐近稳定的充要条件是矩阵 $A$ 的所有特征值都具有负实部。

\subsection{李雅普诺夫第二方法(直接方法)}
\textbf{李雅普诺夫定理}:如果存在标量函数 $V(x)$ 满足:
\begin{enumerate}
    \item $V(x)$ 连续且有连续的一阶偏导数
    \item $V(x_e) = 0$,当 $x \neq x_e$ 时 $V(x) > 0$(正定)
    \item $\dot{V}(x) = \frac{\partial V}{\partial x} f(x) \leq 0$(半负定)
\end{enumerate}
则平衡点 $x_e$ 稳定。

若进一步有 $\dot{V}(x) < 0$(负定),则平衡点渐近稳定。

\subsection{线性系统的李雅普诺夫方程}
对于线性系统 $\dot{x} = Ax$,选择二次型李雅普诺夫函数:
\[V(x) = x^T P x\]

其中 $P$ 为正定矩阵。稳定的充要条件是李雅普诺夫方程:
\[A^T P + PA = -Q\]
对于给定的正定矩阵 $Q$,存在唯一的正定解 $P$。
