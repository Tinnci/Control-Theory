\section{线性控制系统的能控性和能观测性}

\subsection{能控性定义}
系统 $(A, B)$ 在时刻 $t_0$ 是状态能控的,如果存在有限时间 $t_1 > t_0$ 和控制输入 $u(t)$,使得系统能从任意初态 $x(t_0)$ 转移到任意终态 $x(t_1)$。

\subsection{能控性判据}
系统 $(A, B)$ 完全能控的充要条件是能控性矩阵:
\[W_c = [B \quad AB \quad A^2B \quad \cdots \quad A^{n-1}B]\]
满足 $\text{rank}(W_c) = n$。

\subsection{能观测性定义}
系统 $(A, C)$ 在时刻 $t_0$ 是状态能观测的,如果能够根据有限时间区间 $[t_0, t_1]$ 内的输出 $y(t)$ 和输入 $u(t)$ 唯一地确定初始状态 $x(t_0)$。

\subsection{能观测性判据}
系统 $(A, C)$ 完全能观测的充要条件是能观测性矩阵:
\[W_o = \begin{bmatrix}
C \\ CA \\ CA^2 \\ \vdots \\ CA^{n-1}
\end{bmatrix}\]
满足 $\text{rank}(W_o) = n$。

