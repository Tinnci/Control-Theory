\section{状态空间表达式求传递函数}
\label{sec:transfer-function}

\subsection*{引言:搭建经典与现代的桥梁}

你已经掌握了两种系统描述方法:
\begin{itemize}
    \item \textbf{经典方法}:传递函数 $G(s)$(输入输出关系,代数)
    \item \textbf{现代方法}:状态空间 $(A, B, C, D)$(内部状态演化,微分方程组)
\end{itemize}

\textbf{现实问题:}
\begin{itemize}
    \item 已有大量基于传递函数的经典设计工具(根轨迹、频域分析)
    \item 用状态空间建模后,如何利用这些工具?
    \item 两种描述能否互相转换?
\end{itemize}

\textbf{本章核心:}从状态空间 $(A, B, C, D)$ 推导传递函数 $G(s)$。

\textbf{为什么重要?}
\begin{enumerate}
    \item \textbf{工具互通}:用状态空间建模,用传递函数分析
    \item \textbf{验证模型}:两种方法得到的极点、零点应一致
    \item \textbf{降维理解}:状态空间是 $n$ 维内部结构,传递函数是外部输入输出特性
\end{enumerate}

\textbf{本章路线:}
\begin{itemize}
    \item 单输入单输出(SISO)系统的转换公式
    \item 多输入多输出(MIMO)系统的传递函数矩阵
    \item 计算方法与实例
    \item MATLAB实现
\end{itemize}

\subsection{从状态空间到传递函数}

\subsubsection{基本推导}

考虑线性时不变系统:
\begin{align*}
\dot{x} &= Ax + Bu \\
y &= Cx + Du
\end{align*}

\textbf{目标:}求 $G(s) = \frac{Y(s)}{U(s)}$(零初始条件)

\textbf{推导步骤:}

对状态方程取拉普拉斯变换($x(0) = 0$):
\[sX(s) = AX(s) + BU(s)\]

解出 $X(s)$:
\begin{align*}
(sI - A)X(s) &= BU(s) \\
X(s) &= (sI - A)^{-1}BU(s)
\end{align*}

代入输出方程:
\begin{align*}
Y(s) &= CX(s) + DU(s) \\
&= C(sI - A)^{-1}BU(s) + DU(s) \\
&= [C(sI - A)^{-1}B + D]U(s)
\end{align*}

\textbf{传递函数公式:}
\[\boxed{G(s) = C(sI - A)^{-1}B + D}\]

\subsubsection{公式解读}

\begin{itemize}
    \item $(sI - A)^{-1}$:称为\textbf{解析矩阵}或\textbf{状态转移矩阵}(频域)
    \item $C(sI - A)^{-1}B$:状态空间部分的贡献
    \item $D$:直接传输项(输入到输出的直接通路)
    \item 通常 $D = 0$(物理系统很少有瞬时响应)
\end{itemize}

\subsubsection{计算方法}

\textbf{关键步骤:}计算 $(sI - A)^{-1}$

利用矩阵求逆公式:
\[(sI - A)^{-1} = \frac{\text{adj}(sI - A)}{\det(sI - A)}\]

其中:
\begin{itemize}
    \item $\det(sI - A)$:系统的\textbf{特征多项式}(传递函数的分母)
    \item $\text{adj}(sI - A)$:$(sI - A)$ 的伴随矩阵
\end{itemize}

\subsection{单输入单输出(SISO)系统}

\subsubsection{标准形式}

对于 SISO 系统($u, y$ 均为标量):
\[G(s) = C(sI - A)^{-1}B + D = \frac{b_m s^m + \cdots + b_1 s + b_0}{s^n + a_{n-1}s^{n-1} + \cdots + a_1 s + a_0}\]

\textbf{关键性质:}
\begin{itemize}
    \item \textbf{分母}:$\det(sI - A)$,由 $A$ 矩阵决定(系统极点)
    \item \textbf{分子}:$C \cdot \text{adj}(sI - A) \cdot B$,由 $A, B, C$ 共同决定(系统零点)
    \item \textbf{阶数}:分母阶数 = 状态维数 $n$;分子阶数 $\leq n-1$(真分式)
\end{itemize}

\subsubsection{范例1:二阶系统}

\textbf{题目:}求传递函数

给定系统:
\begin{align*}
\dot{x} &= \begin{bmatrix} 0 & 1 \\ -2 & -3 \end{bmatrix} x + \begin{bmatrix} 0 \\ 1 \end{bmatrix} u \\
y &= \begin{bmatrix} 1 & 0 \end{bmatrix} x
\end{align*}

\textbf{解答:}

\textbf{步骤1:}构造 $sI - A$
\[sI - A = \begin{bmatrix} s & -1 \\ 2 & s+3 \end{bmatrix}\]

\textbf{步骤2:}计算 $\det(sI - A)$
\[\det(sI - A) = s(s+3) - (-1)(2) = s^2 + 3s + 2\]

\textbf{步骤3:}计算 $(sI - A)^{-1}$
\[(sI - A)^{-1} = \frac{1}{s^2+3s+2}\begin{bmatrix} s+3 & 1 \\ -2 & s \end{bmatrix}\]

\textbf{步骤4:}计算 $G(s)$
\begin{align*}
G(s) &= C(sI - A)^{-1}B \\
&= \begin{bmatrix} 1 & 0 \end{bmatrix} \cdot \frac{1}{s^2+3s+2}\begin{bmatrix} s+3 & 1 \\ -2 & s \end{bmatrix} \cdot \begin{bmatrix} 0 \\ 1 \end{bmatrix} \\
&= \frac{1}{s^2+3s+2}\begin{bmatrix} 1 & 0 \end{bmatrix}\begin{bmatrix} 1 \\ s \end{bmatrix} \\
&= \boxed{\frac{1}{s^2+3s+2}}
\end{align*}

\textbf{验证:}极点为 $s = -1, -2$(特征值),无零点。

\subsection{多输入多输出(MIMO)系统}

\subsubsection{传递函数矩阵}

对于 $p$ 输出、$m$ 输入系统:
\[G(s) = C(sI - A)^{-1}B + D \in \mathbb{R}^{p \times m}(s)\]

\textbf{矩阵元素:}
\[G_{ij}(s) = \frac{Y_i(s)}{U_j(s)}\Big|_{\text{其他输入为0}}\]

表示第 $j$ 个输入到第 $i$ 个输出的传递函数。

\subsubsection{范例2:MIMO系统}

\textbf{题目:}双输入双输出系统

\begin{align*}
A = \begin{bmatrix} -1 & 0 \\ 0 & -2 \end{bmatrix}, \quad
B = \begin{bmatrix} 1 & 0 \\ 0 & 1 \end{bmatrix}, \quad
C = \begin{bmatrix} 1 & 1 \\ 0 & 1 \end{bmatrix}, \quad
D = 0
\end{align*}

\textbf{解答:}

\textbf{步骤1:}$(sI - A)^{-1}$ 对角矩阵
\[(sI - A)^{-1} = \begin{bmatrix} \frac{1}{s+1} & 0 \\ 0 & \frac{1}{s+2} \end{bmatrix}\]

\textbf{步骤2:}计算 $G(s)$
\begin{align*}
G(s) &= C(sI - A)^{-1}B \\
&= \begin{bmatrix} 1 & 1 \\ 0 & 1 \end{bmatrix} \begin{bmatrix} \frac{1}{s+1} & 0 \\ 0 & \frac{1}{s+2} \end{bmatrix} \begin{bmatrix} 1 & 0 \\ 0 & 1 \end{bmatrix} \\
&= \boxed{\begin{bmatrix} \frac{1}{s+1} & \frac{1}{s+2} \\ 0 & \frac{1}{s+2} \end{bmatrix}}
\end{align*}

\textbf{解释:}
\begin{itemize}
    \item $G_{11}(s) = \frac{1}{s+1}$:输入1 → 输出1
    \item $G_{12}(s) = \frac{1}{s+2}$:输入2 → 输出1
    \item $G_{21}(s) = 0$:输入1对输出2无影响
    \item $G_{22}(s) = \frac{1}{s+2}$:输入2 → 输出2
\end{itemize}

\subsection{MATLAB实现}

\textbf{方法1:直接使用 \texttt{ss2tf}}

\begin{lstlisting}[style=Matlab-editor]
% 定义状态空间模型
A = [0 1; -2 -3];
B = [0; 1];
C = [1 0];
D = 0;

% 转换为传递函数
[num, den] = ss2tf(A, B, C, D);

% 显示结果
G = tf(num, den)
\end{lstlisting}

\textbf{方法2:使用系统对象}

\begin{lstlisting}[style=Matlab-editor]
% 创建状态空间对象
sys_ss = ss(A, B, C, D);

% 转换为传递函数对象
sys_tf = tf(sys_ss);

% 查看传递函数
zpk(sys_tf)  % 零极点增益形式
\end{lstlisting}

\textbf{方法3:MIMO系统}

\begin{lstlisting}[style=Matlab-editor]
% MIMO系统
A = [-1 0; 0 -2];
B = [1 0; 0 1];
C = [1 1; 0 1];
D = zeros(2,2);

sys_ss = ss(A, B, C, D);
sys_tf = tf(sys_ss);

% 查看传递函数矩阵
sys_tf
\end{lstlisting}

\subsection*{本章小结}

\subsubsection*{核心公式}

\begin{tcolorbox}[colback=blue!5!white,colframe=blue!75!black,title=状态空间到传递函数]
\[\boxed{G(s) = C(sI - A)^{-1}B + D}\]

\textbf{关键步骤:}
\begin{enumerate}
    \item 构造 $sI - A$
    \item 计算 $\det(sI - A)$(特征多项式)
    \item 求 $(sI - A)^{-1}$(伴随矩阵法或直接求逆)
    \item 矩阵相乘 $C \cdot (sI-A)^{-1} \cdot B + D$
\end{enumerate}
\end{tcolorbox}

\subsubsection*{两种描述对比}

\begin{center}
\renewcommand{\arraystretch}{1.6}
\begin{tabular}{|l|p{5cm}|p{5cm}|}
\hline
\rowcolor[gray]{0.9}
\textbf{特性} & \textbf{状态空间} & \textbf{传递函数} \\
\hline
描述对象 & 内部状态演化 & 输入输出关系 \\
\hline
维度 & $n$ 维向量微分方程 & 标量代数方程 \\
\hline
适用性 & MIMO、时变、非线性 & SISO、线性时不变 \\
\hline
信息量 & 完整(状态+输入输出) & 部分(仅输入输出) \\
\hline
设计方法 & 极点配置、观测器 & 根轨迹、频域 \\
\hline
\end{tabular}
\end{center}

\subsubsection*{实际应用}

\textbf{何时需要转换?}
\begin{itemize}
    \item 用状态空间建模后,想用根轨迹/伯德图分析
    \item 验证模型:两种方法的极点应一致
    \item 简化理解:传递函数更直观显示输入输出关系
\end{itemize}

\textbf{注意事项:}
\begin{itemize}
    \item 状态空间非唯一,但传递函数唯一
    \item 不可控/不可观部分的极点会被约掉(极零对消)
    \item MIMO系统:传递函数矩阵有 $p \times m$ 个元素
\end{itemize}

\subsubsection*{与后续章节的联系}

\begin{itemize}
    \item \textbf{第\ref{sec:linear-transformation}章}:线性变换改变状态空间表示,但传递函数不变(不变性)
    \item \textbf{第\ref{sec:controllability-observability}章}:能控能观性与传递函数的极零点关系
    \item \textbf{反向转换}:传递函数 → 状态空间(能控/能观标准型)将在第\ref{sec:standard-forms}章讨论
\end{itemize}

\textbf{关键启示:}传递函数是状态空间的\textbf{外部视图},丢失了内部结构信息,但更简洁便于分析。两种描述相辅相成,需要灵活运用。
