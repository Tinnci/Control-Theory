\part{经典控制理论}

\section{自动控制系统的一般概念}

\subsection{控制系统的基本组成}
控制系统一般由以下几个基本组成部分构成:
\begin{itemize}
    \item \textbf{被控对象(控制对象)}:需要被控制的系统或装置
    \item \textbf{控制器}:对输入信号进行处理,产生控制信号
    \item \textbf{传感器}:检测被控量的实际值
    \item \textbf{执行器}:接收控制信号,对被控对象施加控制作用
\end{itemize}

\subsection{控制系统的基本要求}
\begin{enumerate}
    \item \textbf{稳定性}:系统在扰动作用下能够恢复到平衡状态
    \item \textbf{准确性}:系统的稳态误差要小
    \item \textbf{快速性}:系统的动态响应要快
\end{enumerate}

