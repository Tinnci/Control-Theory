\section{稳定性与李雅普诺夫方法}

\subsection{李雅普诺夫稳定性定义}
考虑自治系统 $\dot{x} = f(x)$,设 $x_e$ 为平衡点:
\begin{itemize}
    \item \textbf{稳定}:对任意 $\varepsilon > 0$,存在 $\delta > 0$,当 $\|x(0) - x_e\| < \delta$ 时,有 $\|x(t) - x_e\| < \varepsilon$,$\forall t \geq 0$
    \item \textbf{渐近稳定}:稳定且 $\lim_{t \to \infty} x(t) = x_e$
    \item \textbf{大范围渐近稳定}:渐近稳定且吸引域为整个状态空间
\end{itemize}

\subsection{李雅普诺夫第一方法(线性化方法)}
对于线性系统 $\dot{x} = Ax$,系统渐近稳定的充要条件是矩阵 $A$ 的所有特征值都具有负实部。

\subsection{李雅普诺夫第二方法(直接方法)}
\textbf{李雅普诺夫定理}:如果存在标量函数 $V(x)$ 满足:
\begin{enumerate}
    \item $V(x)$ 连续且有连续的一阶偏导数
    \item $V(x_e) = 0$,当 $x \neq x_e$ 时 $V(x) > 0$(正定)
    \item $\dot{V}(x) = \frac{\partial V}{\partial x} f(x) \leq 0$(半负定)
\end{enumerate}
则平衡点 $x_e$ 稳定。

若进一步有 $\dot{V}(x) < 0$(负定),则平衡点渐近稳定。

\subsection{线性系统的李雅普诺夫方程}
对于线性系统 $\dot{x} = Ax$,选择二次型李雅普诺夫函数:
\[V(x) = x^T P x\]

其中 $P$ 为正定矩阵。稳定的充要条件是李雅普诺夫方程:
\[A^T P + PA = -Q\]
对于给定的正定矩阵 $Q$,存在唯一的正定解 $P$。

\subsection{李雅普诺夫方法的核心思想}

\subsubsection{直观理解}
李雅普诺夫方法的思想非常直观:\textbf{如果能为系统找到一个类似于「能量」的函数 $V(x)$,并且证明这个「能量」总是随时间减少(或至少不增加),那么系统最终必然会稳定下来。}

\begin{itemize}
    \item 这个函数 $V(x)$ 称为\textbf{李雅普诺夫函数}
    \item $V(x)$ 必须是\textbf{正定的}(类似能量总是正的)
    \item $V(x)$ 的导数 $\dot{V}(x)$ 必须是\textbf{负定的或半负定的}(能量随时间衰减或保持不变)
\end{itemize}

\subsubsection{线性系统的二次型李雅普诺夫函数}

对于线性系统 $\dot{x} = Ax$,我们选择二次型的李雅普诺夫函数 $V(x) = x^T P x$,其中 $P$ 是对称正定矩阵。

对 $V(x)$ 求导:
\begin{align*}
\dot{V}(x) &= \frac{d}{dt}(x^T P x) \\
&= \dot{x}^T P x + x^T P \dot{x} \\
&= (Ax)^T P x + x^T P (Ax) \\
&= x^T A^T P x + x^T P A x \\
&= x^T(A^T P + PA)x
\end{align*}

为了让 $\dot{V}(x)$ 是负定的,通常令 $\dot{V}(x) = -x^T Q x$,其中 $Q$ 是我们选择的对称正定矩阵。因此得到:
\[\boxed{A^T P + PA = -Q}\]

这就是著名的\textbf{李雅普诺夫方程}。

\subsubsection{判据总结}

对于给定的系统矩阵 $A$:
\begin{itemize}
    \item 如果对\textbf{任意}正定矩阵 $Q$,李雅普诺夫方程都有唯一的\textbf{正定解} $P$,则系统\textbf{渐近稳定}
    \item 如果对某个正定矩阵 $Q$,方程无解或解不是正定的,则系统\textbf{不是渐近稳定的}
    \item 如果能找到半正定的 $Q$ 使得存在正定的 $P$,但对所有正定 $Q$ 都无解,则系统可能是\textbf{李雅普诺夫稳定}(临界稳定)
\end{itemize}

\subsection{局部稳定性 vs 全局稳定性}

\subsubsection{核心概念辨析}

在李雅普诺夫理论中,\textbf{局部渐近稳定}和\textbf{全局渐近稳定}(又称\textbf{大范围渐近稳定})是两个截然不同的概念。区分它们对于理解非线性系统的稳定性至关重要。

\paragraph{1. 局部渐近稳定 (Local Asymptotic Stability)}

\textbf{直观比喻:}想象一个\textbf{放在巨大桌子中央的小碗}。

\textbf{含义:}
\begin{itemize}
    \item 只要初始状态在平衡点的\textbf{某个邻域}内(碗里面),系统最终会收敛到平衡点
    \item 但如果初始状态在这个邻域之外(碗外面的桌子上),系统行为无法保证
    \item 这个邻域称为\textbf{吸引域 (Domain of Attraction)} 或\textbf{稳定域}
\end{itemize}

\textbf{数学定义:}存在 $\delta > 0$,当 $\|x(0) - x_e\| < \delta$ 时,有 $\lim_{t \to \infty} x(t) = x_e$。

\paragraph{2. 全局渐近稳定 (Global Asymptotic Stability)}

\textbf{直观比喻:}想象\textbf{整个地球就是一个巨大的碗},只有一个最低点。

\textbf{含义:}
\begin{itemize}
    \item \textbf{无论初始状态在状态空间的任何位置},系统最终都会收敛到平衡点
    \item 吸引域是\textbf{整个状态空间}
    \item 这是一个非常强的性质,在实际系统中并不常见
\end{itemize}

\textbf{数学定义:}对于\textbf{任意}初始状态 $x(0)$,都有 $\lim_{t \to \infty} x(t) = x_e$。

\subsubsection{全局稳定性的判断条件}

要证明全局渐近稳定,李雅普诺夫函数必须满足更严格的条件:

\begin{enumerate}
    \item $V(x)$ 在\textbf{整个状态空间}都是正定的
    \item $\dot{V}(x)$ 在\textbf{整个状态空间}都是负定的(除平衡点外)
    \item \textbf{$V(x)$ 是径向无界的 (Radially Unbounded)}:
    \[\lim_{\|x\| \to \infty} V(x) = \infty\]
\end{enumerate}

\paragraph{为什么需要「径向无界」条件?}

\textbf{比喻:}这个条件确保「碗壁是无限高的」。

\textbf{原因:}
\begin{itemize}
    \item 如果 $V(x)$ 在远处趋向某个有限值,可能存在「平坦的高原」
    \item 系统状态可能停留在这个高原上,永远无法回到平衡点
    \item 无限高的碗壁保证了状态总是处于向平衡点倾斜的「斜坡」上
\end{itemize}

对于常用的二次型函数 $V(x) = x^T P x$($P$ 正定),它天然满足径向无界条件。因此,关键是判断 $\dot{V}(x)$ 是否在\textbf{整个空间}都是负定的。

\subsubsection{判断示例:为什么局部≠全局}

考虑系统:
\begin{align*}
\dot{x}_1 &= -2x_1 + 2x_2^4 \\
\dot{x}_2 &= -x_2
\end{align*}

选择李雅普诺夫函数 $V(x) = \frac{1}{2}x_1^2 + x_2^2$,计算得:
\[\dot{V}(x) = -2x_1^2 + 2x_1x_2^4 - 2x_2^2\]

\paragraph{局部分析(原点附近)}

当 $x_1, x_2$ 很小时,二次项 $-2x_1^2 - 2x_2^2$ 占主导,高阶项 $2x_1x_2^4$ 可忽略,因此 $\dot{V}(x) < 0$。

\textbf{结论:}系统在原点附近是\textbf{局部渐近稳定}的。

\paragraph{全局分析(远离原点)}

取 $x_2 = 2$,$x_1 = 10$:
\[\dot{V}(10, 2) = -2(100) + 2(10)(16) - 2(4) = -200 + 320 - 8 = 112 > 0\]

在点 $(10, 2)$ 处,$\dot{V}(x) > 0$,能量\textbf{正在增加}!

\textbf{结论:}$\dot{V}(x)$ 不是全局负定的,系统\textbf{不是全局渐近稳定}的。

\subsubsection{关键要点总结}

\begin{center}
\renewcommand{\arraystretch}{1.6}
\begin{tabular}{|l|p{6cm}|p{6cm}|}
\hline
\rowcolor[gray]{0.9}
\textbf{对比维度} & \textbf{局部渐近稳定} & \textbf{全局渐近稳定} \\
\hline
\textbf{吸引域} & 平衡点的某个邻域 & 整个状态空间 \\
\hline
\textbf{初始条件要求} & 必须在稳定域内 & 任意初始状态 \\
\hline
\textbf{$V(x)$ 正定性} & 在邻域内正定 & 在整个空间正定 \\
\hline
\textbf{$\dot{V}(x)$ 负定性} & 在邻域内负定 & 在整个空间负定 \\
\hline
\textbf{径向无界条件} & 不要求 & \textbf{必须满足} \\
\hline
\textbf{证明难度} & 相对容易 & 通常很困难 \\
\hline
\textbf{实际意义} & 小扰动下稳定 & 任何扰动下都稳定 \\
\hline
\end{tabular}
\end{center}

\textbf{重要提醒:}
\begin{itemize}
    \item 对于\textbf{线性系统},局部渐近稳定等价于全局渐近稳定
    \item 对于\textbf{非线性系统},局部稳定\textbf{不能}推出全局稳定
    \item 在解题时,如果只证明了 $\dot{V}(x)$ 在邻域内负定,结论应是「局部渐近稳定」
    \item 声称「全局稳定」需要更严格的证明
\end{itemize}

\section{李雅普诺夫稳定性判断——范例}

\subsection{范例:临界稳定系统的李雅普诺夫分析}

\subsubsection{题目}

判断系统稳定性:
\begin{align*}
\dot{x} &= \begin{bmatrix} 0 & 1 \\ -1 & 0 \end{bmatrix} x + \begin{bmatrix} 1 \\ 0 \end{bmatrix} u \\
y &= \begin{bmatrix} 0 & 1 \end{bmatrix} x
\end{align*}

其中 $A = \begin{bmatrix} 0 & 1 \\ -1 & 0 \end{bmatrix}$

\subsubsection{解答}

判断稳定性时,我们只关心\textbf{零输入响应}(令 $u=0$),因此只分析系统矩阵 $A$。

\paragraph{方法一:特征值法(快速验证)}

这是最直接的方法,用于判断系统的稳定性。

计算特征值,解 $\det(\lambda I - A) = 0$:
\[\det \begin{bmatrix} \lambda & -1 \\ 1 & \lambda \end{bmatrix} = \lambda^2 + 1 = 0\]

解得特征值为 $\lambda = \pm i$。

\textbf{结论:}特征值的实部为 0,因此系统是\textbf{李雅普诺夫稳定}的(也称\textbf{临界稳定}),但\textbf{不是渐进稳定}。系统受扰动后会产生等幅振荡。

\paragraph{方法二:李雅普诺夫第二方法(题目要求)}

\begin{enumerate}
    \item \textbf{选择对称正定矩阵 $Q$:}
    
    最简单的选择是单位矩阵:
    \[Q = \begin{bmatrix} 1 & 0 \\ 0 & 1 \end{bmatrix}\]
    
    \item \textbf{设定待求的对称矩阵 $P$:}
    
    令 $P = \begin{bmatrix} p_{11} & p_{12} \\ p_{12} & p_{22} \end{bmatrix}$
    
    \item \textbf{构建李雅普诺夫方程 $A^T P + PA = -Q$:}
    
    首先计算各项:
    \[A^T = \begin{bmatrix} 0 & -1 \\ 1 & 0 \end{bmatrix}\]
    
    \[A^T P = \begin{bmatrix} 0 & -1 \\ 1 & 0 \end{bmatrix} \begin{bmatrix} p_{11} & p_{12} \\ p_{12} & p_{22} \end{bmatrix} = \begin{bmatrix} -p_{12} & -p_{22} \\ p_{11} & p_{12} \end{bmatrix}\]
    
    \[PA = \begin{bmatrix} p_{11} & p_{12} \\ p_{12} & p_{22} \end{bmatrix} \begin{bmatrix} 0 & 1 \\ -1 & 0 \end{bmatrix} = \begin{bmatrix} -p_{12} & p_{11} \\ -p_{22} & p_{12} \end{bmatrix}\]
    
    \[A^T P + PA = \begin{bmatrix} -2p_{12} & p_{11}-p_{22} \\ p_{11}-p_{22} & 2p_{12} \end{bmatrix}\]
    
    令其等于 $-Q = \begin{bmatrix} -1 & 0 \\ 0 & -1 \end{bmatrix}$:
    \[\begin{bmatrix} -2p_{12} & p_{11}-p_{22} \\ p_{11}-p_{22} & 2p_{12} \end{bmatrix} = \begin{bmatrix} -1 & 0 \\ 0 & -1 \end{bmatrix}\]
    
    \item \textbf{分析方程组:}
    
    由上式可得:
    \begin{align*}
    -2p_{12} &= -1 \implies p_{12} = \frac{1}{2} & \text{(第(1,1)元素)} \\
    p_{11} - p_{22} &= 0 \implies p_{11} = p_{22} & \text{(第(1,2)元素)} \\
    2p_{12} &= -1 \implies p_{12} = -\frac{1}{2} & \text{(第(2,2)元素)}
    \end{align*}
    
    \textbf{出现矛盾!}从第(1,1)元素得 $p_{12} = \frac{1}{2}$,而从第(2,2)元素得 $p_{12} = -\frac{1}{2}$。
    
    这意味着对于正定矩阵 $Q=I$,\textbf{李雅普诺夫方程无解}。
    
    \item \textbf{得出结论:}
    
    根据李雅普诺夫稳定性判据,如果系统是渐进稳定的,那么对于\textbf{任意}正定 $Q$,都\textbf{必须}能解出唯一的正定 $P$。
    
    现在我们发现,对于最简单的正定 $Q=I$,方程都无解,这直接说明了系统\textbf{不是渐进稳定的}。
\end{enumerate}

\subsubsection{结论汇总}

\begin{center}
\begin{tabular}{|l|l|}
\hline
\textbf{判断方法} & \textbf{结论} \\
\hline
特征值法 & 特征值 $\lambda = \pm i$,实部为 0,系统李雅普诺夫稳定但不渐近稳定 \\
\hline
李雅普诺夫第二方法 & 对正定 $Q=I$,方程无正定解 $P$,系统不渐近稳定 \\
\hline
\textbf{最终结论} & \textbf{系统临界稳定(李雅普诺夫稳定),受扰动后产生等幅振荡} \\
\hline
\end{tabular}
\end{center}
