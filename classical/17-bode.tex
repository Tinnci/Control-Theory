\section{伯德图绘制}

\subsection{伯德图的定义}
伯德图由两个图组成:
\begin{itemize}
    \item 幅频特性图:$L(\omega) = 20\lg|G(\jw)|$ dB vs $\lg\omega$
    \item 相频特性图:$\phi(\omega)$ vs $\lg\omega$
\end{itemize}

\subsection{典型环节的伯德图}
\begin{enumerate}
    \item \textbf{比例环节} $K$:
    \begin{itemize}
        \item $L(\omega) = 20\lg K$ dB(水平线)
        \item $\phi(\omega) = 0°$
    \end{itemize}
    
    \item \textbf{积分环节} $\frac{1}{s}$:
    \begin{itemize}
        \item $L(\omega) = -20\lg\omega$ dB(斜率-20dB/十倍频)
        \item $\phi(\omega) = -90°$
    \end{itemize}
    
    \item \textbf{一阶惯性环节} $\frac{1}{1+Ts}$:
    \begin{itemize}
        \item 转折频率:$\omega_c = \frac{1}{T}$
        \item $\omega < \omega_c$:$L(\omega) \approx 0$ dB
        \item $\omega > \omega_c$:$L(\omega) \approx -20\lg(\omega T)$ dB
        \item $\phi(\omega_c) = -45°$
    \end{itemize}
\end{enumerate}

\subsection{使用 LaTeX 绘制伯德图}
使用 \texttt{bodeplot} 包可以方便地绘制伯德图:

\subsubsection{通过零极点增益形式绘制}
\begin{center}
\begin{tikzpicture}
% 示例:G(s) = 10/((s+1)(s+10))
\begin{semilogxaxis}[
    width=10cm,height=6cm,
    xlabel={频率 $\omega$ (rad/s)},
    ylabel={幅度 (dB)},
    grid=both,
    title={幅频特性曲线}
]
\addplot+[blue,thick] coordinates {
    (1e-2, 40)
    (1e-1, 20)
    (1, 0)
    (10, -20)
    (1e2, -40)
};
\end{semilogxaxis}
\end{tikzpicture}
\end{center}

\subsubsection{相频特性绘制}
\begin{center}
\begin{tikzpicture}
\begin{semilogxaxis}[
    width=10cm,height=5cm,
    xlabel={频率 $\omega$ (rad/s)},
    ylabel={相位 (度)},
    grid=both,
    title={相频特性曲线}
]
\addplot+[red,thick] coordinates {
    (1e-2, 0)
    (1e-1, -45)
    (1, -90)
    (10, -135)
    (1e2, -180)
};
\end{semilogxaxis}
\end{tikzpicture}
\end{center}
