\section{状态观测器}
\label{sec:state-observer}

\subsection*{引言:看不见的状态,如何控制?}

想象你站在一个密闭房间外,房间里有一个人在活动。你\textbf{看不见}房间内部(状态不可测),但你可以:
\begin{itemize}
    \item \textbf{听见}声音从门缝传出(输出 $y$)
    \item \textbf{敲门}与里面的人互动(输入 $u$)
\end{itemize}

你能推断出房间里那个人的位置和动作吗?答案是:\textbf{可以!}通过声音的方向、强度、时间,结合你的敲门动作,你可以\textbf{估计}他的状态。

这正是\textbf{状态观测器}(State Observer)要做的事情:

\begin{quote}
\textbf{当系统的内部状态无法直接测量时,利用可测的输入和输出信息,动态地估计出系统的状态。}
\end{quote}

\subsubsection*{为什么需要状态观测器?}

在上一章(极点配置),我们学会了通过状态反馈 $u = -Kx$ 任意配置极点。但有一个\textbf{关键前提}:
\begin{center}
\textit{所有状态变量 $x$ 都必须是可测的!}
\end{center}

\textbf{实际问题:}
\begin{itemize}
    \item \textbf{传感器成本高}:为每个状态变量安装传感器不现实
    \item \textbf{物理上不可测}:某些状态本质上无法直接测量
    \begin{itemize}
        \item 电机内部的磁通量
        \item 飞行器的侧滑角
        \item 化学反应的中间浓度
    \end{itemize}
    \item \textbf{测量噪声大}:某些传感器信号质量差
    \item \textbf{维护困难}:传感器可能失效或精度下降
\end{itemize}

\textbf{具体例子:}

考虑一个二阶机械系统(如倒立摆):
\[x = \begin{bmatrix} \theta \\ \dot{\theta} \end{bmatrix}\]

\begin{itemize}
    \item $\theta$(角度):容易测量(编码器)
    \item $\dot{\theta}$(角速度):直接测量需要昂贵的陀螺仪,或者对角度\textbf{数值微分}(噪声放大严重)
\end{itemize}

\textbf{观测器的解决方案:}只测量 $\theta$,通过观测器\textbf{估计} $\dot{\theta}$!

\subsubsection*{观测器的核心思想}

状态观测器是一个\textbf{动态系统},它:
\begin{enumerate}
    \item \textbf{模拟}真实系统的动态:$\dot{\hat{x}} = A\hat{x} + Bu$(与真实系统相同的模型)
    \item \textbf{校正}估计误差:利用输出误差 $y - \hat{y} = y - C\hat{x}$ 进行反馈校正
    \item \textbf{收敛}到真实状态:通过设计增益 $L$,使估计误差 $e = x - \hat{x} \to 0$
\end{enumerate}

\textbf{直观类比:}
\begin{itemize}
    \item 真实系统:房间里的人(看不见)
    \item 观测器:你的大脑中的\textit{心理模型}(模拟他的行为)
    \item 输出误差:你预测的声音 vs 实际听到的声音
    \item 校正增益 $L$:你根据误差调整心理模型的速度
\end{itemize}

\subsubsection*{观测器与极点配置的对偶性}

\textbf{极点配置}(上一章):
\begin{itemize}
    \item 控制律:$u = -Kx$
    \item 目标:通过反馈改变系统矩阵 $A \to A - BK$
    \item 前提:系统\textbf{能控}
    \item 设计:配置闭环极点
\end{itemize}

\textbf{状态观测器}(本章):
\begin{itemize}
    \item 观测器:$\dot{\hat{x}} = A\hat{x} + Bu + L(y - C\hat{x})$
    \item 目标:通过校正改变误差动态 $A \to A - LC$
    \item 前提:系统\textbf{能观测}
    \item 设计:配置观测器极点
\end{itemize}

两者是\textbf{对偶}的!能控性 $\leftrightarrow$ 能观测性,$K \leftrightarrow L$,$B \leftrightarrow C^T$。

\subsubsection*{实际应用场景}

\begin{itemize}
    \item \textbf{航天器姿态控制}:只测量姿态角,估计角速度和角加速度
    \item \textbf{电机控制}:测量转速,估计负载转矩和磁通量
    \item \textbf{机器人}:测量关节位置,估计速度和加速度
    \item \textbf{自动驾驶}:传感器融合,估计车辆的侧向速度
    \item \textbf{电力系统}:测量电压电流,估计系统内部状态
\end{itemize}

\subsubsection*{本章内容路线图}

\begin{enumerate}
    \item \textbf{观测器的概念}:观测器的数学形式和工作原理
    \item \textbf{全维状态观测器设计}:如何选择增益 $L$ 配置观测器极点
    \item \textbf{观测器极点选择}:多快才合适?平衡收敛速度与噪声敏感性
    \item \textbf{分离定理}:控制器和观测器可以独立设计!
    \item \textbf{基于观测器的控制}:$u = -K\hat{x}$ 的完整设计
    \item \textbf{设计范例}:从头到尾的实际例子
\end{enumerate}

\textbf{学习目标:}
\begin{itemize}
    \item 理解观测器的必要性和工作原理
    \item 掌握观测器增益矩阵 $L$ 的设计方法
    \item 理解分离定理,学会设计完整的控制系统
    \item 能够解决实际的状态估计问题
\end{itemize}

让我们开始探索\textbf{状态估计}的奇妙世界!

\subsection{状态观测器的概念}

\subsubsection*{本节目的}
建立观测器的数学模型,理解观测器的基本结构和工作原理。

\subsubsection{问题的提出}

考虑线性时不变系统:
\begin{align*}
\dot{x} &= Ax + Bu \\
y &= Cx
\end{align*}

\textbf{已知信息:}
\begin{itemize}
    \item 系统矩阵 $A, B, C$(系统模型)
    \item 输入 $u(t)$(我们施加的控制)
    \item 输出 $y(t)$(传感器测量值)
\end{itemize}

\textbf{未知信息:}
\begin{itemize}
    \item 状态 $x(t)$(部分或全部不可测)
\end{itemize}

\textbf{目标:}构造一个\textbf{估计器},输出状态的估计值 $\hat{x}(t)$,使得:
\[\hat{x}(t) \to x(t) \quad \text{as } t \to \infty\]

\subsubsection{观测器的基本结构}

\textbf{朴素想法(开环观测器):}

既然我们知道系统模型,为什么不直接模拟它?
\[\dot{\hat{x}} = A\hat{x} + Bu\]

\textbf{问题:}
\begin{itemize}
    \item 初值误差:$\hat{x}(0) \neq x(0)$ 会永远存在
    \item 模型误差:$A, B$ 不准确会导致估计偏差累积
    \item 没有利用输出信息 $y$!
\end{itemize}

\textbf{改进想法(闭环观测器):}

利用输出误差 $y - \hat{y}$ 进行\textbf{校正}:
\[\dot{\hat{x}} = A\hat{x} + Bu + L(y - \hat{y})\]

其中 $\hat{y} = C\hat{x}$ 是估计输出。

\textbf{最终形式:}
\[\boxed{\dot{\hat{x}} = A\hat{x} + Bu + L(y - C\hat{x})}\]

\textbf{各项的物理意义:}
\begin{itemize}
    \item $A\hat{x} + Bu$:\textbf{预测项}(根据模型预测下一状态)
    \item $y - C\hat{x}$:\textbf{创新}(innovation)或\textbf{输出误差}(实际与预测的差异)
    \item $L$:\textbf{观测器增益矩阵}(校正力度,类似控制器的 $K$)
\end{itemize}

\textbf{直观理解:}
\begin{itemize}
    \item 如果 $y > C\hat{x}$(实际输出大于估计):说明我们低估了状态,应该\textbf{上调} $\hat{x}$
    \item $L$ 决定调整的速度:$L$ 大 $\to$ 快速校正,$L$ 小 $\to$ 缓慢校正
\end{itemize}

\subsubsection{观测器 vs 传感器}

\textbf{观测器不是传感器!}

\begin{center}
\renewcommand{\arraystretch}{1.6}
\begin{tabular}{|l|p{5cm}|p{5cm}|}
\hline
\rowcolor[gray]{0.9}
& \textbf{传感器} & \textbf{观测器} \\
\hline
\textbf{本质} & 物理设备 & 算法/软件 \\
\hline
\textbf{输入} & 物理信号(温度、位移等) & $u, y$(数据) \\
\hline
\textbf{输出} & 测量值(可能有噪声) & 状态估计 $\hat{x}$ \\
\hline
\textbf{成本} & 硬件成本 & 计算成本 \\
\hline
\textbf{优势} & 直接测量 & 减少传感器数量,滤波效果 \\
\hline
\textbf{劣势} & 昂贵,可能失效 & 依赖模型准确性 \\
\hline
\end{tabular}
\end{center}

\textbf{实际应用中:}
\begin{itemize}
    \item 少量传感器(测量部分状态)+ 观测器(估计其余状态)
    \item 传感器提供 $y$,观测器利用 $y$ 和模型重构完整的 $x$
\end{itemize}

\subsection{全维状态观测器}

\subsubsection*{本节目的}
深入理解观测器的数学原理:误差动态、收敛条件、极点配置方法。

\subsubsection{观测器方程}

\textbf{全维状态观测器的标准形式:}
\[\boxed{\dot{\hat{x}} = A\hat{x} + Bu + L(y - C\hat{x})}\]

改写为:
\[\dot{\hat{x}} = (A - LC)\hat{x} + Bu + Ly\]

\textbf{方程结构分析:}
\begin{itemize}
    \item 输入:$u$(控制输入)和 $y$(系统输出)
    \item 输出:$\hat{x}$(状态估计)
    \item 观测器矩阵:$A - LC$(类比闭环控制中的 $A - BK$)
    \item 待设计参数:$L \in \mathbb{R}^{n \times p}$
\end{itemize}

\textbf{为什么叫\textbf{全维}?}
\begin{itemize}
    \item 全维观测器:估计\textbf{所有} $n$ 个状态变量
    \item 降维观测器:只估计\textbf{不可测}的状态(利用 $y$ 已包含部分状态信息)
    \item 全维观测器更简单,是入门的首选
\end{itemize}

\subsubsection{误差动态分析}

\textbf{核心问题:}观测器的估计值 $\hat{x}$ 能逼近真实状态 $x$ 吗?多快能收敛?

\vspace{0.3cm}

\begin{tcolorbox}[colback=blue!5!white, colframe=blue!75!black, title=\textbf{第一步:定义估计误差}]
\begin{center}
\Large
$\boxed{e(t) = x(t) - \hat{x}(t)}$
\end{center}

\textbf{物理意义:}
\begin{itemize}
    \item $e = 0$ \quad $\Rightarrow$ \quad 完美估计
    \item $\|e(t)\|$ \quad 衡量估计的精度
    \item \textcolor{red!80!black}{\textbf{设计目标}}:使 $e(t) \to 0$(渐近收敛)
\end{itemize}
\end{tcolorbox}

\vspace{0.5cm}

\textbf{第二步:推导误差动态方程}

\textit{思路}:既然 $\dot{x}$ 和 $\dot{\hat{x}}$ 都已知,那么 $\dot{e} = \dot{x} - \dot{\hat{x}}$ 也能算出来!

\begin{align*}
\text{\textcolor{blue}{真实系统:}} \quad & \dot{x} = Ax + Bu \\[0.2cm]
\text{\textcolor{green!60!black}{观测器:}} \quad & \dot{\hat{x}} = A\hat{x} + Bu + \textcolor{red!80!black}{L(y - C\hat{x})} \\
& \quad = A\hat{x} + Bu + LC\underbrace{(x - \hat{x})}_{e} \\[0.3cm]
\text{\textcolor{orange!80!black}{两式相减:}} \quad & \dot{e} = \dot{x} - \dot{\hat{x}}
\end{align*}

\begin{align*}
\dot{e} &= \textcolor{blue}{Ax + Bu} - \textcolor{green!60!black}{[A\hat{x} + Bu + LCe]} \\
&= Ax \textcolor{gray}{+ Bu} - A\hat{x} \textcolor{gray}{- Bu} - LCe \quad \text{\small \textit{($Bu$ 项相消!)}} \\
&= A(x - \hat{x}) - LC(x - \hat{x}) \\
&= (A - LC)e
\end{align*}

\vspace{0.3cm}

\begin{tcolorbox}[colback=red!5!white, colframe=red!75!black, title=\textbf{⚡ 核心结论:误差动态方程}, fonttitle=\bfseries]
\begin{center}
\Huge
$\boxed{\dot{e} = (A - LC)e}$
\end{center}

\vspace{0.2cm}

\textbf{三个震撼性发现:}

\begin{enumerate}
    \item \textcolor{red!80!black}{\textbf{与输入 $u$ 无关!}} 
    \begin{itemize}
        \item 控制输入不影响估计误差
        \item 观测器可以独立于控制器设计
        \item 这是\textit{分离定理}的数学基础
    \end{itemize}
    
    \item \textcolor{blue!80!black}{\textbf{齐次方程}}
    \begin{itemize}
        \item 只依赖于初始误差 $e(0) = x(0) - \hat{x}(0)$
        \item 即使初始估计错误,也能收敛
    \end{itemize}
    
    \item \textcolor{green!60!black}{\textbf{收敛速度由 $A - LC$ 的极点决定}}
    \begin{itemize}
        \item 极点越左,收敛越快
        \item 设计 $L$ 就是配置这些极点
        \item 与极点配置\textit{完全类似}!
    \end{itemize}
\end{enumerate}
\end{tcolorbox}

\vspace{0.3cm}

\begin{center}
\begin{tikzpicture}[>=stealth, thick, node distance=2.5cm]
    % 误差演化示意
    \draw[->] (0,0) -- (8,0) node[right] {时间 $t$};
    \draw[->] (0,0) -- (0,3) node[above] {误差 $\|e(t)\|$};
    
    % 初始误差
    \node[circle, fill=red, inner sep=2pt] at (0.5,2.5) {};
    \node[above] at (0.5,2.8) {\small $e(0)$};
    
    % 收敛曲线
    \draw[red!80!black, very thick] plot[smooth, domain=0.5:7] 
        (\x, {2.5*exp(-0.8*(\x-0.5))});
    
    % 标注
    \node[below, text width=3cm, align=center] at (4,-0.8) {
        \small \textcolor{blue}{由 $A-LC$ 极点决定} \\
        \small \textcolor{green!60!black}{与 $u$ 无关}
    };
    
    % 目标线
    \draw[dashed, gray] (0,0.2) -- (8,0.2) node[right, black] {\small 目标};
\end{tikzpicture}
\end{center}

\begin{tcolorbox}[colback=yellow!10, colframe=orange!75!black, title=\textbf{💡 设计启示}]
\textbf{问题:}如何让 $e(t) \to 0$?

\textbf{答案:}选择增益 $L$,使 $A - LC$ 的所有特征值在\textcolor{green!60!black}{\textbf{左半平面}}!

\begin{center}
\textit{这与极点配置问题一模一样,只是把 $K$ 换成了 $L$!}
\end{center}
\end{tcolorbox}

\subsubsection{观测器收敛条件}

\textbf{目标:}使估计误差 $e(t) \to 0$。

从 $\dot{e} = (A - LC)e$ 可知,误差收敛的\textbf{充要条件}是:

\begin{tcolorbox}[colback=green!5!white, colframe=green!60!black, title=\textbf{收敛条件}]
\begin{center}
\Large
$A - LC$ 的所有特征值在\textcolor{green!60!black}{\textbf{左半平面}}
\end{center}

\vspace{0.2cm}

\textbf{数学表达:}
\[\text{Re}(\lambda_i(A - LC)) < 0, \quad i = 1, 2, \ldots, n\]

\textbf{等价于:}
\[e(t) = e^{(A-LC)t}e(0) \to 0 \quad \text{当 } t \to \infty\]
\end{tcolorbox}

\vspace{0.3cm}

\textbf{关键问题:}如何选择 $L$ 来配置 $A - LC$ 的极点?

\begin{center}
\textit{这个问题是否似曾相识?}
\end{center}

\vspace{0.3cm}

\begin{tcolorbox}[colback=blue!5!white, colframe=blue!75!black, title=\textbf{🔄 对偶性(Duality)—— 观测器设计的"秘密武器"}]

\textbf{惊人发现:}观测器设计 = 极点配置的\textcolor{red!80!black}{\textbf{对偶问题}}!

\begin{center}
\renewcommand{\arraystretch}{1.8}
\begin{tabular}{@{}l c c@{}}
\toprule
\textbf{问题} & \textbf{极点配置} & \textbf{观测器设计} \\
\midrule
\textbf{目标矩阵} & $A - BK$ & $A - LC$ \\[0.2em]
\textbf{极点位置} & 期望控制器极点 & 期望观测器极点 \\[0.2em]
\textbf{前提条件} & $(A, B)$ \textcolor{blue}{能控} & $(A, C)$ \textcolor{green!60!black}{能观测} \\[0.2em]
\textbf{设计参数} & $K \in \mathbb{R}^{m \times n}$ & $L \in \mathbb{R}^{n \times p}$ \\[0.2em]
\midrule
\textbf{对偶变换} & --- & $(A, C) \leftrightarrow (A^T, C^T)$ \\
\bottomrule
\end{tabular}
\end{center}

\vspace{0.3cm}

\begin{center}
\begin{tikzpicture}[>=stealth, thick, node distance=3cm]
    % 极点配置
    \node[draw, rectangle, rounded corners, fill=blue!10, minimum width=3.5cm, minimum height=2cm, align=center] (pole) at (0,0) {
        \textbf{极点配置} \\[0.3cm]
        设计 $K$ \\
        配置 $A - BK$
    };
    
    % 观测器设计
    \node[draw, rectangle, rounded corners, fill=green!10, minimum width=3.5cm, minimum height=2cm, align=center] (obs) at (6,0) {
        \textbf{观测器设计} \\[0.3cm]
        设计 $L$ \\
        配置 $A - LC$
    };
    
    % 对偶箭头
    \draw[<->, very thick, red!80!black, bend left=30] (pole) to node[above, text width=2.5cm, align=center] {\textbf{对偶变换} \\ {\small $A \to A^T$} \\ {\small $B \to C^T$}} (obs);
    
    % 底部标注
    \node[below=1cm of pole, text width=3cm, align=center, fill=yellow!20, rounded corners] {
        {\small 需要能控性}
    };
    
    \node[below=1cm of obs, text width=3cm, align=center, fill=yellow!20, rounded corners] {
        {\small 需要能观测性}
    };
\end{tikzpicture}
\end{center}

\vspace{0.3cm}

\textbf{📌 记忆卡片:对偶性的核心公式}

\begin{center}
\fcolorbox{red!80!black}{yellow!10}{
\begin{minipage}{0.85\textwidth}
\vspace{0.2cm}
\textbf{控制器设计:}
\begin{itemize}
    \item 能控 $(A, B)$ $\Rightarrow$ 存在 $K$ 使 $A - BK$ 有任意极点
    \item 方法:直接法、变换法、Ackermann 公式
\end{itemize}

\textbf{观测器设计(对偶):}
\begin{itemize}
    \item 能观测 $(A, C)$ $\Rightarrow$ 存在 $L$ 使 $A - LC$ 有任意极点
    \item 方法:\textcolor{red!80!black}{\textbf{完全相同}},只需把 $(A, B) \to (A^T, C^T)$,$K \to L^T$
\end{itemize}

\vspace{0.1cm}
\end{minipage}
}
\end{center}

\end{tcolorbox}

\vspace{0.3cm}

\textbf{利用对偶性设计观测器增益 $L$ 的步骤:}

\begin{enumerate}
    \item \textbf{验证能观测性}
    \begin{itemize}
        \item 检查 $(A, C)$ 的能观测矩阵:$W_o = \begin{bmatrix} C \\ CA \\ \vdots \\ CA^{n-1} \end{bmatrix}$
        \item 确认 $\text{rank}(W_o) = n$
    \end{itemize}
    
    \item \textbf{构造对偶系统}
    \begin{itemize}
        \item 对偶系统矩阵:$(A^T, C^T)$
        \item 这个对偶系统是\textbf{能控}的!(因为原系统能观测)
    \end{itemize}
    
    \item \textbf{为对偶系统设计 $K$}
    \begin{itemize}
        \item 使用任何极点配置方法(如 Ackermann)
        \item 使 $A^T - C^T K$ 有期望的观测器极点
    \end{itemize}
    
    \item \textbf{转置得到 $L$}
    \begin{itemize}
        \item $L = K^T$(或 $L^T = K$)
    \end{itemize}
\end{enumerate}

\vspace{0.3cm}

\begin{tcolorbox}[colback=purple!5!white, colframe=purple!75!black, title=\textbf{💻 MATLAB 实现}]

\textbf{方法1:直接使用 place 函数}
\begin{lstlisting}[style=Matlab-editor]
% 为观测器设计增益 L
desired_observer_poles = [-5, -6, -7];  % 期望极点
L = place(A', C', desired_observer_poles)';
% 注意:A' 和 C' 是转置,最后结果再转置
\end{lstlisting}

\textbf{方法2:利用 Ackermann 公式(单输出系统)}
\begin{lstlisting}[style=Matlab-editor]
% 先为对偶系统设计 K
K_dual = acker(A', C', desired_observer_poles);
% 转置得到 L
L = K_dual';
\end{lstlisting}

\textbf{验证设计:}
\begin{lstlisting}[style=Matlab-editor]
% 检查观测器极点
eig(A - L*C)
% 应该等于 desired_observer_poles
\end{lstlisting}

\textbf{为什么这样做?}
\begin{itemize}
    \item MATLAB 的极点配置函数都是为控制器设计的
    \item 通过对偶性,我们"欺骗"MATLAB 为对偶系统设计控制器
    \item 然后转置,得到原系统的观测器增益
\end{itemize}

\end{tcolorbox}

\subsubsection{观测器增益的物理意义}

\textbf{问题:}增益矩阵 $L$ 究竟是什么?它的每个元素代表什么?

\vspace{0.3cm}

\begin{tcolorbox}[colback=blue!5!white, colframe=blue!75!black, title=\textbf{增益 $L$ 的结构}]

\begin{center}
$L = \begin{bmatrix} l_{11} & l_{12} & \cdots & l_{1p} \\ l_{21} & l_{22} & \cdots & l_{2p} \\ \vdots & \vdots & \ddots & \vdots \\ l_{n1} & l_{n2} & \cdots & l_{np} \end{bmatrix} \in \mathbb{R}^{n \times p}$
\end{center}

\textbf{维度解释:}
\begin{itemize}
    \item $n$:状态变量个数(要估计多少个状态)
    \item $p$:输出变量个数(有多少个测量值)
\end{itemize}

\textbf{物理意义:}

\textcolor{red!80!black}{\textbf{$L_{ij}$}}:当第 $j$ 个输出有\textbf{单位误差}时,对第 $i$ 个状态估计的\textcolor{blue}{\textbf{校正强度}}。

\vspace{0.2cm}

\textit{类比:$L$ 就像"信任权重"——每个测量值对每个状态估计的影响有多大。}

\end{tcolorbox}

\vspace{0.3cm}

\textbf{通过极端情况理解 $L$:}

\begin{center}
\begin{tikzpicture}[>=stealth, thick]
    % 光谱图
    \draw[line width=3mm, left color=blue!50, right color=red!50] (0,0) -- (10,0);
    
    % 刻度
    \foreach \x/\label in {0/$L=0$, 5/$L$ 适中, 10/$L \to \infty$} {
        \draw (\x,-0.2) -- (\x,0.2);
        \node[below] at (\x,-0.5) {\label};
    }
    
    % 特性标注
    \node[above, text width=3cm, align=center] at (0,1.2) {
        \textcolor{blue}{\textbf{开环观测器}} \\
        {\small 完全信任模型}
    };
    
    \node[above, text width=3cm, align=center] at (5,1.2) {
        \textcolor{purple}{\textbf{平衡设计}} \\
        {\small 模型+测量}
    };
    
    \node[above, text width=3cm, align=center] at (10,1.2) {
        \textcolor{red!80!black}{\textbf{高增益观测器}} \\
        {\small 完全信任测量}
    };
\end{tikzpicture}
\end{center}

\vspace{0.5cm}

\begin{tcolorbox}[colback=yellow!10, colframe=orange!75!black, title=\textbf{情况1:$L = 0$(开环观测器)}]

\textbf{观测器方程退化为:}
\[\dot{\hat{x}} = A\hat{x} + Bu \quad \text{\textcolor{gray}{(无校正项)}}\]

\textbf{误差动态:}
\[\dot{e} = (A - 0 \cdot C)e = Ae\]

\begin{center}
\begin{tikzpicture}[>=stealth, thick, node distance=2cm]
    \node[draw, rectangle, minimum width=2.5cm, minimum height=1cm] (sys) at (0,0) {真实系统};
    \node[draw, rectangle, minimum width=2.5cm, minimum height=1cm] (obs) at (0,-2.5) {观测器模型};
    
    \node[left=1cm of sys] (u) {$u$};
    \draw[->, very thick] (u) -- (sys);
    \draw[->, very thick] (u) |- (obs);
    
    \node[right=1cm of sys] (y) {$y$};
    \draw[->] (sys) -- (y);
    
    % 无反馈
    \draw[->, dashed, gray] (y) -- ++(0,-1) node[midway, right] {\textcolor{gray}{忽略}};
    
    \node[below, text width=5cm, align=left] at (0,-4) {
        {\small \textcolor{red}{问题}:如果模型不准或初始误差大,} \\
        {\small 误差\textbf{永不收敛}($A$ 可能不稳定)}
    };
\end{tikzpicture}
\end{center}

\textbf{适用场景:}
\begin{itemize}
    \item 模型\textbf{极其准确}(罕见)
    \item 测量噪声巨大,不如不用
\end{itemize}

\textbf{结论:}\textcolor{red!80!black}{几乎不实用!}

\end{tcolorbox}

\vspace{0.3cm}

\begin{tcolorbox}[colback=red!10, colframe=red!75!black, title=\textbf{情况2:$L$ 很大(高增益观测器)}]

\textbf{误差动态:}
\[\dot{e} = (\underbrace{A}_{\text{缓慢}} - \underbrace{LC}_{\text{主导}})e \approx -LCe\]

当 $\|L\| \to \infty$,观测器极点 $\to -\infty$(超快收敛)。

\begin{center}
\begin{tikzpicture}[>=stealth, scale=0.8]
    % 时间轴
    \draw[->] (0,0) -- (6,0) node[right] {$t$};
    \draw[->] (0,0) -- (0,3) node[above] {$\|e\|$};
    
    % 不同L的收敛曲线
    \draw[blue, thick] plot[smooth, domain=0.2:5.5] (\x, {2*exp(-0.5*\x)});
    \draw[green!60!black, thick] plot[smooth, domain=0.2:5.5] (\x, {2*exp(-1.5*\x)});
    \draw[red, ultra thick] plot[smooth, domain=0.2:5.5] (\x, {2*exp(-5*\x)});
    
    % 图例
    \node[right, text=blue] at (3,2) {$L$ 小};
    \node[right, text=green!60!black] at (2,1.2) {$L$ 中等};
    \node[right, text=red!80!black] at (1,0.5) {$L$ 大};
\end{tikzpicture}
\end{center}

\textbf{优点:}
\begin{itemize}
    \item[\textcolor{green!60!black}{\checkmark}] 误差\textbf{极快收敛}
    \item[\textcolor{green!60!black}{\checkmark}] 对模型误差不敏感
\end{itemize}

\textbf{代价:}
\begin{itemize}
    \item[\textcolor{red!80!black}{\times}] 测量噪声被\textbf{严重放大}
    \item[\textcolor{red!80!black}{\times}] 估计值\textbf{抖动}剧烈
    \item[\textcolor{red!80!black}{\times}] 控制输入 $u = -K\hat{x}$ 也跟着抖动
\end{itemize}

\textbf{实际后果:}电机\textit{嗡嗡作响},机械臂\textit{疯狂颤抖}!

\end{tcolorbox}

\vspace{0.3cm}

\begin{tcolorbox}[colback=green!5!white, colframe=green!60!black, title=\textbf{✓ 情况3:$L$ 适中(工程平衡)}]

\textbf{设计准则:}

观测器极点比控制器极点快 \textcolor{blue}{\textbf{2~5 倍}}。

\begin{center}
\begin{tikzpicture}[>=stealth, thick]
    % s平面
    \draw[->] (-6,0) -- (1,0) node[right] {Re};
    \draw[->] (0,-2) -- (0,2) node[above] {Im};
    
    % 控制器极点
    \node[circle, fill=blue, inner sep=2pt] at (-1.5,0.8) {};
    \node[circle, fill=blue, inner sep=2pt] at (-1.5,-0.8) {};
    \node[above right, text=blue] at (-1.5,0.8) {控制器极点};
    
    % 观测器极点
    \node[circle, fill=red, inner sep=2pt] at (-4.5,1.2) {};
    \node[circle, fill=red, inner sep=2pt] at (-4.5,-1.2) {};
    \node[above right, text=red!80!black] at (-4.5,1.2) {观测器极点};
    
    % 标注
    \draw[<->, dashed] (-1.5,-1.5) -- node[below] {\small 快 2-5 倍} (-4.5,-1.5);
\end{tikzpicture}
\end{center}

\textbf{折中效果:}
\begin{itemize}
    \item 误差收敛\textit{足够快}(不拖累控制性能)
    \item 噪声放大\textit{可接受}(不引起抖动)
    \item 鲁棒性\textit{良好}(对模型误差容忍)
\end{itemize}

\textbf{经验公式:}

如果控制器主导极点实部为 $\sigma_c$,则观测器极点选择:
\[\sigma_o \approx (3 \sim 4) \times |\sigma_c|\]

\textbf{示例:}
\begin{itemize}
    \item 控制器极点:$-2 \pm j3$ \quad $\Rightarrow$ \quad $\sigma_c = -2$
    \item 观测器极点:$-8, -10$ \quad $\Rightarrow$ \quad $\sigma_o \approx -9$ \quad (4.5倍)
\end{itemize}

\end{tcolorbox}

\vspace{0.3cm}

\begin{tcolorbox}[colback=purple!5!white, colframe=purple!75!black, title=\textbf{📊 增益选择的权衡曲线}]

\begin{center}
\begin{tikzpicture}[>=stealth, scale=0.9]
    % 坐标轴
    \draw[->] (0,0) -- (8,0) node[right] {$\|L\|$(增益大小)};
    \draw[->] (0,0) -- (0,5) node[above] {性能};
    
    % 收敛速度曲线(单调递增)
    \draw[blue, very thick] plot[smooth, domain=0.5:7.5] (\x, {4 - 3.5*exp(-0.5*\x)});
    \node[above, text=blue] at (7,3.5) {收敛速度};
    
    % 噪声敏感性曲线(单调递增)
    \draw[red, very thick] plot[smooth, domain=0.5:7.5] (\x, {0.5 + 0.4*\x});
    \node[above, text=red!80!black] at (7,3.5) {噪声放大};
    
    % 最优区域
    \fill[green!20, opacity=0.5] (3,0) rectangle (5,5);
    \draw[green!60!black, dashed, thick] (3,0) -- (3,5);
    \draw[green!60!black, dashed, thick] (5,0) -- (5,5);
    \node[below, text=green!60!black] at (4,-0.5) {\textbf{工程最优区}};
\end{tikzpicture}
\end{center}

\textbf{设计哲学:}

\textit{不求最快,但求稳定;不追极致,但求实用。}

\end{tcolorbox}

\subsubsection{观测器设计定理}

\textbf{核心问题:}我们真的能任意配置观测器的极点吗?

\vspace{0.3cm}

\begin{tcolorbox}[colback=blue!5!white, colframe=blue!75!black, title=\textbf{📜 全维状态观测器设计定理}]

\textbf{定理陈述:}

若系统 $(A, C)$ \textcolor{red!80!black}{\textbf{完全能观测}},则对于任意给定的 $n$ 个复数 $\mu_1, \mu_2, \ldots, \mu_n$(复数成对共轭),\textbf{存在}观测器增益矩阵 $L \in \mathbb{R}^{n \times p}$,使得误差矩阵 $A - LC$ 的特征值恰好为 $\mu_1, \mu_2, \ldots, \mu_n$。

\vspace{0.2cm}

\textbf{数学表达:}
\[\text{能观测} \quad \Leftrightarrow \quad \text{极点任意配置}\]

\tcblower

\textbf{定理的三层含义:}

\begin{enumerate}
    \item \textcolor{blue}{\textbf{充分性}}:能观测 $\Rightarrow$ 可配置极点
    
    \textit{(只要系统能观测,我们就有能力让误差以任意速度收敛)}
    
    \item \textcolor{blue}{\textbf{必要性}}:可配置极点 $\Rightarrow$ 能观测
    
    \textit{(如果不能观测,某些状态的估计误差永远无法控制)}
    
    \item \textcolor{blue}{\textbf{构造性}}:定理不仅说"存在",还给出了计算方法!
    
    \textit{(通过对偶性,直接套用极点配置算法)}
\end{enumerate}

\end{tcolorbox}

\vspace{0.3cm}

\begin{tcolorbox}[colback=yellow!10, colframe=orange!75!black, title=\textbf{💡 为什么需要能观测性?——一个直观解释}]

\textbf{场景:}假设某个状态 $x_i$ 对输出 $y$ \textbf{完全没有影响}(不能观测)。

\begin{center}
\begin{tikzpicture}[>=stealth, thick, node distance=2.5cm]
    % 系统框图
    \node[draw, circle, minimum size=1.2cm] (xi) {$x_i$};
    \node[draw, circle, minimum size=1.2cm, right=of xi] (xj) {$x_j$};
    \node[draw, rectangle, minimum width=2cm, minimum height=1cm, right=of xj] (output) {输出 $y$};
    
    % 连接
    \draw[->, dashed, gray] (xi) -- node[above] {\textcolor{gray}{无影响}} (xj);
    \draw[->] (xj) -- (output);
    
    % 观测器部分
    \node[draw, circle, minimum size=1.2cm, below=2cm of xi, fill=red!10] (xi_hat) {$\hat{x}_i$};
    \node[below=0.3cm of xi_hat, text width=3cm, align=center] {
        {\small \textcolor{red}{问题}:$\hat{x}_i$ 的误差} \\
        {\small 在 $y$ 中看不到!}
    };
    
    \draw[->, red, dashed, thick] (output) -- ++(0,-1.5) node[midway, right] {\small 反馈} -- ++(-5,0);
    \draw[->, gray, dashed] (xi_hat) -- ++(-1.5,0) node[left] {\small $e_i = x_i - \hat{x}_i$};
\end{tikzpicture}
\end{center}

\textbf{后果:}
\begin{itemize}
    \item 无论 $L$ 怎么设计,都无法从 $y$ 的误差校正 $\hat{x}_i$
    \item $e_i(t) = x_i(t) - \hat{x}_i(t)$ 的演化只能靠\textbf{开环}($\dot{e}_i = \cdots$)
    \item 如果对应的 $A$ 子块不稳定,$e_i$ 会发散!
\end{itemize}

\textbf{结论:}\textcolor{red!80!black}{\textbf{不能观测 $\Rightarrow$ 无法保证误差收敛!}}

\end{tcolorbox}

\vspace{0.3cm}

\begin{tcolorbox}[colback=green!5!white, colframe=green!60!black, title=\textbf{🔄 对偶性——观测器设计的"免费午餐"}]

\textbf{震撼发现:}观测器设计 = 极点配置的\textcolor{blue}{\textbf{对偶问题}}!

\vspace{0.3cm}

\begin{center}
\renewcommand{\arraystretch}{1.5}
\begin{tabular}{@{} l c c @{}}
\toprule
\textbf{对比项} & \textbf{极点配置(控制器)} & \textbf{观测器设计} \\
\midrule
\textbf{系统方程} & $\dot{x} = Ax + Bu$ & $\dot{e} = (A - LC)e$ \\
\textbf{配置目标矩阵} & $A - BK$ & $A - LC$ \\
\textbf{前提条件} & $(A, B)$ 能控 & $(A, C)$ 能观测 \\
\textbf{增益矩阵} & $K \in \mathbb{R}^{m \times n}$ & $L \in \mathbb{R}^{n \times p}$ \\
\textbf{对偶变换} & --- & $A \to A^T, \, B \to C^T, \, K \to L^T$ \\
\midrule
\textbf{设计方法} & \multicolumn{2}{c}{\textcolor{blue}{完全相同!(pole, acker)}} \\
\bottomrule
\end{tabular}
\end{center}

\vspace{0.3cm}

\begin{center}
\begin{tikzpicture}[>=stealth, thick, node distance=4cm]
    % 左边:控制器设计
    \node[draw, rectangle, fill=blue!10, minimum width=3cm, minimum height=2.5cm, text width=2.8cm, align=center] (control) {
        \textbf{控制器设计} \\[0.2cm]
        配置 $A - BK$ \\[0.1cm]
        需要 $(A,B)$ 能控
    };
    
    % 右边:观测器设计
    \node[draw, rectangle, fill=red!10, minimum width=3cm, minimum height=2.5cm, text width=2.8cm, align=center, right=of control] (observer) {
        \textbf{观测器设计} \\[0.2cm]
        配置 $A - LC$ \\[0.1cm]
        需要 $(A,C)$ 能观测
    };
    
    % 对偶变换箭头
    \draw[->, very thick, blue] (control) to[bend left=20] node[above, text width=3cm, align=center] {
        \small 对偶变换 \\
        \small $A \to A^T$ \\
        \small $B \to C^T$
    } (observer);
    
    \draw[->, very thick, red] (observer) to[bend left=20] node[below, text width=3cm, align=center] {
        \small 逆变换 \\
        \small $K \to L^T$
    } (control);
    
    % 底部标注
    \node[below=0.5cm of control, text=blue] {已有成熟算法};
    \node[below=0.5cm of observer, text=red!80!black] {直接套用!};
\end{tikzpicture}
\end{center}

\vspace{0.3cm}

\fcolorbox{red}{yellow!20}{
\begin{minipage}{0.95\textwidth}
\textbf{📌 记忆卡片:对偶性快速查表}

\begin{itemize}
    \item \textbf{控制器}:能控 $\xrightarrow{\text{设计 } K}$ 配置 $A - BK$
    \item \textbf{观测器}:能观测 $\xrightarrow{\text{对偶变换}}$ 配置 $A - LC$
    \item \textbf{方法一致}:`place(A', C', poles)',结果转置 $\Rightarrow L$
\end{itemize}

\textbf{口诀:}同样的配方,同样的味道,只是原料调换了顺序!
\end{minipage}
}

\end{tcolorbox}

\subsection{观测器的设计}

\subsubsection*{本节目的}
掌握观测器增益 $L$ 的具体计算步骤,以及观测器极点的选择原则。

\subsubsection{设计步骤}

\textbf{核心问题:}知道了理论,如何一步步设计出实际的观测器?

\vspace{0.3cm}

\begin{tcolorbox}[colback=cyan!5!white, colframe=cyan!75!black, title=\textbf{🛠️ 观测器设计标准流程(四步法)}]

\begin{center}
\begin{tikzpicture}[>=stealth, thick, node distance=1.8cm]
    % 步骤节点
    \node[draw, circle, fill=blue!20, minimum size=1.5cm] (step1) {\textbf{1}};
    \node[draw, circle, fill=blue!20, minimum size=1.5cm, below=of step1] (step2) {\textbf{2}};
    \node[draw, circle, fill=blue!20, minimum size=1.5cm, below=of step2] (step3) {\textbf{3}};
    \node[draw, circle, fill=blue!20, minimum size=1.5cm, below=of step3] (step4) {\textbf{4}};
    
    % 步骤描述
    \node[right=0.5cm of step1, text width=6cm, align=left] {
        \textbf{验证能观测性} \\
        {\small 计算 $W_o$,检查 rank$(W_o) = n$}
    };
    
    \node[right=0.5cm of step2, text width=6cm, align=left] {
        \textbf{选择观测器极点} \\
        {\small $\mu_1, \ldots, \mu_n$(左半平面,比控制器快2-5倍)}
    };
    
    \node[right=0.5cm of step3, text width=6cm, align=left] {
        \textbf{计算增益 $L$} \\
        {\small MATLAB: \texttt{L = place(A',C',poles)'}}
    };
    
    \node[right=0.5cm of step4, text width=6cm, align=left] {
        \textbf{验证设计} \\
        {\small 检查 $\lambda(A-LC)$,仿真误差动态}
    };
    
    % 连接箭头
    \draw[->, very thick] (step1) -- (step2);
    \draw[->, very thick] (step2) -- (step3);
    \draw[->, very thick] (step3) -- (step4);
    
    % 失败返回路径
    \draw[->, dashed, red, thick] (step1.west) -- ++(-0.8,0) |- node[left, pos=0.25, text=red] {\small 不能观测} ++(0,-2) node[left] {\small 重新建模};
    \draw[->, dashed, orange, thick] (step4.east) -- ++(0.8,0) |- node[right, pos=0.25, text=orange] {\small 噪声过大} ++(0,-3.5) -- (step2.east);
\end{tikzpicture}
\end{center}

\end{tcolorbox}

\vspace{0.3cm}

\begin{tcolorbox}[colback=blue!5!white, colframe=blue!75!black, title=\textbf{第一步:验证能观测性}]

\textbf{为什么必须先检查?}

如果 $(A,C)$ 不能观测,就\textbf{无法}任意配置极点(定理前提不满足)!

\vspace{0.2cm}

\textbf{检查方法:}

构造\textbf{能观测性矩阵}:
\[\Large W_o = \begin{bmatrix} C \\ CA \\ CA^2 \\ \vdots \\ CA^{n-1} \end{bmatrix} \in \mathbb{R}^{np \times n}\]

\textbf{判据:}$\text{rank}(W_o) = n$ \quad $\Leftrightarrow$ \quad 系统能观测

\vspace{0.2cm}

\textbf{💻 MATLAB 代码:}
\begin{lstlisting}[style=Matlab-editor]
% 方法1:手动计算秩
Wo = obsv(A, C);  % 自动生成能观测性矩阵
r = rank(Wo);
if r == size(A, 1)
    disp('系统能观测,可以继续设计!');
else
    disp(['系统不能观测!秩亏 ', num2str(size(A,1) - r)]);
    return;  % 终止设计
end

% 方法2:直接判断
if rank(obsv(A, C)) == size(A, 1)
    % 继续设计...
end
\end{lstlisting}

\textbf{⚠️ 不能观测怎么办?}
\begin{itemize}
    \item 重新设计传感器布局(增加测量点)
    \item 改变输出选择(测量不同的状态组合)
    \item 使用降维观测器(只估计可观测部分)
\end{itemize}

\end{tcolorbox}

\vspace{0.3cm}

\begin{tcolorbox}[colback=green!5!white, colframe=green!60!black, title=\textbf{第二步:选择观测器极点}]

\textbf{核心原则:}观测器必须比控制器\textcolor{red}{\textbf{更快收敛}}!

\vspace{0.2cm}

\textbf{为什么?——一个时间线故事}

\begin{center}
\begin{tikzpicture}[>=stealth, thick, scale=0.9]
    % 时间轴
    \draw[->, ultra thick] (0,0) -- (12,0) node[right] {时间 $t$};
    
    % 事件标记
    \filldraw (0,0) circle (3pt) node[below] {$t_0$};
    \filldraw (3,0) circle (3pt) node[below] {$t_1$};
    \filldraw (6,0) circle (3pt) node[below] {$t_2$};
    \filldraw (10,0) circle (3pt) node[below] {$t_3$};
    
    % 阶段描述
    \node[above, text width=2.5cm, align=center] at (1.5,1.5) {
        \textcolor{blue}{\textbf{启动阶段}} \\
        {\small $e(t_0)$ 很大}
    };
    
    \node[above, text width=2.5cm, align=center] at (4.5,1.5) {
        \textcolor{orange}{\textbf{观测器收敛}} \\
        {\small $e \to 0$}
    };
    
    \node[above, text width=2.5cm, align=center] at (8,1.5) {
        \textcolor{green!60!black}{\textbf{控制器生效}} \\
        {\small $x \to$ 目标}
    };
    
    % 曲线示意
    \draw[red, very thick] (0,-1.5) .. controls (2,-1) and (4,-0.5) .. (6,-0.2) node[right] {\small $\|e(t)\|$};
    \draw[blue, very thick] (3,-2.5) .. controls (5,-2) and (7,-1) .. (10,-0.2) node[right] {\small $\|x - x_{\text{ref}}\|$};
\end{tikzpicture}
\end{center}

\textbf{关键时刻:}
\begin{itemize}
    \item $t_1$:如果此时 $e$ 还很大,控制器用错误的 $\hat{x}$ 计算 $u = -K\hat{x}$,性能糟糕!
    \item $t_2$:观测器收敛后,控制器才能正常工作
\end{itemize}

\textbf{结论:}观测器收敛速度必须 $\gg$ 控制器,才不拖后腿!

\tcblower

\textbf{📐 经验法则(黄金比例)}

设控制器主导极点为 $s_c = -\sigma_c \pm j\omega_c$,则观测器极点选择:

\[\boxed{s_o = -(k \cdot \sigma_c) \pm j\omega_o, \quad k \in [2, 5]}\]

\begin{center}
\renewcommand{\arraystretch}{1.5}
\begin{tabular}{@{} c c l @{}}
\toprule
\textbf{倍数 $k$} & \textbf{设计类型} & \textbf{特点} \\
\midrule
2 & 保守设计 & 噪声敏感性低,收敛较慢 \\
3-4 & \textcolor{blue}{\textbf{推荐}} & \textcolor{blue}{\textbf{平衡设计,工程常用}} \\
5+ & 激进设计 & 收敛极快,但噪声放大严重 \\
\bottomrule
\end{tabular}
\end{center}

\textbf{✅ 设计范例:}

假设控制器极点为 $-2 \pm j3$($\sigma_c = 2$),则观测器极点可选:

\begin{itemize}
    \item \textbf{方案1(保守)}:$-4, -5$ \quad ($k=2\sim 2.5$)
    \item \textbf{方案2(推荐)}:$-8, -10$ \quad ($k=4\sim 5$)
    \item \textbf{方案3(激进)}:$-12, -15$ \quad ($k=6\sim 7.5$)$\leftarrow$ 噪声敏感!
\end{itemize}

\end{tcolorbox}

\vspace{0.3cm}

\begin{tcolorbox}[colback=purple!5!white, colframe=purple!75!black, title=\textbf{第三步:计算观测器增益 $L$}]

\textbf{好消息:}有了对偶性,计算 $L$ 变得异常简单!

\vspace{0.3cm}

\textbf{💻 方法1:MATLAB 一行搞定(推荐)}

\begin{lstlisting}[style=Matlab-editor]
% 已知:A, C, 期望极点 poles = [-8, -10, ...]
L = place(A', C', poles)';  % 注意三个关键点:
%       ^^  ^^         ^^
%       |   |          |
%       |   |          +--- 最后转置!
%       |   +-------------- C 也要转置
%       +------------------ A 转置

% 为什么这样做?
% place() 本来是设计控制器 K: A-BK
% 通过对偶性 (A,B) <-> (A',C'),我们"欺骗"它设计观测器
% 得到的是 K',再转置回来就是 L

% 验证设计
eig(A - L*C)  % 应该等于 poles(可能顺序不同)
\end{lstlisting}

\vspace{0.2cm}

\textbf{💻 方法2:Ackermann 公式(小规模系统)}

\begin{lstlisting}[style=Matlab-editor]
L = acker(A', C', poles)';  
% acker 适用于 SISO 或小规模系统
% place 更通用,支持多输出系统
\end{lstlisting}

\vspace{0.2cm}

\textbf{💻 方法3:手工计算($n \leq 2$ 时可行)}

对于二阶系统 $(n=2)$,可以直接求解:
\[\det(sI - (A - LC)) = (s - \mu_1)(s - \mu_2)\]
展开后对比系数,解线性方程组得 $L$。

\textbf{示例($n=2$):}
\begin{lstlisting}[style=Matlab-editor]
% 已知 A, C, poles = [mu1, mu2]
syms l1 l2  % 观测器增益元素
A_LC = A - [l1; l2]*C;
char_poly = det(s*eye(2) - A_LC);  % 特征多项式
desired_poly = (s - mu1)*(s - mu2);  % 期望多项式
% 对比系数,求解 l1, l2
\end{lstlisting}

\textbf{⚠️ 注意:}$n \geq 3$ 时手算极其复杂,\textbf{务必使用 MATLAB}!

\end{tcolorbox}

\vspace{0.3cm}

\begin{tcolorbox}[colback=orange!5!white, colframe=orange!75!black, title=\textbf{第四步:验证设计}]

\textbf{设计完成 $\neq$ 任务结束!}必须验证是否符合预期。

\end{tcolorbox}

\vspace{0.3cm}

\textbf{✅ 验证清单:}

\vspace{0.2cm}

\noindent\textbf{1️⃣ 极点检查}

\begin{lstlisting}[style=Matlab-editor]
designed_poles = eig(A - L*C);
fprintf('desired poles: '); disp(poles');
fprintf('actual poles: '); disp(designed_poles);
% Allow small numerical errors, position should be close
\end{lstlisting}

\vspace{0.3cm}

\noindent\textbf{2️⃣ 误差收敛仿真}

\begin{lstlisting}[style=Matlab-editor]
% Set large initial error
e0 = [5; -3; 2];
% Simulate error dynamics: de/dt = (A-LC)*e
[t, e] = ode45(@(t,e) (A-L*C)*e, [0 5], e0);
plot(t, e); xlabel('time (s)'); ylabel('error e(t)');
title('Observer error convergence');
% Check: convergence to 0 within expected time?
\end{lstlisting}

\vspace{0.3cm}

\noindent\textbf{3️⃣ 噪声敏感性测试}

\begin{lstlisting}[style=Matlab-editor]
% Add noise to output y
y_noisy = y + 0.01*randn(size(y));  % 1% noise
% Run observer, observe chattering in x_hat
% If x_hat chatters severely -> L too large, reselect poles
\end{lstlisting}

\vspace{0.3cm}

\noindent\textbf{4️⃣ 鲁棒性测试(可选)}

\begin{lstlisting}[style=Matlab-editor]
% Introduce model errors intentionally
A_actual = A * 1.1;  % Assume 10% error in A
% Run observer with original L, check if still converges
\end{lstlisting}

\vspace{0.3cm}

\textbf{⚠️ 常见问题诊断:}

\begin{center}
\renewcommand{\arraystretch}{1.5}
\begin{tabular}{@{} p{4cm} p{6cm} @{}}
\toprule
\textbf{现象} & \textbf{可能原因与解决方案} \\
\midrule
误差不收敛 & • 检查能观测性 \\
 & • 检查极点是否在左半平面 \\
 & • 验证 \texttt{eig(A-L*C)} \\
\midrule
收敛过慢 & • 观测器极点离虚轴太近 \\
 & • 增大极点实部(更靠左) \\
\midrule
$\hat{x}$ 剧烈抖动 & • 增益 $L$ 过大,噪声放大 \\
 & • 减小极点实部(不要太激进) \\
 & • 考虑滤波器 \\
\midrule
计算溢出/不稳定 & • 极点过于极端(如 $-1000$) \\
 & • 数值精度问题,调整极点 \\
\bottomrule
\end{tabular}
\end{center}

\subsubsection{观测器极点选择准则}

\begin{tcolorbox}[colback=red!5!white, colframe=red!75!black, title=\textbf{⚠️ 极点配置的实际限制——不是越快越好!}]

虽然理论上可以任意配置极点,但实际工程中存在\textbf{四大约束}:

\vspace{0.3cm}

\textbf{1️⃣ 测量噪声约束}

\begin{center}
\begin{tikzpicture}[>=stealth, thick]
    % 关系链
    \node[draw, rectangle, fill=blue!10] (poles) {极点左移};
    \node[draw, rectangle, fill=orange!10, right=1.5cm of poles] (L) {增益 $L$ 增大};
    \node[draw, rectangle, fill=red!10, right=1.5cm of L] (noise) {噪声放大};
    
    \draw[->, very thick] (poles) -- (L);
    \draw[->, very thick] (L) -- (noise);
    
    \node[below=0.5cm of L, text width=8cm, align=center] {
        {\small 观测器方程:$\dot{\hat{x}} = A\hat{x} + Bu + L(y - C\hat{x})$} \\
        {\small $L$ 大 $\Rightarrow$ 测量噪声直接放大到 $\hat{x}$ 中}
    };
\end{tikzpicture}
\end{center}

\textbf{后果:}$\hat{x}$ 抖动 $\to$ $u = -K\hat{x}$ 抖动 $\to$ 执行器磨损、能耗增加

\textbf{折中:}收敛速度 vs 噪声抑制

\vspace{0.3cm}

\textbf{2️⃣ 计算负担约束}

\begin{itemize}
    \item 观测器需要\textbf{实时}运行(每个采样周期更新 $\hat{x}$)
    \item 极点极快 $\Rightarrow$ 需要极小的采样时间 $T_s$(数字实现时)
    \item 嵌入式处理器可能\textbf{算不过来}
\end{itemize}

\textbf{经验:}采样频率应至少为观测器最快极点频率的 \textbf{10~20倍}

\vspace{0.3cm}

\textbf{3️⃣ 模型误差约束}

\begin{center}
\begin{tikzpicture}[>=stealth, thick, node distance=2cm]
    \node[draw, ellipse, fill=green!10] (model) {理论模型 $A, B, C$};
    \node[draw, ellipse, fill=red!10, right=3cm of model] (real) {实际系统};
    
    \draw[<->, dashed, thick, red] (model) -- node[above] {误差!} (real);
    
    \node[below=0.5cm of model, text width=3cm, align=center] {
        {\small 观测器依赖这些}
    };
    
    \node[below=0.5cm of real, text width=3cm, align=center] {
        {\small 参数漂移、非线性}
    };
\end{tikzpicture}
\end{center}

\textbf{问题:}高增益观测器对模型误差\textcolor{red}{\textbf{不鲁棒}}(误差会被放大)

\textbf{解决:}
\begin{itemize}
    \item 不追求过快的极点
    \item 使用鲁棒观测器设计(如 $H_\infty$ 观测器)
    \item 在线参数辨识
\end{itemize}

\vspace{0.3cm}

\textbf{4️⃣ 执行器饱和约束}

\begin{itemize}
    \item 观测器初始误差 $e(0)$ 可能很大(如启动时刻)
    \item 快速极点 $\Rightarrow$ $\hat{x}$ 剧烈变化
    \item 控制律 $u = -K\hat{x}$ 可能\textbf{超出执行器范围}(如电机最大电压)
\end{itemize}

\textbf{后果:}饱和 $\to$ 实际控制效果与设计偏离 $\to$ 性能下降甚至不稳定

\textbf{解决:}
\begin{itemize}
    \item 抗饱和(anti-windup)设计
    \item 初始化观测器状态(如有可能)
    \item 软启动策略
\end{itemize}

\end{tcolorbox}

\vspace{0.3cm}

\fcolorbox{blue}{cyan!10}{
\begin{minipage}{0.95\textwidth}
\textbf{📌 记忆卡片:极点选择的黄金准则}

\begin{enumerate}
    \item \textbf{基准:}观测器极点 = $3 \sim 4 \times$ 控制器极点实部
    \item \textbf{上限:}受测量噪声、计算能力、模型精度限制
    \item \textbf{下限:}不能比控制器慢(否则拖累性能)
    \item \textbf{迭代:}设计 $\to$ 仿真 $\to$ 调整,反复优化
\end{enumerate}

\textbf{记住:}\textit{极点选择是艺术,不是科学——需要经验和权衡!}
\end{minipage}
}

\textbf{3. 观测器极点的典型配置模式}

\textbf{实极点配置(无超调):}
\begin{itemize}
    \item 适用于噪声较大的系统
    \item 避免振荡放大噪声
    \item 例:控制器极点 $-3, -4$,观测器极点 $-10, -12$
\end{itemize}

\textbf{Bessel型配置(最优延迟):}
\begin{itemize}
    \item 最小化估计延迟
    \item 适合快速跟踪应用
\end{itemize}

\textbf{经验公式(二阶系统):}

控制器:$s_{c} = -\zeta_c\omega_{nc} \pm j\omega_{nc}\sqrt{1-\zeta_c^2}$

观测器:$s_o = -\alpha\zeta_c\omega_{nc} \pm j\alpha\omega_{nc}\sqrt{1-\zeta_c^2}$

其中 $\alpha = 3 \sim 5$(加速因子)。

\subsubsection{对偶性的具体应用}

\textbf{利用极点配置工具设计观测器:}

\textbf{步骤:}
\begin{enumerate}
    \item 构造对偶系统:$(A^T, C^T, B^T)$
    \item 用极点配置方法(如阿克曼公式)为对偶系统设计控制器增益:
    \[K_{\text{dual}} = [0 \quad \cdots \quad 0 \quad 1] W_o^{-1} \alpha_o(A^T)\]
    其中 $W_o = [C^T \quad A^TC^T \quad \cdots \quad (A^T)^{n-1}C^T]$
    \item 转换回观测器增益:
    \[L = K_{\text{dual}}^T\]
\end{enumerate}

\textbf{MATLAB实现:}
\begin{verbatim}
% 方法1:直接设计观测器
L = place(A', C', observer_poles)';

% 方法2:利用对偶性
K_dual = acker(A', C', observer_poles);
L = K_dual';

% 验证
eig(A - L*C)  % 应等于 observer_poles
\end{verbatim}

\textbf{为什么对偶性如此重要?}
\begin{itemize}
    \item 所有极点配置的算法(直接法、变换法、阿克曼公式)都可以\textbf{直接用于}观测器设计
    \item 理论统一:能控性理论 $\leftrightarrow$ 能观测性理论
    \item 工具复用:同一套代码,只需转置矩阵
\end{itemize}

\subsubsection{观测器增益计算范例:倒立摆系统}

\paragraph*{范例说明}
通过一个经典的倒立摆系统,详细展示观测器增益 $L$ 的计算步骤,并深入分析其物理意义及对实际问题的考量。

\paragraph{1. 问题背景与系统模型}

考虑一个线性化的倒立摆系统:
\begin{itemize}
    \item 状态变量:$x_1 = \theta$(摆角),$x_2 = \dot{\theta}$(角速度)
    \item 控制输入:$u$(施加在底座上的水平力)
    \item 测量输出:$y = \theta$(\textbf{只能}通过编码器测量摆角)
\end{itemize}

系统模型如下:
\begin{align*}
\dot{x} &= \begin{bmatrix} 0 & 1 \\ 2 & 0 \end{bmatrix} x + \begin{bmatrix} 0 \\ 1 \end{bmatrix} u \\
y &= \begin{bmatrix} 1 & 0 \end{bmatrix} x
\end{align*}

\textbf{系统特性分析:}

计算开环极点:
\[\det(sI - A) = \det\begin{bmatrix} s & -1 \\ -2 & s \end{bmatrix} = s^2 - 2 = 0\]

极点:$s = \pm\sqrt{2} \approx \pm 1.414$

\textbf{关键观察:}
\begin{itemize}
    \item 存在一个\textbf{正实部极点} $+\sqrt{2}$,说明系统开环\textbf{不稳定}
    \item 物理意义:倒立摆在没有控制时会倾倒
    \item 角速度 $\dot{\theta}$ 无法直接测量(需要昂贵的陀螺仪或噪声敏感的数值微分)
\end{itemize}

\paragraph{2. 设计任务与系统分析}

\textbf{任务:}设计一个状态观测器,用于估计不可直接测量的角速度 $x_2 = \dot{\theta}$。

\textbf{能观测性检查:}
\[W_o = \begin{bmatrix} C \\ CA \end{bmatrix} = \begin{bmatrix} 1 & 0 \\ 0 & 1 \end{bmatrix}\]

$\det(W_o) = 1 \neq 0$,系统\textbf{完全能观测},因此可以任意配置观测器极点。

\textbf{能观测性的物理意义:}
\begin{itemize}
    \item 从摆角 $\theta$ 的测量值可以推断角速度 $\dot{\theta}$
    \item 第一个状态直接可测:$y = x_1$
    \item 第二个状态可从 $y$ 的变化率推断:$\dot{y} = Cx_2$
\end{itemize}

\paragraph{3. 选择观测器极点}

\textbf{设计考虑:}
\begin{itemize}
    \item 由于原系统\textbf{不稳定}(极点 $+\sqrt{2}$),估计误差必须\textbf{快速收敛}
    \item 观测器要为控制器提供可靠的状态估计
    \item 同时要避免增益过大,以抑制测量噪声
\end{itemize}

\textbf{极点选择策略:}

我们选择一对位于实轴上且比系统不稳定极点快得多的极点:
\[\mu_1 = -6, \quad \mu_2 = -8\]

\textbf{选择理由:}
\begin{itemize}
    \item \textbf{实极点}:避免振荡,减少噪声放大
    \item \textbf{足够快}:$|-6| \gg |\sqrt{2}|$,保证估计误差快速收敛
    \item \textbf{不过度激进}:不是 -20, -30 这样的极端值,平衡收敛速度与噪声抑制
\end{itemize}

\textbf{收敛时间估算:}
\[t_s \approx \frac{4}{|\text{Re}(\mu_{\min})|} = \frac{4}{6} \approx 0.67 \text{ 秒}\]

误差在不到 1 秒内收敛到 2\% 以内。

\textbf{期望特征多项式:}
\[\alpha_o(s) = (s + 6)(s + 8) = s^2 + 14s + 48\]

\paragraph{4. 计算观测器增益 L}

\textbf{方法:直接法}

设 $L = \begin{bmatrix} l_1 \\ l_2 \end{bmatrix}$,计算误差动态矩阵 $A - LC$:
\begin{align*}
A - LC &= \begin{bmatrix} 0 & 1 \\ 2 & 0 \end{bmatrix} - \begin{bmatrix} l_1 \\ l_2 \end{bmatrix} \begin{bmatrix} 1 & 0 \end{bmatrix} \\
&= \begin{bmatrix} 0 & 1 \\ 2 & 0 \end{bmatrix} - \begin{bmatrix} l_1 & 0 \\ l_2 & 0 \end{bmatrix} \\
&= \begin{bmatrix} -l_1 & 1 \\ 2 - l_2 & 0 \end{bmatrix}
\end{align*}

其特征方程为:
\begin{align*}
\det(sI - (A - LC)) &= \det\begin{bmatrix} s + l_1 & -1 \\ l_2 - 2 & s \end{bmatrix} \\
&= (s + l_1) \cdot s - (-1) \cdot (l_2 - 2) \\
&= s^2 + l_1 s + l_2 - 2
\end{align*}

将此多项式与期望的 $\alpha_o(s)$ 进行系数匹配:
\[s^2 + l_1 s + l_2 - 2 = s^2 + 14s + 48\]

比较系数可得:
\begin{align*}
l_1 &= 14 \\
l_2 - 2 &= 48 \quad \Rightarrow \quad l_2 = 50
\end{align*}

\textbf{结果:}
\[\boxed{L = \begin{bmatrix} 14 \\ 50 \end{bmatrix}}\]

\paragraph{5. 验证设计}

\textbf{检查闭环极点:}
\[A - LC = \begin{bmatrix} -14 & 1 \\ -48 & 0 \end{bmatrix}\]

特征值计算:
\[\det(sI - (A-LC)) = s^2 + 14s + 48 = (s+6)(s+8)\]

极点:$s = -6, -8$ ✓(与设计目标一致)

\paragraph{6. 结果分析与讨论}

\textbf{观测器增益的物理意义:}

观测器方程为:
\[\dot{\hat{x}} = A\hat{x} + Bu + L(y - C\hat{x})\]

展开为:
\begin{align*}
\dot{\hat{x}}_1 &= \hat{x}_2 + 14(y - \hat{x}_1) \\
\dot{\hat{x}}_2 &= 2\hat{x}_1 + u + 50(y - \hat{x}_1)
\end{align*}

增益 $L = [14, 50]^T$ 的含义:
\begin{itemize}
    \item $l_1 = 14$:当测量角度 $y$ 与估计角度 $\hat{x}_1$ 存在误差时,该误差会以 \textbf{14 倍}的增益来校正角度估计的变化率 $\dot{\hat{x}}_1$
    \item $l_2 = 50$:同时,该角度误差会以 \textbf{50 倍}的更大增益来校正角速度估计的变化率 $\dot{\hat{x}}_2$
    \item \textbf{角速度估计更依赖校正}:因为 $x_2$ 不可测,需要更强的校正力度
\end{itemize}

\textbf{噪声敏感性分析:}

假设测量值 $y$ 存在一个幅值为 $w$ 的高频噪声:$y = \theta + w$。

这个噪声会通过增益 $L$ 被放大并影响状态估计:
\begin{itemize}
    \item 对 $\dot{\hat{x}}_1$ 的影响:$14w$
    \item 对 $\dot{\hat{x}}_2$ 的影响:$50w$(\textbf{放大 50 倍}!)
\end{itemize}

\textbf{实际权衡:}

如果测量噪声 $w$ 的标准差为 0.01 弧度(约 0.57°),则:
\begin{itemize}
    \item 角速度估计的噪声:$50 \times 0.01 = 0.5$ rad/s
    \item 这可能导致控制输入抖动
\end{itemize}

\textbf{调整策略:}

如果实际噪声过大,可以考虑:
\begin{enumerate}
    \item \textbf{放慢观测器极点}:如选择 $-4, -5$(代价:收敛变慢)
    \item \textbf{增加滤波器}:对测量信号 $y$ 预先滤波
    \item \textbf{使用卡尔曼滤波器}:在随机噪声存在时的最优观测器
\end{enumerate}

\textbf{对比不同极点选择:}

\begin{center}
\renewcommand{\arraystretch}{1.6}
\begin{tabular}{|c|c|c|c|}
\hline
\rowcolor[gray]{0.9}
\textbf{观测器极点} & \textbf{$L$} & \textbf{收敛时间} & \textbf{噪声放大} \\
\hline
$-4, -5$ & $[9, 17]^T$ & 1.0 s & 低 \\
\hline
$-6, -8$ & $[14, 50]^T$ & 0.67 s & 中等 \\
\hline
$-10, -12$ & $[22, 118]^T$ & 0.4 s & 高 \\
\hline
\end{tabular}
\end{center}

\textbf{范例总结:}

这个倒立摆例子展示了:
\begin{itemize}
    \item 观测器设计的\textbf{完整流程}:能观测性检查 → 极点选择 → 增益计算 → 验证
    \item 增益矩阵的\textbf{物理意义}:校正力度与状态的可测性相关
    \item \textbf{实际权衡}:收敛速度 vs 噪声敏感性
    \item 不同极点选择的\textbf{对比分析}
\end{itemize}

\subsection{降维状态观测器}

\subsubsection*{本节目的}
当输出 $y = Cx$ 中已经直接包含部分状态信息时,我们\textbf{不需要估计所有状态},只需估计不可测的状态,从而构造计算量更小的\textbf{降维观测器}。

\subsubsection{问题的提出}

\textbf{全维观测器的"浪费":}

考虑一个简单例子:
\begin{itemize}
    \item 系统状态:$x = \begin{bmatrix} x_1 \\ x_2 \end{bmatrix}$
    \item 输出:$y = x_1$ \quad (\textcolor{blue}{$x_1$ 可以直接测量!})
    \item 全维观测器:估计 $\hat{x}_1$ 和 $\hat{x}_2$ (2个状态)
\end{itemize}

\textbf{问题:}既然 $x_1$ 已知,为什么还要"估计"它?这不是浪费计算资源吗?

\begin{tcolorbox}[colback=yellow!10, colframe=orange!75!black, title=\textbf{降维观测器的核心思想}]
\textbf{只估计不可测的状态,直接使用可测的状态}

\begin{itemize}
    \item 全维观测器:估计 $n$ 个状态(阶数 $n$)
    \item \textcolor{red!80!black}{\textbf{降维观测器}}:只估计 $(n-p)$ 个不可测状态(阶数 $n-p$)
    \item 其中 $p$ 是输出维数(可测状态数量)
\end{itemize}

\textbf{优势:}
\begin{itemize}
    \item 计算量减少(低阶微分方程)
    \item 避免对已知状态的"重复估计"
    \item 特别适合传感器部分失效的情况
\end{itemize}
\end{tcolorbox}

\begin{center}
\begin{tikzpicture}[>=stealth, thick, node distance=2.5cm]
    % 全维观测器
    \node[draw, rectangle, rounded corners, fill=blue!10, minimum width=3cm, minimum height=2cm, align=center] (full) at (0,0) {
        \textbf{全维观测器} \\[0.3cm]
        估计所有状态 \\
        $\hat{x} \in \mathbb{R}^n$ \\[0.2cm]
        {\small 阶数:$n$}
    };
    
    % 降维观测器
    \node[draw, rectangle, rounded corners, fill=green!10, minimum width=3cm, minimum height=2cm, align=center] (reduced) at (6,0) {
        \textbf{降维观测器} \\[0.3cm]
        只估计不可测 \\
        $\hat{x}_b \in \mathbb{R}^{n-p}$ \\[0.2cm]
        {\small 阶数:$n-p$}
    };
    
    % 输入输出
    \node[above=0.5cm of full] (input) {输入:$u, y$};
    \draw[->, very thick] (input) -- (full);
    
    \node[above=0.5cm of reduced] (input2) {输入:$u, y$};
    \draw[->, very thick] (input2) -- (reduced);
    
    % 优势标注
    \node[below=0.8cm of reduced, text width=3.5cm, align=center, fill=yellow!20, rounded corners] {
        {\small \textcolor{green!60!black}{\checkmark} 计算量小} \\
        {\small \textcolor{green!60!black}{\checkmark} 效率高}
    };
    
    % 对比箭头
    \draw[->, dashed, thick, red!60!black] (full.south) -- ++(0,-1.5) -| node[near start, below, text width=2cm, align=center] {\small 降低维数} (reduced.south);
\end{tikzpicture}
\end{center}

\subsubsection{状态分块与系统重构}

\textbf{基本假设:}系统 $(A, B, C)$ 完全能观测,输出方程为 $y = Cx$。

\textbf{状态划分:}

将状态向量 $x$ 划分为两部分:
\begin{itemize}
    \item $x_a \in \mathbb{R}^p$:\textcolor{blue}{\textbf{可测状态}}(从输出 $y$ 可以直接获得)
    \item $x_b \in \mathbb{R}^{n-p}$:\textcolor{red!80!black}{\textbf{不可测状态}}(需要估计)
\end{itemize}

\[\boxed{x = \begin{bmatrix} x_a \\ x_b \end{bmatrix}, \quad x_a \in \mathbb{R}^p, \quad x_b \in \mathbb{R}^{n-p}}\]

\textbf{简化假设:}为了简化推导,不失一般性地假设:
\[y = \begin{bmatrix} I_p & 0 \end{bmatrix} \begin{bmatrix} x_a \\ x_b \end{bmatrix} = x_a\]

这意味着输出\textbf{直接等于}可测状态(通过坐标变换总能实现)。

\textbf{系统矩阵分块:}

相应地,将系统矩阵分块:
\[\begin{bmatrix} \dot{x}_a \\ \dot{x}_b \end{bmatrix} = \begin{bmatrix} A_{aa} & A_{ab} \\ A_{ba} & A_{bb} \end{bmatrix} \begin{bmatrix} x_a \\ x_b \end{bmatrix} + \begin{bmatrix} B_a \\ B_b \end{bmatrix} u\]

其中:
\begin{itemize}
    \item $A_{aa} \in \mathbb{R}^{p \times p}$:可测状态之间的耦合
    \item $A_{ab} \in \mathbb{R}^{p \times (n-p)}$:不可测对可测的影响
    \item $A_{ba} \in \mathbb{R}^{(n-p) \times p}$:可测对不可测的影响
    \item $A_{bb} \in \mathbb{R}^{(n-p) \times (n-p)}$:不可测状态之间的耦合
    \item $B_a \in \mathbb{R}^p$,$B_b \in \mathbb{R}^{n-p}$:输入对各部分的作用
\end{itemize}

\begin{center}
\begin{tikzpicture}[>=stealth, thick]
    % 矩阵结构可视化
    \draw[thick] (0,0) rectangle (4,4);
    \draw[thick] (0,2) -- (4,2);
    \draw[thick] (2,0) -- (2,4);
    
    % 填充颜色
    \fill[blue!20] (0,2) rectangle (2,4);
    \fill[red!20] (2,0) rectangle (4,2);
    \fill[green!20] (0,0) rectangle (2,2);
    \fill[orange!20] (2,2) rectangle (4,4);
    
    % 标注
    \node at (1,3) {$A_{aa}$};
    \node at (3,3) {$A_{ab}$};
    \node at (1,1) {$A_{ba}$};
    \node at (3,1) {$A_{bb}$};
    
    % 维度标注
    \node[above] at (1,4.2) {$p$};
    \node[above] at (3,4.2) {$n-p$};
    \node[left] at (-0.3,3) {$p$};
    \node[left] at (-0.3,1) {$n-p$};
    
    % 说明
    \node[below, text width=8cm, align=left] at (2,-0.8) {
        \textcolor{blue}{蓝色}:可测→可测 \quad
        \textcolor{orange}{橙色}:不可测→可测 \\
        \textcolor{green!60!black}{绿色}:可测→不可测 \quad
        \textcolor{red!80!black}{红色}:不可测→不可测
    };
\end{tikzpicture}
\end{center}

\subsubsection{降维观测器设计方法}

\textbf{核心思路:}从不可测状态的动态方程出发。

从分块方程的第二行:
\[\dot{x}_b = A_{ba}x_a + A_{bb}x_b + B_b u\]

这个方程告诉我们 $x_b$ 如何演化,但它依赖于未知的 $x_b$。

\textbf{巧妙的辅助变量法:}

引入辅助变量 $z \in \mathbb{R}^{n-p}$:
\[\boxed{z = \hat{x}_b - Lx_a}\]

其中:
\begin{itemize}
    \item $\hat{x}_b$:对不可测状态的估计
    \item $L \in \mathbb{R}^{(n-p) \times p}$:\textcolor{red!80!black}{\textbf{降维观测器增益}}(待设计)
    \item $x_a = y$:可测状态(已知)
\end{itemize}

\textbf{为什么要引入 z?}
\begin{enumerate}
    \item 避免直接微分 $y = x_a$(微分会放大测量噪声)
    \item 将问题转化为设计 $z$ 的动态方程
    \item $z$ 包含了历史信息的"滤波积累"
\end{enumerate}

\textbf{推导 z 的动态方程:}

对 $z$ 求导:
\begin{align*}
\dot{z} &= \dot{\hat{x}}_b - L\dot{x}_a \\
&= \dot{\hat{x}}_b - L(A_{aa}x_a + A_{ab}x_b + B_a u)
\end{align*}

我们希望 $\dot{z}$ 的方程\textbf{不依赖于未知的 $x_b$},因此设计:
\[\dot{\hat{x}}_b = A_{ba}x_a + A_{bb}\hat{x}_b + B_b u + L\dot{x}_a\]

代入后:
\begin{align*}
\dot{z} &= A_{ba}x_a + A_{bb}\hat{x}_b + B_b u + L\dot{x}_a - L\dot{x}_a \\
&= A_{ba}x_a + A_{bb}(\underbrace{z + Lx_a}_{\hat{x}_b}) + B_b u \\
&= A_{ba}x_a + A_{bb}z + A_{bb}Lx_a + B_b u \\
&= A_{bb}z + (A_{ba} + A_{bb}L)x_a + B_b u
\end{align*}

但我们还要利用 $\dot{x}_a = A_{aa}x_a + A_{ab}x_b + B_a u$,消去其中的 $x_b$。

\textbf{最终形式:}经过精心设计,降维观测器动态方程为:

\begin{tcolorbox}[colback=green!5!white, colframe=green!60!black, title=\textbf{降维观测器标准形式}]

\textbf{辅助变量动态方程:}
\[\boxed{\dot{z} = (A_{bb} - LA_{ab})z + (A_{ba} - LA_{aa})x_a + (B_b - LB_a)u}\]

\textbf{不可测状态估计:}
\[\boxed{\hat{x}_b = z + Lx_a}\]

\textbf{增益设计准则:}

选择 $L$ 使得矩阵 $(A_{bb} - LA_{ab})$ 的特征值为期望的观测器极点。

\textbf{关键观察:}
\begin{itemize}
    \item 这是一个 $(n-p)$ 阶系统(比全维的 $n$ 阶小!)
    \item $x_a = y$ 是可测的,可以直接使用
    \item 增益 $L$ 的设计类似于全维观测器的极点配置
\end{itemize}

\end{tcolorbox}

\begin{center}
\begin{tikzpicture}[>=stealth, thick, node distance=2.5cm]
    % 系统框图
    \node[circle, draw, minimum size=0.8cm] (sum1) at (0,0) {$\Sigma$};
    \node[draw, rectangle, minimum width=2cm, minimum height=1cm, fill=blue!10] (sys) at (3,0) {真实系统};
    
    % 输出
    \draw[->, very thick] (sys) -- ++(2,0) node[right] {$y = x_a$} coordinate (out);
    
    % 降维观测器
    \node[draw, rectangle, rounded corners, minimum width=3.5cm, minimum height=2.5cm, fill=green!10] (obs) at (3,-3.5) {
        \begin{tabular}{c}
            \textbf{降维观测器} \\[0.2cm]
            $\dot{z} = (A_{bb}-LA_{ab})z$ \\
            $+ (A_{ba}-LA_{aa})x_a$ \\
            $+ (B_b-LB_a)u$ \\[0.2cm]
            $\hat{x}_b = z + Lx_a$
        \end{tabular}
    };
    
    % 输入u
    \node[left=1cm of sum1] (u_in) {$u$};
    \draw[->, very thick] (u_in) -- (sum1);
    \draw[->, very thick] (sum1) -- (sys);
    
    % 反馈到观测器
    \draw[->, thick, blue] (out) -- ++(0,-2) -| node[near start, right] {可测} (obs);
    \draw[->, thick, blue] (1.5,0) |- (obs);
    
    % 输出估计
    \node[right=0.5cm of obs] (est) {$\hat{x}_b$};
    \draw[->, thick, red!80!black] (obs) -- node[above] {估计} (est);
\end{tikzpicture}
\end{center}

\subsubsection{降维观测器设计步骤}

\begin{tcolorbox}[colback=blue!5!white, colframe=blue!75!black, title=\textbf{降维观测器设计流程}]

\textbf{步骤1:状态分块}
\begin{itemize}
    \item 确定可测状态 $x_a$ 和不可测状态 $x_b$
    \item 对系统矩阵进行分块:$A \to \begin{bmatrix} A_{aa} & A_{ab} \\ A_{ba} & A_{bb} \end{bmatrix}$
    \item 对输入矩阵分块:$B \to \begin{bmatrix} B_a \\ B_b \end{bmatrix}$
\end{itemize}

\textbf{步骤2:选择观测器极点}
\begin{itemize}
    \item 根据性能要求选择 $(n-p)$ 个期望极点 $\lambda_1, \ldots, \lambda_{n-p}$
    \item 通常选择比系统极点快 2-5 倍
\end{itemize}

\textbf{步骤3:计算增益 L}
\begin{itemize}
    \item 设计 $L$ 使 $\det(sI - (A_{bb} - LA_{ab}))$ 的根为期望极点
    \item 等价于为系统 $(A_{bb}, A_{ab})$ 设计观测器增益
    \item 可以用对偶性:转化为极点配置问题
\end{itemize}

\textbf{步骤4:构造观测器方程}
\begin{itemize}
    \item $\dot{z} = (A_{bb} - LA_{ab})z + (A_{ba} - LA_{aa})x_a + (B_b - LB_a)u$
    \item $\hat{x}_b = z + Lx_a$
\end{itemize}

\textbf{步骤5:验证设计}
\begin{itemize}
    \item 检查 $(A_{bb} - LA_{ab})$ 的特征值
    \item 仿真验证估计误差收敛性
\end{itemize}

\end{tcolorbox}

\subsubsection{降维观测器设计范例}

\textbf{问题:}设计降维状态观测器,使其极点为 -10。

\textbf{系统描述:}
\[\dot{x} = \begin{bmatrix} 0 & 1 \\ 0 & -5 \end{bmatrix}x + \begin{bmatrix} 0 \\ 100 \end{bmatrix}u, \quad y = \begin{bmatrix} 1 & 0 \end{bmatrix}x\]

其中 $x = \begin{bmatrix} x_1 \\ x_2 \end{bmatrix}$,$x_1$ 可测,$x_2$ 不可测。

---

\textbf{解答:}

\paragraph{步骤1:状态分块}

\begin{itemize}
    \item 可测:$x_a = [x_1]$ (1维)
    \item 不可测:$x_b = [x_2]$ (1维)
    \item 输出:$y = x_1 = x_a$ ✓
\end{itemize}

系统矩阵分块:
\[\begin{bmatrix} \dot{x}_1 \\ \dot{x}_2 \end{bmatrix} = \begin{bmatrix} \textcolor{blue}{0} & \textcolor{orange}{1} \\ \textcolor{green!60!black}{0} & \textcolor{red!80!black}{-5} \end{bmatrix} \begin{bmatrix} x_1 \\ x_2 \end{bmatrix} + \begin{bmatrix} 0 \\ 100 \end{bmatrix} u\]

得到:
\begin{align*}
A_{aa} &= [0], \quad A_{ab} = [1] \\
A_{ba} &= [0], \quad A_{bb} = [-5] \\
B_a &= [0], \quad B_b = [100]
\end{align*}

\paragraph{步骤2:设计增益 L}

降维观测器是1阶系统(只估计 $x_2$),增益 $L$ 是标量 $l$。

期望极点为 -10,因此:
\[\det(sI - (A_{bb} - lA_{ab})) = s - (-5 - l \cdot 1) = s + 5 + l\]

令 $s + 5 + l = s + 10$,得:
\[\boxed{l = 5}\]

\paragraph{步骤3:构造观测器方程}

代入各分块矩阵:
\begin{align*}
A_{bb} - LA_{ab} &= -5 - 5 \cdot 1 = -10 \\
A_{ba} - LA_{aa} &= 0 - 5 \cdot 0 = 0 \\
B_b - LB_a &= 100 - 5 \cdot 0 = 100
\end{align*}

\textbf{降维观测器动态方程:}
\[\boxed{\dot{z} = -10z + 100u}\]

\textbf{状态估计方程:}
\[\boxed{\hat{x}_2 = z + 5x_1}\]

其中 $x_1$ 是可直接测量的($x_1 = y$)。

\paragraph{步骤4:物理意义解释}

\begin{center}
\begin{tikzpicture}[>=stealth, thick]
    % 信息流
    \node[draw, circle, minimum size=1.2cm, fill=blue!10] (x1) at (0,2) {$x_1$};
    \node[draw, circle, minimum size=1.2cm, fill=red!10] (x2) at (0,0) {$x_2$};
    \node[draw, rectangle, rounded corners, minimum width=2.5cm, minimum height=1.5cm, fill=green!10] (z) at (5,1) {
        \begin{tabular}{c}
            $z$ \\[0.1cm]
            {\small $\dot{z}=-10z+100u$}
        \end{tabular}
    };
    \node[draw, circle, minimum size=1.2cm, fill=orange!10] (est) at (9,0) {$\hat{x}_2$};
    
    % 箭头
    \draw[->, very thick, blue] (x1) -- node[above] {\small 可测} (3,2);
    \draw[->, very thick, red!80!black, dashed] (x2) -- node[left] {\small 不可测} (0,1);
    \draw[->, very thick] (1,2) |- (z);
    \draw[->, very thick] (z) -- node[above] {\small $+5x_1$} (est);
    
    % 输入u
    \node[above=0.3cm of z] (u) {$u$};
    \draw[->, thick] (u) -- (z);
    
    % 说明
    \node[below, text width=9cm, align=left] at (4.5,-1.5) {
        \textbf{工作原理:} \\
        1. 传感器测量 $x_1$ \\
        2. 观测器用 $u$ 和 $x_1$ 更新内部状态 $z$ \\
        3. 通过 $\hat{x}_2 = z + 5x_1$ 重构不可测状态
    };
\end{tikzpicture}
\end{center}

\textbf{极点-10的意义:}
\begin{itemize}
    \item 估计误差以 $e^{-10t}$ 速度衰减
    \item 比系统自身极点 -5 快一倍,能及时跟踪状态变化
    \item 时间常数:$\tau = 0.1$s($0.1$秒内误差减少到 $1/e \approx 37\%$)
\end{itemize}

\textbf{与全维观测器对比:}
\begin{center}
\renewcommand{\arraystretch}{1.5}
\begin{tabular}{@{}l c c@{}}
\toprule
\textbf{特性} & \textbf{全维观测器} & \textbf{降维观测器} \\
\midrule
估计状态数 & 2个 ($x_1, x_2$) & 1个 ($x_2$) \\
观测器阶数 & 2阶 & \textcolor{green!60!black}{\textbf{1阶}} \\
微分方程数 & 2个 & \textcolor{green!60!black}{\textbf{1个}} \\
增益矩阵 & $L \in \mathbb{R}^{2 \times 1}$ & $l \in \mathbb{R}$ \\
计算复杂度 & 较高 & \textcolor{green!60!black}{\textbf{低}} \\
\bottomrule
\end{tabular}
\end{center}

\paragraph{步骤5:MATLAB实现}

\begin{lstlisting}[style=Matlab-editor, caption=降维观测器仿真]
% 系统参数
A = [0 1; 0 -5];
B = [0; 100];
C = [1 0];

% 状态分块
A_aa = 0; A_ab = 1;
A_ba = 0; A_bb = -5;
B_a = 0; B_b = 100;

% 降维观测器增益(极点-10)
L = 5;

% 观测器矩阵
A_obs = A_bb - L*A_ab;  % = -10
B_obs_u = B_b - L*B_a;  % = 100
B_obs_y = A_ba - L*A_aa; % = 0

% 仿真
t = 0:0.01:2;
u = ones(size(t));  % 阶跃输入

% 真实系统
sys_real = ss(A, B, C, 0);
[y_real, ~, x_real] = lsim(sys_real, u, t);

% 降维观测器
sys_obs = ss(A_obs, [B_obs_u B_obs_y], 1, 0);
[z, ~] = lsim(sys_obs, [u; y_real'], t);

% 重构估计状态
x2_hat = z + L*y_real;

% 绘图对比
figure;
plot(t, x_real(:,2), 'b-', 'LineWidth', 2); hold on;
plot(t, x2_hat, 'r--', 'LineWidth', 1.5);
legend('真实 x_2', '估计 \hat{x}_2');
title('降维观测器估计效果');
xlabel('时间 (s)'); ylabel('状态 x_2');
grid on;
\end{lstlisting}

\textbf{预期结果:}估计值 $\hat{x}_2$ 快速收敛到真实值 $x_2$,收敛速度由极点 -10 决定。

---

\subsubsection{降维 vs 全维观测器:何时使用?}

\begin{tcolorbox}[colback=yellow!5!white, colframe=orange!75!black, title=\textbf{选择指南}]

\textbf{使用降维观测器的场景:}
\begin{itemize}
    \item[\checkmark] 部分状态可以\textbf{直接且准确}地测量
    \item[\checkmark] 计算资源有限(嵌入式系统、实时控制)
    \item[\checkmark] 传感器数量有限,但足够覆盖关键状态
    \item[\checkmark] 系统阶数较高,降维效果显著
\end{itemize}

\textbf{使用全维观测器的场景:}
\begin{itemize}
    \item[\checkmark] 设计简单,易于实现和调试
    \item[\checkmark] 所有状态都存在测量噪声(需要统一滤波)
    \item[\checkmark] 计算资源充足,不在意额外开销
    \item[\checkmark] 系统阶数本身就很低($n \leq 3$)
\end{itemize}

\textbf{实际工程建议:}
\begin{itemize}
    \item 原型设计阶段:先用全维观测器(快速验证)
    \item 优化阶段:根据性能瓶颈考虑降维
    \item 生产部署:平衡计算量、精度、可维护性
\end{itemize}

\end{tcolorbox}

\begin{center}
\begin{tikzpicture}[>=stealth, thick]
    % 决策树
    \node[draw, rectangle, rounded corners, fill=blue!10, minimum width=3cm] (start) at (0,0) {
        测量部分状态?
    };
    
    \node[draw, rectangle, rounded corners, fill=green!10, below left=1.5cm and 1.5cm of start] (reduced) {
        \textbf{降维观测器} \\
        {\small 效率高}
    };
    
    \node[draw, rectangle, rounded corners, fill=orange!10, below right=1.5cm and 1.5cm of start] (full) {
        \textbf{全维观测器} \\
        {\small 简单通用}
    };
    
    \draw[->, very thick] (start) -- node[left, sloped, above] {是} (reduced);
    \draw[->, very thick] (start) -- node[right, sloped, above] {否} (full);
    
    % 进一步考虑
    \node[below=0.8cm of reduced, text width=3.5cm, align=center, fill=yellow!20, rounded corners] (q1) {
        计算资源\\紧张?
    };
    
    \node[below=0.8cm of full, text width=3.5cm, align=center, fill=yellow!20, rounded corners] (q2) {
        系统阶数\\很高?
    };
    
    \draw[->, dashed] (reduced) -- (q1);
    \draw[->, dashed] (full) -- (q2);
\end{tikzpicture}
\end{center}

---

\subsubsection*{降维观测器小结}

\textbf{核心思想:}
\begin{itemize}
    \item 只估计不可测的 $(n-p)$ 个状态,避免浪费计算资源
    \item 通过辅助变量 $z$ 构造低阶观测器
    \item 增益设计等价于 $(A_{bb}, A_{ab})$ 系统的观测器问题
\end{itemize}

\textbf{设计要点:}
\begin{itemize}
    \item 状态分块:明确可测 $x_a$ 和不可测 $x_b$
    \item 矩阵分块:$A, B$ 按维度正确划分
    \item 增益计算:配置 $(A_{bb} - LA_{ab})$ 的极点
    \item 方程构造:$\dot{z}$ 和 $\hat{x}_b$ 的标准形式
\end{itemize}

\textbf{实际优势:}
\begin{itemize}
    \item 计算量:$(n-p)$ vs $n$(降维显著)
    \item 实时性:低阶微分方程求解更快
    \item 灵活性:适应传感器配置变化
\end{itemize}

降维观测器是现代控制工程中的重要工具,特别是在资源受限的嵌入式系统中!

\subsection{分离定理}

\subsubsection*{本节目的}
理解现代控制理论的\textbf{核心定理}之一:控制器和观测器可以独立设计,且联合系统的性能可以预测。

\subsubsection{基于观测器的状态反馈}

\textbf{问题背景:}

\begin{itemize}
    \item 极点配置需要:$u = -Kx$(需要完整状态 $x$)
    \item 实际情况:$x$ 不可测,只有估计值 $\hat{x}$
\end{itemize}

\textbf{实际控制律:}
\[\boxed{u = -K\hat{x} + v}\]

用\textbf{估计状态}代替\textbf{真实状态}!

\textbf{直观担忧:}
\begin{itemize}
    \item 用 $\hat{x}$ 代替 $x$ 会破坏闭环极点吗?
    \item 控制器和观测器的极点会相互干扰吗?
    \item 我们需要重新设计 $K$ 和 $L$ 吗?
\end{itemize}

\textbf{分离定理的答案:}
\begin{quote}
\textit{不用担心!控制器和观测器可以\textbf{完全独立}设计,就像 $x$ 真的可测一样。}
\end{quote}

\subsubsection{分离定理(Separation Principle)}

\textbf{定理陈述:}

对于系统:
\begin{align*}
\dot{x} &= Ax + Bu \\
y &= Cx
\end{align*}

采用基于观测器的状态反馈:
\begin{align*}
u &= -K\hat{x} + v \\
\dot{\hat{x}} &= A\hat{x} + Bu + L(y - C\hat{x})
\end{align*}

闭环系统的特征多项式等于:
\[\boxed{\det(sI - A + BK) \cdot \det(sI - A + LC)}\]

即:\textbf{控制器特征多项式 × 观测器特征多项式}

\textbf{定理的深刻含义:}

\begin{enumerate}
    \item \textbf{极点分离}:闭环系统的极点 = 控制器极点 + 观测器极点
    \begin{itemize}
        \item 控制器极点:由 $K$ 单独决定(如上一章设计)
        \item 观测器极点:由 $L$ 单独决定(本章设计)
        \item 两者\textbf{不相互影响}
    \end{itemize}
    
    \item \textbf{独立设计}:可以分两步设计
    \begin{itemize}
        \item 步骤1:假设 $x$ 可测,设计 $K$(极点配置)
        \item 步骤2:设计观测器,配置 $L$(观测器设计)
        \item 无需迭代或联合优化
    \end{itemize}
    
    \item \textbf{性能保证}:闭环稳定性由两者的\textit{最慢}极点决定
    \begin{itemize}
        \item 若控制器和观测器都稳定 $\to$ 整体系统稳定
        \item 响应速度由两者的\textit{瓶颈}决定
    \end{itemize}
\end{enumerate}

\subsubsection{分离定理的证明(思路)}

\textbf{定义增广状态向量:}
\[\xi = \begin{bmatrix} x \\ e \end{bmatrix}, \quad e = x - \hat{x}\]

\textbf{推导增广系统:}

真实系统:
\[\dot{x} = Ax + Bu = Ax + B(-K\hat{x}) = Ax - BK(x - e) = (A - BK)x + BKe\]

误差动态(与 $u$ 无关):
\[\dot{e} = (A - LC)e\]

\textbf{增广形式:}
\[\begin{bmatrix} \dot{x} \\ \dot{e} \end{bmatrix} = \begin{bmatrix} A - BK & BK \\ 0 & A - LC \end{bmatrix} \begin{bmatrix} x \\ e \end{bmatrix}\]

\textbf{关键观察:}矩阵是\textbf{块上三角}形式!

特征多项式:
\[\det(sI - \begin{bmatrix} A - BK & BK \\ 0 & A - LC \end{bmatrix}) = \det(sI - A + BK) \cdot \det(sI - A + LC)\]

$\square$(证毕)

\textbf{块上三角的意义:}
\begin{itemize}
    \item $e$ 的动态\textbf{不依赖} $x$(误差自己演化)
    \item $x$ 的动态受 $e$ 影响(但不改变极点位置)
    \item 极点 = 对角块的极点之并集
\end{itemize}

\subsubsection{分离定理的实际意义}

\textbf{1. 简化设计流程}

\textbf{无需分离定理的情况:}
\begin{itemize}
    \item 联合设计 $K$ 和 $L$
    \item 求解耦合的非线性方程
    \item 计算量巨大,难以直观理解
\end{itemize}

\textbf{有分离定理:}
\begin{itemize}
    \item 步骤1:设计 $K$(假设状态可测)
    \item 步骤2:设计 $L$(配置观测器极点)
    \item 两步独立,清晰明了
\end{itemize}

\textbf{2. 设计自由度}

\begin{itemize}
    \item $K$:根据\textbf{性能要求}选择(响应速度、超调量)
    \item $L$:根据\textbf{估计需求}选择(收敛速度、噪声抑制)
    \item 两者有不同的设计准则,互不干扰
\end{itemize}

\textbf{3. 鲁棒性考虑}

\textbf{理想情况:}
\begin{itemize}
    \item 模型准确:$A, B, C$ 完全已知
    \item 无噪声:测量 $y$ 无误差
    \item 分离定理\textbf{严格成立}
\end{itemize}

\textbf{实际情况:}
\begin{itemize}
    \item 模型误差:$\Delta A, \Delta B$
    \item 测量噪声:$y = Cx + w$
    \item 分离定理\textbf{近似成立}(鲁棒性分析需要额外工具)
\end{itemize}

\textbf{启示:}
\begin{itemize}
    \item 观测器极点不宜过快(对模型误差敏感)
    \item 需要在收敛速度与鲁棒性间折中
    \item 可能需要\textit{鲁棒观测器}设计(如 $H_\infty$ 观测器)
\end{itemize}

\subsubsection{完整系统的设计流程}

\textbf{标准设计程序:}

\begin{enumerate}
    \item \textbf{系统分析}
    \begin{itemize}
        \item 建立状态空间模型 $(A, B, C)$
        \item 验证能控性:$\text{rank}([B \quad AB \quad \cdots \quad A^{n-1}B]) = n$
        \item 验证能观测性:$\text{rank}(\begin{bmatrix} C \\ CA \\ \vdots \\ CA^{n-1} \end{bmatrix}) = n$
    \end{itemize}
    
    \item \textbf{控制器设计}
    \begin{itemize}
        \item 根据性能要求选择期望闭环极点 $\lambda_1, \ldots, \lambda_n$
        \item 使用极点配置方法计算 $K$
        \item 验证:$\det(sI - A + BK) = \prod (s - \lambda_i)$
    \end{itemize}
    
    \item \textbf{观测器设计}
    \begin{itemize}
        \item 选择观测器极点 $\mu_1, \ldots, \mu_n$(通常比 $\lambda_i$ 快2-5倍)
        \item 利用对偶性计算 $L$
        \item 验证:$\det(sI - A + LC) = \prod (s - \mu_i)$
    \end{itemize}
    
    \item \textbf{实现控制律}
    \begin{itemize}
        \item 观测器:$\dot{\hat{x}} = A\hat{x} + Bu + L(y - C\hat{x})$
        \item 控制律:$u = -K\hat{x} + v$
        \item 数字实现时需要离散化
    \end{itemize}
    
    \item \textbf{验证与调试}
    \begin{itemize}
        \item 仿真增广系统:$2n$ 个状态($x$ 和 $e$)
        \item 检查瞬态响应(初始误差、阶跃输入)
        \item 评估对噪声和扰动的敏感性
        \item 必要时调整极点位置
    \end{itemize}
\end{enumerate}

\subsubsection{分离定理的局限性}

\textbf{定理成立的前提:}
\begin{itemize}
    \item \textbf{线性系统}(非线性系统不适用)
    \item \textbf{确定性系统}(无随机干扰)
    \item \textbf{模型准确}($A, B, C$ 已知且精确)
\end{itemize}

\textbf{实际中的挑战:}
\begin{itemize}
    \item 模型误差导致性能下降
    \item 测量噪声被高增益观测器放大
    \item 非线性和饱和效应
    \item 计算延迟(数字实现)
\end{itemize}

\textbf{扩展方向:}
\begin{itemize}
    \item \textbf{卡尔曼滤波器}(随机噪声情况)
    \item \textbf{鲁棒观测器}(模型不确定性)
    \item \textbf{非线性观测器}(扩展卡尔曼滤波、UKF)
    \item \textbf{自适应观测器}(在线参数估计)
\end{itemize}

\subsection{综合设计范例:直流电机位置控制}

\subsubsection*{本节目的}
将本章所学的所有概念——\textbf{控制器设计、观测器设计、分离定理}——应用于一个具体的物理系统,完整地走一遍从问题分析到系统验证的全过程。

\subsubsection{问题描述与系统建模}

\textbf{场景:}为一个直流电机设计高精度位置控制器。

\textbf{控制目标:}
\begin{itemize}
    \item 使电机角度 $\theta(t)$ 快速、准确地跟踪参考角度 $\theta_{\text{ref}}$
    \item 调节时间 $T_s \approx 1.5$ 秒
    \item 超调量 $M_p \leq 5\%$
\end{itemize}

\textbf{系统变量:}
\begin{itemize}
    \item \textbf{输入} $u(t)$:电枢电压 (V)
    \item \textbf{状态变量} $x(t)$:$x_1 = \theta(t)$ (角度, rad),$x_2 = \dot{\theta}(t)$ (角速度, rad/s)
    \item \textbf{输出} $y(t)$:$y = \theta(t)$ (仅能测量角度)
\end{itemize}

\textbf{关键约束:}
\begin{center}
\textit{角速度 $\dot{\theta}(t)$ 无法直接测量(传感器成本高或不可用)}
\end{center}

\textbf{状态空间模型}(参数已简化):
\[A = \begin{bmatrix} 0 & 1 \\ 0 & -1 \end{bmatrix}, \quad B = \begin{bmatrix} 0 \\ 1 \end{bmatrix}, \quad C = \begin{bmatrix} 1 & 0 \end{bmatrix}\]

\textbf{物理意义:}
\begin{itemize}
    \item $\dot{x}_1 = x_2$:角度的变化率是角速度(运动学)
    \item $\dot{x}_2 = -x_2 + u$:角速度受摩擦阻尼($-x_2$)和电压驱动($u$)影响
    \item $y = x_1$:只测量角度
\end{itemize}

\subsubsection{系统分析与性能指标}

\textbf{开环特性分析:}

特征方程:
\[\det(sI - A) = \det\begin{bmatrix} s & -1 \\ 0 & s+1 \end{bmatrix} = s(s+1) = 0\]

开环极点:$s = 0, -1$

\textbf{系统特性:}
\begin{itemize}
    \item 极点 $s=0$:\textbf{临界稳定}(积分器特性)
    \item 极点 $s=-1$:稳定的阻尼项
    \item 开环系统无法自动回到期望位置,需要闭环控制
\end{itemize}

\textbf{能控性检查:}
\[W_c = [B \quad AB] = \begin{bmatrix} 0 & 1 \\ 1 & -1 \end{bmatrix}\]

$\det(W_c) = -1 \neq 0$,系统\textbf{完全能控} ✓

\textbf{能观测性检查:}
\[W_o = \begin{bmatrix} C \\ CA \end{bmatrix} = \begin{bmatrix} 1 & 0 \\ 0 & 1 \end{bmatrix}\]

$\det(W_o) = 1 \neq 0$,系统\textbf{完全能观测} ✓

\textbf{结论:}可以独立设计控制器和观测器!

\textbf{性能指标转化为极点位置:}

根据二阶系统理论,要求 $T_s \approx 1.5$s,$M_p \leq 5\%$:
\begin{itemize}
    \item 超调量 $M_p = 5\%$ $\Rightarrow$ 阻尼比 $\zeta \approx 0.69$
    \item 调节时间 $T_s = \frac{4}{\zeta\omega_n} = 1.5$ $\Rightarrow$ $\omega_n \approx 3.86$ rad/s
\end{itemize}

期望的\textbf{控制器主导极点}:
\[\lambda_{1,2} = -\zeta\omega_n \pm j\omega_n\sqrt{1-\zeta^2} \approx -2.67 \pm j2.77\]

为了计算简便,我们选择接近的极点:
\[\boxed{\lambda_{1,2} = -2.5 \pm j2.5}\]

\subsubsection{控制器设计(基于分离定理的第一步)}

我们首先\textbf{假设所有状态均可测量},设计状态反馈增益 $K$。

\textbf{控制律:}$u = -Kx + v = -[k_1 \quad k_2]x + v$

\textbf{目标:}使 $A-BK$ 的特征值为 $-2.5 \pm j2.5$。

\textbf{期望特征多项式:}
\[\alpha_c(s) = (s+2.5-j2.5)(s+2.5+j2.5) = s^2 + 5s + 12.5\]

\textbf{计算 $K$(直接法):}

\[A - BK = \begin{bmatrix} 0 & 1 \\ 0 & -1 \end{bmatrix} - \begin{bmatrix} 0 \\ 1 \end{bmatrix} [k_1 \quad k_2] = \begin{bmatrix} 0 & 1 \\ -k_1 & -1-k_2 \end{bmatrix}\]

特征方程:
\[\det(sI - (A-BK)) = s(s+1+k_2) + k_1 = s^2 + (1+k_2)s + k_1\]

令其等于 $s^2 + 5s + 12.5$:
\begin{align*}
1 + k_2 &= 5 \quad \Rightarrow \quad k_2 = 4 \\
k_1 &= 12.5
\end{align*}

\textbf{结果:}
\[\boxed{K = \begin{bmatrix} 12.5 & 4 \end{bmatrix}}\]

\textbf{理想控制律:}$u = -12.5\theta - 4\dot{\theta} + v$

\textbf{问题:}$\dot{\theta}$ 是未知的!这就是为什么我们需要观测器。

\subsubsection{观测器设计(基于分离定理的第二步)}

为解决 $\dot{\theta}$ 不可测的问题,我们设计一个观测器来提供状态估计 $\hat{x}$。

\textbf{选择观测器极点:}

\textbf{原则:}观测器的收敛速度必须比控制器快,以保证估计误差 $e(t)$ 能迅速衰减。

\textbf{经验法则:}观测器极点的实部应为控制器极点实部的 2~5 倍。

控制器极点实部为 $-2.5$,我们选择 \textbf{4 倍}:$-10$。

为避免振荡(减少噪声放大),选择两个\textbf{实极点}:
\[\mu_1 = -10, \quad \mu_2 = -12\]

\textbf{收敛时间:}
\[t_s \approx \frac{4}{10} = 0.4 \text{ 秒}\]

观测器误差在 0.4 秒内收敛,远快于控制器的 1.5 秒响应时间 ✓

\textbf{期望误差动态特征多项式:}
\[\alpha_o(s) = (s+10)(s+12) = s^2 + 22s + 120\]

\textbf{计算观测器增益 $L$(直接法):}

设 $L = \begin{bmatrix} l_1 \\ l_2 \end{bmatrix}$:

\[A - LC = \begin{bmatrix} 0 & 1 \\ 0 & -1 \end{bmatrix} - \begin{bmatrix} l_1 \\ l_2 \end{bmatrix} [1 \quad 0] = \begin{bmatrix} -l_1 & 1 \\ -l_2 & -1 \end{bmatrix}\]

特征方程:
\[\det(sI - (A-LC)) = (s+l_1)(s+1) + l_2 = s^2 + (1+l_1)s + (l_1+l_2)\]

令其等于 $s^2 + 22s + 120$:
\begin{align*}
1 + l_1 &= 22 \quad \Rightarrow \quad l_1 = 21 \\
l_1 + l_2 &= 120 \quad \Rightarrow \quad l_2 = 120 - 21 = 99
\end{align*}

\textbf{结果:}
\[\boxed{L = \begin{bmatrix} 21 \\ 99 \end{bmatrix}}\]

\subsubsection{组合系统与最终实现}

根据\textbf{分离定理},我们可以将上述独立设计的控制器和观测器安全地组合起来。

\textbf{观测器方程(实现):}
\[\dot{\hat{x}} = A\hat{x} + Bu + L(y - C\hat{x})\]

展开为:
\begin{align*}
\dot{\hat{x}}_1 &= \hat{x}_2 + 21(y - \hat{x}_1) \\
\dot{\hat{x}}_2 &= -\hat{x}_2 + u + 99(y - \hat{x}_1)
\end{align*}

\textbf{实际控制律(实现):}
\[u = -K\hat{x} + v = -12.5\hat{x}_1 - 4\hat{x}_2 + v\]

其中 $v = 12.5\theta_{\text{ref}}$ 是参考输入(使稳态误差为零)。

\textbf{完整控制系统框图:}

\begin{center}
\begin{tikzpicture}[auto, node distance=2cm, >=stealth]
    \node [draw, circle] (sum1) {$+$};
    \node [draw, rectangle, right of=sum1, node distance=2cm] (K) {$-K$};
    \node [draw, rectangle, right of=K, node distance=2.5cm] (plant) {系统 $(A,B,C)$};
    \node [draw, rectangle, below of=plant, node distance=2cm] (obs) {观测器};
    \node [coordinate, right of=plant, node distance=2cm] (output) {};
    \node [coordinate, left of=sum1, node distance=2cm] (input) {};
    
    \draw [->] (input) -- node {$v$} (sum1);
    \draw [->] (sum1) -- node {$u$} (K);
    \draw [->] (K) -- (plant);
    \draw [->] (plant) -- node [name=y] {$y$} (output);
    \draw [->] (y) |- (obs);
    \draw [->] (obs) -| node [pos=0.9] {$\hat{x}$} (K);
    \draw [->] (K) |- (obs);
\end{tikzpicture}
\end{center}

\subsubsection{系统整体特性分析}

\textbf{系统阶数:}

最终的闭环系统是一个 \textbf{4 阶系统}(2个物理状态 + 2个观测器状态)。

\textbf{系统总极点}(根据分离定理):
\[\text{极点} = \{ \underbrace{-2.5 \pm j2.5}_{\text{控制器极点,决定系统响应}}, \underbrace{-10, -12}_{\text{观测器极点,决定估计误差收敛}} \}\]

\textbf{定性分析:}

\begin{enumerate}
    \item \textbf{初始阶段}($t < 0.4$s):
    \begin{itemize}
        \item 观测器快速收敛(由 $-10, -12$ 决定)
        \item 估计状态 $\hat{x}$ 迅速逼近真实状态 $x$
        \item 此时控制效果还未完全发挥
    \end{itemize}
    
    \item \textbf{主导阶段}($t > 0.4$s):
    \begin{itemize}
        \item 观测器误差 $e \approx 0$,即 $\hat{x} \approx x$
        \item 系统动态主要由控制器极点($-2.5 \pm j2.5$)主导
        \item 展现设计的响应性能:$T_s \approx 1.5$s,$M_p \leq 5\%$
    \end{itemize}
\end{enumerate}

\textbf{稳定性分析:}

所有极点均在左半平面 $\Rightarrow$ 系统\textbf{渐近稳定}!

\textbf{响应速度:}

\begin{itemize}
    \item 观测器误差收敛:$t_s^{(o)} \approx 0.4$s(非常快)
    \item 系统输出响应:$t_s^{(c)} \approx 1.5$s(符合设计目标)
\end{itemize}

由于观测器远快于控制器,系统整体响应由控制器主导 ✓

\subsubsection{设计验证与性能评估}

\textbf{验证步骤:}

\begin{enumerate}
    \item \textbf{极点验证}
    \begin{itemize}
        \item 控制器闭环极点:$\det(sI - A + BK) = s^2 + 5s + 12.5$ ✓
        \item 观测器极点:$\det(sI - A + LC) = s^2 + 22s + 120$ ✓
    \end{itemize}
    
    \item \textbf{仿真测试}
    \begin{itemize}
        \item 初始条件:$x(0) = [0, 0]^T$,$\hat{x}(0) = [0.5, 0]^T$(初始估计误差)
        \item 参考输入:$\theta_{\text{ref}} = 1$ rad(阶跃信号)
        \item 观察:估计误差 $e(t)$、系统输出 $\theta(t)$、控制输入 $u(t)$
    \end{itemize}
\end{enumerate}

\textbf{预期结果:}

\begin{itemize}
    \item 估计误差 $e(t)$ 在 0.4s 内衰减到零
    \item 系统输出 $\theta(t)$ 在 1.5s 内到达 1 rad,超调量 $< 5\%$
    \item 控制输入 $u(t)$ 无剧烈抖动(观测器增益适中)
\end{itemize}

\textbf{MATLAB实现代码:}
\begin{verbatim}
% 系统矩阵
A = [0 1; 0 -1]; B = [0; 1]; C = [1 0];

% 控制器设计
K = [12.5 4];
eig(A - B*K)  % 验证: -2.5±2.5j

% 观测器设计
L = [21; 99];
eig(A - L*C)  % 验证: -10, -12

% 增广系统仿真(4阶)
A_aug = [A-B*K, B*K; zeros(2), A-L*C];
B_aug = [B; zeros(2,1)];
C_aug = [C, zeros(1,2)];

sys = ss(A_aug, B_aug, C_aug, 0);
step(sys * 12.5);  % 阶跃响应
\end{verbatim}

\subsubsection{实际考虑与改进方向}

\textbf{实际问题:}

\begin{enumerate}
    \item \textbf{测量噪声}
    \begin{itemize}
        \item 编码器可能有噪声($\pm 0.01$ rad)
        \item 观测器增益 $L = [21, 99]^T$ 会放大噪声
        \item 角速度估计可能抖动
    \end{itemize}
    \textbf{解决:}增加低通滤波器或使用卡尔曼滤波器
    
    \item \textbf{执行器饱和}
    \begin{itemize}
        \item 电压限制:$|u| \leq u_{\max}$
        \item 初始阶段控制输入可能饱和
    \end{itemize}
    \textbf{解决:}抗饱和设计,限制参考输入变化率
    
    \item \textbf{模型误差}
    \begin{itemize}
        \item 实际摩擦系数可能不是精确的 $-1$
        \item 可能存在未建模动态
    \end{itemize}
    \textbf{解决:}鲁棒控制方法,增加积分项消除稳态误差
\end{enumerate}

\textbf{改进方向:}

\begin{itemize}
    \item \textbf{LQR控制}:用最优控制理论自动选择 $K$
    \item \textbf{卡尔曼滤波器}:在噪声环境下的最优观测器
    \item \textbf{降维观测器}:只估计 $\dot{\theta}$,减少计算量
    \item \textbf{自适应控制}:在线估计参数变化
\end{itemize}

\subsubsection*{范例总结}

这个直流电机例子完美展示了\textbf{分离定理}的威力:

\begin{enumerate}
    \item \textbf{模块化设计}:
    \begin{itemize}
        \item 步骤1:独立设计控制器 $K$(假设状态可测)
        \item 步骤2:独立设计观测器 $L$(配置快速极点)
        \item 步骤3:直接组合,无需重新设计
    \end{itemize}
    
    \item \textbf{性能可预测}:
    \begin{itemize}
        \item 系统极点 = 控制器极点 + 观测器极点
        \item 响应特性由控制器主导(观测器足够快)
    \end{itemize}
    
    \item \textbf{实际可行}:
    \begin{itemize}
        \item 减少了传感器需求(只测角度)
        \item 降低了硬件成本
        \item 通过软件实现状态估计
    \end{itemize}
\end{enumerate}

\textbf{关键经验:}
\begin{itemize}
    \item 观测器极点选择为控制器的 3-5 倍,平衡收敛与噪声
    \item 验证能控性和能观测性是设计的前提
    \item 实际系统需要考虑噪声、饱和等非理想因素
    \item 分离定理简化了复杂控制系统的设计流程
\end{itemize}

这个例子从\textbf{问题分析}到\textbf{系统建模},从\textbf{独立设计}到\textbf{系统集成},从\textbf{理论验证}到\textbf{实际考虑},完整展示了现代控制理论在实际工程中的应用!

\subsection*{本章总结}

\subsubsection*{核心要点回顾}

\textbf{1. 状态观测器的本质}

\begin{itemize}
    \item 观测器是一个\textbf{动态系统},通过模拟真实系统并利用输出误差校正来估计状态
    \item 观测器方程:$\dot{\hat{x}} = A\hat{x} + Bu + L(y - C\hat{x})$
    \item 关键创新:利用\textbf{创新}(innovation)$y - C\hat{x}$ 进行闭环校正
    \item 与传感器的区别:算法/软件 vs 物理设备
\end{itemize}

\textbf{2. 误差动态与收敛}

\begin{itemize}
    \item 估计误差:$e = x - \hat{x}$
    \item 误差动态:$\dot{e} = (A - LC)e$(\textbf{与控制输入 $u$ 无关})
    \item 收敛条件:$A - LC$ 的所有特征值在左半平面
    \item 设计目标:选择 $L$ 配置观测器极点
\end{itemize}

\textbf{3. 对偶性(Duality)}

\begin{center}
\renewcommand{\arraystretch}{1.6}
\begin{tabular}{|l|c|c|}
\hline
\rowcolor[gray]{0.9}
& \textbf{极点配置} & \textbf{观测器设计} \\
\hline
目标矩阵 & $A - BK$ & $A - LC$ \\
\hline
前提条件 & $(A, B)$ 能控 & $(A, C)$ 能观测 \\
\hline
增益设计 & $K$ & $L$ \\
\hline
对偶关系 & --- & $(A, C) \leftrightarrow (A^T, C^T)$ \\
\hline
\end{tabular}
\end{center}

\textbf{利用对偶性:}所有极点配置算法(直接法、变换法、阿克曼公式)都可直接用于观测器设计!

\textbf{4. 降维观测器(Reduced-Order Observer)}

\begin{itemize}
    \item \textbf{核心思想}:只估计不可测的 $(n-p)$ 个状态,避免浪费计算资源
    \item \textbf{状态划分}:$x_a$(可测,维数 $p$)+ $x_b$(不可测,维数 $n-p$)
    \item \textbf{辅助变量}:$z = \hat{x}_b - Lx_a$(避免直接微分输出)
    \item \textbf{降维观测器方程}:
    \[\dot{z} = (A_{bb} - LA_{ab})z + (A_{ba} - LA_{aa})x_a + (B_b - LB_a)u\]
    \[\hat{x}_b = z + Lx_a\]
    \item \textbf{优势}:计算量小($(n-p)$ vs $n$),实时性好,适合嵌入式系统
    \item \textbf{应用场景}:部分状态可直接准确测量,计算资源受限
\end{itemize}

\begin{center}
\renewcommand{\arraystretch}{1.5}
\begin{tabular}{@{}l c c@{}}
\toprule
\textbf{特性} & \textbf{全维观测器} & \textbf{降维观测器} \\
\midrule
估计状态数 & $n$ 个(所有状态) & $(n-p)$ 个(仅不可测) \\
观测器阶数 & $n$ 阶 & $(n-p)$ 阶 \\
设计复杂度 & 简单直观 & 需要矩阵分块 \\
计算效率 & 一般 & \textcolor{green!60!black}{\textbf{高}} \\
适用场景 & 通用 & 部分状态可测 \\
\bottomrule
\end{tabular}
\end{center}

\textbf{5. 分离定理(Separation Principle)}

\textbf{定理陈述:}

基于观测器的状态反馈系统特征多项式 = 控制器特征多项式 × 观测器特征多项式

\[\det(sI - A + BK) \cdot \det(sI - A + LC)\]

\textbf{定理意义:}
\begin{itemize}
    \item 控制器和观测器可以\textbf{完全独立}设计
    \item 闭环极点 = 控制器极点 + 观测器极点
    \item 无需联合优化,大大简化设计流程
\end{itemize}

\subsubsection*{设计流程总结}

\textbf{完整的基于观测器的控制系统设计:}

\begin{enumerate}
    \item \textbf{系统分析}
    \begin{itemize}
        \item 建立状态空间模型 $(A, B, C)$
        \item 验证能控性:$\text{rank}(W_c) = n$
        \item 验证能观测性:$\text{rank}(W_o) = n$
    \end{itemize}
    
    \item \textbf{控制器设计}(假设状态可测)
    \begin{itemize}
        \item 根据性能要求选择期望极点 $\lambda_1, \ldots, \lambda_n$
        \item 计算反馈增益 $K$(如阿克曼公式)
        \item 验证闭环极点
    \end{itemize}
    
    \item \textbf{观测器设计}
    \begin{itemize}
        \item 选择观测器极点 $\mu_1, \ldots, \mu_n$(通常比控制器快 2-5 倍)
        \item 利用对偶性计算 $L$
        \item 验证观测器极点
    \end{itemize}
    
    \item \textbf{系统实现}
    \begin{itemize}
        \item 观测器:$\dot{\hat{x}} = A\hat{x} + Bu + L(y - C\hat{x})$
        \item 控制律:$u = -K\hat{x} + v$
        \item 数字实现时需要离散化
    \end{itemize}
    
    \item \textbf{验证与调试}
    \begin{itemize}
        \item 仿真增广系统($2n$ 阶)
        \item 检查瞬态响应和稳态性能
        \item 评估噪声敏感性
        \item 必要时调整极点位置
    \end{itemize}
\end{enumerate}

\subsubsection*{观测器极点选择准则}

\textbf{经验法则:}

\[\text{观测器极点实部} = (2 \sim 5) \times \text{控制器极点实部}\]

\textbf{选择考虑:}

\begin{center}
\renewcommand{\arraystretch}{1.6}
\begin{tabular}{|l|p{5cm}|p{5cm}|}
\hline
\rowcolor[gray]{0.9}
\textbf{因素} & \textbf{快速极点(5倍)} & \textbf{保守极点(2倍)} \\
\hline
收敛速度 & 非常快 & 较慢 \\
\hline
噪声敏感性 & 高(大增益) & 低(小增益) \\
\hline
计算负担 & 高(小采样周期) & 低 \\
\hline
鲁棒性 & 对模型误差敏感 & 较鲁棒 \\
\hline
适用场景 & 低噪声、准确模型 & 高噪声、模型不确定 \\
\hline
\end{tabular}
\end{center}

\textbf{实用建议:}
\begin{itemize}
    \item 初次设计:选择 3-4 倍(中庸之道)
    \item 实际调试:根据噪声水平和响应速度权衡调整
    \item 实极点:适合噪声环境(避免振荡)
    \item 复数极点:适合快速跟踪应用
\end{itemize}

\subsubsection*{重要概念辨析}

\textbf{1. 观测器 vs 滤波器}

\begin{itemize}
    \item \textbf{观测器}:基于系统模型的状态估计(需要知道 $A, B, C$)
    \item \textbf{卡尔曼滤波器}:在随机噪声环境下的最优观测器(需要噪声统计特性)
    \item \textbf{低通滤波器}:简单的信号平滑(不利用系统模型)
\end{itemize}

\textbf{2. 全维观测器 vs 降维观测器}

\begin{itemize}
    \item \textbf{全维}:估计所有 $n$ 个状态(简单,本章重点)
    \item \textbf{降维}:只估计不可测的 $(n-p)$ 个状态(计算量更小)
    \item 当 $y = Cx$ 中部分状态直接可测时,降维观测器更高效
\end{itemize}

\textbf{3. 确定性观测器 vs 随机观测器}

\begin{itemize}
    \item \textbf{确定性}(本章):无随机噪声,极点配置方法
    \item \textbf{随机}(卡尔曼滤波):考虑过程噪声和测量噪声,最小化估计方差
\end{itemize}

\subsubsection*{常见误区与正确做法}

\begin{itemize}
    \item ✗ \textbf{观测器极点配置过快}
    \begin{itemize}
        \item 后果:对噪声极度敏感,控制输入抖动
        \item ✓ 正确:平衡收敛速度与噪声抑制,选择 2-5 倍控制器极点
    \end{itemize}
    
    \item ✗ \textbf{忽略能观测性检查}
    \begin{itemize}
        \item 后果:无法任意配置极点,观测器可能发散
        \item ✓ 正确:设计前必须验证 $\text{rank}(W_o) = n$
    \end{itemize}
    
    \item ✗ \textbf{认为分离定理对非线性系统成立}
    \begin{itemize}
        \item 后果:非线性系统中分离定理\textbf{不成立},需要联合设计
        \item ✓ 正确:线性系统才能独立设计控制器和观测器
    \end{itemize}
    
    \item ✗ \textbf{用数值微分代替观测器}
    \begin{itemize}
        \item 后果:噪声严重放大(如 $\dot{\theta}$ 从 $\theta$ 微分)
        \item ✓ 正确:使用观测器估计导数,自带滤波效果
    \end{itemize}
    
    \item ✗ \textbf{观测器初值随意设置}
    \begin{itemize}
        \item 后果:初始误差过大可能导致控制饱和
        \item ✓ 正确:尽量根据先验知识设置接近真实值的初值
    \end{itemize}
\end{itemize}

\subsubsection*{MATLAB工具箱}

\textbf{能观测性检查:}
\begin{verbatim}
Wo = obsv(A, C);
rank(Wo)  % 应等于 n
\end{verbatim}

\textbf{观测器设计:}
\begin{verbatim}
% 方法1:直接设计
L = place(A', C', poles)';

% 方法2:利用对偶性
L = acker(A', C', poles)';

% 验证
eig(A - L*C)  % 应等于期望极点
\end{verbatim}

\textbf{基于观测器的控制系统仿真:}
\begin{verbatim}
% 增广系统 [x; e]
A_aug = [A-B*K, B*K; zeros(n), A-L*C];
B_aug = [B; zeros(n,1)];
C_aug = [C, zeros(1,n)];

sys = ss(A_aug, B_aug, C_aug, 0);
step(sys);
\end{verbatim}

\subsubsection*{与前后章节的联系}

\textbf{与第\ref{sec:controllability-observability}章(能控性和能观测性):}
\begin{itemize}
    \item 第\ref{sec:controllability-observability}章:定义能观测性,判别准则
    \item 本章:\textbf{应用}能观测性——观测器设计
    \item 能观测性是任意配置观测器极点的\textbf{充要条件}
\end{itemize}

\textbf{与第\ref{sec:pole-placement}章(极点配置):}
\begin{itemize}
    \item 第\ref{sec:pole-placement}章:状态反馈 $u = -Kx$(假设状态可测)
    \item 本章:解决状态不可测问题——用 $\hat{x}$ 代替 $x$
    \item 分离定理:两者可以独立设计
\end{itemize}

\textbf{与后续章节(最优控制和卡尔曼滤波):}
\begin{itemize}
    \item 本章:极点配置方法设计观测器(设计者指定极点)
    \item 最优控制:LQR + 卡尔曼滤波器(自动优化极点)
    \item 线性二次高斯(LQG):随机环境下的最优设计
\end{itemize}

\subsubsection*{实际应用场景}

\textbf{1. 航空航天}
\begin{itemize}
    \item 飞行器姿态控制:测量姿态角,估计角速度
    \item 火箭导航:GPS + 惯性测量单元(IMU)融合
\end{itemize}

\textbf{2. 机器人}
\begin{itemize}
    \item 关节控制:编码器测位置,观测器估计速度和加速度
    \item 减少陀螺仪和加速度计数量
\end{itemize}

\textbf{3. 电机驱动}
\begin{itemize}
    \item 无传感器控制:只测电流和电压,估计转速和转矩
    \item 降低成本,提高可靠性
\end{itemize}

\textbf{4. 自动驾驶}
\begin{itemize}
    \item 车辆状态估计:GPS + 轮速传感器 + 观测器
    \item 估计侧向速度、横摆角速度等难以直接测量的状态
\end{itemize}

\subsubsection*{学习检查清单}

\textbf{理论理解:}
\begin{itemize}
    \item ☐ 能解释观测器的工作原理和校正机制
    \item ☐ 能推导误差动态方程 $\dot{e} = (A-LC)e$
    \item ☐ 理解对偶性:$(A, C)$ 能观测 $\leftrightarrow$ $(A^T, C^T)$ 能控
    \item ☐ 理解分离定理及其适用条件
\end{itemize}

\textbf{设计能力:}
\begin{itemize}
    \item ☐ 能验证系统的能观测性
    \item ☐ 能根据性能要求选择观测器极点
    \item ☐ 能计算观测器增益矩阵 $L$
    \item ☐ 能设计完整的控制器+观测器系统
\end{itemize}

\textbf{实践技能:}
\begin{itemize}
    \item ☐ 会使用 MATLAB 的 \texttt{place}, \texttt{acker} 设计观测器
    \item ☐ 能仿真和验证观测器性能
    \item ☐ 能分析噪声对观测器的影响
    \item ☐ 能调试和优化实际系统
\end{itemize}

\subsubsection*{延伸阅读}

\begin{itemize}
    \item \textbf{卡尔曼滤波}:随机环境下的最优观测器
    \item \textbf{滑模观测器}:对模型不确定性鲁棒的观测器
    \item \textbf{高增益观测器}:非线性系统的观测器设计
    \item \textbf{扩展卡尔曼滤波(EKF)}:非线性系统的状态估计
    \item \textbf{无迹卡尔曼滤波(UKF)}:更精确的非线性状态估计
\end{itemize}

\subsubsection*{本章核心公式}

\begin{tcolorbox}[colback=blue!5!white, colframe=blue!75!black, title=状态观测器核心公式]

\textbf{观测器方程:}
\[\dot{\hat{x}} = A\hat{x} + Bu + L(y - C\hat{x})\]

\textbf{误差动态:}
\[\dot{e} = (A - LC)e, \quad e = x - \hat{x}\]

\textbf{观测器设计定理:}

$(A, C)$ 完全能观测 $\Leftrightarrow$ 可任意配置 $A-LC$ 的极点

\textbf{分离定理:}

闭环特征多项式 = $\det(sI - A + BK) \cdot \det(sI - A + LC)$

\textbf{基于观测器的控制:}
\[u = -K\hat{x} + v\]

\end{tcolorbox}

\textbf{结语:}

状态观测器是现代控制理论的\textbf{基石之一}。它将第\ref{sec:controllability-observability}章的能观测性理论转化为实际的估计算法,与第\ref{sec:pole-placement}章的极点配置方法结合,通过分离定理构成了完整的状态反馈控制系统。

观测器的核心思想——\textbf{利用输出误差动态校正模型估计}——不仅在控制理论中至关重要,在信号处理、机器学习、导航定位等众多领域也有广泛应用。

掌握观测器设计,就掌握了从\textbf{部分信息重构完整状态}的强大工具!
