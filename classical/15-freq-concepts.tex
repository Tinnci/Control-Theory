\section{频率特性基本概念}

\subsection{频率特性的定义}
对于一个稳定的线性时不变系统,当输入一个正弦信号 $r(t) = A_{\text{in}}\sin(\omega t)$ 时,其稳态输出 $y_{\text{ss}}(t)$ 必然是同频率的正弦信号,形式为:
\[y_{\text{ss}}(t) = A_{\text{out}}\sin(\omega t + \phi)\]

线性定常系统的频率特性是指系统稳态输出与输入的复数比:
\[G(\jw) = G(s)|_{s=\jw} = |G(\jw)|e^{j\phi(\omega)}\]

其中:
\begin{itemize}
    \item \textbf{幅频特性 $A(\omega)$}:输出响应的稳态分量幅值与输入信号幅值之比。
    \[A(\omega) = \frac{A_{\text{out}}}{A_{\text{in}}} = |G(\jw)|\]
    \item \textbf{相频特性 $\phi(\omega)$}:输出响应的稳态分量与输入信号的相位之差。
    \[\phi(\omega) = \angle G(\jw)\]
\end{itemize}

幅频特性 $A(\omega)$ 和相频特性 $\phi(\omega)$ 统称为\textbf{频率特性}。

\subsection{频率特性的物理意义}
\begin{itemize}
    \item 幅频特性表示不同频率正弦输入信号通过系统后幅值的变化
    \item 相频特性表示不同频率正弦输入信号通过系统后相位的变化
\end{itemize}

\subsection{频率特性的计算}

\subsubsection{计算步骤}
\begin{enumerate}
    \item 将传递函数 $G(s)$ 中的 $s$ 替换为 $j\omega$,得到 $G(j\omega)$
    \item 计算幅频特性:$A(\omega) = |G(j\omega)|$
    \item 计算相频特性:$\phi(\omega) = \angle G(j\omega)$
\end{enumerate}

\subsubsection{例题}

\textbf{例1:已知传递函数为 $G(s)=\frac{4(s+1)}{s(s+2)}$,写出该传递函数的频率特性。}

\textit{解:}
\begin{enumerate}
    \item 将 $s$ 替换为 $j\omega$:
    \[G(j\omega)=\frac{4(j\omega+1)}{j\omega(j\omega+2)}\]
    
    \item 计算幅频特性 $A(\omega)$:
    \[A(\omega) = |G(j\omega)| = \frac{|4(1+j\omega)|}{|j\omega||2+j\omega|} = \frac{4\sqrt{1^2+\omega^2}}{\omega\sqrt{2^2+\omega^2}} = \frac{4\sqrt{1+\omega^2}}{\omega\sqrt{4+\omega^2}}\]
    
    \item 计算相频特性 $\phi(\omega)$:
    \begin{align*}
    \phi(\omega) &= \angle G(j\omega) = \angle 4 + \angle(1+j\omega) - \angle(j\omega) - \angle(2+j\omega) \\
    &= 0° + \arctan(\omega) - 90° - \arctan(\frac{\omega}{2}) \\
    &= \arctan(\omega) - \arctan(\frac{\omega}{2}) - 90°
    \end{align*}
\end{enumerate}

\textbf{例2:某单位负反馈的开环传递函数为 $G(s)=\frac{4}{s(s+2)}$。若输入信号 $r(t)=2\sin(2t)$,试求系统的稳态输出。}

\textit{解:}
\begin{enumerate}
    \item 求闭环传递函数 $\Phi(s)$:
    \[\Phi(s) = \frac{G(s)}{1+G(s)} = \frac{\frac{4}{s(s+2)}}{1+\frac{4}{s(s+2)}} = \frac{4}{s^2+2s+4}\]
    
    \item 分析输入信号:幅值 $A=2$,角频率 $\omega=2$ rad/s
    
    \item 计算闭环系统在 $\omega=2$ 处的频率响应 $\Phi(j2)$:
    \[\Phi(j2) = \frac{4}{(j2)^2+2(j2)+4} = \frac{4}{-4+j4+4} = \frac{4}{j4} = -j\]
    
    \item 计算幅值和相角:
    \begin{align*}
    |\Phi(j2)| &= |-j| = 1 \\
    \angle\Phi(j2) &= \angle(-j) = -90°
    \end{align*}
    
    \item 写出稳态输出 $c_{\text{ss}}(t)$:
    \begin{align*}
    c_{\text{ss}}(t) &= A \cdot |\Phi(j2)| \cdot \sin(\omega t + \angle\Phi(j2)) \\
    &= 2 \cdot 1 \cdot \sin(2t - 90°) \\
    &= 2\sin(2t-90°) = -2\cos(2t)
    \end{align*}
\end{enumerate}
