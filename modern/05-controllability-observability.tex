\section{线性控制系统的能控性和能观测性}
\label{sec:controllability-observability}

\subsection*{引言:控制系统设计的两个基本问题}

想象你正在设计一个火箭姿态控制系统。你面临两个最基本的问题:

\textbf{问题1:我能控制它吗?}如果火箭的某些状态(如角速度)完全不受控制输入影响,那么无论你设计多么精妙的控制算法,都无法让火箭按预期运动。这就是\textbf{能控性}(Controllability)问题。

\textbf{问题2:我能观测它吗?}如果火箭的某些状态(如姿态角)无法从传感器输出中推断出来,那么你根本不知道当前状态,又如何实施状态反馈控制?这就是\textbf{能观测性}(Observability)问题。

这两个概念是\textbf{现代控制理论的基石},由匈牙利数学家\textbf{卡尔曼(Kalman)}在1960年提出。它们回答了控制系统设计中最根本的问题:
\begin{itemize}
    \item \textbf{能控性}:系统的哪些状态可以通过控制输入改变?
    \item \textbf{能观测性}:系统的哪些状态可以从输出测量中推断?
\end{itemize}

\textbf{为什么这些概念重要?}
\begin{itemize}
    \item \textbf{控制器设计的前提}:只有完全能控的系统,才能任意配置极点(后续章节)
    \item \textbf{观测器设计的前提}:只有完全能观的系统,才能设计状态观测器
    \item \textbf{最小实现}:判断系统描述是否冗余,能否简化
    \item \textbf{系统分析}:识别系统中不可控或不可观的\textbf{死角}
\end{itemize}

\textbf{实际应用场景:}
\begin{itemize}
    \item \textbf{航天器姿态控制}:判断推进器配置能否控制所有姿态自由度
    \item \textbf{电力系统}:判断传感器布置能否观测到所有关键状态
    \item \textbf{机器人控制}:设计执行器和传感器的最优配置
    \item \textbf{化工过程}:判断测量点能否监测整个反应过程
\end{itemize}

本章将系统介绍:
\begin{itemize}
    \item 能控性和能观测性的严格定义
    \item 简单实用的判据(秩判据)
    \item 直观理解:几何意义和物理意义
    \item 两者之间的对偶关系
    \item 实际应用范例
\end{itemize}

\subsection{能控性(Controllability)}

\subsubsection*{本节目的}
回答问题:\textbf{给定控制输入,我们能把系统状态驱动到任意期望位置吗?}

\subsubsection{能控性定义}

\textbf{直观理解:}想象一辆只能前进不能后退的汽车。虽然你可以通过转弯到达很多位置,但有些位置(如正后方)永远无法到达。这辆车的位置就\textbf{不是完全能控的}。

\textbf{严格定义:}

系统 $(A, B)$ 在时刻 $t_0$ 是\textbf{状态能控的},如果存在有限时间 $t_1 > t_0$ 和控制输入 $u(t)$,使得系统能从任意初态 $x(t_0)$ 转移到任意终态 $x(t_1)$。

\textbf{数学表述:}对于系统 $\dot{x} = Ax + Bu$,如果
\[\forall x(t_0), x(t_1) \in \mathbb{R}^n, \exists u(t), t \in [t_0, t_1] \text{ 使得 } x(t_1) = e^{A(t_1-t_0)}x(t_0) + \int_{t_0}^{t_1} e^{A(t_1-\tau)}Bu(\tau)d\tau\]
则称系统完全能控。

\textbf{通俗解释:}
\begin{itemize}
    \item \textbf{完全能控}:控制输入能影响所有状态变量,状态空间中任何点都可达
    \item \textbf{不完全能控}:存在某些状态分量,无论如何施加控制都无法改变
    \item \textbf{物理意义}:执行器(控制输入)的配置能否覆盖所有自由度
\end{itemize}

\subsubsection{能控性判据}

\textbf{好消息:}虽然定义涉及复杂的积分,但判断能控性非常简单!

\textbf{卡尔曼秩判据:}

系统 $(A, B)$ 完全能控的充要条件是\textbf{能控性矩阵}:
\[W_c = [B \quad AB \quad A^2B \quad \cdots \quad A^{n-1}B]\]
满足 $\text{rank}(W_c) = n$。

\textbf{判据解读:}
\begin{itemize}
    \item $W_c$ 的列数为 $n \times p$($p$ 为输入个数)
    \item 如果 $\text{rank}(W_c) = n$,说明这些列张成整个 $n$ 维状态空间
    \item 物理意义:$B$ 是输入的直接作用,$AB$ 是经过一步传递的作用,$A^2B$ 是两步传递...
    \item 如果这些作用能覆盖整个状态空间,系统就完全能控
\end{itemize}

\textbf{为什么只需要 $n$ 项?}根据凯莱-哈密顿定理,$A^n$ 可以表示为 $A^0, A^1, \ldots, A^{n-1}$ 的线性组合,所以更高次项不提供新信息。

\textbf{计算步骤:}
\begin{enumerate}
    \item 构造能控性矩阵 $W_c$(计算 $B, AB, A^2B, \ldots$)
    \item 计算 $W_c$ 的秩(用行阶梯形或行列式)
    \item 判断:$\text{rank}(W_c) = n \Rightarrow$ 完全能控;否则不完全能控
\end{enumerate}

\subsection{能观测性(Observability)}

\subsubsection*{本节目的}
回答问题:\textbf{通过观测输出,我们能推断出所有内部状态吗?}

\subsubsection{能观测性定义}

\textbf{直观理解:}想象一个黑盒子里有三个齿轮,但你只能看到最外层齿轮的转动。如果三个齿轮通过固定传动比连接,你可以从外层推断内层的状态——这是\textbf{能观的}。但如果某个内层齿轮是独立的(不传动到外层),你永远无法知道它的状态——这个状态就\textbf{不能观}。

\textbf{严格定义:}

系统 $(A, C)$ 在时刻 $t_0$ 是\textbf{状态能观测的},如果能够根据有限时间区间 $[t_0, t_1]$ 内的输出 $y(t)$ 和输入 $u(t)$ 唯一地确定初始状态 $x(t_0)$。

\textbf{通俗解释:}
\begin{itemize}
    \item \textbf{完全能观}:所有状态变量都能从输出中推断出来
    \item \textbf{不完全能观}:存在某些\textbf{隐藏}状态,无论观测多久都无法确定
    \item \textbf{物理意义}:传感器(输出测量)的配置能否\textbf{看到}所有内部状态
\end{itemize}

\subsubsection{能观测性判据}

\textbf{对偶判据:}

系统 $(A, C)$ 完全能观测的充要条件是\textbf{能观测性矩阵}:
\[W_o = \begin{bmatrix}
C \\ CA \\ CA^2 \\ \vdots \\ CA^{n-1}
\end{bmatrix}\]
满足 $\text{rank}(W_o) = n$。

\textbf{判据解读:}
\begin{itemize}
    \item $W_o$ 的行数为 $n \times q$($q$ 为输出个数)
    \item 如果 $\text{rank}(W_o) = n$,说明这些行能区分所有状态
    \item 物理意义:$C$ 是输出对状态的直接观测,$CA$ 是观测经过一步传递的效果,$CA^2$ 是两步传递...
    \item 如果这些观测能覆盖整个状态空间,系统就完全能观
\end{itemize}

\textbf{计算步骤:}
\begin{enumerate}
    \item 构造能观测性矩阵 $W_o$(计算 $C, CA, CA^2, \ldots$)
    \item 计算 $W_o$ 的秩
    \item 判断:$\text{rank}(W_o) = n \Rightarrow$ 完全能观;否则不完全能观
\end{enumerate}

\subsection{对偶性原理}

\subsubsection*{本节目的}
能控性和能观测性不是孤立的概念,它们之间存在深刻的\textbf{对偶关系}(Duality)。这个关系不仅优美,而且实用——关于能控性的结论可以\textbf{翻译}成能观测性的结论。

\textbf{对偶定理:}

系统 $(A, B, C)$ 能控 $\Leftrightarrow$ 对偶系统 $(A^T, C^T, B^T)$ 能观

\textbf{数学表达:}
\begin{itemize}
    \item 原系统的能控性矩阵:$W_c = [B \quad AB \quad \cdots \quad A^{n-1}B]$
    \item 对偶系统的能观测性矩阵:$W_o^T = [C^T \quad A^TC^T \quad \cdots \quad (A^T)^{n-1}C^T]$
    \item 注意:$W_o^T$ 与 $W_c$ 具有相同的秩(只是 $A \to A^T, B \to C^T$)
\end{itemize}

\textbf{实用价值:}
\begin{itemize}
    \item 能控性的定理、算法可直接用于能观测性(只需转置矩阵)
    \item 简化证明:证明一个性质即可,另一个由对偶性自动得到
    \item 对称美:输入和输出在数学上是对称的
\end{itemize}

\subsection{能控性和能观测性的几何理解}

\subsubsection{能控性的几何意义}

\textbf{可达子空间:}能控性矩阵 $W_c$ 的列空间称为\textbf{可达子空间}(Reachable Subspace),记为 $\mathcal{R}$。

\[\mathcal{R} = \text{span}\{B, AB, A^2B, \ldots, A^{n-1}B\}\]

\textbf{物理意义:}
\begin{itemize}
    \item $\mathcal{R}$ 中的任意状态都可以从原点通过适当的控制输入到达
    \item $\text{dim}(\mathcal{R}) = \text{rank}(W_c)$ 是可控状态的维数
    \item 如果 $\text{dim}(\mathcal{R}) = n$,则整个状态空间可达,系统完全能控
    \item 如果 $\text{dim}(\mathcal{R}) < n$,存在不可达的\textbf{禁区}
\end{itemize}

\textbf{例子:}对于二阶系统,如果 $\text{rank}(W_c) = 1$,可达子空间是一条直线——你只能在这条线上移动,无法到达平面上的其他点。

\subsubsection{能观测性的几何意义}

\textbf{不可观子空间:}能观测性矩阵 $W_o$ 的零空间称为\textbf{不可观子空间}(Unobservable Subspace),记为 $\mathcal{N}$。

\[\mathcal{N} = \text{ker}(W_o) = \{x : W_o x = 0\}\]

\textbf{物理意义:}
\begin{itemize}
    \item $\mathcal{N}$ 中的状态对输出没有任何影响($Cx = 0, CAx = 0, \ldots$)
    \item 这些状态是\textbf{隐藏}的,无法从输出中观测到
    \item $\text{dim}(\mathcal{N}) = n - \text{rank}(W_o)$ 是不可观状态的维数
    \item 如果 $\text{dim}(\mathcal{N}) = 0$,没有隐藏状态,系统完全能观
\end{itemize}

\textbf{例子:}对于二阶系统,如果 $\text{rank}(W_o) = 1$,不可观子空间是一条直线——沿着这条直线的任何状态变化,输出都看不到。

\subsubsection{能控性与能观测性:对比与联系}

为了更好地理解这两个对偶概念,下表总结了它们的对比关系:

\begin{table}[htbp]
\centering
\caption{能控性与能观测性对比}
\begin{tabular}{|l|c|c|}
\hline
\textbf{特性} & \textbf{能控性(Controllability)} & \textbf{能观测性(Observability)} \\
\hline
\textbf{核心问题} & 我能控制它吗? & 我能观测它吗? \\
\hline
\textbf{涉及矩阵} & $(A, B)$ & $(A, C)$ \\
\hline
\textbf{判据矩阵} & $W_c = [B \; AB \; \cdots \; A^{n-1}B]$ & $W_o = \begin{bmatrix} C \\ CA \\ \vdots \\ CA^{n-1} \end{bmatrix}$ \\
\hline
\textbf{判据条件} & $\text{rank}(W_c) = n$ & $\text{rank}(W_o) = n$ \\
\hline
\textbf{几何子空间} & 可达子空间 $\mathcal{R} = \text{span}(W_c)$ & 不可观子空间 $\mathcal{N} = \text{ker}(W_o)$ \\
\hline
\textbf{物理意义} & 输入能否影响所有状态? & 输出能否反映所有状态? \\
\hline
\textbf{不满足后果} & 无法任意配置极点 & 无法设计状态观测器 \\
\hline
\textbf{对偶关系} & $(A,B)$ 能控 $\Leftrightarrow$ $(A^T,C^T)$ 能观 & $(A,C)$ 能观 $\Leftrightarrow$ $(A^T,B^T)$ 能控 \\
\hline
\textbf{应用场景} & 控制器设计、极点配置 & 观测器设计、状态估计 \\
\hline
\textbf{硬件对应} & 执行器配置 & 传感器配置 \\
\hline
\end{tabular}
\end{table}

\textbf{关键洞察:}
\begin{itemize}
    \item \textbf{独立性}:能控和能观是独立的,一个系统可能:
    \begin{itemize}
        \item 能控但不能观(有执行器但传感器不足)
        \item 能观但不能控(有传感器但执行器不足)
        \item 既能控又能观(理想情况)
        \item 既不能控也不能观(最糟糕)
    \end{itemize}
    \item \textbf{对偶性}:数学结构完全对称,转置即可互换
    \item \textbf{系统性}:两者都满足才能实现完整的状态反馈控制
\end{itemize}

\subsection{实际应用范例}

\subsubsection*{范例说明}
以下两个范例展示:
\begin{itemize}
    \item \textbf{范例1}:判断系统的能控性和能观测性
    \item \textbf{范例2}:不能控/不能观对系统的影响
\end{itemize}

\subsubsection{范例1:能控性和能观测性判断}

\textbf{题目:}判断以下系统的能控性和能观测性
\begin{align*}
\dot{x} &= \begin{bmatrix} 0 & 1 \\ -2 & -3 \end{bmatrix} x + \begin{bmatrix} 0 \\ 1 \end{bmatrix} u \\
y &= \begin{bmatrix} 1 & 0 \end{bmatrix} x
\end{align*}

\textbf{解答:}

\paragraph{1. 判断能控性}

构造能控性矩阵:
\[W_c = [B \quad AB]\]

计算 $AB$:
\[AB = \begin{bmatrix} 0 & 1 \\ -2 & -3 \end{bmatrix} \begin{bmatrix} 0 \\ 1 \end{bmatrix} = \begin{bmatrix} 1 \\ -3 \end{bmatrix}\]

因此:
\[W_c = \begin{bmatrix} 0 & 1 \\ 1 & -3 \end{bmatrix}\]

计算行列式:
\[\det(W_c) = 0 \times (-3) - 1 \times 1 = -1 \neq 0\]

$\text{rank}(W_c) = 2 = n$,\textbf{系统完全能控}。

\paragraph{2. 判断能观测性}

构造能观测性矩阵:
\[W_o = \begin{bmatrix} C \\ CA \end{bmatrix}\]

计算 $CA$:
\[CA = \begin{bmatrix} 1 & 0 \end{bmatrix} \begin{bmatrix} 0 & 1 \\ -2 & -3 \end{bmatrix} = \begin{bmatrix} 0 & 1 \end{bmatrix}\]

因此:
\[W_o = \begin{bmatrix} 1 & 0 \\ 0 & 1 \end{bmatrix}\]

显然 $\text{rank}(W_o) = 2 = n$,\textbf{系统完全能观}。

\paragraph{结论}

这是一个\textbf{完全能控且完全能观}的系统,可以:
\begin{itemize}
    \item 通过控制输入任意配置闭环极点
    \item 设计状态观测器估计所有状态
    \item 实现基于观测器的状态反馈控制
\end{itemize}

\subsubsection{范例2:不能控系统的分析}

\textbf{题目:}分析系统
\begin{align*}
\dot{x} &= \begin{bmatrix} 1 & 0 \\ 0 & 2 \end{bmatrix} x + \begin{bmatrix} 1 \\ 0 \end{bmatrix} u \\
y &= \begin{bmatrix} 1 & 1 \end{bmatrix} x
\end{align*}

\textbf{解答:}

\paragraph{1. 能控性判断}

\[W_c = [B \quad AB] = \begin{bmatrix} 1 & 1 \\ 0 & 0 \end{bmatrix}\]

$\text{rank}(W_c) = 1 < 2$,\textbf{系统不完全能控}。

\paragraph{2. 物理解释}

系统方程可以写成:
\begin{align*}
\dot{x}_1 &= x_1 + u \\
\dot{x}_2 &= 2x_2
\end{align*}

\textbf{关键观察:}
\begin{itemize}
    \item $x_1$ 受控制输入 $u$ 影响——\textbf{能控}
    \item $x_2$ 完全独立于 $u$,自己演化($\dot{x}_2 = 2x_2$)——\textbf{不能控}
    \item $x_2$ 会指数发散($x_2(t) = e^{2t}x_2(0)$),无法通过控制阻止!
\end{itemize}

\paragraph{3. 能观测性判断}

\[W_o = \begin{bmatrix} C \\ CA \end{bmatrix} = \begin{bmatrix} 1 & 1 \\ 1 & 2 \end{bmatrix}\]

$\det(W_o) = 2 - 1 = 1 \neq 0$,$\text{rank}(W_o) = 2$,\textbf{系统完全能观}。

\paragraph{4. 实际意义}

\begin{itemize}
    \item 虽然能观测到所有状态(包括 $x_2$),但\textbf{无法控制 $x_2$}
    \item 这个系统是\textbf{不稳定且不能稳定化}的
    \item 在实际设计中,必须重新配置执行器(修改 $B$ 矩阵)
\end{itemize}

\subsubsection*{范例总结}

\textbf{范例1的启示:}
\begin{itemize}
    \item 完全能控+完全能观 = 理想情况,可以实现任何控制目标
    \item 实际系统设计应尽量满足这两个条件
\end{itemize}

\textbf{范例2的警示:}
\begin{itemize}
    \item 不能控意味着存在\textbf{失控}的状态分量
    \item 即使能观测到问题,也无法通过控制解决
    \item 系统设计阶段必须保证能控性(合理配置执行器)
\end{itemize}

\subsection*{本章总结}

\subsubsection*{Kalman秩判据:核心工具}

能控性和能观测性的判定都归结为矩阵的秩计算。下表总结了两个判据的详细信息:

\begin{table}[htbp]
\centering
\caption{Kalman秩判据详解}
\begin{tabular}{|l|p{5.5cm}|p{5.5cm}|}
\hline
\textbf{判据要素} & \textbf{能控性判据} & \textbf{能观测性判据} \\
\hline
\textbf{判据矩阵构造} & 
$W_c = [B \; AB \; A^2B \; \cdots \; A^{n-1}B]$ 
\newline 维度:$n \times (n \cdot p)$($p$为输入数)
& 
$W_o = \begin{bmatrix} C \\ CA \\ CA^2 \\ \vdots \\ CA^{n-1} \end{bmatrix}$ 
\newline 维度:$(n \cdot q) \times n$($q$为输出数)
\\
\hline
\textbf{判据条件} & $\text{rank}(W_c) = n$ & $\text{rank}(W_o) = n$ \\
\hline
\textbf{秩计算方法} & 
• 行阶梯形(推荐)
\newline • Gram矩阵:$\det(W_c W_c^T) \neq 0$
\newline • MATLAB: \texttt{rank(ctrb(A,B))}
& 
• 行阶梯形(推荐)
\newline • Gram矩阵:$\det(W_o^T W_o) \neq 0$
\newline • MATLAB: \texttt{rank(obsv(A,C))}
\\
\hline
\textbf{物理意义} & 
输入通过系统传递($B \to AB \to A^2B \to \cdots$)能否覆盖整个状态空间
& 
状态通过输出映射($C \to CA \to CA^2 \to \cdots$)能否区分所有初始状态
\\
\hline
\textbf{不满足时的维数} & 
$\text{rank}(W_c) = r < n$
\newline 意味着只有 $r$ 维可控
& 
$\text{rank}(W_o) = s < n$
\newline 意味着有 $n-s$ 维不可观
\\
\hline
\textbf{简化技巧(对角系统)} & 
若 $A$ 对角化:检查 $B$ 的哪些行非零
\newline (非零行对应能控模态)
& 
若 $A$ 对角化:检查 $C$ 的哪些列非零
\newline (非零列对应能观模态)
\\
\hline
\end{tabular}
\end{table}

\subsubsection*{核心要点回顾}

\textbf{1. 两个基本概念}
\begin{itemize}
    \item \textbf{能控性}:控制输入能否驱动所有状态?(执行器够不够?)
    \item \textbf{能观测性}:输出测量能否推断所有状态?(传感器够不够?)
\end{itemize}

\textbf{2. 判据(最重要)}
\begin{itemize}
    \item 能控性:$\text{rank}[B \quad AB \quad \cdots \quad A^{n-1}B] = n$
    \item 能观测性:$\text{rank}\begin{bmatrix} C \\ CA \\ \vdots \\ CA^{n-1} \end{bmatrix} = n$
\end{itemize}

\textbf{3. 对偶性}
\begin{itemize}
    \item $(A, B)$ 能控 $\Leftrightarrow$ $(A^T, C^T)$ 能观($B \leftrightarrow C^T$)
    \item 能控性和能观测性在数学上是对称的
\end{itemize}

\textbf{4. 实际意义}
\begin{itemize}
    \item 能控性是\textbf{极点配置}的前提条件
    \item 能观测性是\textbf{状态观测器}的前提条件
    \item 两者都满足才能实现完整的状态反馈控制
\end{itemize}

\subsubsection*{学习清单(掌握程度自检)}

\textbf{基础理解:}
\begin{itemize}
    \item[$\square$] 能用直观语言解释能控性和能观测性(不看书)
    \item[$\square$] 理解为什么需要 $n$ 项($B, AB, \ldots, A^{n-1}B$)
    \item[$\square$] 知道对偶性的含义($A \to A^T, B \to C^T$)
    \item[$\square$] 理解可达子空间和不可观子空间的几何意义
\end{itemize}

\textbf{计算能力:}
\begin{itemize}
    \item[$\square$] 给定 $(A, B)$,能快速构造 $W_c$ 并计算秩
    \item[$\square$] 给定 $(A, C)$,能快速构造 $W_o$ 并计算秩
    \item[$\square$] 会用行阶梯形方法计算秩(手算)
    \item[$\square$] 会用MATLAB的\texttt{ctrb()}和\texttt{obsv()}函数
    \item[$\square$] 对于对角系统,能直接观察 $B$ 和 $C$ 判断能控/能观
\end{itemize}

\textbf{深入应用:}
\begin{itemize}
    \item[$\square$] 能分析不完全能控/能观系统的物理原因(哪个状态失控/隐藏)
    \item[$\square$] 理解能控性如何影响极点配置(后续章节)
    \item[$\square$] 理解能观测性如何影响观测器设计(后续章节)
    \item[$\square$] 能设计执行器/传感器配置使系统满足能控/能观性
    \item[$\square$] 理解结构分解中的四个子系统(可控可观、可控不可观等)
\end{itemize}

\subsubsection*{学习建议}

\textbf{判断流程:}
\begin{enumerate}
    \item 写出系统矩阵 $A, B, C$
    \item 构造 $W_c$ 和 $W_o$ 矩阵(计算 $AB, A^2B, \ldots$ 和 $CA, CA^2, \ldots$)
    \item 计算秩(行阶梯形或行列式)
    \item 根据秩判断能控性和能观测性
\end{enumerate}

\textbf{物理直觉:}
\begin{itemize}
    \item 看到对角化的 $A$ 矩阵:检查 $B$ 的哪些行非零(对应能控的模态)
    \item 看到对角化的 $A$ 矩阵:检查 $C$ 的哪些列非零(对应能观的模态)
    \item 如果某个特征值对应的模态既不能控也不能观,可以简化系统(最小实现)
\end{itemize}

\subsubsection*{常见误区与易错点}

\textbf{概念误区:}
\begin{itemize}
    \item ✗ \textbf{误区1}:认为能控就一定能观(两者独立,需分别判断!)
    \item ✓ \textbf{正确}:能控/能观是两个独立性质,四种组合都可能出现
    
    \item ✗ \textbf{误区2}:认为 $\text{rank}(B) = n$ 就是能控
    \item ✓ \textbf{正确}:必须计算 $\text{rank}(W_c) = \text{rank}[B \; AB \; \cdots]$
    
    \item ✗ \textbf{误区3}:认为能控/能观与控制器设计有关
    \item ✓ \textbf{正确}:这是系统固有性质,与控制器无关(只与 $A, B, C$ 有关)
    
    \item ✗ \textbf{误区4}:认为不能控的系统一定不稳定
    \item ✓ \textbf{正确}:不能控但稳定的系统可以正常工作(只是无法任意配置极点)
\end{itemize}

\textbf{计算易错点:}
\begin{itemize}
    \item ✗ 计算 $A^2B$ 时误算成 $A^2 \cdot B^2$(矩阵幂运算顺序)
    \item ✓ 正确:$A^2B = A \cdot A \cdot B$(先算 $A^2$,再右乘 $B$)
    
    \item ✗ 构造 $W_o$ 时写成 $[C \; CA \; CA^2]$(应该是行排列)
    \item ✓ 正确:$W_o = \begin{bmatrix} C \\ CA \\ CA^2 \end{bmatrix}$(竖着排)
    
    \item ✗ 秩计算时只看行列式($W_c$ 可能不是方阵)
    \item ✓ 正确:用行阶梯形或Gram矩阵 $W_c W_c^T$ 的行列式
    
    \item ✗ 对角系统忘记检查 $B$ 的\textbf{行}(误看成列)
    \item ✓ 正确:对角 $A$ 时,$B$ 的第 $i$ \textbf{行}非零 $\Rightarrow$ 第 $i$ 个模态能控
\end{itemize}

\textbf{物理理解易错点:}
\begin{itemize}
    \item ✗ 把不能控理解成\textquotedblleft 控制效果不好\textquotedblright
    \item ✓ 正确:不能控意味着\textbf{根本无法}通过控制改变某些状态(结构问题)

    \item ✗ 把不能观理解成\textquotedblleft 测量误差大\textquotedblright
    \item ✓ 正确:不能观意味着\textbf{根本无法}从输出推断某些状态(结构问题)
\end{itemize}

\textbf{后续章节预告:}
\begin{itemize}
    \item \textbf{第\ref{sec:structural-decomposition}章(结构分解)}:将系统分解为能控/不能控、能观/不能观的四个子系统,理解不可控/不可观部分对传递函数的影响
    \item \textbf{第\ref{sec:pole-placement}章(极点配置)}:利用能控性任意配置闭环极点,设计期望的动态响应
    \item \textbf{第\ref{sec:state-observer}章(状态观测器)}:利用能观测性从输出估计不可测状态,实现基于观测器的状态反馈
\end{itemize}

\textbf{实践建议:}
\begin{itemize}
    \item 在系统设计阶段就考虑能控性和能观测性(执行器和传感器配置)
    \item 对于大型系统,先检查能控/能观性,避免设计不可实现的控制器
    \item 如果系统不满足能控/能观性,考虑:
    \begin{itemize}
        \item 增加执行器或传感器(改变 $B$ 或 $C$)
        \item 重新建模(可能遗漏了物理耦合)
        \item 放弃某些控制目标(只控制可控部分)
    \end{itemize}
\end{itemize}

能控性和能观测性是现代控制理论的\textbf{门槛概念}——只有真正理解它们,才能深入理解后续的控制器设计方法。它们回答了最基本但最关键的问题:\textbf{我能控制吗?我能观测吗?}

