\section{控制器设计与校正}

\subsection{概述}

控制器设计(或称系统校正)是指通过设计合适的补偿器(校正器),使闭环系统满足性能指标要求。常见的性能指标包括相位裕度(PM)、增益裕度(GM)、稳态误差、响应速度等。

\subsubsection{超前校正}

超前校正用于提高系统的相位裕度,改善稳定性和快速性,但不改善稳态精度。

\textbf{超前校正器传递函数:}
\begin{align*}
G_c(s) = K_c \frac{1 + \alpha T s}{1 + T s}, \quad \alpha > 1
\end{align*}

其中:$K_c$ 为补偿增益,$\alpha$ 为超前比,$T$ 为时间常数。

\begin{minipage}[t]{0.52\textwidth}
\textbf{频率特性:}

\textbf{1. 转折频率:}
\begin{itemize}
    \item 低频转折点:$\omega_1 = \frac{1}{\alpha T}$(零点)
    \item 高频转折点:$\omega_2 = \frac{1}{T}$(极点)
    \item 关系:$\omega_2 = \alpha \omega_1$
\end{itemize}

\textbf{2. 最大相位超前角:}
\begin{align*}
\phi_m &= \arcsin\frac{\alpha - 1}{\alpha + 1} \\
\text{对应频率:} \quad \omega_m &= \frac{1}{T\sqrt{\alpha}} = \sqrt{\omega_1 \omega_2}
\end{align*}

\textbf{3. 幅值特性:}
\begin{itemize}
    \item 在 $\omega_m$ 处增益:$20\lg\sqrt{\alpha}$ dB
    \item 高频增益:$20\lg(\alpha K_c)$ dB
\end{itemize}

\vspace{0.3cm}
\textbf{设计步骤:}

\textbf{步骤1:}确定所需相位超前量
\begin{align*}
\phi_m = \text{PM}_d - \text{PM}_0 + (5° \sim 15°)
\end{align*}
其中 PM$_d$ 为期望相位裕度,PM$_0$ 为原系统相位裕度,$5°$-$15°$ 为补偿量。

\textbf{步骤2:}计算超前比
\begin{align*}
\alpha = \frac{1 + \sin\phi_m}{1 - \sin\phi_m}
\end{align*}

\textbf{步骤3:}确定新剪切频率 $\omega_c'$

使原系统在 $\omega_c'$ 处的增益为 $-20\lg\sqrt{\alpha}$ dB。

\textbf{步骤4:}计算时间常数
\begin{align*}
T = \frac{1}{\omega_c' \sqrt{\alpha}}
\end{align*}

\textbf{步骤5:}选择补偿增益 $K_c$ 使低频增益满足要求。

\end{minipage}\hfill
\begin{minipage}[t]{0.45\textwidth}
\vspace{0pt}
\textbf{超前校正器伯德图:}

\begin{center}
\begin{tikzpicture}
\begin{groupplot}[
    group style={group size=1 by 2, vertical sep=0.3cm},
    width=6.5cm, height=3cm,
    xmode=log,
    grid=both,
    xmin=0.1, xmax=100,
]
\nextgroupplot[ylabel={$L(\omega)$ (dB)}, ytick={0,5,10}, ymin=-2, ymax=12]
% 幅频曲线(α=10)
\addplot[blue, thick, samples=200, domain=0.1:100] 
    {20*log10(sqrt(1+(10*x)^2)/sqrt(1+x^2))};
% 渐近线
\addplot[red, dashed, domain=0.1:1] {0};
\addplot[red, dashed, domain=1:10] {20*log10(x)};
\addplot[red, dashed, domain=10:100] {20*log10(10)};
% 标注
\node[green!60!black] at (axis cs:3.16,5) {\textbullet};
\node[green!60!black, above] at (axis cs:3.16,6.5) {\tiny $\omega_m, 20\lg\sqrt{\alpha}$};
\draw[<-, green!60!black] (axis cs:1,0) -- (axis cs:0.4,2);
\node[green!60!black, left] at (axis cs:0.4,2) {\tiny $\omega_1$};
\draw[<-, green!60!black] (axis cs:10,10) -- (axis cs:20,11);
\node[green!60!black, right] at (axis cs:20,11) {\tiny $\omega_2$};

\nextgroupplot[ylabel={$\phi(\omega)$ (°)}, xlabel={$\omega$ (rad/s)}, 
    ytick={0,30,60}, ymin=-5, ymax=65]
% 相频曲线
\addplot[blue, thick, samples=200, domain=0.1:100] 
    {deg(atan(10*x))-deg(atan(x))};
% 标注
\node[red] at (axis cs:3.16,54.9) {\textbullet};
\node[red, above] at (axis cs:3.16,58) {\tiny $\phi_m$};
\draw[->, red] (axis cs:3.16,0) -- (axis cs:3.16,54);
\end{groupplot}
\end{tikzpicture}
\end{center}

\small\textit{示例:$\alpha = 10$,$T = 0.316$,$\phi_m \approx 55°$}

\vspace{0.3cm}
\textbf{超前校正的作用:}

\begin{itemize}
    \item \textbf{增加相位裕度}:提高稳定性
    \item \textbf{提高剪切频率}:加快响应速度
    \item \textbf{减小超调量}:改善动态性能
    \item \textbf{不改善稳态精度}:低频增益不变
\end{itemize}

\vspace{0.2cm}
\textbf{注意事项:}

\begin{itemize}
    \item 通常 $\alpha < 20$($\phi_m < 65°$)
    \item $\alpha$ 过大会导致高频噪声放大
    \item 需在 $\omega_m$ 处补偿增益损失
\end{itemize}
\end{minipage}

\vspace{0.3cm}
\textbf{例题:}系统 $G(s) = \frac{4}{s(s+1)(0.5s+1)}$,要求 PM $\geq 45°$,设计超前校正器。

\textit{解:}

\textbf{1. 分析原系统}

开环截止频率 $\omega_c \approx 1.6$ rad/s,相位裕度 PM$_0 = 180° - 90° - 58° - 38° \approx -6°$(不稳定)

\textbf{2. 确定所需相位超前量}

$\phi_m = 45° - (-6°) + 10° = 61°$(取补偿量 $10°$)

\textbf{3. 计算超前比}

$\alpha = \frac{1 + \sin 61°}{1 - \sin 61°} = \frac{1.875}{0.125} = 15$

\textbf{4. 确定新剪切频率}

原系统在 $\omega_c'$ 处增益应为 $-20\lg\sqrt{15} \approx -11.8$ dB

从伯德图找到:$\omega_c' \approx 2.5$ rad/s

\textbf{5. 计算时间常数}

$T = \frac{1}{\omega_c'\sqrt{\alpha}} = \frac{1}{2.5 \times \sqrt{15}} \approx 0.103$

\textbf{6. 校正器传递函数}

$G_c(s) = \frac{1 + 1.545s}{1 + 0.103s}$(取 $K_c = 1$)

\textbf{验证:}校正后 PM $\approx 45°$,满足要求。

\subsubsection{滞后校正}

滞后校正用于提高系统的稳态精度,同时保持或略微改善稳定性,但会降低响应速度。

\textbf{滞后校正器传递函数:}
\begin{align*}
G_c(s) = K_c \frac{1 + T s}{1 + \beta T s}, \quad \beta > 1
\end{align*}

其中:$K_c$ 为补偿增益,$\beta$ 为滞后比,$T$ 为时间常数。

\begin{minipage}[t]{0.52\textwidth}
\textbf{频率特性:}

\textbf{1. 转折频率:}
\begin{itemize}
    \item 低频转折点:$\omega_1 = \frac{1}{\beta T}$(极点)
    \item 高频转折点:$\omega_2 = \frac{1}{T}$(零点)
    \item 关系:$\omega_2 = \beta \omega_1$
\end{itemize}

\textbf{2. 相位滞后角:}
\begin{itemize}
    \item 最大滞后角:$\phi_{\min} = -\arcsin\frac{\beta - 1}{\beta + 1}$(负值)
    \item 对应频率:$\omega_m = \frac{1}{T\sqrt{\beta}} = \sqrt{\omega_1 \omega_2}$
    \item 特点:相位滞后,但幅值低频增加
\end{itemize}

\textbf{3. 幅值特性:}
\begin{itemize}
    \item 低频增益:$20\lg(K_c)$ dB(通常取 $K_c = \beta$)
    \item 高频增益:$20\lg(K_c / \beta)$ dB
    \item 低频段提升:$20\lg\beta$ dB
\end{itemize}

\vspace{0.3cm}
\textbf{设计步骤:}

\textbf{步骤1:}确定所需低频增益提升量

根据稳态误差要求确定 $\beta$:
\begin{align*}
20\lg\beta = \text{所需低频增益提升(dB)}
\end{align*}

\textbf{步骤2:}确定新剪切频率 $\omega_c'$

使新剪切频率满足相位裕度要求(通常比原 $\omega_c$ 小)。

\textbf{步骤3:}选择转折频率

使滞后环节在 $\omega_c'$ 处影响很小:
\begin{align*}
\omega_2 = \frac{1}{T} = \frac{\omega_c'}{5 \sim 10}
\end{align*}

\textbf{步骤4:}计算时间常数
\begin{align*}
T = \frac{5 \sim 10}{\omega_c'}, \quad \beta = \text{由步骤1确定}
\end{align*}

\textbf{步骤5:}通常取 $K_c = \beta$ 以保持原剪切频率附近增益。

\end{minipage}\hfill
\begin{minipage}[t]{0.45\textwidth}
\vspace{0pt}
\textbf{滞后校正器伯德图:}

\begin{center}
\begin{tikzpicture}
\begin{groupplot}[
    group style={group size=1 by 2, vertical sep=0.3cm},
    width=6.5cm, height=3cm,
    xmode=log,
    grid=both,
    xmin=0.01, xmax=100,
]
\nextgroupplot[ylabel={$L(\omega)$ (dB)}, ytick={0,10,20}, ymin=-2, ymax=22]
% 幅频曲线(β=10, Kc=10)
\addplot[blue, thick, samples=200, domain=0.01:100] 
    {20*log10(10*sqrt(1+x^2)/sqrt(1+(10*x)^2))};
% 渐近线
\addplot[red, dashed, domain=0.01:0.1] {20*log10(10)};
\addplot[red, dashed, domain=0.1:1] {20*log10(10)-20*log10(x/0.1)};
\addplot[red, dashed, domain=1:100] {0};
% 标注
\draw[<-, green!60!black] (axis cs:0.1,20) -- (axis cs:0.03,21);
\node[green!60!black, left] at (axis cs:0.03,21) {\tiny $\omega_1$};
\draw[<-, green!60!black] (axis cs:1,0) -- (axis cs:3,2);
\node[green!60!black, right] at (axis cs:3,2) {\tiny $\omega_2$};
\node[purple] at (axis cs:0.01,20) {};
\draw[<->, purple, thick] (axis cs:0.015,0) -- (axis cs:0.015,20);
\node[purple, left] at (axis cs:0.012,10) {\tiny $20\lg\beta$};

\nextgroupplot[ylabel={$\phi(\omega)$ (°)}, xlabel={$\omega$ (rad/s)}, 
    ytick={-60,-30,0}, ymin=-65, ymax=5]
% 相频曲线
\addplot[blue, thick, samples=200, domain=0.01:100] 
    {deg(atan(x))-deg(atan(10*x))};
% 标注
\node[red] at (axis cs:0.316,-54.9) {\textbullet};
\node[red, below] at (axis cs:0.316,-60) {\tiny $\phi_{\min}$};
\draw[<-, red] (axis cs:0.316,-54.9) -- (axis cs:0.8,-50);
\end{groupplot}
\end{tikzpicture}
\end{center}

\small\textit{示例:$\beta = 10$,$K_c = 10$,$T = 1$}

\vspace{0.3cm}
\textbf{滞后校正的作用:}

\begin{itemize}
    \item \textbf{提高低频增益}:改善稳态精度
    \item \textbf{保持稳定性}:相位裕度基本不变或略增
    \item \textbf{降低响应速度}:剪切频率降低
    \item \textbf{抑制高频噪声}:高频衰减特性
\end{itemize}

\vspace{0.2cm}
\textbf{注意事项:}

\begin{itemize}
    \item $\omega_2$ 应远小于 $\omega_c'$(通常 $\omega_2 \leq \omega_c'/5$)
    \item 避免在中频段引入过多相位滞后
    \item 适用于原系统相位裕度已满足要求的情况
\end{itemize}
\end{minipage}

\vspace{0.3cm}
\textbf{例题:}系统 $G(s) = \frac{K}{s(s+1)(0.2s+1)}$,$K=5$ 时 PM $= 50°$,要求稳态速度误差系数 $K_v = 50$ s$^{-1}$,设计滞后校正器。

\textit{解:}

\textbf{1. 确定所需增益提升}

原系统:$K_v = \lim_{s\to 0} s \cdot G(s) = 5$

所需提升:$\beta = \frac{50}{5} = 10$ $\implies$ $20\lg\beta = 20$ dB

\textbf{2. 分析原系统}

$K = 5$ 时,$\omega_c \approx 1.5$ rad/s,PM $= 50°$(满足要求)

\textbf{3. 选择新剪切频率}

为保持 PM $\approx 50°$,希望滞后环节对 $\omega_c$ 影响小,选择 $\omega_c' \approx \omega_c = 1.5$ rad/s

\textbf{4. 确定转折频率}

$\omega_2 = \frac{\omega_c'}{10} = 0.15$ rad/s $\implies T = \frac{1}{\omega_2} = 6.67$ s

$\omega_1 = \frac{\omega_2}{\beta} = 0.015$ rad/s

\textbf{5. 滞后校正器}

$G_c(s) = 10 \cdot \frac{1 + 6.67s}{1 + 66.7s}$

\textbf{6. 校正后系统}

$G_c(s)G(s) = \frac{50(1+6.67s)}{s(s+1)(0.2s+1)(1+66.7s)}$,$K_v = 50$ s$^{-1}$,PM $\approx 48°$(满足)

\textbf{验证:}稳态精度提高10倍,稳定性基本保持。

\subsubsection{超前-滞后综合校正}

当系统既需要改善稳态精度,又需要提高稳定性和快速性时,采用超前-滞后综合校正。

\textbf{综合校正器传递函数:}
\begin{align*}
G_c(s) = K_c \cdot \frac{(1 + \alpha T_1 s)(1 + T_2 s)}{(1 + T_1 s)(1 + \beta T_2 s)}, \quad \alpha > 1, \beta > 1
\end{align*}

\begin{minipage}[t]{0.52\textwidth}
\textbf{结构分析:}

综合校正器 = 超前部分 $\times$ 滞后部分

\textbf{1. 超前部分:}$\frac{1 + \alpha T_1 s}{1 + T_1 s}$
\begin{itemize}
    \item 转折频率:$\omega_1' = \frac{1}{\alpha T_1}$,$\omega_2' = \frac{1}{T_1}$
    \item 作用:提供相位超前,提高 PM
\end{itemize}

\textbf{2. 滞后部分:}$K_c \frac{1 + T_2 s}{1 + \beta T_2 s}$
\begin{itemize}
    \item 转折频率:$\omega_1'' = \frac{1}{\beta T_2}$,$\omega_2'' = \frac{1}{T_2}$
    \item 作用:提高低频增益,改善稳态精度
\end{itemize}

\textbf{3. 频段分离:}
\begin{itemize}
    \item 滞后部分在低频段起作用($\omega < \omega_c/10$)
    \item 超前部分在中频段起作用($\omega \approx \omega_c$)
    \item 两部分互不干扰:$\omega_2'' \ll \omega_1'$
\end{itemize}

\vspace{0.3cm}
\textbf{设计步骤:}

\textbf{步骤1:}根据稳态误差要求确定 $\beta$

\textbf{步骤2:}根据动态性能要求确定 $\alpha$ 和超前角 $\phi_m$

\textbf{步骤3:}设计滞后部分
\begin{itemize}
    \item 确定新剪切频率 $\omega_c'$(考虑 $\beta$ 引起的增益提升)
    \item 选择 $\omega_2'' = \omega_c'/10$,计算 $T_2$
\end{itemize}

\textbf{步骤4:}设计超前部分
\begin{itemize}
    \item 在 $\omega_c'$ 处提供所需相位超前 $\phi_m$
    \item 计算 $\alpha$ 和 $T_1$
\end{itemize}

\textbf{步骤5:}选择 $K_c$ 调整总增益

\textbf{步骤6:}验证校正后系统性能

\end{minipage}\hfill
\begin{minipage}[t]{0.45\textwidth}
\vspace{0pt}
\textbf{综合校正器伯德图:}

\begin{center}
\begin{tikzpicture}
\begin{groupplot}[
    group style={group size=1 by 2, vertical sep=0.3cm},
    width=6.5cm, height=3.2cm,
    xmode=log,
    grid=both,
    xmin=0.01, xmax=100,
]
\nextgroupplot[ylabel={$L(\omega)$ (dB)}, ytick={0,10,20}, ymin=-2, ymax=25, 
    legend style={at={(0.5,-0.15)}, anchor=north, font=\tiny, legend columns=2}]
% 滞后部分(低频)
\addplot[green!60!black, dashed, samples=100, domain=0.01:10] 
    {20*log10(10*sqrt(1+x^2)/sqrt(1+(10*x)^2))};
\addlegendentry{滞后部分}
% 超前部分(中高频)
\addplot[orange, dashed, samples=100, domain=0.1:100] 
    {20*log10(sqrt(1+(10*x)^2)/sqrt(1+x^2))};
\addlegendentry{超前部分}
% 综合(叠加)
\addplot[blue, thick, samples=200, domain=0.01:100] 
    {20*log10(10*sqrt(1+x^2)/sqrt(1+(10*x)^2)) + 20*log10(sqrt(1+(10*x)^2)/sqrt(1+x^2))};
\addlegendentry{综合}

\nextgroupplot[ylabel={$\phi(\omega)$ (°)}, xlabel={$\omega$ (rad/s)}, 
    ytick={-60,-30,0,30,60}, ymin=-70, ymax=70]
% 相频特性
\addplot[blue, thick, samples=200, domain=0.01:100] 
    {deg(atan(x))-deg(atan(10*x)) + deg(atan(10*x))-deg(atan(x))};
\addplot[green!60!black, dashed, samples=100, domain=0.01:10] 
    {deg(atan(x))-deg(atan(10*x))};
\addplot[orange, dashed, samples=100, domain=0.1:100] 
    {deg(atan(10*x))-deg(atan(x))};
\end{groupplot}
\end{tikzpicture}
\end{center}

\small\textit{滞后 $\beta=10$,超前 $\alpha=10$}

\vspace{0.3cm}
\textbf{综合校正的特点:}

\begin{itemize}
    \item \textbf{兼顾稳态和动态}:同时改善精度和稳定性
    \item \textbf{设计灵活}:可独立调整两部分参数
    \item \textbf{应用广泛}:适用于性能要求全面的系统
    \item \textbf{参数较多}:需要仔细设计和调试
\end{itemize}

\vspace{0.2cm}
\textbf{三种校正方式比较:}

\begin{center}
\small
\begin{tabular}{|c|c|c|c|}
\hline
\textbf{校正方式} & \textbf{稳态精度} & \textbf{动态性能} & \textbf{适用场合} \\
\hline
超前 & 不变 & 改善 & PM不足,精度满足 \\
\hline
滞后 & 提高 & 略降 & PM满足,精度不足 \\
\hline
超前-滞后 & 提高 & 改善 & 精度和PM均不足 \\
\hline
\end{tabular}
\end{center}
\end{minipage}

\vspace{0.3cm}
\textbf{例题:}系统 $G(s) = \frac{K}{s(s+1)(0.5s+1)}$,$K=4$,要求 $K_v = 20$ s$^{-1}$,PM $\geq 45°$。原系统 PM $\approx 17°$。设计综合校正器。

\textit{解:}

\textbf{1. 确定所需增益提升(滞后部分)}

$\beta = \frac{20}{4} = 5$ $\implies$ $20\lg\beta = 14$ dB

\textbf{2. 确定所需相位超前(超前部分)}

所需:$\phi_m = 45° - 17° + 10° = 38°$ $\implies$ $\alpha = \frac{1+\sin 38°}{1-\sin 38°} \approx 4.2$

\textbf{3. 设计滞后部分}

原 $\omega_c \approx 1.5$ rad/s,增益提升后新 $\omega_c' \approx 3$ rad/s

选择:$\omega_2'' = 0.3$ rad/s $\implies T_2 = 3.33$ s

$\omega_1'' = \omega_2''/\beta = 0.06$ rad/s

滞后部分:$G_{c1}(s) = 5 \cdot \frac{1+3.33s}{1+16.65s}$

\textbf{4. 设计超前部分}

在 $\omega_c' = 3$ rad/s 处提供 $38°$ 相位超前

$T_1 = \frac{1}{\omega_c'\sqrt{\alpha}} = \frac{1}{3 \times 2.05} \approx 0.163$ s

$\omega_1' = \frac{1}{\alpha T_1} = 1.46$ rad/s,$\omega_2' = \frac{1}{T_1} = 6.13$ rad/s

超前部分:$G_{c2}(s) = \frac{1+0.68s}{1+0.163s}$

\textbf{5. 综合校正器}

$G_c(s) = 5 \cdot \frac{(1+0.68s)(1+3.33s)}{(1+0.163s)(1+16.65s)}$

\textbf{验证:}校正后 $K_v = 20$ s$^{-1}$,PM $\approx 45°$,满足要求。
