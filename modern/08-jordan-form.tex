\section{约当型实现}
\label{sec:jordan-form}

\subsection*{引言:当对角化失败时的 \textquotedblleft 备用方案 \textquotedblright}

在\ref{sec:linear-transformation}节中,我们学会了对角化:将系统矩阵 $A$ 转换为对角矩阵 $\Lambda$。但是,\textbf{并非所有矩阵都能对角化!}当系统有\textbf{重特征值}且线性无关的特征向量不足时,对角化就会失败。

\subsubsection*{为什么需要约当型?}

考虑一个简单的例子:
\[A = \begin{bmatrix} 2 & 1 \\ 0 & 2 \end{bmatrix}\]

它有重特征值 $\lambda = 2$(重数为2),但只有一个线性无关的特征向量 $v = \begin{bmatrix} 1 \\ 0 \end{bmatrix}$。\textbf{无法构造2×2的特征向量矩阵来对角化!}

这时就需要\textbf{约当标准型}(Jordan Canonical Form)——它是对角化的推广,能够处理所有矩阵。

\subsubsection*{约当型 vs 对角型}

\begin{table}[h]
\centering
\begin{tabular}{|l|l|l|}
\hline
\rowcolor{gray!20}
& \textbf{对角型} & \textbf{约当型} \\
\hline
矩阵形式 & $\Lambda = \text{diag}(\lambda_1, \ldots, \lambda_n)$ & 对角线+主对角线上方的1 \\
\hline
适用条件 & 特征值互不相同 & \textbf{所有矩阵} \\
\hline
特征向量 & $n$ 个线性无关特征向量 & 可能需要广义特征向量 \\
\hline
系统解耦 & 完全解耦 & 部分耦合(约当块内) \\
\hline
\end{tabular}
\end{table}

\subsubsection*{本章路线图}

\begin{enumerate}
    \item \textbf{约当块}:基本单元的定义和性质
    \item \textbf{约当矩阵}:由约当块组成的分块对角矩阵
    \item \textbf{广义特征向量}:构造变换矩阵的关键
    \item \textbf{例题}:简单特征值和重特征值的两个例子
    \item \textbf{MATLAB实现}:\texttt{jordan()} 函数的使用
\end{enumerate}

\subsection{约当块的定义}

\subsubsection*{基本约当块}

与特征值 $\lambda$ 对应的 $k$ 阶\textbf{约当块}(Jordan Block)定义为:

\begin{empheq}[box=\fbox]{equation}
J_k(\lambda) = \begin{bmatrix}
\lambda & 1 & 0 & \cdots & 0 \\
0 & \lambda & 1 & \cdots & 0 \\
0 & 0 & \lambda & \cdots & 0 \\
\vdots & \vdots & \vdots & \ddots & \vdots \\
0 & 0 & 0 & \cdots & \lambda
\end{bmatrix}_{k \times k}
\end{equation}

\subsubsection*{约当块的特点}

\begin{itemize}
    \item \textbf{对角线}:全部为特征值 $\lambda$
    \item \textbf{超对角线}:主对角线正上方全部为1
    \item \textbf{其余元素}:全部为0
    \item \textbf{阶数}:$k$ 表示该特征值的几何重数(有几个线性无关的特征向量)
\end{itemize}

\subsubsection*{特殊情况}

当 $k=1$ 时,约当块退化为标量:
\[J_1(\lambda) = [\lambda]\]

这就是对角矩阵的元素!因此,\textbf{对角矩阵是约当矩阵的特殊情况}。

\subsection{约当标准型}

\subsubsection*{定义}

任意 $n \times n$ 矩阵 $A$ 都可以通过相似变换化为\textbf{约当标准型}:

\begin{empheq}[box=\fbox]{equation}
J = P^{-1}AP = \begin{bmatrix}
J_1 & 0 & \cdots & 0 \\
0 & J_2 & \cdots & 0 \\
\vdots & \vdots & \ddots & \vdots \\
0 & 0 & \cdots & J_r
\end{bmatrix}
\end{equation}

其中 $J_1, J_2, \ldots, J_r$ 是约当块。

\subsubsection*{约当块的个数与大小}

\begin{itemize}
    \item 如果特征值 $\lambda_i$ 的\textbf{代数重数}为 $m_i$(特征多项式中的重数)
    \item 几何重数为 $g_i$(线性无关特征向量的个数)
    \item 则对应 $g_i$ 个约当块,总大小为 $m_i$
\end{itemize}

\textbf{关键关系}:$g_i \leq m_i$,当 $g_i = m_i$ 时可对角化。

\subsection{广义特征向量}

\subsubsection*{定义}

对于特征值 $\lambda$ 和阶数为 $k$ 的约当块,需要构造 $k$ 个向量:

\begin{align*}
(A - \lambda I)v_1 &= 0 \quad \text{(特征向量)} \\
(A - \lambda I)v_2 &= v_1 \\
(A - \lambda I)v_3 &= v_2 \\
&\vdots \\
(A - \lambda I)v_k &= v_{k-1}
\end{align*}

向量 $v_1, v_2, \ldots, v_k$ 称为\textbf{广义特征向量链}。

\subsubsection*{物理意义}

广义特征向量 $v_2, v_3, \ldots$ 可以理解为 \textquotedblleft 次优的特征向量 \textquotedblright ——虽然不满足 $(A-\lambda I)v=0$,但能逐步逼近特征向量。

\subsection{例题1:简单特征值(可对角化)}

\textbf{问题}:求矩阵的约当标准型
\[A = \begin{bmatrix} 1 & 2 \\ 3 & 2 \end{bmatrix}\]

\textbf{解}:

\textbf{步骤1}:求特征值
\[\det(\lambda I - A) = \det \begin{bmatrix} \lambda - 1 & -2 \\ -3 & \lambda - 2 \end{bmatrix} = (\lambda - 1)(\lambda - 2) - 6 = \lambda^2 - 3\lambda - 4 = (\lambda - 4)(\lambda + 1)\]

特征值:$\lambda_1 = 4$,$\lambda_2 = -1$(互不相同)

\textbf{步骤2}:求特征向量

对于 $\lambda_1 = 4$:
\[(A - 4I)v_1 = \begin{bmatrix} -3 & 2 \\ 3 & -2 \end{bmatrix} \begin{bmatrix} v_{11} \\ v_{12} \end{bmatrix} = 0 \quad \Rightarrow \quad v_1 = \begin{bmatrix} 2 \\ 3 \end{bmatrix}\]

对于 $\lambda_2 = -1$:
\[(A + I)v_2 = \begin{bmatrix} 2 & 2 \\ 3 & 3 \end{bmatrix} \begin{bmatrix} v_{21} \\ v_{22} \end{bmatrix} = 0 \quad \Rightarrow \quad v_2 = \begin{bmatrix} 1 \\ -1 \end{bmatrix}\]

\textbf{步骤3}:构造约当型

由于两个特征值互不相同,约当型就是对角矩阵:
\[J = \begin{bmatrix} 4 & 0 \\ 0 & -1 \end{bmatrix}, \quad P = \begin{bmatrix} 2 & 1 \\ 3 & -1 \end{bmatrix}\]

\textbf{验证}:
\[P^{-1}AP = \frac{1}{-5} \begin{bmatrix} -1 & -1 \\ -3 & 2 \end{bmatrix} \begin{bmatrix} 1 & 2 \\ 3 & 2 \end{bmatrix} \begin{bmatrix} 2 & 1 \\ 3 & -1 \end{bmatrix} = \begin{bmatrix} 4 & 0 \\ 0 & -1 \end{bmatrix} \quad \checkmark\]

\subsection{例题2:重特征值(不可对角化)}

\textbf{问题}:求矩阵的约当标准型
\[A = \begin{bmatrix} 2 & 1 & 0 \\ 0 & 2 & 0 \\ 0 & 0 & 3 \end{bmatrix}\]

\textbf{解}:

\textbf{步骤1}:求特征值
\[\det(\lambda I - A) = (\lambda - 2)^2 (\lambda - 3)\]

特征值:$\lambda_1 = 2$(重数2),$\lambda_2 = 3$(重数1)

\textbf{步骤2}:求 $\lambda_1 = 2$ 的特征向量
\[(A - 2I) = \begin{bmatrix} 0 & 1 & 0 \\ 0 & 0 & 0 \\ 0 & 0 & 1 \end{bmatrix}, \quad \text{rank}(A - 2I) = 2\]

零空间维数 = $3 - 2 = 1$,只有一个线性无关的特征向量:
\[v_1 = \begin{bmatrix} 1 \\ 0 \\ 0 \end{bmatrix}\]

\textbf{步骤3}:求广义特征向量

由于只有一个特征向量但重数为2,需要一个广义特征向量 $v_2$ 满足:
\[(A - 2I)v_2 = v_1\]

解方程:
\[\begin{bmatrix} 0 & 1 & 0 \\ 0 & 0 & 0 \\ 0 & 0 & 1 \end{bmatrix} \begin{bmatrix} v_{21} \\ v_{22} \\ v_{23} \end{bmatrix} = \begin{bmatrix} 1 \\ 0 \\ 0 \end{bmatrix}\]

得 $v_{22} = 1$,其余自由,取:
\[v_2 = \begin{bmatrix} 0 \\ 1 \\ 0 \end{bmatrix}\]

\textbf{步骤4}:求 $\lambda_2 = 3$ 的特征向量
\[v_3 = \begin{bmatrix} 0 \\ 0 \\ 1 \end{bmatrix}\]

\textbf{步骤5}:构造约当型
\[J = \begin{bmatrix} 2 & 1 & 0 \\ 0 & 2 & 0 \\ 0 & 0 & 3 \end{bmatrix}, \quad P = \begin{bmatrix} 1 & 0 & 0 \\ 0 & 1 & 0 \\ 0 & 0 & 1 \end{bmatrix} = I\]

注意:前两列对应 $\lambda_1 = 2$ 的2阶约当块,第三列对应 $\lambda_2 = 3$ 的1阶约当块。

\subsection{约当型与系统动态}

\subsubsection*{约当块的矩阵指数}

对于约当块 $J_k(\lambda)$,其矩阵指数为:

\begin{empheq}[box=\fbox]{equation}
e^{J_k(\lambda)t} = e^{\lambda t} \begin{bmatrix}
1 & t & \frac{t^2}{2!} & \cdots & \frac{t^{k-1}}{(k-1)!} \\
0 & 1 & t & \cdots & \frac{t^{k-2}}{(k-2)!} \\
0 & 0 & 1 & \cdots & \frac{t^{k-3}}{(k-3)!} \\
\vdots & \vdots & \vdots & \ddots & \vdots \\
0 & 0 & 0 & \cdots & 1
\end{bmatrix}
\end{equation}

\subsubsection*{稳定性影响}

\begin{itemize}
    \item 如果 $\text{Re}(\lambda) < 0$,则 $e^{\lambda t} \to 0$,约当块对应的状态收敛
    \item 如果 $\text{Re}(\lambda) > 0$,则 $e^{\lambda t} \to \infty$,系统不稳定
    \item 如果 $\text{Re}(\lambda) = 0$:
    \begin{itemize}
        \item 1阶约当块:$e^{J_1(0)t} = 1$(临界稳定)
        \item $k>1$ 阶约当块:包含 $t, t^2, \ldots$(不稳定!)
    \end{itemize}
\end{itemize}

\textbf{关键结论}:纯虚特征值的重根会导致不稳定(即使特征值在虚轴上)。

\subsection{MATLAB实现}

\begin{lstlisting}[style=Matlab-editor, caption=约当标准型的MATLAB实现]
% 例题1:简单特征值
A1 = [1 2; 3 2];
[P1, J1] = jordan(A1);
fprintf('例题1 - 约当型(对角):\n');
J1
P1

% 例题2:重特征值
A2 = [2 1 0; 0 2 0; 0 0 3];
[P2, J2] = jordan(A2);
fprintf('例题2 - 约当型(有约当块):\n');
J2
P2

% 验证变换
A2_reconstructed = P2 * J2 * inv(P2);
fprintf('验证: P*J*P^{-1} =\n');
A2_reconstructed

% 可视化约当块结构
figure;
spy(J2);
title('约当矩阵的稀疏结构');
xlabel('列');
ylabel('行');

% 方法2:使用eig检查对角化可能性
[V, D] = eig(A2);
fprintf('特征值:\n');
diag(D)
fprintf('特征向量矩阵的秩: %d (应为 %d 才能对角化)\n', rank(V), size(A2,1));
\end{lstlisting}

\subsection*{本章小结}

\subsubsection*{核心公式}

\begin{tcolorbox}[colback=green!5!white, colframe=green!75!black, title=约当标准型的核心概念]
\textbf{约当块}:
\[J_k(\lambda) = \begin{bmatrix} \lambda & 1 & & \\ & \lambda & \ddots & \\ & & \ddots & 1 \\ & & & \lambda \end{bmatrix}_{k \times k}\]

\textbf{约当矩阵}:
\[J = \text{diag}(J_1, J_2, \ldots, J_r)\]

\textbf{变换矩阵}:$P$ 的列由特征向量和广义特征向量组成

\textbf{矩阵指数}:
\[e^{J_k(\lambda)t} = e^{\lambda t} \begin{bmatrix} 1 & t & \frac{t^2}{2!} & \cdots \\ & 1 & t & \cdots \\ & & \ddots & \ddots \\ & & & 1 \end{bmatrix}\]
\end{tcolorbox}

\subsubsection*{约当型 vs 对角型对比}

\begin{table}[h]
\centering
\caption{约当标准型与对角标准型对比}
\begin{tabular}{|l|l|l|}
\hline
\rowcolor{gray!20}
\textbf{特性} & \textbf{对角型} & \textbf{约当型} \\
\hline
适用范围 & 特征值互不相同 & \textbf{所有矩阵} \\
\hline
矩阵形式 & 纯对角矩阵 & 对角+超对角线的1 \\
\hline
系统解耦 & 完全解耦 & 约当块内部耦合 \\
\hline
特征向量 & $n$ 个线性无关 & 可能需要广义特征向量 \\
\hline
计算复杂度 & 低 & 中等 \\
\hline
稳定性判定 & 直接看特征值 & 需考虑约当块大小 \\
\hline
\end{tabular}
\end{table}

\subsubsection*{应用场景}

\begin{itemize}
    \item \textbf{不可对角化系统}:处理重特征值且几何重数<代数重数的情况
    \item \textbf{状态方程求解}:计算 $e^{At}$ 时,约当型比原矩阵简单(参考\ref{sec:solving-state-space}节)
    \item \textbf{稳定性分析}:判断纯虚特征值重根的稳定性(参考\ref{sec:lyapunov-stability}节)
    \item \textbf{理论研究}:约当型是矩阵相似理论的基石
\end{itemize}

\subsubsection*{常见误区}

\begin{itemize}
    \item ❌ \textbf{误区1}:所有矩阵都能对角化(错!重特征值可能无法对角化)
    \item ❌ \textbf{误区2}:约当块的大小等于特征值的重数(错!可能有多个小约当块)
    \item ❌ \textbf{误区3}:纯虚特征值重根是临界稳定(错!若约当块阶数>1则不稳定)
\end{itemize}

\subsubsection*{与其他章节的联系}

\begin{itemize}
    \item \textbf{向后链接}:
    \begin{itemize}
        \item \ref{sec:linear-transformation}节:约当型是对角化失败时的替代方案
        \item \ref{sec:solving-state-space}节:约当型简化 $e^{At}$ 的计算
        \item \ref{sec:standard-forms}节:约当型是另一种重要的标准型
    \end{itemize}
    \item \textbf{向前链接}:
    \begin{itemize}
        \item \ref{sec:lyapunov-stability}节:约当块影响稳定性判定
        \item 高级课程:约当型在泛函分析、矩阵理论中的应用
    \end{itemize}
\end{itemize}

\subsubsection*{学习检查清单}

\begin{itemize}
    \item[$\square$] 理解约当块的定义(对角线+超对角线的1)
    \item[$\square$] 知道什么时候需要约当型(重特征值且不可对角化)
    \item[$\square$] 掌握广义特征向量的求解方法($(A-\lambda I)v_k = v_{k-1}$)
    \item[$\square$] 能够判断约当块的个数和大小(几何重数 vs 代数重数)
    \item[$\square$] 理解约当型对系统稳定性的影响(纯虚重根的危险)
    \item[$\square$] 能够使用MATLAB的 \texttt{jordan()} 函数
\end{itemize}
