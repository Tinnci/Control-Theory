\part{现代控制理论}

\section{状态空间表达式及其建立}

\subsection{状态空间的基本概念}
\begin{itemize}
    \item \textbf{状态}:系统在某一时刻的状态是指确定系统该时刻以后行为所必需的最少信息
    \item \textbf{状态变量}:描述系统状态的一组变量
    \item \textbf{状态向量}:由状态变量组成的向量
    \item \textbf{状态空间}:以状态变量为坐标的n维空间
\end{itemize}

\subsection{状态空间表达式}
线性定常系统的状态空间表达式:
\begin{align}
\dot{x}(t) &= Ax(t) + Bu(t) \\
\dot{y}(t) &= Cx(t) + Du(t)
\end{align}

其中:
\begin{itemize}
    \item $x(t) \in \mathbb{R}^n$:状态向量
    \item $u(t) \in \mathbb{R}^p$:输入向量
    \item $y(t) \in \mathbb{R}^q$:输出向量
    \item $A \in \mathbb{R}^{n \times n}$:系统矩阵
    \item $B \in \mathbb{R}^{n \times p}$:输入矩阵
    \item $C \in \mathbb{R}^{q \times n}$:输出矩阵
    \item $D \in \mathbb{R}^{q \times p}$:前馈矩阵
\end{itemize}

\subsection{状态空间表达式的建立方法}
\begin{enumerate}
    \item 根据物理规律建立微分方程组
    \item 选择状态变量(通常选择能量存储元件的变量)
    \item 将高阶微分方程化为一阶微分方程组
    \item 写出输出方程
\end{enumerate}
