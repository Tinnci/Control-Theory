\section{极点配置——状态反馈}

\subsection*{引言:为什么我们需要极点配置?}

想象你正在驾驶一辆汽车。如果车辆的动力学特性很差——转向反应迟钝、加速太慢、或者容易失控,你会怎么办?理想情况下,你希望\textbf{改造}这辆车,让它响应更快、更稳定、更好操控。

在控制系统中,我们面临同样的问题:
\begin{itemize}
    \item 原始系统可能\textbf{不稳定}(极点在右半平面)
    \item 即使稳定,响应可能\textbf{太慢}(极点太靠近虚轴)
    \item 或者\textbf{振荡太大}(极点的阻尼比太小)
\end{itemize}

我们能改变系统的动态特性吗?答案是:\textbf{能!}通过\textbf{状态反馈},我们可以\textbf{任意配置闭环系统的极点}(前提是系统能控)。

\textbf{极点配置的核心思想:}
\begin{quote}
通过反馈控制,将不满意的开环极点\textit{搬到}我们期望的位置,从而获得理想的动态性能。
\end{quote}

\textbf{这有多强大?}
\begin{itemize}
    \item 不稳定系统 $\to$ 可以变成稳定的
    \item 慢系统 $\to$ 可以变得快速响应
    \item 振荡系统 $\to$ 可以变得平稳
    \item 你可以\textbf{精确指定}系统的超调量、调节时间等性能指标
\end{itemize}

\textbf{但有一个重要前提:}系统必须\textbf{完全能控}!这就是为什么我们在上一章花大量篇幅讨论能控性——它是极点配置的\textbf{充要条件}。

\textbf{实际应用场景:}
\begin{itemize}
    \item \textbf{航天器姿态控制}:配置极点以获得快速无超调的姿态调整
    \item \textbf{机器人关节控制}:通过极点配置实现精确的轨迹跟踪
    \item \textbf{电力系统稳定器}:将不稳定的电力系统稳定化
    \item \textbf{汽车主动悬架}:改善舒适性和操控性
\end{itemize}

本章将介绍:
\begin{itemize}
    \item 状态反馈的基本原理
    \item 极点配置定理(为什么能控性是关键)
    \item 三种极点配置方法(直接法、变换法、阿克曼公式)
    \item 实际设计范例
    \item 极点选择的实用准则
\end{itemize}

\subsection{状态反馈}

\subsubsection*{本节目的}
在讨论如何配置极点之前,我们需要理解\textbf{状态反馈}的概念——这是极点配置的基本工具。

\subsubsection{状态反馈的基本思想}

\textbf{经典反馈 vs 状态反馈:}

\begin{itemize}
    \item \textbf{经典输出反馈}:$u = -K_p y$(只反馈输出,如PID控制器)
    \item \textbf{状态反馈}:$u = -Kx$(反馈所有状态变量)
\end{itemize}

\textbf{为什么需要状态反馈?}
\begin{itemize}
    \item 输出反馈只利用了部分信息($y = Cx$,可能只是状态的一部分)
    \item 状态反馈利用了\textbf{完整的系统信息}
    \item 只有状态反馈才能\textbf{任意配置极点}
\end{itemize}

\subsubsection{状态反馈控制律}

\textbf{基本形式:}
\[u = -Kx + v\]

其中:
\begin{itemize}
    \item $x \in \mathbb{R}^n$:状态向量(需要全部可测或通过观测器估计)
    \item $K \in \mathbb{R}^{p \times n}$:\textbf{反馈增益矩阵}(待设计的参数)
    \item $v \in \mathbb{R}^p$:\textbf{参考输入}(期望的目标值)
    \item 负号表示\textbf{负反馈}(稳定控制的标准配置)
\end{itemize}

\textbf{直观理解:}
\begin{itemize}
    \item $-Kx$ 是\textbf{校正项}:根据当前状态偏差进行调整
    \item $v$ 是\textbf{目标项}:驱动系统到期望状态
    \item $K$ 的每个元素决定了对应状态变量的\textit{权重}
\end{itemize}

\subsubsection{闭环系统}

将 $u = -Kx + v$ 代入开环系统 $\dot{x} = Ax + Bu$:
\begin{align*}
\dot{x} &= Ax + B(-Kx + v) \\
&= Ax - BKx + Bv \\
&= (A - BK)x + Bv
\end{align*}

\textbf{闭环系统:}
\[\boxed{\dot{x} = (A - BK)x + Bv}\]

\textbf{关键观察:}
\begin{itemize}
    \item 开环系统矩阵:$A$
    \item 闭环系统矩阵:$A_{\text{cl}} = A - BK$
    \item 通过选择 $K$,我们改变了系统矩阵,从而改变了极点!
\end{itemize}

\textbf{极点的变化:}
\begin{itemize}
    \item 开环极点:$\det(\lambda I - A) = 0$ 的根
    \item 闭环极点:$\det(\lambda I - A + BK) = 0$ 的根
    \item $K$ 是我们的\textit{旋钮},可以调节极点位置
\end{itemize}

\textbf{重要提醒:}
\begin{itemize}
    \item 状态反馈\textbf{不改变系统的能控性和能观测性}
    \item 状态反馈\textbf{不改变系统的零点}(只改变极点)
    \item 状态反馈要求状态\textbf{可测}或\textbf{可估计}(这就是为什么需要观测器)
\end{itemize}

\subsection{极点配置定理}

\subsubsection*{本节目的}
这是本章的\textbf{核心定理},它回答了一个关键问题:我们能把极点配置到任意位置吗?答案是:\textbf{能控就能配!}

\subsubsection{定理陈述}

\textbf{极点配置定理(Pole Placement Theorem):}

对于单输入系统 $(A, B)$,若系统\textbf{完全能控},则对于任意给定的 $n$ 个复数 $\lambda_1, \lambda_2, \ldots, \lambda_n$(复数必须成对共轭出现),\textbf{存在}反馈增益矩阵 $K$,使得闭环系统矩阵 $A - BK$ 的特征值恰好为 $\lambda_1, \lambda_2, \ldots, \lambda_n$。

\textbf{定理的含义:}
\begin{itemize}
    \item \textbf{充要条件}:完全能控 $\Leftrightarrow$ 可任意配置极点
    \item \textbf{任意性}:极点位置完全由你决定(只要满足共轭对要求)
    \item \textbf{构造性}:不仅说存在,还有具体的计算方法
\end{itemize}

\subsubsection{定理的深刻意义}

\textbf{1. 能控性是充要条件}

\begin{itemize}
    \item \textbf{充分性}:能控 $\Rightarrow$ 可任意配置极点(定理保证)
    \item \textbf{必要性}:可任意配置极点 $\Rightarrow$ 能控(反证法:不能控意味着某些模态不受控制影响,无法移动)
\end{itemize}

\textbf{为什么能控性如此关键?}
\begin{itemize}
    \item 能控性保证 $B, AB, \ldots, A^{n-1}B$ 张成整个状态空间
    \item 这意味着通过 $BK$ 可以在所有方向上施加影响
    \item 从而可以将 $A$ 的特征值\textit{推到}任意位置
\end{itemize}

\textbf{2. 复数共轭对要求}

由于系统矩阵 $A, B, K$ 都是\textbf{实矩阵},闭环矩阵 $A - BK$ 的特征多项式的系数也是实数。因此:
\begin{itemize}
    \item 实数极点:可以单独指定
    \item 复数极点:必须成对共轭出现(如 $-2 \pm 3j$)
\end{itemize}

\textbf{物理意义:}实系统不能产生单个复数极点,这会导致非物理的复数状态轨迹。

\textbf{3. 多输入系统的推广}

定理对多输入系统($p > 1$)也成立,但:
\begin{itemize}
    \item $K$ 不是唯一的(有更多自由度)
    \item 可以利用额外自由度优化其他性能指标(如鲁棒性、输入能量)
    \item 这是\textbf{最优控制}理论的起点
\end{itemize}

\subsubsection{设计步骤概览}

使用极点配置定理设计控制器的典型流程:

\begin{enumerate}
    \item \textbf{验证能控性}:计算 $\text{rank}(W_c)$,确保 $= n$
    \item \textbf{选择期望极点}:根据性能要求确定 $\lambda_1, \ldots, \lambda_n$
    \item \textbf{计算反馈增益}:使用直接法、变换法或阿克曼公式求 $K$
    \item \textbf{验证设计}:检查闭环极点是否正确,分析性能
\end{enumerate}

\textbf{定理的局限性:}
\begin{itemize}
    \item 只适用于\textbf{能控}系统(这是硬性要求)
    \item 没有告诉你\textbf{如何选择}期望极点(需要经验和性能分析)
    \item 假设状态\textbf{完全可测}(实际中可能需要观测器)
    \item 可能导致很大的控制输入(\textbf{饱和问题})
\end{itemize}

\subsection{极点配置的方法}

\subsubsection*{本节目的}
定理告诉我们极点\textbf{可以}任意配置,但\textbf{怎么}计算反馈增益矩阵 $K$ 呢?本节介绍三种实用方法。

\subsubsection{方法概览}

\begin{enumerate}
    \item \textbf{直接方法}:解特征方程 $\det(sI - A + BK) = 0$
    \begin{itemize}
        \item 适用于低阶系统($n \leq 3$)
        \item 直观但计算繁琐
    \end{itemize}
    
    \item \textbf{变换方法}:将系统化为能控标准型后配置极点
    \begin{itemize}
        \item 适用于任意阶次
        \item 需要坐标变换
    \end{itemize}
    
    \item \textbf{阿克曼公式}:$K = [0 \quad 0 \quad \cdots \quad 0 \quad 1] W_c^{-1} \alpha_c(A)$
    \begin{itemize}
        \item 单输入系统的显式公式
        \item 最简洁,适合程序实现
    \end{itemize}
\end{enumerate}

\subsubsection{方法1:直接方法}

\textbf{基本思路:}直接让闭环特征多项式等于期望特征多项式。

\textbf{步骤:}
\begin{enumerate}
    \item 写出期望特征多项式:
    \[\alpha_d(s) = (s - \lambda_1)(s - \lambda_2) \cdots (s - \lambda_n) = s^n + a_1 s^{n-1} + \cdots + a_n\]
    
    \item 计算闭环特征多项式:
    \[\det(sI - A + BK) = s^n + b_1(K) s^{n-1} + \cdots + b_n(K)\]
    
    \item 令两者相等:
    \[b_i(K) = a_i, \quad i = 1, 2, \ldots, n\]
    
    \item 解方程组得到 $K$
\end{enumerate}

\textbf{优点:}
\begin{itemize}
    \item 概念直接,容易理解
    \item 适合手算低阶系统
\end{itemize}

\textbf{缺点:}
\begin{itemize}
    \item 高阶系统计算量大
    \item 展开 $\det(sI - A + BK)$ 很繁琐
\end{itemize}

\subsubsection{方法2:变换方法}

\textbf{核心思想:}在能控标准型下,极点配置变得非常简单。

\textbf{能控标准型的特殊性质:}

对于能控标准型系统:
\[\bar{A} = \begin{bmatrix}
0 & 1 & 0 & \cdots & 0 \\
0 & 0 & 1 & \cdots & 0 \\
\vdots & \vdots & \vdots & \ddots & \vdots \\
0 & 0 & 0 & \cdots & 1 \\
-a_0 & -a_1 & -a_2 & \cdots & -a_{n-1}
\end{bmatrix}, \quad \bar{B} = \begin{bmatrix} 0 \\ 0 \\ \vdots \\ 0 \\ 1 \end{bmatrix}\]

特征多项式直接从最后一行读出:
\[\det(sI - \bar{A}) = s^n + a_{n-1}s^{n-1} + \cdots + a_1s + a_0\]

\textbf{设计步骤:}
\begin{enumerate}
    \item 找到变换矩阵 $T$,使原系统变为能控标准型
    \[\bar{A} = TAT^{-1}, \quad \bar{B} = TB\]
    
    \item 在标准型下设计反馈增益 $\bar{K}$(非常简单!):
    \[\bar{K} = [\alpha_0 - a_0 \quad \alpha_1 - a_1 \quad \cdots \quad \alpha_{n-1} - a_{n-1}]\]
    其中 $\alpha_i$ 是期望特征多项式的系数
    
    \item 变换回原坐标:
    \[K = \bar{K}T\]
\end{enumerate}

\textbf{优点:}
\begin{itemize}
    \item 计算规范,不易出错
    \item 适用于任意阶次
\end{itemize}

\textbf{缺点:}
\begin{itemize}
    \item 需要计算变换矩阵 $T$
    \item 当系统接近不能控时,$T$ 可能数值不稳定
\end{itemize}

\subsubsection{方法3:阿克曼公式(Ackermann's Formula)}

\textbf{最简洁的方法!}对于单输入系统,有显式公式:

\[\boxed{K = [0 \quad 0 \quad \cdots \quad 0 \quad 1] W_c^{-1} \alpha_c(A)}\]

其中:
\begin{itemize}
    \item $W_c = [B \quad AB \quad A^2B \quad \cdots \quad A^{n-1}B]$:能控性矩阵
    \item $\alpha_c(s) = s^n + \alpha_1 s^{n-1} + \cdots + \alpha_n$:期望特征多项式
    \item $\alpha_c(A) = A^n + \alpha_1 A^{n-1} + \cdots + \alpha_n I$:矩阵多项式
\end{itemize}

\textbf{注意:}$\alpha_c(A)$ 是将多项式中的 $s$ 替换为矩阵 $A$!

\textbf{优点:}
\begin{itemize}
    \item \textbf{最简洁}:一个公式搞定
    \item 易于编程实现
    \item 不需要坐标变换
\end{itemize}

\textbf{缺点:}
\begin{itemize}
    \item 仅适用于\textbf{单输入系统}
    \item 需要求逆 $W_c^{-1}$(能控时必可逆)
    \item 数值稳定性依赖于 $W_c$ 的条件数
\end{itemize}

\textbf{MATLAB实现:}
\begin{verbatim}
K = acker(A, B, desired_poles)
\end{verbatim}

\subsubsection{三种方法的对比}

\begin{center}
\renewcommand{\arraystretch}{1.6}
\begin{tabular}{|l|p{4cm}|p{4cm}|p{4cm}|}
\hline
\rowcolor[gray]{0.9}
\textbf{方法} & \textbf{直接方法} & \textbf{变换方法} & \textbf{阿克曼公式} \\
\hline
\textbf{适用范围} & 低阶系统 & 任意阶次 & 单输入系统 \\
\hline
\textbf{计算复杂度} & 中等(需展开行列式) & 较高(需变换矩阵) & 低(一个公式) \\
\hline
\textbf{直观性} & 高(直接配置) & 中等 & 低(公式较抽象) \\
\hline
\textbf{数值稳定性} & 好 & 取决于 $T$ & 取决于 $W_c$ \\
\hline
\textbf{推荐场景} & 手算 $n \leq 3$ & 理论分析 & 程序实现 \\
\hline
\end{tabular}
\end{center}

\subsection{极点选择的实用准则}

\subsubsection*{本节目的}
理论上可以任意配置极点,但\textbf{实际中如何选择}期望极点?这需要平衡性能、鲁棒性和实现成本。

\subsubsection{性能指标与极点位置的关系}

\textbf{二阶系统的标准形式:}
\[s^2 + 2\zeta\omega_n s + \omega_n^2 = 0\]

极点位置:$s_{1,2} = -\zeta\omega_n \pm j\omega_n\sqrt{1-\zeta^2}$

\begin{itemize}
    \item $\omega_n$:自然频率(影响\textbf{响应速度})
    \item $\zeta$:阻尼比(影响\textbf{超调量和振荡})
\end{itemize}

\textbf{性能指标:}
\begin{itemize}
    \item 调节时间:$t_s \approx \frac{4}{\zeta\omega_n}$(越小越快)
    \item 超调量:$M_p \approx e^{-\pi\zeta/\sqrt{1-\zeta^2}}$(越小越平稳)
    \item 峰值时间:$t_p = \frac{\pi}{\omega_n\sqrt{1-\zeta^2}}$
\end{itemize}

\subsubsection{极点配置的实用原则}

\textbf{1. 稳定性要求(基本要求)}
\begin{itemize}
    \item 所有极点必须在\textbf{左半平面}($\text{Re}(s_i) < 0$)
    \item 距离虚轴越远,稳定裕度越大
\end{itemize}

\textbf{2. 响应速度要求}
\begin{itemize}
    \item 主导极点的实部决定响应速度
    \item 更快响应 $\to$ 极点更靠左(但需要更大的控制能量)
\end{itemize}

\textbf{3. 超调和振荡要求}
\begin{itemize}
    \item 实极点:无超调,单调响应
    \item 复数极点:有超调和振荡
    \item 常用选择:$\zeta = 0.5 \sim 0.707$(适度阻尼)
\end{itemize}

\textbf{4. 高阶系统的极点配置}
\begin{itemize}
    \item \textbf{主导极点}:最靠近虚轴的极点对(决定主要动态特性)
    \item \textbf{非主导极点}:远离虚轴(衰减快,影响小)
    \item 典型配置:2个主导极点 + $(n-2)$ 个快衰减极点
\end{itemize}

\textbf{5. 实际限制}
\begin{itemize}
    \item \textbf{控制饱和}:极点太左 $\to$ 控制信号过大
    \item \textbf{测量噪声}:极点太左 $\to$ 对噪声敏感
    \item \textbf{模型误差}:极点太精确 $\to$ 鲁棒性差
\end{itemize}

\subsubsection{经验法则}

\textbf{巴特沃斯配置(Butterworth Pattern):}

将极点均匀分布在以原点为中心的圆弧上,角度为:
\[\theta_k = \frac{(2k+1)\pi}{2n}, \quad k = 0, 1, \ldots, n-1\]

半径选择为 $r = \omega_n$(期望的响应速度)。

\textbf{贝塞尔配置(Bessel Pattern):}
\begin{itemize}
    \item 优化阶跃响应的延迟时间
    \item 极点分布更集中
    \item 适合对延迟敏感的应用
\end{itemize}

\textbf{ITAE最优配置(Integral of Time-weighted Absolute Error):}
\begin{itemize}
    \item 最小化 $\int_0^\infty t|e(t)|dt$
    \item 有标准的极点位置表(查表即可)
\end{itemize}

\subsubsection{极点配置的常见错误}

\begin{itemize}
    \item ✗ 极点配置过于激进(远离虚轴)$\to$ 控制饱和
    \item ✗ 忽略零点的影响(系统零点不变,可能抵消部分极点效果)
    \item ✗ 所有极点聚在同一位置 $\to$ 数值不稳定
    \item ✓ 平衡性能和实现成本
    \item ✓ 验证闭环系统对参数变化的鲁棒性
\end{itemize}

\subsection{实际应用范例}

\subsubsection*{范例说明}
通过具体例子展示:
\begin{itemize}
    \item 如何使用阿克曼公式设计控制器
    \item 如何选择期望极点
    \item 如何验证设计结果
\end{itemize}

\subsubsection{范例:二阶系统的极点配置}

\textbf{题目:}设计状态反馈控制器,使系统具有快速无超调响应

\textbf{系统描述:}
\begin{align*}
\dot{x} &= \begin{bmatrix} 0 & 1 \\ -2 & -3 \end{bmatrix} x + \begin{bmatrix} 0 \\ 1 \end{bmatrix} u \\
y &= \begin{bmatrix} 1 & 0 \end{bmatrix} x
\end{align*}

\textbf{性能要求:}
\begin{itemize}
    \item 调节时间 $t_s < 2$s
    \item 无超调(实极点)
\end{itemize}

\textbf{解答:}

\paragraph{第一步:验证能控性}

\[W_c = [B \quad AB] = \begin{bmatrix} 0 & 1 \\ 1 & -3 \end{bmatrix}\]

$\det(W_c) = -1 \neq 0$,系统\textbf{完全能控},可以任意配置极点。

\paragraph{第二步:选择期望极点}

要求无超调 $\to$ 选择\textbf{实极点}

要求 $t_s < 2$s,由 $t_s \approx \frac{4}{|\text{Re}(s)|}$ 得:
\[|\text{Re}(s)| > \frac{4}{2} = 2\]

选择两个快速实极点:$s_1 = -3, s_2 = -4$

期望特征多项式:
\[\alpha_d(s) = (s+3)(s+4) = s^2 + 7s + 12\]

\paragraph{第三步:使用阿克曼公式}

计算 $\alpha_d(A)$:
\begin{align*}
\alpha_d(A) &= A^2 + 7A + 12I \\
&= \begin{bmatrix} -2 & -3 \\ 6 & 7 \end{bmatrix} + 7\begin{bmatrix} 0 & 1 \\ -2 & -3 \end{bmatrix} + 12\begin{bmatrix} 1 & 0 \\ 0 & 1 \end{bmatrix} \\
&= \begin{bmatrix} -2 & -3 \\ 6 & 7 \end{bmatrix} + \begin{bmatrix} 0 & 7 \\ -14 & -21 \end{bmatrix} + \begin{bmatrix} 12 & 0 \\ 0 & 12 \end{bmatrix} \\
&= \begin{bmatrix} 10 & 4 \\ -8 & -2 \end{bmatrix}
\end{align*}

计算 $W_c^{-1}$:
\[W_c^{-1} = \frac{1}{-1}\begin{bmatrix} -3 & -1 \\ -1 & 0 \end{bmatrix} = \begin{bmatrix} 3 & 1 \\ 1 & 0 \end{bmatrix}\]

应用阿克曼公式:
\begin{align*}
K &= [0 \quad 1] W_c^{-1} \alpha_d(A) \\
&= [0 \quad 1] \begin{bmatrix} 3 & 1 \\ 1 & 0 \end{bmatrix} \begin{bmatrix} 10 & 4 \\ -8 & -2 \end{bmatrix} \\
&= [1 \quad 0] \begin{bmatrix} 10 & 4 \\ -8 & -2 \end{bmatrix} \\
&= [10 \quad 4]
\end{align*}

因此,反馈增益:$\boxed{K = [10 \quad 4]}$

\paragraph{第四步:验证设计}

闭环系统矩阵:
\[A - BK = \begin{bmatrix} 0 & 1 \\ -2 & -3 \end{bmatrix} - \begin{bmatrix} 0 \\ 1 \end{bmatrix} [10 \quad 4] = \begin{bmatrix} 0 & 1 \\ -12 & -7 \end{bmatrix}\]

特征方程:
\[\det(sI - (A-BK)) = s^2 + 7s + 12 = (s+3)(s+4)\]

极点确实为 $-3, -4$,\textbf{设计成功}!

\paragraph{性能分析}

\begin{itemize}
    \item 调节时间:$t_s \approx \frac{4}{3} \approx 1.33$s(满足 $< 2$s)
    \item 无超调(实极点)
    \item 控制律:$u = -10x_1 - 4x_2 + v$
\end{itemize}

\subsubsection*{范例总结}

\textbf{关键步骤回顾:}
\begin{enumerate}
    \item 验证能控性(必须满足)
    \item 根据性能要求选择极点(平衡速度和成本)
    \item 计算反馈增益(阿克曼公式最简洁)
    \item 验证闭环极点(检查计算正确性)
\end{enumerate}

\textbf{实际考虑:}
\begin{itemize}
    \item 如果状态不可测,需要设计观测器(下一章)
    \item 极点位置影响控制能量($K$ 越大,$u$ 越大)
    \item 实际系统需要考虑执行器饱和限制
\end{itemize}

\subsection*{本章总结}

\subsubsection*{核心要点回顾}

\textbf{1. 极点配置的本质}
\begin{itemize}
    \item 通过状态反馈 $u = -Kx + v$ 改变系统矩阵
    \item 开环 $A$ $\to$ 闭环 $A - BK$
    \item 从而改变系统的动态特性(极点位置)
\end{itemize}

\textbf{2. 极点配置定理(核心)}
\begin{itemize}
    \item \textbf{完全能控} $\Leftrightarrow$ \textbf{可任意配置极点}
    \item 这是能控性概念的直接应用
    \item 复数极点必须成对共轭出现
\end{itemize}

\textbf{3. 三种计算方法}
\begin{itemize}
    \item \textbf{直接法}:展开特征方程,适合低阶手算
    \item \textbf{变换法}:转为能控标准型,适合理论分析
    \item \textbf{阿克曼公式}:$K = [0 \cdots 0 \quad 1] W_c^{-1} \alpha_d(A)$,\textbf{最实用}
\end{itemize}

\textbf{4. 极点选择原则}
\begin{itemize}
    \item 所有极点在左半平面(稳定性)
    \item 实部大小决定响应速度($t_s \approx 4/|\text{Re}(s)|$)
    \item 虚部/实部比决定超调量($\zeta$)
    \item 平衡性能、鲁棒性和实现成本
\end{itemize}

\subsubsection*{设计流程总结}

\textbf{标准极点配置流程:}
\begin{enumerate}
    \item \textbf{系统分析}
    \begin{itemize}
        \item 写出系统方程 $\dot{x} = Ax + Bu$
        \item 计算开环极点(了解原始特性)
        \item 验证能控性 $\text{rank}(W_c) = n$
    \end{itemize}
    
    \item \textbf{性能要求转化}
    \begin{itemize}
        \item 调节时间 $t_s$ $\to$ 极点实部
        \item 超调量 $M_p$ $\to$ 阻尼比 $\zeta$
        \item 确定期望极点 $\lambda_1, \ldots, \lambda_n$
    \end{itemize}
    
    \item \textbf{控制器计算}
    \begin{itemize}
        \item 选择计算方法(推荐阿克曼公式)
        \item 计算反馈增益 $K$
        \item 控制律 $u = -Kx + v$
    \end{itemize}
    
    \item \textbf{验证与调整}
    \begin{itemize}
        \item 验证闭环极点位置
        \item 仿真闭环响应
        \item 检查控制信号幅值(是否饱和)
        \item 必要时调整极点位置
    \end{itemize}
\end{enumerate}

\subsubsection*{重要概念辨析}

\textbf{极点配置 vs 最优控制:}
\begin{itemize}
    \item \textbf{极点配置}:直接指定极点位置(设计者经验)
    \item \textbf{最优控制(LQR)}:优化性能指标得到极点(自动平衡)
    \item 极点配置更直观,LQR更系统化
\end{itemize}

\textbf{状态反馈 vs 输出反馈:}
\begin{itemize}
    \item 状态反馈需要所有状态(可能需要观测器)
    \item 输出反馈只用输出(但不能任意配置极点)
    \item 实际常用:状态观测器 + 状态反馈
\end{itemize}

\subsubsection*{常见误区}

\begin{itemize}
    \item ✗ 认为能控性不重要(能控性是极点配置的\textbf{充要条件})
    \item ✗ 极点配置得越快越好(要考虑控制饱和和噪声)
    \item ✗ 忽略系统零点(零点不变,会影响实际响应)
    \item ✗ 期望所有极点重合(数值不稳定)
    \item ✓ 仔细选择极点位置,平衡性能和鲁棒性
    \item ✓ 验证闭环系统对参数摄动的敏感性
\end{itemize}

\subsubsection*{实际应用建议}

\textbf{工程实践:}
\begin{itemize}
    \item 从保守的极点位置开始(不要过于激进)
    \item 逐步调整,观察实际系统响应
    \item 考虑测量噪声对高增益的放大效应
    \item 使用仿真验证非线性因素(饱和、死区等)
\end{itemize}

\textbf{MATLAB工具:}
\begin{itemize}
    \item \texttt{ctrb(A,B)}:计算能控性矩阵
    \item \texttt{acker(A,B,p)}:阿克曼公式(单输入)
    \item \texttt{place(A,B,p)}:极点配置(多输入)
    \item \texttt{step(sys)}:阶跃响应仿真
\end{itemize}

\subsubsection*{后续章节预告}

极点配置解决了\textbf{控制器设计}问题,但实际中常面临:
\begin{itemize}
    \item \textbf{状态不可测} $\to$ 需要状态观测器(第11章)
    \item \textbf{性能指标优化} $\to$ 最优控制理论(LQR)
    \item \textbf{鲁棒性要求} $\to$ 鲁棒控制方法
\end{itemize}

极点配置是现代控制理论的\textbf{第一个综合性设计方法}。它将前面学习的能控性概念转化为实际的控制器,体现了从理论到实践的完整链条:

\begin{center}
\textbf{能控性} $\to$ \textbf{极点配置} $\to$ \textbf{性能实现}
\end{center}

掌握极点配置,就掌握了状态反馈控制的核心技术!
