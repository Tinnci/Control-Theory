\section{状态空间表达式及其建立}
\label{sec:state-space}

\subsection*{引言:从输入输出到内部状态}

当你使用手机时,你只关心按下按钮后屏幕的响应——这是\textbf{输入输出关系}。但手机内部CPU有数十亿晶体管,每个晶体管的状态(开或关)共同决定了系统的行为。这些内部状态对用户不可见,却是理解和设计系统的关键。

\textbf{经典控制理论的局限:}传统的频域方法(传递函数、频率响应)只关注系统的\textbf{外部行为}——输入如何影响输出。这种方法对于简单的单输入单输出(SISO)系统非常有效,但面对现代工程中的复杂系统时遇到了困难:

\begin{itemize}
    \item \textbf{多变量系统}:飞机有多个控制面(副翼、升降舵、方向舵),如何协调控制?
    \item \textbf{初始条件}:火箭发射时的初始状态(位置、速度)如何影响轨迹?传递函数假设初始条件为零,无法处理这类问题。
    \item \textbf{时变与非线性}:卫星轨道随时间变化,机器人关节有摩擦非线性,传递函数方法难以适用。
\end{itemize}

\textbf{现代控制理论的突破:}20世纪60年代,\textbf{卡尔曼(Kalman)}等学者提出了\textbf{状态空间方法}(State Space Approach),将系统的\textbf{内部状态}显式地纳入描述。这种方法不仅能处理多变量系统,还为最优控制、自适应控制等高级主题奠定了基础。

\textbf{本章的核心目标:}
\begin{itemize}
    \item 理解\textbf{状态}的概念:为什么需要它?它代表什么?
    \item 掌握\textbf{状态空间表达式}:现代控制理论的标准数学语言
    \item 学会\textbf{建立状态空间模型}:从物理系统到数学描述
    \item 认识\textbf{状态选择的非唯一性}:同一系统可以有不同的状态描述
\end{itemize}

\textbf{章节路线图:}
\begin{enumerate}
    \item 状态的基本概念(什么是状态?为什么需要状态?)
    \item 状态空间表达式的标准形式(数学框架)
    \item 建立方法与实例(从物理规律到状态方程)
    \item 状态选择的多样性(同一系统的不同描述)
\end{enumerate}

\subsection{状态空间的基本概念}

\subsubsection*{本节目的}
在引入严格的数学定义之前,我们需要建立对「状态」的直观理解:它是什么?为什么需要它?如何选择它?

\subsubsection{状态的定义与物理意义}

\textbf{状态的严格定义:}

\textbf{状态(State)}是系统在某一时刻的\textbf{最少信息集合},它能够与输入一起,\textbf{唯一确定}系统未来的行为。

\textbf{通俗解释:}
\begin{itemize}
    \item \textbf{最少信息}:不多不少,恰好够用。比如描述单摆运动,需要角度和角速度两个量,缺一不可。
    \item \textbf{唯一确定未来}:给定当前状态 $x(t_0)$ 和未来的输入 $u(t)$,就能计算出任意未来时刻 $t > t_0$ 的状态 $x(t)$。
    \item \textbf{与历史无关}:只要知道当前状态,就不需要知道系统过去的轨迹。这是\textbf{马尔可夫性质}。
\end{itemize}

\textbf{相关概念:}

\begin{itemize}
    \item \textbf{状态变量(State Variables)}:描述系统状态的一组变量,记为 $x_1, x_2, \ldots, x_n$。
    \item \textbf{状态向量(State Vector)}:由状态变量组成的列向量:
    \[
    x(t) = \begin{bmatrix} x_1(t) \\ x_2(t) \\ \vdots \\ x_n(t) \end{bmatrix} \in \mathbb{R}^n
    \]
    \item \textbf{状态空间(State Space)}:以状态变量为坐标轴的 $n$ 维欧几里得空间 $\mathbb{R}^n$。系统在某时刻的状态对应状态空间中的一个点,系统的演化对应状态空间中的一条轨迹。
\end{itemize}

\subsubsection{状态的物理意义:能量存储的视角}

\textbf{核心洞察:}在物理系统中,\textbf{状态变量通常与能量存储元件相关}。

\begin{itemize}
    \item \textbf{电路系统}:
    \begin{itemize}
        \item 电容电压 $v_C$:存储电场能量 $E = \frac{1}{2}CV_C^2$
        \item 电感电流 $i_L$:存储磁场能量 $E = \frac{1}{2}LI_L^2$
    \end{itemize}
    \item \textbf{机械系统}:
    \begin{itemize}
        \item 位置 $x$:存储势能(如弹簧)$E = \frac{1}{2}kx^2$
        \item 速度 $v$:存储动能 $E = \frac{1}{2}mv^2$
    \end{itemize}
    \item \textbf{热力学系统}:温度、压力(存储热能)
    \item \textbf{化学系统}:浓度(存储化学能)
\end{itemize}

\textbf{为什么能量存储很重要?}因为能量不能瞬间改变——电容电压不能突变(需要充电时间),质量速度不能突变(需要加速度)。正是这些\textbf{能量存储机制}使得系统具有\textbf{动态特性},而状态变量就是描述这些动态的数学工具。

\textbf{实用原则:}
\begin{quote}
    \textit{系统的状态变量个数} $n$ = \textit{独立能量存储元件个数}
\end{quote}

例如,一个包含 2 个电容和 3 个电感的电路,通常有 $n = 5$ 个状态变量。

\subsection{状态空间表达式的标准形式}

\subsubsection*{本节目的}
建立状态空间方法的数学框架——这是现代控制理论的标准语言。

\subsubsection{线性定常系统的状态空间表达式}

\textbf{标准形式:}

对于线性时不变(LTI, Linear Time-Invariant)系统,状态空间表达式由两个方程组成:

\begin{empheq}[box=\fbox]{align}
\dot{x}(t) &= Ax(t) + Bu(t) \quad \text{(状态方程)} \label{eq:ss:state} \\
y(t) &= Cx(t) + Du(t) \quad \text{(输出方程)} \label{eq:ss:output}
\end{empheq}

\textbf{各变量的含义:}

\begin{itemize}
    \item $x(t) \in \mathbb{R}^n$:\textbf{状态向量}($n$ 个状态变量)
    \item $u(t) \in \mathbb{R}^p$:\textbf{输入向量}($p$ 个控制输入)
    \item $y(t) \in \mathbb{R}^q$:\textbf{输出向量}($q$ 个可测量输出)
    \item $A \in \mathbb{R}^{n \times n}$:\textbf{系统矩阵}(描述系统自身动态)
    \item $B \in \mathbb{R}^{n \times p}$:\textbf{输入矩阵}(描述输入如何影响状态)
    \item $C \in \mathbb{R}^{q \times n}$:\textbf{输出矩阵}(描述哪些状态可测量)
    \item $D \in \mathbb{R}^{q \times p}$:\textbf{前馈矩阵}(描述输入到输出的直接传递)
\end{itemize}

\textbf{方程的物理意义:}

\begin{itemize}
    \item \textbf{状态方程} \eqref{eq:ss:state}:描述\textbf{状态如何随时间演化}。$\dot{x}$ 是状态的变化率,由当前状态 $x$ 和输入 $u$ 共同决定。
    \begin{quote}
        \textit{状态方程回答:「系统下一时刻会变成什么样?」}
    \end{quote}
    
    \item \textbf{输出方程} \eqref{eq:ss:output}:描述\textbf{哪些量是可观测的}。并非所有状态都能直接测量(如电机内部温度),输出方程告诉我们能测到什么。
    \begin{quote}
        \textit{输出方程回答:「我们能看到什么?」}
    \end{quote}
\end{itemize}

\textbf{关键特性:}

\begin{itemize}
    \item \textbf{线性}:方程右边是 $x$ 和 $u$ 的线性组合(没有 $x^2$, $\sin(x)$ 等非线性项)
    \item \textbf{定常}:矩阵 $A, B, C, D$ 不随时间变化(系统参数恒定)
    \item \textbf{因果性}:当前输出只依赖于当前状态和输入,不依赖未来
\end{itemize}

\subsubsection{系统矩阵的作用分解}

为了更深入理解,我们可以将状态方程 $\dot{x} = Ax + Bu$ 写成分量形式:

\[
\begin{bmatrix} \dot{x}_1 \\ \dot{x}_2 \\ \vdots \\ \dot{x}_n \end{bmatrix}
= \begin{bmatrix} 
a_{11} & a_{12} & \cdots & a_{1n} \\
a_{21} & a_{22} & \cdots & a_{2n} \\
\vdots & \vdots & \ddots & \vdots \\
a_{n1} & a_{n2} & \cdots & a_{nn}
\end{bmatrix}
\begin{bmatrix} x_1 \\ x_2 \\ \vdots \\ x_n \end{bmatrix}
+ \begin{bmatrix} 
b_{11} & \cdots & b_{1p} \\
b_{21} & \cdots & b_{2p} \\
\vdots & \ddots & \vdots \\
b_{n1} & \cdots & b_{np}
\end{bmatrix}
\begin{bmatrix} u_1 \\ u_2 \\ \vdots \\ u_p \end{bmatrix}
\]

\textbf{矩阵元素的物理意义:}

\begin{itemize}
    \item $a_{ij}$:状态 $x_j$ 对状态 $x_i$ 变化率的影响系数(\textbf{状态耦合})
    \item $b_{ij}$:输入 $u_j$ 对状态 $x_i$ 变化率的影响系数(\textbf{控制增益})
    \item $c_{ij}$:状态 $x_j$ 对输出 $y_i$ 的贡献系数(\textbf{观测增益})
    \item $d_{ij}$:输入 $u_j$ 对输出 $y_i$ 的直接影响(\textbf{前馈项},物理系统中常为零)
\end{itemize}

\textbf{直观理解:}
\begin{itemize}
    \item $A$ 矩阵的\textbf{对角元} $a_{ii}$ 描述状态 $x_i$ 自身的衰减或增长
    \item $A$ 矩阵的\textbf{非对角元} $a_{ij}$ 描述状态之间的相互作用
    \item $B$ 矩阵的第 $i$ 行描述哪些输入能影响状态 $x_i$
    \item $C$ 矩阵的第 $i$ 行描述输出 $y_i$ 由哪些状态组成
\end{itemize}

\subsection{状态空间表达式的建立方法}

\subsubsection*{本节目的}
将物理系统转化为状态空间描述的系统性方法——从牛顿定律、基尔霍夫定律到状态方程。

\subsubsection{建立步骤}

\textbf{标准流程:}

\begin{enumerate}
    \item \textbf{根据物理规律建立系统的微分方程}
    \begin{itemize}
        \item 机械系统:牛顿第二定律 $F = ma$
        \item 电路系统:基尔霍夫电压定律(KVL)、电流定律(KCL)
        \item 热力学系统:能量守恒定律
    \end{itemize}
    
    \item \textbf{选择状态变量}
    \begin{itemize}
        \item \textbf{原则}:选择独立能量存储元件的相关变量
        \item 电路:电容电压、电感电流
        \item 机械:位置、速度(对于二阶系统)
        \item \textbf{注意}:状态变量必须\textbf{线性独立}
    \end{itemize}
    
    \item \textbf{将高阶微分方程化为一阶微分方程组}
    \begin{itemize}
        \item 引入中间变量降阶
        \item 整理成 $\dot{x} = Ax + Bu$ 的形式
    \end{itemize}
    
    \item \textbf{写出输出方程}
    \begin{itemize}
        \item 确定哪些量是可测量的
        \item 用状态变量表示输出:$y = Cx + Du$
    \end{itemize}
\end{enumerate}

\subsubsection{范例1:RLC串联电路}

\textbf{系统描述:}

考虑如图所示的RLC串联电路,输入为电压源 $u(t)$,输出为电容电压 $y(t) = v_C(t)$。

\begin{center}
\begin{circuitikz}
    \draw (0,0) to[V, v=$u(t)$] (0,3)
    to[R, l=$R$] (3,3)
    to[L, l=$L$, i=$i(t)$] (6,3)
    to[C, l=$C$, v=$v_C(t)$] (6,0) -- (0,0);
\end{circuitikz}
\end{center}

\textbf{步骤1:建立微分方程}

根据基尔霍夫电压定律(KVL):
\[
u(t) = R i(t) + L \frac{di(t)}{dt} + v_C(t)
\]

根据电容的伏安关系:
\[
i(t) = C \frac{dv_C(t)}{dt}
\]

\textbf{步骤2:选择状态变量}

系统有两个能量存储元件(电感和电容),因此选择 $n=2$ 个状态变量:
\[
x_1(t) = v_C(t) \quad \text{(电容电压)}, \quad x_2(t) = i(t) \quad \text{(电感电流)}
\]

\textbf{步骤3:建立状态方程}

对电容电压求导:
\[
\dot{x}_1 = \frac{dv_C}{dt} = \frac{i}{C} = \frac{1}{C} x_2
\]

对电感电流求导,从KVL方程得:
\[
\dot{x}_2 = \frac{di}{dt} = \frac{1}{L}(u - Ri - v_C) = -\frac{1}{L}x_1 - \frac{R}{L}x_2 + \frac{1}{L}u
\]

写成矩阵形式:
\[
\begin{bmatrix} \dot{x}_1 \\ \dot{x}_2 \end{bmatrix}
= \begin{bmatrix} 0 & \frac{1}{C} \\ -\frac{1}{L} & -\frac{R}{L} \end{bmatrix}
\begin{bmatrix} x_1 \\ x_2 \end{bmatrix}
+ \begin{bmatrix} 0 \\ \frac{1}{L} \end{bmatrix} u
\]

\textbf{步骤4:输出方程}

输出为电容电压:
\[
y = v_C = x_1 = \begin{bmatrix} 1 & 0 \end{bmatrix} \begin{bmatrix} x_1 \\ x_2 \end{bmatrix}
\]

\textbf{最终结果:}
\[
A = \begin{bmatrix} 0 & \frac{1}{C} \\ -\frac{1}{L} & -\frac{R}{L} \end{bmatrix}, \quad
B = \begin{bmatrix} 0 \\ \frac{1}{L} \end{bmatrix}, \quad
C = \begin{bmatrix} 1 & 0 \end{bmatrix}, \quad
D = 0
\]

\textbf{物理意义解读:}
\begin{itemize}
    \item $a_{12} = \frac{1}{C}$:电感电流增大使电容电压上升(充电)
    \item $a_{21} = -\frac{1}{L}$:电容电压增大使电感电流减小(反向电动势)
    \item $a_{22} = -\frac{R}{L}$:电阻导致电流自然衰减
    \item $b_2 = \frac{1}{L}$:输入电压直接驱动电感电流变化
\end{itemize}

\subsubsection{范例2:质量-弹簧-阻尼系统}

\textbf{系统描述:}

质量 $m$ 通过弹簧(刚度 $k$)和阻尼器(阻尼系数 $c$)连接到固定端,外力 $u(t)$ 作用在质量上。

\textbf{步骤1:建立微分方程}

根据牛顿第二定律:
\[
m \ddot{x} + c \dot{x} + k x = u(t)
\]

\textbf{步骤2:选择状态变量}

这是二阶系统,选择位置和速度作为状态:
\[
x_1 = x \quad \text{(位置)}, \quad x_2 = \dot{x} \quad \text{(速度)}
\]

\textbf{步骤3:降阶为一阶方程组}

\[
\begin{cases}
\dot{x}_1 = x_2 \\
\dot{x}_2 = \ddot{x} = \frac{1}{m}(u - cx_2 - kx_1) = -\frac{k}{m}x_1 - \frac{c}{m}x_2 + \frac{1}{m}u
\end{cases}
\]

矩阵形式:
\[
\begin{bmatrix} \dot{x}_1 \\ \dot{x}_2 \end{bmatrix}
= \begin{bmatrix} 0 & 1 \\ -\frac{k}{m} & -\frac{c}{m} \end{bmatrix}
\begin{bmatrix} x_1 \\ x_2 \end{bmatrix}
+ \begin{bmatrix} 0 \\ \frac{1}{m} \end{bmatrix} u
\]

\textbf{步骤4:输出方程}

假设输出为位置:
\[
y = x_1 = \begin{bmatrix} 1 & 0 \end{bmatrix} \begin{bmatrix} x_1 \\ x_2 \end{bmatrix}
\]

\textbf{观察:二阶系统的标准模式}

注意到 $A$ 矩阵的结构:
\[
A = \begin{bmatrix} 0 & 1 \\ -\omega_n^2 & -2\zeta\omega_n \end{bmatrix}
\]
其中 $\omega_n = \sqrt{k/m}$ 是固有频率,$\zeta = c/(2\sqrt{km})$ 是阻尼比。这是\textbf{二阶系统的能控标准型}。

\subsection{状态选择的非唯一性}

\subsubsection*{本节目的}
认识到同一系统可以有无穷多种状态空间描述——理解这种非唯一性及其影响。

\subsubsection{核心问题}

\textbf{关键洞察:}对于同一物理系统,状态变量的选择\textbf{不是唯一的}。不同的状态选择会导致不同的矩阵 $(A, B, C, D)$,但它们描述的是\textbf{同一系统}。

\textbf{为什么会有非唯一性?}

状态是描述系统的\textbf{坐标系},就像地理位置可以用经纬度、也可以用直角坐标描述一样。数学上,不同的状态选择通过\textbf{线性变换}联系:
\[
\bar{x} = Tx, \quad T \text{ 为可逆矩阵}
\]

\textbf{什么保持不变?}

虽然矩阵 $(A, B, C, D)$ 会变,但系统的本质特性不变:
\begin{itemize}
    \item \textbf{特征值}(极点):决定稳定性和响应速度
    \item \textbf{能控性}:能否通过输入控制所有状态
    \item \textbf{能观性}:能否从输出推断所有状态
    \item \textbf{传递函数}:输入输出关系
\end{itemize}

\subsubsection{范例:同一RLC电路的不同状态选择}

回到前面的RLC电路,我们还可以做不同的状态选择。

\textbf{方案1(前面已用)}:$x_1 = v_C$, $x_2 = i$

\[
A_1 = \begin{bmatrix} 0 & \frac{1}{C} \\ -\frac{1}{L} & -\frac{R}{L} \end{bmatrix}
\]

\textbf{方案2}:引入电感磁链 $\lambda = Li$ 作为状态

选择 $\bar{x}_1 = v_C$, $\bar{x}_2 = \lambda = Li$,则:
\[
\begin{cases}
\dot{\bar{x}}_1 = \frac{i}{C} = \frac{1}{LC}\bar{x}_2 \\
\dot{\bar{x}}_2 = L\dot{i} = u - Ri - v_C = -\bar{x}_1 - \frac{R}{L}\bar{x}_2 + u
\end{cases}
\]

得到新的系统矩阵:
\[
A_2 = \begin{bmatrix} 0 & \frac{1}{LC} \\ -1 & -\frac{R}{L} \end{bmatrix}
\]

\textbf{方案3}:能量坐标(高级)

选择 $z_1 = \sqrt{\frac{C}{L}}v_C$, $z_2 = \sqrt{L}i$,这是归一化的能量坐标。

\textbf{三种方案的比较:}

\begin{center}
\begin{tabular}{lccc}
\hline
特性 & 方案1 & 方案2 & 方案3 \\
\hline
物理直观 & ✓✓ & ✓ & △ \\
数值稳定性 & 依赖参数 & 依赖参数 & ✓ \\
特征值 & \multicolumn{3}{c}{相同} \\
传递函数 & \multicolumn{3}{c}{相同} \\
\hline
\end{tabular}
\end{center}

\textbf{实际选择原则:}

\begin{itemize}
    \item \textbf{物理意义清晰}:便于理解和调试(工程首选)
    \item \textbf{数值稳定性}:矩阵元素量级相近,避免病态问题
    \item \textbf{标准型}:便于分析和控制器设计(理论分析首选)
    \item \textbf{测量可行性}:选择可直接或间接测量的变量
\end{itemize}

\subsection*{本章总结}

\subsubsection*{核心要点}

\begin{enumerate}
    \item \textbf{状态的本质}:系统在某时刻确定未来行为所需的最少信息,通常与能量存储相关。
    
    \item \textbf{状态空间表达式}:现代控制理论的标准语言
    \[
    \dot{x} = Ax + Bu, \quad y = Cx + Du
    \]
    
    \item \textbf{建立方法}:物理规律 → 微分方程 → 状态选择 → 一阶方程组 → 输出方程
    
    \item \textbf{非唯一性}:状态选择不唯一,但系统本质特性(特征值、能控性、能观性、传递函数)不变。
\end{enumerate}

\subsubsection*{与其他章节的联系}

\begin{itemize}
    \item \textbf{第\ref{sec:transfer-function}章}:如何从状态空间表达式求传递函数?两种描述方法如何转换?
    \item \textbf{第\ref{sec:linear-transformation}章}:线性变换如何联系不同的状态描述?
    \item \textbf{第\ref{sec:solving-state-space}章}:如何求解状态方程,得到系统的时间响应?
    \item \textbf{第\ref{sec:controllability-observability}章}:能控性和能观性——系统可控和可观测的充要条件
    \item \textbf{第\ref{sec:pole-placement}章}:状态反馈控制——如何设计反馈增益 $K$ 使 $u = -Kx$ 达到期望性能?
\end{itemize}

\subsubsection*{常见误区与澄清}

\begin{enumerate}
    \item \textbf{误区}:状态就是输出。\\
    \textbf{澄清}:状态是内部变量,输出是可测量的外部变量。例如,电机的内部温度是状态,但不一定是输出。
    
    \item \textbf{误区}:状态变量个数等于微分方程阶数。\\
    \textbf{澄清}:通常是这样,但要注意约束条件。如果变量间有代数约束(如 $x_1 + x_2 = 0$),实际自由度会减少。
    
    \item \textbf{误区}:矩阵 $D$ 总是零。\\
    \textbf{澄清}:多数物理系统 $D = 0$(输入不直接影响输出),但某些系统(如带前馈的控制系统)$D \neq 0$。
    
    \item \textbf{误区}:不同的状态选择导致不同的系统。\\
    \textbf{澄清}:它们描述的是\textbf{同一系统},只是坐标系不同。就像同一个向量用不同基表示,向量本身没变。
\end{enumerate}

\subsubsection*{MATLAB工具箱}

\begin{lstlisting}[style=Matlab-editor, caption={状态空间模型的创建与操作}]
% 创建状态空间模型
A = [0 1; -2 -3];
B = [0; 1];
C = [1 0];
D = 0;
sys = ss(A, B, C, D);

% 查看系统属性
pole(sys)          % 系统极点
tzero(sys)         % 系统零点
dcgain(sys)        % 直流增益

% 时域仿真
t = 0:0.01:10;
u = ones(size(t));   % 阶跃输入
x0 = [1; 0];         % 初始状态
[y, t, x] = lsim(sys, u, t, x0);

% 绘制状态轨迹
figure;
plot(x(:,1), x(:,2));
xlabel('x_1'); ylabel('x_2');
title('State Trajectory');
grid on;
\end{lstlisting}

\subsubsection*{学习检查清单}

完成本章学习后,你应该能够:

\begin{itemize}
    \item[$\square$] 解释状态的概念,并说明为什么需要状态空间方法
    \item[$\square$] 写出状态空间表达式的标准形式,并解释各矩阵的物理意义
    \item[$\square$] 给定物理系统(电路、机械),能建立其状态空间模型
    \item[$\square$] 识别能量存储元件,并合理选择状态变量
    \item[$\square$] 理解状态选择的非唯一性,并判断不同描述是否等价
    \item[$\square$] 使用MATLAB创建和操作状态空间模型
    \item[$\square$] 区分状态、输入、输出的概念
    \item[$\square$] 理解状态空间方法相比传递函数的优势
\end{itemize}

\textbf{下一步:}在建立了状态空间模型后,我们需要解决两个核心问题:(1)如何从状态空间得到传递函数?(2)如何求解状态方程?这将在接下来的第\ref{sec:transfer-function}、\ref{sec:solving-state-space}章中讨论。
