\section{频率特性基本概念}

\subsection{频率特性的定义}
线性定常系统的频率特性是指当输入为正弦信号时,系统稳态输出与输入的复数比:
\[G(\jw) = G(s)|_{s=\jw} = |G(\jw)|e^{j\phi(\omega)}\]

其中:
\begin{itemize}
    \item $|G(\jw)|$:幅频特性
    \item $\phi(\omega) = \angle G(\jw)$:相频特性
\end{itemize}

\subsection{频率特性的物理意义}
\begin{itemize}
    \item 幅频特性表示不同频率正弦输入信号通过系统后幅值的变化
    \item 相频特性表示不同频率正弦输入信号通过系统后相位的变化
\end{itemize}
